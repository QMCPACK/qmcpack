\documentclass[oneside,11pt]{memoir}

\usepackage{amsmath}
\usepackage{graphicx}

\usepackage[usenames,dvipsnames]{color}
\usepackage[pdftex, pdfstartview = {FitH}]{hyperref}

\usepackage{anyfontsize}

\usepackage{makeidx}


% to use minted
%   install the pygments package for python
%   sudo apt-get install python-pygments
%   get the style file, minted.sty
%     http://code.google.com/p/minted/downloads/detail?name=minted.sty
%   resources
%     http://stackoverflow.com/questions/1966425/source-code-highlighting-in-latex#1985330
%     http://mirror.unl.edu/ctan/macros/latex/contrib/minted/minted.pdf
%     also see minted.pdf in practical documentation
%   running pdflatex:
%     pdflatex -shell-escape nexus_user_guide.tex
\usepackage{minted}


\hypersetup{
    colorlinks=true,
    citecolor = Blue,
    linkcolor = Blue,
    urlcolor  = Blue,
    pdfborder={0 0 0},
}


\makeindex

\chapterstyle{hangnum}
%\chapterstyle{verville}


\numberwithin{equation}{section}

\setlength{\hoffset}{0in}
\setlength{\voffset}{-0.5in}
\setlength{\marginparwidth}{1.0in}
\setlength{\oddsidemargin}{1.0in}
\setlength{\evensidemargin}{1.0in}
\addtolength{\textwidth}{1.0in}
\addtolength{\textheight}{1.0in}
\setlength{\spinemargin}{1.25in}
\setlength{\foremargin}{1.25in}
\addtolength{\textwidth}{-8pt}
\checkandfixthelayout


\newcommand{\hide}[1]{}
%\newcommand{\hide}[1]{{#1}}

%\newcommand{\condense}[1]{This content has been temporarily redacted to produce a shorter document.}
\newcommand{\condense}[1]{{#1}}


  \newcommand{\abs}[1]{\lvert #1 \rvert}
  \newcommand{\norm}[1]{\lVert #1 \rVert}
  \newcommand{\pnorm}[2]{\lVert #1 \rVert_{#2}}
  \newcommand{\mean}[1]{\langle #1 \rangle}

  \newcommand{\ket}[1]{\lvert #1 \rangle}
  \newcommand{\bra}[1]{\langle #1 \rvert}
  \newcommand{\expval}[3]{\bra{#1}\hat{#2}\ket{#3}}
  \newcommand{\expvalnh}[3]{\bra{#1}#2\ket{#3}}
  \newcommand{\overlap}[2]{\langle #1 \lvert #2 \rangle}


  \newcommand{\sstitle}[1]{\textbf{\underline{#1}}\newline}
  \newcommand{\bu}[1]{\textbf{\underline{#1}}}

  \newcommand{\HRule}{\rule{\linewidth}{0.5mm}}



\frontmatter


\newenvironment{changemargin}[2]{%
\begin{list}{}{%
%\setlength{\topsep}{0pt}%
\setlength{\leftmargin}{#1}%
\setlength{\rightmargin}{#2}%
%\setlength{\listparindent}{\parindent}%
%\setlength{\itemindent}{\parindent}%
%\setlength{\parsep}{\parskip}%
}%
\item[]}{\end{list}}




\begin{document}

\definecolor{shadecolor}{gray}{0.9}


\thispagestyle{empty}
\begin{changemargin}{-1cm}{-1cm}
  \begin{center}
    \hspace{1cm}\\
    \hspace{1cm}\\
    \hspace{1cm}\\
    \hspace{1cm}\\
    \hspace{1cm}\\
    \hspace{1cm}\\
    \hspace{1cm}\\
    \hspace{1cm}\\
    \hspace{1cm}\\
    \hspace{1cm}\\
    \hspace{1cm}\\
    \hspace{1cm}\\
    \HRule\\
    \vspace{4mm}
    \textbf{\fontsize{40}{45}\selectfont The Nexus User Guide} \\ 
    \HRule\\
    \vspace{1cm}
    %\textbf{\fontsize{35}{40}\selectfont User Guide}\\
    \vspace{6cm}
    By Jaron T. Krogel \\
    \hspace{1cm}\\
    22 May 2013
  \end{center}
\end{changemargin}
\pagebreak

\tableofcontents



\mainmatter

\pagebreak
\chapter{Using this document} \label{usedoc}

The Nexus User Guide provides an overview of Nexus 
(\ref{overview}), instructions on how to install it (\ref{installation}), 
complete examples of electronic structure calculations using it 
(\ref{examples}), a complete reference section (\ref{reference}), 
a brief overview of Quantum Monte Carlo (QMC) from an applied perspective 
(\ref{theory}), and directions on where to go to learn more (\ref{reading}).  
If you are new to QMC, consider reading the
 ``QMC Practice in a Nutshell'' section (\ref{theory}) and the 
review articles and online resources listed under
 ``Quantum Monte Carlo: Theory and Practice'' (\ref{learn_qmc}) 
before proceeding to the overview (\ref{overview}) and the examples 
(\ref{examples}).  For those more experienced in QMC, or the impatient, 
quickly visit ``Nexus Installation'' (\ref{installation}) and see the 
examples section (\ref{examples}) for template calculations to begin 
using Nexus immediately.  For fine-grained information 
about Nexus's many features, consult 
the ``Nexus User Reference'' (\ref{reference}).
If you cannot find what you need in this document, contact the main 
developer of Nexus (Jaron Krogel), at krogeljt@ornl.gov 
(but please make a thorough search first!).



\pagebreak
\chapter{Overview of Nexus} \label{overview}
\section{What Nexus is}
Nexus is a collection of tools, written in Python, to perform 
complex electronic structure calculations and analyze the results.  The main 
focus is currently on performing arbitrary Quantum Monte Carlo (QMC) 
calculations with QMCPACK.  A single QMC calculation typically requires several 
previous calculations with other codes to produce a starting guess for the 
many-body wavefunction and convert it into a form that QMCPACK understands.  
Managing the resulting array of calculations, and the flow of information 
between them, quickly becomes unweildy to the researcher, demands a great 
deal of human time, and increases the potential for human error.  Nexus 
reduces both the human time required and potential for error by 
automating the total simulation process.  

\section{What Nexus can do}
The capabilities of Nexus currently include crystal structure 
generation, standalone Density Functional Theory (DFT) calculations with PWSCF, 
Hartree-Fock (HF) calculations of atoms with the SQD code (packaged with 
QMCPACK), complete QMC calculations with QMCPACK (including wavefunction 
optimization, Variational Monte Carlo (VMC), and Diffusion Monte Carlo (DMC) in 
periodic or open boundary conditions), automated job management on workstations 
(by acting as a virtual queue) and clusters/supercomputers (such as OIC5, 
Edison, Blue Waters, with Titan coming) including handling of dependencies 
between calculations and bundling of jobs,  and extraction of results from 
completed calculations for analysis.  The integration of these capabilities 
permits the user to focus on the high-level tasks of problem formulation and 
interpretation of the results without (in principle) becoming too involved 
in the time-consuming, lower level details.

\section{How Nexus is used}
Use of Nexus currently involves writing a short Python script 
describing the calculations to be performed.  This small script formed by the 
user closely resembles an input file for electronic structure codes.  A key 
difference is that this ``input file'' represents executable code, and so 
variables are easily defined for use in expressions and more complicated 
simulation workflows (\emph{e.g.} an equation of state) can be constructed 
with if/else logic and for loops.  Knowledge of the Python programming language 
is helpful to perform complex calculations, but not essential for use of  
Nexus.  Starting from working ``input files'' such as those covered 
in the ``Complete Examples'' section (\ref{examples}) is a good way to proceed. 


\pagebreak
\chapter{Nexus Installation} \label{installation}
Installation of Nexus can be accomplished by a single download 
with Subversion (SVN) and setting a single environment variable provided a 
working python environment exists. Follow the example below to download  
Nexus:
\begin{shaded}
\begin{verbatim}
cd /your_download_path
svn co https://subversion.assembla.com/svn/qmcdev/trunk/nexus
\end{verbatim}
\end{shaded}
If you do not have access to the Assembla SVN repository, please make an 
account on assembla.com and email the lead developer of QMCPACK (Jeongnim Kim) 
at jnkim@ornl.gov to obtain access.

To make your Python installation (must be Python 2.x as 3.x is not supported) 
aware of Nexus, simply set the 
PYTHONPATH environment variable.  For example, in bash this would look like:
\begin{shaded}
\begin{verbatim}
export PYTHONPATH=/your_download_path/nexus/library
\end{verbatim}
\end{shaded}
If you want to use the command line tools, add them to your path:
\begin{shaded}
\begin{verbatim}
export PATH=/your_download_path/nexus/executables:$PATH
\end{verbatim}
\end{shaded}
Add these to \emph{e.g.} your .bashrc file to make Nexus available 
to future sessions.

In addition to the standard Python installation, the \texttt{numpy} module must 
be installed for Nexus to function at a basic level.  To realize 
the full range of functionality available, it is recommended that the 
\texttt{scipy}, \texttt{matplotlib}, and \texttt{h5py} modules be installed as 
well.  Many of these packages are already available in various supercomputing 
environments.  On a debian-based Linux system, such as Ubuntu, installation of 
these python modules is easily accomplished by invoking the following at the 
command line:
\begin{shaded}
\begin{verbatim}
sudo apt-get install python-numpy
sudo apt-get install python-scipy python-matplotlib python-h5py 
\end{verbatim}
\end{shaded}
For installing the Python modules on other platforms, please see 
``Helpful Links for Installing Python Modules'' (section \ref{install_python}).

Of course, to run full calculations, the simulation codes and converters 
involved must be installed as well.  These include a modified version of 
Quantum Espresso (\texttt{pw.x}, \texttt{pw2qmcpack.x}, optionally 
\texttt{pw2casino.x}), QMCPACK (\texttt{qmcapp}, \texttt{qmcapp\_complex}), 
SQD (\texttt{sqd}, packaged with QMCPACK), and, optionally, \texttt{wfconvert}.  
Complete coverage of this task is beyond the scope of the current document, but 
please see ``Helpful Links for Installing Electronic Structure Codes'' 
(section \ref{install_code}).


\pagebreak
\chapter{Complete Examples} \label{examples}

\bu{Disclaimer:} Please note that the examples given here do not generally qualify as production 
calculations because the supercell size, optimization process, DMC timestep and 
other key parameters may not be converged.  Pseudopotentials are provided 
``as is'' and should not be trusted without explicit validation.

Complete examples of calculations performed with Nexus are provided 
in the following sections.  These examples are intended to highlight basic 
features of Nexus and act as templates for future calculations.  
A complete description of the available features can be found in ``Nexus 
User Reference'' (section \ref{reference}).  If there is an example you 
would like to contribute, or if you feel an example on a particular topic is 
needed, please contact the developer at krogeljt@ornl.gov to discuss the 
possibilities.  

To perform the example calculations yourself, consult the \texttt{examples} 
directory in your Nexus installation:
\begin{shaded}
\begin{verbatim}
/your_download_path/nexus/examples
\end{verbatim}
\end{shaded}
The examples assume that you have working versions of \texttt{pw.x}, 
\texttt{pw2qmcpack.x}, \texttt{qmcapp} (real version), and 
\texttt{qmcapp\_complex} (complex version) installed and in your \texttt{PATH}. 
A brief description of each example is given below.  

\begin{description}
  \item[Graphene Sheet DMC] \hfill \\
    A representative bulk calculation.  The total DMC energy of a graphene 
    ``sheet'' consisting of 8 atoms is computed.  DFT is performed with 
    PWSCF on the primitive cell followed by Jastrow optimization by QMCPACK 
    and finally a supercell VMC+DMC calculation by QMCPACK.  

  \item[C 20 Molecule DMC] \hfill  \\
    A representative molecular calculation.  The total DMC energy of an ideal 
    C 20 molecule is computed.  DFT is performed with PWSCF on a periodic cell 
    with some vacuum surrounding the molecule.  QMCPACK optimization and 
    VMC+DMC follow on the system with open boundary conditions.  

    (Note that without the crystal field splitting afforded by the initial 
    artificial periodicity, the Kohn-Sham HOMO would be degenerate, and so a 
    production calculation would likely require more care in appropriately 
    setting up the wavefunction.)
\end{description}


%\pagebreak
%\section{DFT Calculations with PWSCF}
%
%\pagebreak
%\subsection{Example: Converging Ge T interstitial structure w.r.t. k-points}
%
%  Pseudopotentials                                     # read in pseudopotentials
%    reading pp:  ./pseudopotentials/Ge.pbe-kjpaw.UPF
%upfPP Warning:                                         # ignore this warning
%    ability to read new UPF format has not yet been implemented
%      attempted to read ./pseudopotentials/Ge.pbe-kjpaw.UPF
%
%Project starting                              # start the project
%  checking for file collisions                # ensure files don't overlap   
%  loading cascade images                      # load simulation cascades
%    cascade 0 checking in
%  checking cascade dependencies               # check simulation dependencies 
%    all simulation dependencies satisfied
%
%  starting runs:                              # run all project jobs
%  ~~~~~~~~~~~~~~~~~~~~~~~~~~~~~~
%  poll 0  memory 56.40 MB
%    Entering ./runs/relax/kgrid_111 0         # 1x1x1 mesh relaxation starts
%      writing input files  0 relax            # write input files
%    Entering ./runs/relax/kgrid_111 0
%      sending required files  0 relax         # copy PP files
%      submitting job  0 relax
%    Entering ./runs/relax/kgrid_111 0
%      Executing:                              # run the job
%        export OMP_NUM_THREADS=1              # with 1 thread
%        mpirun -np 16 pw.x -input relax.in    # for each of 16 mpi tasks
%
%  poll 1  memory 56.51 MB                     # check on the jobs 
%  poll 2  memory 56.52 MB                     # every 3 seconds
%  poll 3  memory 56.52 MB                     # to see if they are finished
%  ...
%  ...
%  poll 101  memory 56.77 MB                   # 1x1x1 mesh relaxation finishes
%  poll 102  memory 56.77 MB                   # a couple of minutes later
%    Entering ./runs/relax/kgrid_111 0
%      copying results  0 relax                # save output files
%    Entering ./runs/relax/kgrid_111 0
%      analyzing  0 relax                      # get relaxed structure
%
%  poll 103  memory 57.54 MB
%    Entering ./runs/relax/kgrid_222 1         # 2x2x2 mesh relaxation starts
%      writing input files  1 relax            # write input files
%    Entering ./runs/relax/kgrid_222 1
%      sending required files  1 relax         # copy PP files
%      submitting job  1 relax
%    Entering ./runs/relax/kgrid_222 1
%      Executing:                              # run the job on 16 cores
%        export OMP_NUM_THREADS=1
%        mpirun -np 16 pw.x -input relax.in
%
%  poll 104  memory 57.55 MB
%  poll 105  memory 57.55 MB
%  ...
%  ...
%  poll 373  memory 63.80 MB                   # 2x2x2 mesh relaxation finishes 
%  poll 374  memory 63.80 MB                   # about 5 minutes later
%    Entering ./runs/relax/kgrid_222 1
%      copying results  1 relax                # save output files
%    Entering ./runs/relax/kgrid_222 1
%      analyzing  1 relax                      # get relaxed structure
%
%  poll 376  memory 63.80 MB
%    Entering ./runs/relax/kgrid_444 2         # 4x4x4 mesh relaxation starts
%      writing input files  2 relax
%      ...
%
%  poll 377  memory 63.80 MB
%  poll 378  memory 63.80 MB
%  ...
%  ...
%  poll 1539  memory 63.80 MB                  # 4x4x4 mesh relaxation finishes
%  poll 1540  memory 63.80 MB                  # about an hour later
%    Entering ./runs/relax/kgrid_444 2
%      copying results  2 relax
%      ...
%
%  poll 1541  memory 63.80 MB
%    Entering ./runs/relax/kgrid_666 3         # 6x6x6 mesh relaxation starts
%      writing input files  3 relax
%      ...
%
%  poll 1542  memory 63.80 MB
%  poll 1543  memory 63.80 MB
%  ...
%  ...
%  poll 2322  memory 63.80 MB                  # 6x6x6 mesh relaxation finishes
%  poll 2323  memory 63.80 MB                  # about 40 minutes later
%    Entering ./runs/relax/kgrid_666 3
%      copying results  3 relax
%    Entering ./runs/relax/kgrid_666 3
%      analyzing  3 relax                      # final structure is obtained
%
%Project finished                              # all jobs in the project are done
%
%
%Relaxation results:                           # display relaxation results
%-------------------
%    kgrid     starting force   max force    # of cycles
%  (1, 1, 1)     0.127915       0.127915        35
%  (2, 2, 2)      0.09733        0.09733        34
%  (4, 4, 4)     0.054516       0.143809        24
%  (6, 6, 6)     0.005833       0.036184         5
%
%
%The final structure is:                       # display the final structure
%
%[[  5.30732311   5.30835278   5.30189172]
% [ -2.18761868  -2.18873545  -2.18155455]
% [  1.57188665   1.57140679   1.57407291]
% [  5.03674643   5.03702837   0.35598249]
% [  8.79869042   8.79878175   2.76635262]
% [  0.35901756   5.0430751    5.04141008]
% [  2.76298517   8.79651294   8.79649574]
% [  5.5816783   10.89440134   5.58336983]
% [  8.2739258   13.61486785   8.27247705]
% [  5.04404696   0.35961504   5.04266772]
% [  8.79625105   2.76242822   8.79631982]
% [ 10.8938261    5.58159787   5.58361909]
% [ 13.61461938   8.27385039   8.27211794]
% [  5.58357278   5.58324177  10.89167699]
% [  8.27223497   8.27251628  13.6138305 ]
% [ 10.96556729  10.96540859  10.96610725]
% [ 13.63167079  13.63207444  13.62958692]]
%


\pagebreak
\section{Simple QMC Calculations} \label{simple_qmc}
The simplest QMC calculations that can be performed with Nexus 
involve five main stages:
\begin{description}
  \item[Configure Nexus settings] \hfill \\
    The \texttt{settings} function allows you to specify where pseudopotentials 
    are located, whether to generate input files without running jobs, details 
    of the machine you are on, and how often to have Nexus check 
    on the status of running jobs.  \index{settings}

  \item[Describe the physical system] \hfill \\
    Generate a crystal structure with the \texttt{generate\_physical\_system} 
    convenience function (see ``Graphene Sheet DMC'' \ref{graphene_dmc}), 
    or load an XYZ file into a \texttt{Structure} object 
    (see ``C 20 Molecule DMC'' \ref{c20_dmc}).  This is where 
    information about k-points, net system charge, or net system spin are 
    recorded.\index{generate\_physical\_system}\index{Structure}

  \item[Describe the calculations] \hfill \\
    Use the \texttt{standard\_qmc} (see ``Graphene Sheet DMC'' 
    \ref{graphene_dmc}) or \texttt{basic\_qmc} (see ``C 20 Molecule DMC'' 
    \ref{c20_dmc}) convenience functions 
    to select specific pseudopotentials, specify PWSCF and QMCPACK input 
    parameters, and job details, like how many nodes/cores/threads to use.
    \index{standard\_qmc}\index{basic\_qmc}

  \item[Run the simulations] \hfill \\
    Pass simulation objects created by \texttt{standard\_qmc/basic\_qmc} to 
    the \texttt{ProjectManager} and run the jobs by calling the 
    \texttt{run\_project} function.
    \index{run\_project}

  \item[Collect simulation results] \hfill \\
    Load results of simulation objects using the \texttt{load\_analyzer\_image} 
    function.  This gives you access to several/most (PWSCF/QMCPACK) physical 
    results produced by the simulation with statistical analysis already 
    performed (though the responsibility is still yours to verify absolute 
    correctness).
    \index{load\_analyzer\_image}
\end{description}


For more information about the functions/objects mentioned above, consider 
the examples in the following sections or consult ``Nexus User 
Reference'' (section \ref{reference}).


\pagebreak
\subsection{Example: Graphene Sheet DMC} \label{graphene_dmc}
The files for this example are found in:
\begin{shaded}
\begin{verbatim}
/your_download_path/nexus/examples/simple_qmc/graphene_example
\end{verbatim}
\end{shaded}

Take a moment to study the ``input file'' script 
(\texttt{graphene\_example.py}) and the attendant comments 
(prefixed with \#). The five stages listed in section \ref{simple_qmc} should 
be apparent.


\HRule
\begin{minted}{python}
#! /usr/bin/env python

from nexus import settings,ProjectManager,Job
from nexus import generate_physical_system
from nexus import loop,linear,vmc,dmc
from qmcpack_calculations import standard_qmc


#general settings for Nexus
settings(
    pseudo_dir    = './pseudopotentials',# directory with all pseudopotentials
    sleep         = 3,                   # check on runs every 'sleep' seconds
    generate_only = 0,                   # only make input files
    status_only   = 0,                   # only show status of runs
    machine       = 'node16',            # local machine is 16 core workstation
    )



#generate the graphene physical system
graphene = generate_physical_system(
    structure = 'graphite_aa',  # graphite keyword
    cell      = 'hex',          # hexagonal cell shape
    tiling    = (2,2,1),        # tiling of primitive cell
    constants = (2.462,10.0),   # a,c constants
    units     = 'A',            # in Angstrom
    kgrid     = (1,1,1),        # Monkhorst-Pack grid
    kshift    = (.5,.5,.5),     # and shift
    C         = 4               # C has 4 valence electrons
    ) 


#generate the simulations for the qmc workflow
qsims = standard_qmc(
    # subdirectory of runs 
    directory       = 'graphene_test',
    # description of the physical system
    system          = graphene,
    pseudos         = ['C.BFD.upf',  # pwscf PP file
                       'C.BFD.xml'], # qmcpack PP file
    # job parameters
    scfjob          = Job(cores=16), # cores to run scf 
    nscfjob         = Job(cores=16), # cores to run non-scf 
    optjob          = Job(cores=16), # cores for optimization 
    qmcjob          = Job(cores=16), # cores for qmc
    # dft parameters (pwscf)
    functional      = 'lda',         # dft functional
    ecut            =  150 ,         # planewave energy cutoff (Ry)
    conv_thr        =  1e-6,         # scf convergence threshold (Ry)
    mixing_beta     =    .7,         # charge mixing factor
    scf_kgrid       = (8,8,8),       # MP grid of primitive cell
    scf_kshift      = (1,1,1),       #  to converge charge density
    # qmc wavefunction parameters (qmcpack)
    meshfactor      = 1.0,           # bspline grid spacing, larger is finer
    jastrows        = [
        dict(type     = 'J1',        # 1-body
             function = 'bspline',   # bspline jastrow
             size     = 8),          # with 8 knots
        dict(type     = 'J2',        # 2-body
             function = 'bspline',   # bspline jastrow
             size     = 8)           # with 8 knots
        ],
    # opt parameters (qmcpack)
    perform_opt     = True,     # produce optimal jastrows
    block_opt       = False,    # if true, ignores opt and qmc
    skip_submit_opt = False,    # if true, writes input files, does not run opt
    opt_kpoint      = 'L',      # supercell k-point for the optimization
    opt_calcs       = [         # qmcpack input parameters for opt
        loop(max = 4,                        # No. of loop iterations
             qmc = linear(                   # linearized optimization method
                energy               =  0.0, # cost function
                unreweightedvariance =  1.0, #   is all unreweighted variance
                reweightedvariance   =  0.0, #   no energy or r.w. var. 
                timestep             =  0.5, # vmc timestep (1/Ha)
                warmupsteps          =  100, # MC steps before data collected 
                samples              = 16000,# samples used for cost function 
                stepsbetweensamples  =   10, # steps between uncorr. samples
                blocks               =   10, # ignore this  
                minwalkers           =   0.1,#  and this
                bigchange            =  15.0,#  and this
                alloweddifference    =  1e-4 #  and this, for now
                )
             ),
        loop(max = 4,
             qmc = linear(                   # same as above, except
                energy               =  0.5, # cost function
                unreweightedvariance =  0.0, #   is 50/50 energy and r.w. var.
                reweightedvariance   =  0.5, 
                timestep             =  0.5,  
                warmupsteps          =  100, 
                samples              = 64000,# and there are more samples 
                stepsbetweensamples  =   10, 
                blocks               =   10,   
                minwalkers           =   0.1, 
                bigchange            =  15.0,
                alloweddifference    =  1.0e-4
                )
             )
        ],
    # qmc parameters (qmcpack)
    block_qmc       = False,    # if true, ignores qmc
    skip_submit_qmc = False,    # if true, writes input file, does not run qmc
    qmc_calcs       = [         # qmcpack input parameters for qmc
        vmc(                      # vmc parameters 
            timestep      = 0.5,  # vmc timestep (1/Ha)
            warmupsteps   = 100,  # No. of MC steps before data is collected
            blocks        = 200,  # No. of data blocks recorded in scalar.dat
            steps         =  10,  # No. of steps per block
            substeps      =   3,  # MC steps taken w/o computing E_local
            samplesperthread = 40 # No. of dmc walkers per thread
            ),                    
        dmc(                      # dmc parameters
            timestep      = 0.01, # dmc timestep (1/Ha)
            warmupsteps   =  50,  # No. of MC steps before data is collected
            blocks        = 400,  # No. of data blocks recorded in scalar.dat
            steps         =   5,  # No. of steps per block
            nonlocalmoves = True  # use Casula's T-moves
            ),                    #  (retains variational principle for NLPP's)
        ],
    # return a list or object containing simulations
    return_list = False
    )


#the project manager monitors all runs
pm = ProjectManager()  

# give it the simulation objects
pm.add_simulations(qsims.list()) 

# run all the simulations    
pm.run_project()  



# print out the total energy
performed_runs = not settings.generate_only and not settings.status_only
if performed_runs:
    # get the qmcpack analyzer object
    # it contains all of the statistically analyzed data from the run
    qa = qsims.qmc.load_analyzer_image()
    # get the local energy from dmc.dat
    le = qa.dmc[1].dmc.LocalEnergy  # dmc series 1, dmc.dat, local energy
    #  print the total energy for the 8 atom system
    print 'The DMC ground state energy for graphene is:'
    print '    {0} +/- {1} Ha'.format(le.mean,le.error)
#end if
\end{minted}
\HRule

To run the example, navigate to the example directory and type
\begin{shaded}
\begin{verbatim}
./graphene_example.py
\end{verbatim}
\end{shaded}
or, alternatively, 
\begin{shaded}
\begin{verbatim}
python ./graphene_example.py
\end{verbatim}
\end{shaded}

You should see output like this (without the added \# comments):
\begin{shaded}
\begin{verbatim}
  Pseudopotentials   # reading pseudopotential files
    reading pp:  ./pseudopotentials/C.BFD.upf
    reading pp:  ./pseudopotentials/C.BFD.xml

Project starting 
  checking for file collisions  # ensure created files don't overlap
  loading cascade images        # load previous simulation state
    cascade 0 checking in 
  checking cascade dependencies # ensure sim.'s have needed dep.'s
    all simulation dependencies satisfied 
  
  starting runs:                # start submitting jobs
  ~~~~~~~~~~~~~~~~~~~~~~~~~~~~~~ 
  poll 0  memory 56.28 MB 
    Entering ./runs/graphene_test/scf 0 # scf job
      writing input files  0 scf        # input file written 
    Entering ./runs/graphene_test/scf 0 
      sending required files  0 scf     # PP files copied
      submitting job  0 scf             # job is in virtual queue
    Entering ./runs/graphene_test/scf 0 
      Executing:                        # job executed on workstation
        export OMP_NUM_THREADS=1
        mpirun -np 16 pw.x -input scf.in 

  poll 1  memory 56.30 MB               # waiting for job to finish
  poll 2  memory 56.30 MB 
  poll 3  memory 56.30 MB 
  poll 4  memory 56.30 MB 
    Entering ./runs/graphene_test/scf 0 
      copying results  0 scf            # job is finished, copy results
    Entering ./runs/graphene_test/scf 0 
      analyzing  0 scf                  # analyze output data

                                        # now do the same for
                                        # nscf job for Jastrow opt
                                        #   single k-point
                                        # nscf job for VMC/DMC
                                        #   multiple k-points

  poll 5  memory 56.31 MB               
    Entering ./runs/graphene_test/nscf 1 # nscf dmc 
      writing input files  1 nscf 
      ...
    Entering ./runs/graphene_test/nscfopt 4 # nscf opt 
      writing input files  4 nscf 
      ...

                                        # now convert KS orbitals
                                        # from planewave to bspline
                                        # with pw2qmcpack.x
                                        # for nscf opt & nscf dmc

  poll 7  memory 56.32 MB 
    Entering ./runs/graphene_test/nscf 2 # convert dmc orbitals
      sending required files  2 p2q 
      ...
    Entering ./runs/graphene_test/nscfopt 4 # convert opt orbitals
      copying results  4 nscf 
      ...

  poll 10  memory 56.32 MB 
    Entering ./runs/graphene_test/opt 6 # submit jastrow opt
      writing input files  6 opt        # write input file
    Entering ./runs/graphene_test/opt 6 
      sending required files  6 opt     # copy PP files
      submitting job  6 opt             # job is in virtual queue
    Entering ./runs/graphene_test/opt 6 
      Executing:                        # run qmcpack
        export OMP_NUM_THREADS=1        # w/ complex arithmetic
        mpirun -np 16 qmcapp_complex opt.in.xml 

  poll 11  memory 56.32 MB 
  poll 12  memory 56.32 MB 
  poll 13  memory 56.32 MB 
  ...
  ...
  ...
  poll 793  memory 56.32 MB   # qmcpack opt finishes
  poll 794  memory 56.32 MB   # nearly an hour later
  poll 795  memory 56.32 MB 
    Entering ./runs/graphene_test/opt 6 
      copying results  6 opt            # copy output files
    Entering ./runs/graphene_test/opt 6 
      analyzing  6 opt                  # analyze the results

  poll 796  memory 56.41 MB 
    Entering ./runs/graphene_test/qmc 3 # submit dmc
      writing input files  3 qmc        # write input file
    Entering ./runs/graphene_test/qmc 3 
      sending required files  3 qmc     # copy PP files
      submitting job  3 qmc             # job is in virtual queue
    Entering ./runs/graphene_test/qmc 3 
      Executing:                        # run qmcpack
        export OMP_NUM_THREADS=1
        mpirun -np 16 qmcapp_complex qmc.in.xml 

  poll 797  memory 57.31 MB 
  poll 798  memory 57.31 MB 
  poll 799  memory 57.31 MB 
  ...
  ...
  ...
  poll 1041  memory 57.31 MB   # qmcpack dmc finishes
  poll 1042  memory 57.31 MB   # about 15 minutes later
  poll 1043  memory 57.31 MB 
    Entering ./runs/graphene_test/qmc 3 
      copying results  3 qmc            # copy output files
    Entering ./runs/graphene_test/qmc 3 
      analyzing  3 qmc                  # analyze the results

Project finished                        # all jobs are finished
 
The DMC ground state energy for graphene is:
    -45.824960552 +/- 0.00498990689364 Ha    # one value from
                                             # qmcpack analyzer
\end{verbatim}
\end{shaded}

If successful, you have just performed a start-to-finish QMC calculation.  
The total energy quoted above probably will not match the one you produce 
due to different compilation environments and the probabilistic nature of 
QMC.  They should not, however differ by three sigma.

Take some time to inspect the input files generated by Nexus and 
the output files from PWSCF and QMCPACK.  The runs were performed in 
sub-directories of the \texttt{runs} directory.  The order of execution of 
the simulations is roughly \texttt{scf}, \texttt{nscf}, \texttt{nscfopt}, 
\texttt{opt}, then \texttt{qmc}.

\begin{shaded}
\begin{verbatim}
runs
└── graphene_test
    ├── nscf
    │   ├── nscf.in
    │   └── nscf.out
    ├── nscfopt
    │   ├── nscf.in
    │   └── nscf.out
    ├── opt
    │   ├── opt.in.xml
    │   └── opt.out
    ├── qmc
    │   ├── qmc.in.xml
    │   └── qmc.out
    └── scf
        ├── scf.in
        └── scf.out
\end{verbatim}
\end{shaded}

The directories above contain all the files generated by the simulations.  
Often one only wants to save the files with the most important data, which 
are generally small.  These are copied to the \texttt{results} directory 
which mirrors the structure of \texttt{runs}. 

\begin{shaded}
\begin{verbatim}
results
└── runs
    └── graphene_test
        ├── nscf
        │   ├── nscf.in
        │   └── nscf.out
        ├── nscfopt
        │   ├── nscf.in
        │   └── nscf.out
        ├── opt
        │   ├── opt.in.xml
        │   └── opt.out
        ├── qmc
        │   ├── qmc.in.xml
        │   └── qmc.out
        └── scf
            ├── scf.in
            └── scf.out
\end{verbatim}
\end{shaded}

Although this QMC run was performed at a single k-point, a twist-averaged run 
could be performed simply by changing \texttt{kgrid} in 
\texttt{generate\_physical\_system} from \texttt{(1,1,1)} to \texttt{(4,4,1)}, 
or similar.
 

\pagebreak
\subsection{Example: C 20 Molecule DMC}  \label{c20_dmc}
The files for this example are found in:
\begin{shaded}
\begin{verbatim}
/your_download_path/nexus/examples/simple_qmc/c20_example
\end{verbatim}
\end{shaded}

Take a moment to study the ``input file'' script 
(\texttt{c20\_example.py}) and the attendant comments 
(prefixed with \#). The relevant differences from the graphene example
mostly involve how the structure is procured (it is read from an XYZ file
rather than being generated), the boundary conditions (open BC's, see 
\texttt{bconds} in the QMCPACK input parameters), and the workflow involved 
(as opposed to \texttt{standard\_qmc}, \texttt{basic\_qmc} does not perform 
non-self-consistent DFT calculations).


\HRule
\begin{minted}{python}
#! /usr/bin/env python

from nexus import settings,ProjectManager,Job
from nexus import Structure,PhysicalSystem
from nexus import loop,linear,vmc,dmc
from qmcpack_calculations import basic_qmc


#general settings for Nexus
settings(
    pseudo_dir    = './pseudopotentials',# directory with all pseudopotentials
    sleep         = 3,                   # check on runs every 'sleep' seconds
    generate_only = 0,                   # only make input files
    status_only   = 0,                   # only show status of runs
    machine       = 'node16',            # local machine is 16 core workstation
    )



#generate the C20 physical system
# specify the xyz file
structure_file = 'c20.cage.xyz'
# make an empty structure object
structure = Structure()
# read in the xyz file
structure.read_xyz(structure_file)
# place a bounding box around the structure
structure.bounding_box(
    box   = 'cubic',         # cube shaped cell
    scale = 1.5              # 50% extra space
    )
# make it a gamma point cell
structure.add_kmesh(
    kgrid      = (1,1,1),    # Monkhorst-Pack grid
    kshift     = (0,0,0)     # and shift
    )
# add electronic information
c20 = PhysicalSystem(
    structure = structure,   # C20 structure
    net_charge = 0,          # net charge in units of e
    net_spin   = 0,          # net spin in units of e-spin
    C          = 4           # C has 4 valence electrons
    ) 


#generate the simulations for the qmc workflow
qsims = basic_qmc(
    # subdirectory of runs 
    directory       = 'c20_test',
    # description of the physical system
    system          = c20,
    pseudos         = ['C.BFD.upf',  # pwscf PP file
                       'C.BFD.xml'], # qmcpack PP file
    # job parameters
    scfjob          = Job(cores=16), # cores to run scf 
    optjob          = Job(cores    = 16,        # cores for optimization 
                          app_name = 'qmcapp'), # use real-valued qmcpack
    qmcjob          = Job(cores    = 16,        # cores for qmc
                          app_name = 'qmcapp'), # use real-valued qmcpack
    # dft parameters (pwscf)
    functional      = 'lda',         # dft functional
    ecut            =  150 ,         # planewave energy cutoff (Ry)
    conv_thr        =  1e-6,         # scf convergence threshold (Ry)
    mixing_beta     =    .7,         # charge mixing factor
    # qmc wavefunction parameters (qmcpack)
    bconds          = 'nnn',         # open boundary conditions
    meshfactor      = 1.0,           # bspline grid spacing, larger is finer
    jastrows        = [
        dict(type     = 'J1',        # 1-body
             function = 'bspline',   # bspline jastrow
             size     = 8,           # with 8 knots
             rcut     = 6.0),        # and a radial cutoff of 6 bohr
        dict(type     = 'J2',        # 2-body
             function = 'bspline',   # bspline jastrow
             size     = 8,           # with 8 knots
             rcut     = 8.0),        # and a radial cutoff of 8 bohr
        ],
    # opt parameters (qmcpack)
    perform_opt     = True,     # produce optimal jastrows
    block_opt       = False,    # if true, ignores opt and qmc
    skip_submit_opt = False,    # if true, writes input files, does not run opt
    opt_calcs       = [         # qmcpack input parameters for opt
        loop(max = 4,                        # No. of loop iterations
             qmc = linear(                   # linearized optimization method
                energy               =  0.0, # cost function
                unreweightedvariance =  1.0, #   is all unreweighted variance
                reweightedvariance   =  0.0, #   no energy or r.w. var. 
                timestep             =  0.5, # vmc timestep (1/Ha)
                warmupsteps          =  100, # MC steps before data collected 
                samples              = 16000,# samples used for cost function 
                stepsbetweensamples  =   10, # steps between uncorr. samples
                blocks               =   10, # ignore this  
                minwalkers           =   0.1,#  and this
                bigchange            =  15.0,#  and this
                alloweddifference    =  1e-4 #  and this, for now
                )
             )
        ],
    # qmc parameters (qmcpack)
    block_qmc       = False,    # if true, ignores qmc
    skip_submit_qmc = False,    # if true, writes input file, does not run qmc
    qmc_calcs       = [         # qmcpack input parameters for qmc
        vmc(                      # vmc parameters 
            timestep      = 0.5,  # vmc timestep (1/Ha)
            warmupsteps   = 100,  # No. of MC steps before data is collected
            blocks        = 200,  # No. of data blocks recorded in scalar.dat
            steps         =  10,  # No. of steps per block
            substeps      =   3,  # MC steps taken w/o computing E_local
            samplesperthread = 40 # No. of dmc walkers per thread
            ),                    
        dmc(                      # dmc parameters
            timestep      = 0.01, # dmc timestep (1/Ha)
            warmupsteps   =  50,  # No. of MC steps before data is collected
            blocks        = 400,  # No. of data blocks recorded in scalar.dat
            steps         =   5,  # No. of steps per block
            nonlocalmoves = True  # use Casula's T-moves
            ),                    #  (retains variational principle for NLPP's)
        ],
    # return a list or object containing simulations
    return_list = False
    )




#the project manager monitors all runs
pm = ProjectManager()  

# give it the simulation objects
pm.add_simulations(qsims.list()) 

# run all the simulations    
pm.run_project()  



# print out the total energy
performed_runs = not settings.generate_only and not settings.status_only
if performed_runs:
    # get the qmcpack analyzer object
    # it contains all of the statistically analyzed data from the run
    qa = qsims.qmc.load_analyzer_image()
    # get the local energy from dmc.dat
    le = qa.dmc[1].dmc.LocalEnergy  # dmc series 1, dmc.dat, local energy
    #  print the total energy for the 20 atom system
    print 'The DMC ground state energy for C20 is:'
    print '    {0} +/- {1} Ha'.format(le.mean,le.error)
#end if
\end{minted}
\HRule


To run the example, navigate to the example directory and type
\begin{shaded}
\begin{verbatim}
./c20_example.py
\end{verbatim}
\end{shaded}
or, alternatively, 
\begin{shaded}
\begin{verbatim}
python ./c20_example.py
\end{verbatim}
\end{shaded}

You should see output like this (without the added \# comments):
\begin{shaded}
\begin{verbatim}
  Pseudopotentials   # reading pseudopotential files
    reading pp:  ./pseudopotentials/C.BFD.upf
    reading pp:  ./pseudopotentials/C.BFD.xml

Project starting 
  checking for file collisions  # ensure created files don't overlap 
  loading cascade images        # load previous simulation state 
    cascade 0 checking in 
  checking cascade dependencies # ensure sim.'s have needed dep.'s
    all simulation dependencies satisfied 
  
  starting runs:                # start submitting jobs
  ~~~~~~~~~~~~~~~~~~~~~~~~~~~~~~
  poll 0  memory 56.21 MB 
    Entering ./runs/c20_test/scf 0  # scf job
      writing input files  0 scf    # input file written
    Entering ./runs/c20_test/scf 0 
      sending required files  0 scf # PP files copied
      submitting job  0 scf         # job is in the virtual queue
    Entering ./runs/c20_test/scf 0 
      Executing:                    # job executed on workstation
        export OMP_NUM_THREADS=1
        mpirun -np 16 pw.x -input scf.in 

  poll 1  memory 56.23 MB           # waiting for job to finish
  poll 2  memory 56.23 MB 
  poll 3  memory 56.23 MB 
  poll 4  memory 56.23 MB 
  poll 5  memory 56.23 MB 
  poll 6  memory 56.23 MB 
  poll 7  memory 56.23 MB 
  poll 8  memory 56.23 MB 
    Entering ./runs/c20_test/scf 0 
      copying results  0 scf        # job is finished, copy results
    Entering ./runs/c20_test/scf 0 
      analyzing  0 scf              # analyze output data
                                    
  poll 9  memory 56.23 MB           # now convert KS orbitals
    Entering ./runs/c20_test/scf 1  # from planewave to bspline
      writing input files  1 p2q    # with pw2qmcpack.x
      ...

  poll 12  memory 56.23 MB 
    Entering ./runs/c20_test/opt 3  # submit jastrow opt
      writing input files  3 opt    # write input file
    Entering ./runs/c20_test/opt 3 
      sending required files  3 opt # copy PP files
      submitting job  3 opt         # job is in virtual queue
    Entering ./runs/c20_test/opt 3 
      Executing:                    # run qmcpack
        export OMP_NUM_THREADS=1    # w/ real arithmetic
        mpirun -np 16 qmcapp opt.in.xml 

  poll 13  memory 56.24 MB 
  poll 14  memory 56.24 MB 
  poll 15  memory 56.24 MB 
  ...
  ...
  ...
  poll 204  memory 56.24 MB   # qmcpack opt finishes
  poll 205  memory 56.24 MB   # about 10 minutes later
  poll 206  memory 56.24 MB 
    Entering ./runs/c20_test/opt 3 
      copying results  3 opt        # copy output files
    Entering ./runs/c20_test/opt 3 
      analyzing  3 opt              # analyze the results

  poll 207  memory 56.27 MB 
    Entering ./runs/c20_test/qmc 2  # submit dmc
      writing input files  2 qmc    # write input file
    Entering ./runs/c20_test/qmc 2 
      sending required files  2 qmc # copy PP files
      submitting job  2 qmc         # job is in virtual queue
    Entering ./runs/c20_test/qmc 2 
      Executing:                    # run qmcpack
        export OMP_NUM_THREADS=1
        mpirun -np 16 qmcapp qmc.in.xml 

  poll 208  memory 56.49 MB 
  poll 209  memory 56.49 MB 
  poll 210  memory 56.49 MB 
  ...
  ...
  ...
  poll 598  memory 56.49 MB   # qmcpack dmc finishes
  poll 599  memory 56.49 MB   # about 20 minutes later
  poll 600  memory 56.49 MB 
    Entering ./runs/c20_test/qmc 2 
      copying results  2 qmc        # copy output files
    Entering ./runs/c20_test/qmc 2 
      analyzing  2 qmc              # analyze the results

Project finished                    # all jobs are finished
 
The DMC ground state energy for C20 is:
    -112.890695404 +/- 0.0151688786226 Ha  # one value from
                                           # qmcpack analyzer
\end{verbatim}
\end{shaded}

Again, the total energy quoted above probably will not match the one you produce 
due to different compilation environments and the probabilistic nature of 
QMC.  The results should still be statistically comparable.

The directory trees generated by Nexus for C 20 have a similar structure 
to the graphene example.  Note the absence of the \texttt{nscf} runs.  The order 
of execution of the simulations is \texttt{scf}, \texttt{opt}, then \texttt{qmc}.

\begin{shaded}
\begin{verbatim}
runs
└── c20_test
    ├── opt
    │   ├── opt.in.xml
    │   └── opt.out
    ├── qmc
    │   ├── qmc.in.xml
    │   └── qmc.out
    └── scf
        ├── scf.in
        └── scf.out
results
└── runs
    └── c20_test
        ├── opt
        │   ├── opt.in.xml
        │   └── opt.out
        ├── qmc
        │   ├── qmc.in.xml
        │   └── qmc.out
        └── scf
            ├── scf.in
            └── scf.out
\end{verbatim}
\end{shaded}



\pagebreak
\chapter{Nexus User Reference} \label{reference}
Pending.

\section{Reading what you wrote}

\section{Nexus settings: global state and user-specific information}
The first section of a project script is often dedicated to providing 
information regarding the local machine, the location of various files, and 
the desired behavior of the \texttt{ProjectManager}.  This information is 
communicated to Nexus through the \texttt{settings} function. 
The settings function is available in the \texttt{project} module.  To make 
\texttt{settings} available in your project script, use the following import 
statement:\newline
\HRule
\begin{minted}{python}
from nexus import settings
\end{minted}
\HRule

\subsection{\bu{How to use the \texttt{settings} function}}
In most cases, it is sufficient to supply only four pieces of information 
through the \texttt{settings} function: whether to run all jobs or just create 
the input files, how often to check jobs for completion, the location of 
pseudopotential files, and a description of the local machine.
\HRule
\begin{minted}{python}
settings(
    generate_only = True,                 # only write input files, do not run
    sleep         = 3,                    # check on jobs every 3 seconds
    pseudo_dir    = './pseudopotentials', # path to PP file collection
    machine       = 'node8'               # local machine is an 8 core workstation
    )
\end{minted}
\HRule

A few additional parameters are available in \texttt{settings} to control where 
runs are performed, where output data is gathered, and whether to print job 
status information.  More detailed information about both local and target 
machines can be provided, such as allocation account numbers, filesystem 
structure, and where executables are located.
\HRule
\begin{minted}{python}
settings(
    generate_only = True,                 # only write input files, do not run
    sleep         = 3,                    # check on jobs every 3 seconds
    pseudo_dir    = './pseudopotentials', # path to PP file collection
    machine       = 'node8'               # local machine is an 8 core workstation
    )
\end{minted}
\HRule


\subsection{\bu{Accessing settings data}}


%
%\section{Making structures}
%
%\section{Making simulations}
%
%\section{Linking simulations together in cascades}
%
%
%
%\section{Working with objects}
%\subsection{\bu{General properties of objects}}
%At a conceptual level, objects are a generalization of normal variables.  
%They are really collections of variables taking integer, real, string, or 
%even object values.  An object is also more than structured data, as it 
%makes available operations, or functions, that act on that data.  The data 
%(member data) and functions (member functions) of an object are accessed with 
%a ``.'' following the object's name.  So when you see code like
%\begin{shaded}
%\begin{verbatim}
%   from mymod import MyObj  # make MyObj available from the mymod module
%   i = 5                    # make an integer
%   s = 'abc'                # make a string
%   o = MyObj()              # make a MyObj called o
%   j = i + o.an_int         # access member data an_int, created with MyObj() 
%   r = o.create_real()      # access member function create_real
%   o.a = i                  # create a new member data field called a
%   take_action(             # a normal function call
%      a = 3,                #   with keyword arguments
%      i = i,                #   left i is argument, right i is variable above
%      more = True           #   and more function arguments
%      )                     #   closing the function call ()
%\end{verbatim}
%\end{shaded}
%it should be clear what is happening.
%
%\subsection{\bu{Interacting with Nexus objects}}
%All objects in Nexus behave and can be interacted with in a 
%similar way.  Each object you encounter that is made by Nexus 
%can be used in the same way as \texttt{obj} objects from the \texttt{generic} 
%module
%
%
%\section{Objects describing matter}
%\subsection{\bu{Structure}}
%\subsection{\bu{PhysicalSystem}}
%
%
%\section{Objects describing simulations}
%\subsection{\bu{General properties of \texttt{Simulation} objects}}
%\subsection{\bu{Sqd}}
%\subsection{\bu{Pwscf}}
%\subsection{\bu{Pw2casino}}
%\subsection{\bu{Pw2qmcpack}}
%\subsection{\bu{Wfconvert}}
%\subsection{\bu{Qmcpack}}
%
%
%\section{Objects describing simulation input files}
%\subsection{\bu{SqdInput}}
%\subsection{\bu{PwscfInput}}
%\subsection{\bu{Pw2casinoInput}}
%\subsection{\bu{Pw2qmcpackInput}}
%\subsection{\bu{WfconvertInput}}
%\subsection{\bu{QmcpackInput}}
%
%
%\section{Objects describing machines}
%\subsection{\bu{Workstation}}
%\subsection{\bu{Supercomputer} }
%
%
%\section{Objects describing/affecting jobs and job submission}
%\subsection{\bu{Job}}
%\subsection{\bu{bundle}}
%\subsection{\bu{ProjectManager}}
%
%
%\section{Objects and functions for data analysis}
%\subsection{\bu{Unit conversion: convert}}
%\subsection{\bu{PWCSF data analysis: PwscfAnalyzer}}
%\subsection{\bu{QMCPACK data analysis: QmcpackAnalyzer}}
%\subsection{\bu{Fitting equation of state data: eos\_fit}}
%\subsection{\bu{Making plots: plotting}}
%
%
%\section{Runtime behavior of the ProjectManager}
%\subsection{\bu{Analysis of simulation dependencies and workflow}}
%\subsection{\bu{Loading saved data about simulations}}
%\subsection{\bu{Frequent traversal of simulation dependency tree (cascades)}}
%\subsection{\bu{Modifying runtime behavior}}
%
%
%\section{Supporting classes}
%\subsection{\bu{Particles}}
%\subsection{\bu{PeriodicTable}}
%\subsection{\bu{Pseudopotential}}
%\subsection{\bu{Generator}}
%\subsection{\bu{HDFreader}}
%\subsection{\bu{XMLreader}}



\pagebreak
\chapter{QMC Practice \\in a Nutshell}\label{theory}
The aim of this section is to provide a very brief overview of the 
essential concepts undergirding Quantum Monte Carlo calculations 
of electronic structure with a particular focus on the key 
approximations and quantities to converge to achieve high
accuracy.  The discussion here is not intended to be comprehensive. 
For deeper perspectives on QMC, please see the review articles 
listed in the ``Recommended Reading'' section (\ref{learn_qmc}).


\section{VMC and DMC in the abstract}
Ground state QMC methods, such as Variational (VMC) and Diffusion (DMC) Monte 
Carlo, attempt to obtain the ground state energy of a many-body quantum system.
\begin{align}
  E_0 = \frac{\expval{\Psi_0}{H}{\Psi_0}}{\overlap{\Psi_0}{\Psi_0}}
\end{align}
The VMC method obtains an upper bound on the ground state energy (guaranteed 
by the Variational Principle) by introducing a guess at the ground state wavefunction, known as the trial wavefunction $\Psi_T$:
\begin{align}
  E_{VMC} = \frac{\expval{\Psi_T}{H}{\Psi_T}}{\overlap{\Psi_T}{\Psi_T}} \ge E_0
\end{align}
The DMC method improves on this variational bound by projecting out component 
eigenstates of the trial wavefunction lying higher in energy than the ground 
state.  The operator that acts as a projector is the imaginary time, or 
thermodynamic, density matrix:
\begin{align}
  \ket{\Psi_t} & = e^{-t\hat{H}}\ket{\Psi_T} \nonumber \\
               & = e^{-tE_0}\left( \ket{\Psi_0} + \sum_{n>0}e^{-t(E_n-E_0)}\ket{\Psi_n} \right) \nonumber \\
               &\xrightarrow[t\rightarrow\infty]{} e^{-tE_0} \ket{\Psi_0} 
\end{align}
The DMC energy approaches the ground state energy from above as the imaginary 
time becomes large. 
\begin{align}
  E_{DMC}  & = \lim_{t\rightarrow\infty}\frac{\expval{\Psi_t}{H}{\Psi_t}}{\overlap{\Psi_t}{\Psi_t}} = E_0
\end{align}
However from the equations above, one can already anticipate that the DMC 
method will struggle in the face of degeneracy or near-degeneracy.

In principle, the DMC method is exact for the ground state, but further 
complications arise for systems that are extended, comprised of fermions, 
or contain heavy nuclei, pseudized or otherwise.  Approximations arising 
from the numerical implementation of the method also require care to keep 
under control.


\section{From expectation values to random walks}
Evaluating expectation values of a many-body system involves performing high 
dimensional integrals (the dimensionality is at least the dimensions of the 
physical space times the number of particles).  In VMC, for example, the 
expectation value of the total energy is represented succinctly as:
\begin{align}
  E_{VMC} &= \int dR \abs{\Psi_T}^2 E_L
\end{align}
where $E_L$ is the local energy $E_L=\Psi_T^{-1}\hat{H}\Psi_T$.  The other 
factor in the integral $\abs{\Psi_T}^2$ can clearly be thought of as a 
probability distribution and can therefore be sampled by Monte Carlo 
methods (such as the Metropolis algorithm) to evaluate the integral exactly.

The sampling procedure takes the form of random walks.  A ``walker'' is just 
a set of particle positions, along with a weight, that evolves (or moves) to 
new positions according to a set of statisical rules.  In VMC as few walkers 
are used as possible to reduce the equilibration time (the number of steps or 
moves required to lose a memory of the potentially poor starting guess for 
particle posistions).  In DMC, the walker population is a dynamic feature of 
the calculation and must be large enough to avoid introducing bias in 
expectation values.

The tradeoff of moving to a the sampling procedure for the integration is that 
it introduces statistical error 
into the calculation which diminishes slowly with the number of samples 
(it falls off like $1/(\# of samples)$ by the Central Limit Theorem).  The 
good news for ground state QMC is that this error can be reduced more rapidly 
through the discovery of better guesses at the detailed nature of the 
many-body wavefunction.


\section{Quality orbitals: planewaves, cutoffs, splines, and meshes}
Acting on an understanding of perturbation theory, the zeroth order 
representation of the wavefunction of an interacting system takes the form 
of a Slater determinant of single particle orbitals.  In practice, QMC 
calculations often obtain a starting guess at these orbitals from Hartree-Fock 
or Density Functional Theory calculations (which already contain 
non-perturbative contributions from correlation).  An important factor 
in the generation and use of these orbitals is to ensure that they are 
described to high accuracy within the parent theory.

For example, when taking orbitals from a planewave DFT calculation, one must 
take care to converge the planewave energy cutoff to a sufficient level of 
accuracy (usually far beyond what is required to obtain an accurate DFT 
energy).  One criterion to use it to converge the kinetic energy of the 
Kohn-Sham wavefunction with respect to the planewave energy cutoff until it 
is accurate to the energy scale you care about in your production QMC 
calcuation.  For systems with a small number of valence electrons, a cutoff of 
around 200 Ry is often sufficient.  To obtain the kinetic energy from a PWSCF 
calculation the \texttt{pw2casino.x} post-processing tool can be used.  In 
Nexus one has the option to compute the kinetic energy by setting 
the \texttt{kinetic\_E} flag in the \texttt{standard\_qmc} or 
\texttt{basic\_qmc} convenience functions. 

For efficiency reasons, QMC codes often use a real-space representation of the 
wavefunction.  It is common to represent the orbitals in terms of B-splines 
which have control points, or knots, that fall on a regular 3-D mesh.  
Analogous to the planewave cutoff, the fineness of the B-spline mesh controls 
the quality of the represented orbitals.  To verify that the quality of the 
orbitals has not been compromised during the conversion process from planewave 
to B-spline, one often performs a VMC calculation with the B-spline Slater 
determinant wavefunction to obtain the kinetic energy.  This value should agree 
with the kinetic energy of the planewave representation within the energy scale 
of interest.  

In QMCPACK, the B-spline mesh is controlled with the \texttt{meshfactor} 
keyword.  Larger values correspond to finer meshes.  A value of $1.0$ usually 
gives a similar quality representation as the original planewave calculation. 
Control of this parameter is made available in Nexus through the 
\texttt{meshfactor} keyword in the \texttt{standard\_qmc} or 
\texttt{basic\_qmc} convenience functions. 


\section{Quality Jastrows: less variance = more efficient}
Taking a further que from perturbation theory, the first order correction to 
the Slater determinant wavefunction is the Jastrow correlation prefactor.  
\begin{align}
  \Psi_T\approx e^{-J}\Psi_{Slater\, Det.}
\end{align}
In a quantum liquid, an appropriate form for the Jastrow factor is:
\begin{align}
  J = \sum_{i<j} u_{ij}(\abs{r_i-r_j})
\end{align}
This form is often used without modification in electronic structure 
calculations.  Note that the correlation factors $u_{ij}$ can be different for 
particles of differing species, or, if one of the particles in the pair is 
classical (such as a heavy atomic nucleus), the local electronic environment 
varies across the system.

The primary role of the Jastrow factor is to increase the efficiency of the 
QMC calculation.  The variance of the local energy across all samples of the 
random walk is directly related to the statistical error of the final results: 
\begin{align}
  v_{\Psi_T} &= \frac{1}{N_{samples}}\sum_{s\in samples} E_L(s)^2 - \left[\frac{1}{N_{samples}}\sum_{s\in samples} E_L(s)\right]^2 \\
  \sigma_{error} &\approx \sqrt{\frac{v_{\Psi_T}}{N_{samples}}}
\end{align}
The variance of local energy is usually minimized by performing a statistical 
optimization of the Jastrow factor with QMC.  

In addition to selecting a good 
form for the pair correlation functions $u_{ij}$ (which are represented in 
QMCPACK as 1-D B-spline functions with a finite cutoff radius), the 
(iterative) optimization procedure must be performed with a sufficient number 
of samples to converge all the free parameters.  Starting with a small number 
of samples ($\approx 20,000$) is usually preferable for early iterations, 
followed by a larger number for later iterations.  This larger number is 
something close to $100,000\times (\#~of~free~parameters)^2$.  For B-spline 
functions, the number of free parameters is the number of control points, or knots.

The number of samples is controlled with the \texttt{samples} keyword in 
QMCPACK.  Control of this parameter is made available in Nexus 
through the \texttt{samples} keyword in the \texttt{linear} or 
\texttt{cslinear} convenience functions (Which are often used in conjunction 
with \texttt{standard\_qmc} or \texttt{basic\_qmc}).  For a B-spline 
correlation factor, the number of free parameters/knots is indicated by the 
\texttt{size} keyword in either QMCPACK or Nexus.


\section{Finite size effects: k-points, supercells, and corrections}
For extended systems, finite size errors are a key consideration.  In addition 
to the finite size effects that are typically seen in DFT (k-points related).  
Correlated, many-body methods such as QMC also must contend with 
correlation-related finite size effects.  Both types of finite-size effects 
are reduced by simply using larger supercells.  The complete 
elimination of finite size effects using this approach can be prohibitively 
costly since the finite size error typically falls off like $1/\Omega_C$, where 
$\Omega_C$ is the volume of the supercell.  A more sophisticated approach 
involves a combination of the supercell size, k-point grid, and additional 
estimated corrections for correlation finite size effects.

Although there is no firm rule on the selection of these three elements, 
adhering to some general guidelines is usually helpful.  For a production 
calculation of an extended system, the minimum supercell size is around 50 
atoms.  The size of the supercell k-point grid can then be determined by 
proxy with a DFT calculation (converge the energy down to the scale of 
interest).  Note that although the cost of a DFT calculation scales linearly 
with the number of k-points, the cost of the corresponding QMC calculation 
is hardly increased due to the statistical averaging of the results 
(the QMC calculation at each separate supercell k-point is simply performed 
with fewer samples so that the total number of samples remains fixed w.r.t. 
the number of k-points).  Finally, corrections for correlation-related 
finite size effects are computed during the QMC run and added to the result 
by hand in post-processing the data.

In Nexus, the supercell size is controlled through the 
\texttt{tiling} parameter in the \texttt{generate\_physical\_system}, 
\texttt{generate\_structure}, \texttt{Structure}, or \texttt{Crystal} 
convenience functions.  Supercells can also be constructed by tiling exising 
structures through the \texttt{tile} member function of \texttt{Structure} or 
\texttt{PhysicalSystem} objects.  The k-point grid is controlled through the 
\texttt{kgrid} parameter in the 
\texttt{generate\_physical\_system}, 
\texttt{generate\_structure}, \texttt{Structure}, or \texttt{Crystal} 
convenience functions.  K-point grids can also be added to existing structures 
through the \texttt{add\_kmesh} member function of \texttt{Structure} or 
\texttt{PhysicalSystem} objects.


\section{Imaginary time discretization: the DMC timestep}
An analytic form for the imaginary time projection operator is not known, but 
real-space approximations to it can be obtained in the small time limit.  
With importance sampling included (not covered here), the short-time projector 
splits into two parts, known as the drift-diffusion and branching factors 
(shown below in atomic units):
\begin{align}
   \rho(R',R;t)  &= \expvalnh{R'}{\hat{\Psi_T}e^{-t\hat{H}}\hat{\Psi_T}^{-1}}{R}\\ 
    &= G_d(R',R;t)G_b(R',R,t) +\mathcal{O}(t^2) \\
  G_d(R',R;t) &\equiv \exp{\left(-\tfrac{1}{2t}\left[R'-R-t\nabla_R\log\Psi_T(R)\right]^2\right)} \\
  G_b(R',R;t) &\equiv \exp{\left(\tfrac{1}{2}\left[E_L(R')+E_L(R)\right]\right)}
\end{align}
The long-time projector is found as the product of many approximate short-time 
solutions, which takes the form of a many-body path integral in real space:
\begin{align}
  \rho(R_M,R_0; M\tau) = \int dR_1dR_{M-1}\ldots \prod_{m=0}^{M-1}\rho(R_{m+1},R_m;\tau)
\end{align}
The short-time parameter $\tau$ is known as the DMC timestep and accurate 
quantities are obtained only in the limit as $\tau$ approaches zero.

Ensuring that the timestep error is sufficiently small usually involves 
performing many DMC calculations over a range of timesteps (sometimes on 
a smaller supercell than the production calculation).  The largest timestep 
is chosen that produces a bias smaller than the energy scale of interest.  
For very high accuracy, one uses the total energy as a function of timestep to 
extrapolate to the zero time limit.

The DMC timestep is made available in Nexus through the 
\texttt{timestep} parameter of the \texttt{dmc} convenience function 
(which is often used in conjuction with the \texttt{standard\_qmc},  
\texttt{basic\_qmc}, \texttt{generate\_qmcpack}, or \texttt{Qmcpack} 
functions). 


\section{Population control bias: safety in numbers}
While the drift-diffusion factor $G_d(R',R;\tau)$ can be sampled exactly using 
Gaussian distributed random numbers (this generates the DMC random walk), the 
branching factor $G_b(R',R;\tau)$ is handled a different way for efficiency.  
The product of branching factors over an imaginary time trajectory (random 
walk) serves as a statistical weight for each walker.  The fluctuations in 
this weight rapidly become quite large as the random walk progresses 
(because it approaches an infinite product of real numbers).  As its name 
suggests, this weight factor is used to ``branch'' walkers every few steps.  
If the weight is small the walker is deleted, but if the weight is large the 
walker is copied many times (``branched'') with each copy carrying a weight 
close to unity.  This is more efficient because more walkers are created (and 
thus more statistics are gathered) in the high weight regions of phase space 
that contribute most to the integral.

The branching process in DMC naturally leads to a fluctuating population of 
walkers.  The fluctuations in the walker population, if left to its own 
dynamics, are unbounded.  This means that the walker population can grow very 
large, or even become zero.  To prevent collapse of the walker population, 
population control techniques (not covered here) are added to the algorithm.
The practical upshot of population control is that it introduces a systematic 
bias in the DMC results that scales like $1/(\# of walkers)$ (Although note 
that another route to reduce the population control bias is to improve the 
trial wavefunction, since the fluctuations in the branching weights will 
become zero for the exact ground state).  

For many production calculations, population control bias is not much of an 
issue because the simulations are performed on supercomputers with thousands of 
cores per run, and thus tens of thousands of walkers.  As a rule of thumb, the 
walker population should at least number in the thousands.  One should 
occasionally explicitly check the magnitude of the population control bias for 
the system under study since predictions have been made that it will 
eventually diverge exponentially with the number of particles in the system.

The DMC walker population can be directly controlled in QMCPACK or Nexus 
through the \texttt{samples} (total walker population) or 
\texttt{samplesperthread} (walkers per OpenMP thread) keywords in the VMC 
block directly proceeding DMC (\texttt{vmc} convenience function in  
Nexus).  If you opt to use the \texttt{samples} keyword, check that 
each thread in the calculation will have at least a few walkers.


\section{The fixed node/phase approximation: varying the nodes/phase}
For every fermionic system, the bosonic ground state lies lower in energy than 
the fermionic ground state.  This means that projection methods like DMC 
will approach the bosonic ground state exponentially fast in imaginary time if 
unconstrained (this would show up as an exponentially diverging statistical 
error).  In order to guarantee that the projected wavefunction remains in the 
space of fermionic functions (and consequently that the projected energy 
remains an upper bound to the fermionic ground state energy), the projected 
wavefunction is constrained to share the nodes (if it is real-valued) or the 
phase (if it is complex-valued) of the trial wavefunction.  The fixed 
node/phase approximation represents one of the two most important 
approximations for electronic structure calculations (the other is the 
pseudopotential approximation covered in the next section).  

The fixed node/phase error can be reduced, but it cannot be completely 
eliminated unless the exact nodes/phase is known.  A common approach to reduce 
the fixed node/phase error is to perform several DMC calculations (sometimes 
on a smaller supercell) with different sets of orbitals (perhaps generated 
with different functionals).  Another, more expensive approach, is to include 
the backflow transformation (this is the second order correction to the 
wavefunction; it is not covered in any detail here) to get a lower bound on 
how large the fixed node error is in standard Slater-Jastrow calculations.

To perform a calculation of this type (scanning over orbitals from different 
functionals) with Nexus, the DFT functional can be selected 
with the \texttt{functional} keyword in the  \texttt{standard\_qmc} or 
\texttt{basic\_qmc} convenience functions.  If you are using pseudopotentials 
generated for use in DFT, you should maintain consistency between the 
functional and pseudopotential.  Even if such consistency is maintained, the 
impact of using DFT pseudopotentials (or those made with many other theories) 
in QMC can be significant.


\section{Pseudopotentials: theoretical dissonance, the locality approximation, and T-moves}
The accurate use of pseudopotentials in electronic structure QMC calculations 
remains one of the largest challenges in current practice.  The necessity for 
pseudopotentials arises from the rapidly increasing computational cost with 
increasing nuclear charge (it scales like $Z^6$, compared with the 
$N_{electrons}^3$ scaling with $Z$ fixed).  The challenge in using 
pseudopotentials in QMC is that practically no pseudopotentials exist that have 
been generated self-consistently with QMC.  In other words, QMC is currently 
reliant on other theories to provide the pseudopotentials, which can be a 
critical source of error.

The current state-of-the-art is not without rigor, however.  One source of 
Dirac-Fock based pseudopotentials, the Burkatzki-Filippi-Dolg database (see 
\url{http://www.burkatzki.com/pseudos/index.2.html}), has been explicitly 
vetted against quantum chemistry calculations of atoms (a higher-fidelity 
proxy for QMC calculations of small systems).  It must be stressed that 
these pseudopotentials should still be validated for use in a particular 
target system.  Another collection of Dirac-Fock pseudopotentials that have 
been created for use in QMC can be found in the Trail-Needs database 
(see \url{http://www.tcm.phy.cam.ac.uk/~mdt26/casino2_pseudopotentials.html}). 
Many current calculations also use the OPIUM package (see 
\url{http://opium.sourceforge.net/}) to generate DFT pseudopotentials and then 
port them directly to QMC.

Whatever the source of pseudopotentials (but perhaps especially so for those 
derived from DFT), testing and validation remains an important step preceeding 
production calculations.  One option is to perform parallel pseudopotential 
and all-electron DMC calculations of atoms with varying electron count 
(\emph{i.e.} ionization potential/electron affinity calculations).  As with 
any electronic structure calculation, it is also advisable to devise a test 
in or close to the target host environment.
Validating pseudopotentials remains a difficult task, and while the suggestions 
presented here may be of some help, they do not amount to a panacea for the 
issue. 

Beyond the central approximation of using a pseudopotential at all, two 
approximations unique to pseudopotential use in DMC merit discussion.  The 
direct use of non-local pseudopotentials in DMC leads to a second sign-problem 
(akin to the fixed-node issue) in the imaginary time projector.  One solution, 
devised first, is known as the locality approximation.  In the locality 
approximation, the non-local pseudopotential is replaced by a ``localized'' 
form: $V_{NLPP}\rightarrow \Psi_T^{-1}V_{NLPP}\Psi_T$.  This approximation 
becomes exact as the trial wavefunction approaches the pseudo ground state, 
however the Variational Principle of the pseudo-system is lost (though it 
should be acknowledged that a non-variational portion of the energy has been 
discarded by using pseudopotentials at all).  The Variational Principle for 
the pseudo-system can be restored with an advanced sampling technique known 
as T-moves (although the first incarnation of the technique reduces to the 
locality approximation as the system becomes larger than several atoms, the 
second version fixes this oversight).

One can select whether to use the locality approximation or T-moves 
(version 1!) in QMCPACK from within Nexus by setting the 
parameter \texttt{nonlocalmoves} to True or False in the \texttt{dmc} 
convenience function.


\section{Other approximations: what else is missing?}
Though a few points could be selected for mention at this point, only one 
additional approximation will be highlighted here.  In most modern QMC 
calculations of electronic structure, relativistic effects have been neglected 
entirely (there have been a few exceptions) or simply assumed to be covered 
by the pseudopotential.  Clearly this will become an issue for systems with 
large effective core charges.  At present, relativistic corrections are not 
available within QMCPACK.



\pagebreak
\chapter{Recommended Reading} \label{reading}
The sections below contain information, or at least links to information, 
that should be helpful for anyone who wants to use Nexus, but who 
is not an expert in one of the following areas: installing python and related 
modules, installing PWSCF and QMCPACK, the Python programming language, and 
the theory and practice of Quantum Monte Carlo.  


\section{Helpful Links for Installing Python Modules} \label{install_python}
\begin{description}
  \item[Python itself] \hfill \\
    Download: \url{http://www.python.org/download/}\\  
    Be sure to get Python 2.x, not 3.x.
  \item[Numpy and Scipy] \hfill \\
    Download and installation: \url{http://www.scipy.org/Installing_SciPy}.
  \item[Matplotlib] \hfill \\
    Download: \url{http://matplotlib.org/downloads.html}\\
    Installation: \url{http://matplotlib.org/users/installing.html}
  \item[H5py] \hfill \\
    Download and installation: \url{http://www.h5py.org/}
\end{description}


\section{Helpful Links for Installing Electronic Structure Codes} \label{install_code}
\subsection{PWSCF: pw.x, pw2qmcpack.x, pw2casino.x}
  \subsubsection{Publicly available version}
  Download: svn co http://qmctools.googlecode.com/svn/dft/espresso-4.2 \\
  See also: \url{http://qmcpack.cmscc.org/}\\
  \subsubsection{Developer's version:} 
  Download: svn co https://subversion.assembla.com/svn/qmcdev/qe4.3.2 \\
  (QE 5.0 is not currently supported in Nexus)\\
  Installation instructions: See section 2 of the User Guide (user\_guide.pdf) 
  found in the Doc directory of the distribution.

\subsection{Wfconvert: wfconvert}
  Download: svn co http://qmctools.googlecode.com/svn/trunk/wfconvert \\
  See also: \url{http://qmcpack.cmscc.org/}
\subsection{QMCPACK: sqd, qmcapp, qmcapp\_complex}
  Install: \url{http://users.nccs.gov/~jnkim/qmcpack/ug/a00001.html}\\
  See also: \url{http://qmcpack.cmscc.org/getting-started}


\section{Brushing Up On Python}\label{learn_python}
\subsection{Python}
Python is a flexible, multi-paradigm, interpreted programming language with 
powerful intrinsic datatypes and a large library of modules that greatly expand 
its functionality.  A good way to learn the language is through the extensive 
Documentation provided on the python.org website. If you have never worked with 
Python before, be sure to go through the Tutorial. To learn more about the 
intrinsic data types and standard libraries look at Library Reference.
\begin{center}
  \begin{tabular}{|l|l|}
    \hline
    Documentation      & \url{http://docs.python.org/2/}\\ \hline
    Tutorial           & \url{http://docs.python.org/2/tutorial/index.html}\\ \hline
    Library Reference  & \url{http://docs.python.org/2/library/index.html}\\ \hline
  \end{tabular}
\end{center}

\subsection{NumPy}
Other than the Python Standard Library, the main library/module Nexus 
makes heavy use of is NumPy.  NumPy provides a convenient and fairly 
fast implementation of multi-dimensional arrays and related functions, much like 
MATLAB.  If you want to learn about NumPy arrays, the NumPy 
Tutorial is recommended.  For more detailed information, see the NumPy User Guide 
and the NumPy Reference Manual. If MATLAB is one of your native languages, check out 
NumPy for MATLAB Users.
\begin{center}
  \begin{tabular}{|l|l|}
    \hline
    Tutorial   & \url{http://www.scipy.org/Tentative_NumPy_Tutorial}\\ \hline
    User Guide & \url{http://docs.scipy.org/doc/numpy/user/index.html#user}\\ \hline
    Reference  & \url{http://docs.scipy.org/doc/numpy/reference/}\\ \hline
    MATLAB     & \url{http://www.scipy.org/NumPy_for_Matlab_Users}\\ \hline
 \end{tabular}
\end{center}

\subsection{Matplotlib}
Plotting in Nexus is currently handled by Matplotlib.  If you want 
to learn more about plotting with Matplotlib, the Pyplot Tutorial is a good place 
to start.  More detailed information is in the User's Guide.  Sometimes Examples 
provide the fastest way to learn.
\begin{center}
  \begin{tabular}{|l|l|}
    \hline
    Tutorial     & \url{http://matplotlib.org/users/pyplot_tutorial.html}\\ \hline
    User's Guide & \url{http://matplotlib.org/users/index.html}\\ \hline
    Examples     & \url{http://matplotlib.org/examples/} \\ \hline
  \end{tabular}
\end{center}

\subsection{Scipy and H5Py}
Nexus also occasionally uses functionality from SciPy and H5Py.  
Learning more about them is unlikely to help you interact with Nexus.  
However, they are quite valuable on their own.  SciPy provides access to 
special functions, numerical integration, optimization, interpolation, fourier 
transforms, eigenvalue solvers, and statistical analysis.  To get an overview, 
try the SciPy Tutorial.  More detailed material is found in the Scipy Reference.
H5Py provides a NumPy-like interface to HDF5 data files, which QMCPACK creates.  
To learn more about interacting with HDF5 files through H5Py, try the Quick Start 
Guide.  For more information, see the General Documentation.
\begin{center}
  \begin{tabular}{|l|l|}
    \hline
    SciPy Tutorial    & \url{http://docs.scipy.org/doc/scipy/reference/tutorial/index.html}\\ \hline
    SciPy Reference   & \url{http://docs.scipy.org/doc/scipy/reference/}\\ \hline
    H5Py Quick Guide  & \url{http://www.h5py.org/docs/intro/quick.html#quick}\\ \hline
    H5Py General Docs & \url{http://www.h5py.org/docs/}\\ \hline
  \end{tabular}
\end{center}



\section{Quantum Monte Carlo: Theory and Practice}\label{learn_qmc}
Currently, review articles may be the best way to get an overview of Quantum 
Monte Carlo methods and practices.  The review article by Foulkes, \emph{et al.} 
from 2001 remains quite relevant and is lucidly written.  Other review articles 
also provide a broader perspective on QMC, including more recent developments. 
Another resource that can be useful for newcomers (and needs to be updated) 
is the QMC Wiki.
If you are aware of resources that fill a gap in the information presented 
here (almost a certainty), please contact the developer at krogeljt@ornl.gov 
to add your contribution.
\begin{center}
  \begin{tabular}{|l|l|}
    \hline
    \multicolumn{2}{|c|}{QMC Review Articles} \\ \hline
    Foulkes,  2001  &  \url{http://rmp.aps.org/abstract/RMP/v73/i1/p33_1}  \\ \hline
    Bajdich,  2009  &  \url{http://www.physics.sk/aps/pub.php?y=2009&pub=aps-09-02}  \\ \hline
    Needs,    2010  &  \url{http://iopscience.iop.org/0953-8984/22/2/023201/}  \\ \hline
    Kolorenc, 2011  &  \url{http://iopscience.iop.org/0034-4885/74/2/026502/}  \\ \hline

    \multicolumn{2}{|c|}{Online Resources} \\ \hline
    QMCWiki                &  \url{www.qmcwiki.org} \\ \hline
    QMC Summer School 2012 & \url{http://www.mcc.uiuc.edu/summerschool/2012/program.html} \\ \hline
  \end{tabular}
\end{center}






\backmatter


\printindex


\end{document}
