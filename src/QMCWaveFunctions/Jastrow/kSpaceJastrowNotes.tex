\documentclass{article}
\usepackage{amsmath}
\newcommand{\vG}{\mathbf{G}}
\newcommand{\vr}{\mathbf{r}}
\newcommand{\vI}{\mathbf{I}}
\title{Notes on Reciprocal-Space Jastrow Factors}
\date{\today}
\author{Ken Esler}
\begin{document}
\maketitle
\section{Two-body Jastrow}
\begin{equation}
J_2 = \sum_{\vG\neq \mathbf{0}}\sum_{i\neq j} a_\vG e^{i\vG\cdot(\vr_i-\vr_j)}
\end{equation}
This may be rewritten as
\begin{eqnarray}
J_2 & = & \sum_{\vG\neq \mathbf{0}}\sum_{i\neq j} a_\vG e^{i\vG\cdot\vr_i}e^{-i\vG\cdot\vr_j} \\
& = & \sum_{\vG\neq \mathbf{0}} a_\vG \left\{
\underbrace{\left[\sum_i e^{i\vG\cdot\vr_i} \right]}_{\rho_\vG}
\underbrace{\left[\sum_j e^{-i\vG\cdot\vr_j} \right]}_{\rho_{-\vG}}  -1 \right\}
\end{eqnarray}
The $-1$ is just a constant term and may be subsumed into the $a_\vG$
coefficient by a simple redefintion.  This leaves a simple, but
general, form:
\begin{equation}
J_2 = \sum_{\vG\neq\mathbf{0}} a_\vG \rho_\vG \rho_{-\vG}
\end{equation}
We may now further constrain this on physical grounds.  First, we
recognize that $J_2$ should be real.  Since $\rho_{-\vG} =
\rho_\vG^*$, it follows that $\rho_{\vG}\rho_{-\vG} = |\rho_\vG|^2$ is
real, so that $a_\vG$ must be real.  Furthermore, we group the $\vG$'s
into $(+\vG, -\vG)$ pairs, and sum over only the positive vectors to
save time.

\section{One-body Jastrow}
The one-body Jastrow has a similar form, but depends on the
displacment from the electrons to the ions in the system.
\begin{equation}
J_1 = \sum_{\vG\neq\mathbf{0}} \sum_{\alpha}
\sum_{i\in\vI^\alpha}\sum_{j\in\text{elec.}} b^{\alpha}_\vG
  e^{i\vG\cdot(\vI^{\alpha}_i - \vr_j)},
\end{equation}
where $\alpha$ denotes the different ionic species.
We may rewrite this in terms of $\rho^{\alpha}_\vG$, 
\begin{equation}
J_1 = \sum_{\vG\neq\mathbf{0}} \left[\sum_\alpha b^\alpha_\vG
  \rho_\vG^\alpha\right] \rho_{-\vG},
\end{equation}
where
\begin{equation}
\rho^\alpha_\vG = \sum_{i\in\vI^\alpha} e^{i\vG\cdot\vI^\alpha_i}.
\end{equation}
We note that in the above equation, for a single configuration of the
ions, the sum in brackets can be rewritten as a single constant.  This
implies that the per-species one-body coefficents, $b^\alpha_\vG$, are
underdetermined for single configuration of the ions.  In general, if
we have $N$ species, we need $N$ linearly independent ion
configurations to uniquely determine $b^{\alpha}_\vG$.  For this
reason, we will drop the $\alpha$ superscript of $b_\vG$ for now.  

If we do desire to find a reciprocal space one-body Jastrow that is
transferable to systems with different ion positions and $N$ 
ionic species, we must perform compute $b_\vG$ for $N$ different ion
configurations.  We may then construct $N$ equations at each value of
$\vG$ to solve for the $N$ unknown values, $b^\alpha_\vG$.

In the two-body case, $a_\vG$ was constrained to be real by the fact
that $\rho_\vG \rho_{-\vG}$ was real.  However, in the one-body case,
there is no such guarantee about $\rho^\alpha_\vG \rho_\vG$.
Therefore, in general, $b_\vG$ may be complex.

\section{Symmetry considerations}
For a crystal, many of the $\vG$-vectors will be equivalent by
symmetry.  It is useful then, to divide the $\vG$-vectors into
symmetry-related groups and then to require that they share a common
coefficient.  Two vectors, $\vG$ and $\vG'$, may be considered
symmetry related if, for all $\alpha$ and $\beta$,
\begin{equation}
\rho^\alpha_\vG \rho^\beta_{-\vG} = \rho^\alpha_{\vG'} \rho^\beta_{-\vG'}. 
\end{equation}
For the one-body term, we may also omit from our list of $\vG$-vectors
those for which all species structure factors are zero.  This is
equivalent to saying that, if we are tiling a primitive cell, we
should include only the $\vG$-vectors of the primitive cell, and not
the supercell.  Note that this is not the case for the two-body term,
since the exchange-correlation hole should not have the periodicity of
the primitive cell.

\section{Gradients and Laplacians}
\begin{eqnarray}
\nabla_{\vr_i} J_2 & = & \sum_{\vG \neq 0} a_\vG \left[\left(\nabla_{\vr_i}\rho_\vG\right) \rho_{-\vG} + \text{c.c.}\right] \\
& = & \sum_{\vG\neq \mathbf{0}} 2\vG a_\vG \mathbf{Re}\left(i e^{i\vG\cdot\vr_i} \rho_{-\vG} \right) \\
& = & \sum_{\vG\neq \mathbf{0}} -2\vG a_\vG\mathbf{Im}\left(e^{i\vG\cdot\vr_i} \rho_{-\vG} \right)
\end{eqnarray}
The Laplacian is then given by
\begin{eqnarray}
  \nabla^2 J_2 & = & \sum_{\vG\neq\mathbf{0}} a_\vG \left[\left(\nabla^2 \rho_\vG\right) \rho_{-\vG} + \text{c.c.} 
  + 2\left(\nabla \rho_\vG)\cdot(\nabla \rho_{-\vG}\right)\right] \\
& = & \sum_{\vG\neq\mathbf{0}} a_\vG \left[ -2G^2\mathbf{Re}(e^{i\vG\cdot\vr_i}\rho_{-\vG}) + 
    2\left(i\vG e^{i\vG\cdot\vr_i}\right) \cdot \left(-i\vG e^{-i\vG\cdot\vr_i}\right)
\right] \\
& = & 2 \sum_{\vG\neq\mathbf{0}} G^2 a_\vG  \left[-\mathbf{Re}\left(e^{i\vG\cdot\vr_i}\rho_{-\vG}\right) + 1\right] 
%  \nabla^2_{\vr_i} J_2 & = & \nabla_{\vr_i} \cdot \nabla_{\vr_i} J_2 \\
%  & = & -2\sum_{\vG \neq \mathbf{0}} a_\vG \vG \cdot \nabla_{\vr_i} \mathbf{Im}\left(e^{i\vG\cdot\vr_i} \rho_{-\vG}\right)
%  & = & -2\sum_{\vG \neq \mathbf{0}} a_\vG \vG \cdot \mathbf{Im}\left(i\vG e^{i\vG\cdot\vr_i}\rho_{-\vG} -i\vG \right)
\end{eqnarray}



\end{document}
