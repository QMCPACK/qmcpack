\documentclass{article}
\usepackage{amsmath}
\author{Kenneth P. Esler Jr.}
\title{Computing the Zero-Variance term for Forces}
\begin{document}
\maketitle
The zero-variance term is of the form
\begin{equation}
\int \psi_T^2 (\hat{H} - E_L) \frac{\partial_\alpha \psi_T}{\psi_T}.
\end{equation}
The kinetic contribution will involve the term,
\begin{equation}
\frac{\nabla^2 \psi_T^\alpha}{\psi_T} = \nabla^2 \partial_\alpha \ln
\psi_T + 2 (\nabla \partial_\alpha \ln \psi_T)\cdot (\nabla \ln \psi_T)
\end{equation}
Let us begin with the term
\begin{equation}
\nabla \partial_\alpha \ln \psi_T =
\nabla\left(\frac{\psi^\alpha_T}{\psi_T}\right)
\end{equation}
For a Dirac determinant, we can write, dropping the subscript ``T'', 
\begin{equation}
\frac{\psi^\alpha}{\psi} = \phi^\alpha_{ij} M^{-1}_{ji},
\end{equation}
where $\phi^{\alpha}_{ij} = \partial_\alpha \phi_j(r_i)$ and $M_{ij} =
\phi_j(r_i)$.  We now want to compute the gradient of this term
w.r.t. the n$^\text{th}$ electron:
\begin{equation}
\nabla_n \left(\frac{\psi^\alpha_T}{\psi_T}\right)  = 
(\nabla_n \phi_{ij}^\alpha)M^{-1}_{ji} + \phi^\alpha_{ij} (\nabla_n M^{-1}_{ji}).
\end{equation}
The first term on the R.H.S. can be computed directly if the gradient
of the $\phi^\alpha$s are known.  For the second term, we make use of
the identity 
\begin{equation}
\nabla_n (M^{-1}) = -M^{-1} (\nabla_n M) M^{-1}.
\end{equation}
Then, defining 
\begin{eqnarray}
\phi^\alpha_{ij} (\nabla_n M^{-1}_{ji}) & = & -\text{Tr}\left[\phi^\alpha M^{-1}
  (\nabla_n M)M^{-1}\right] \\
& = & -\phi^\alpha_{ij} M^{-1}_{jk} (\nabla_n \phi_{kl}) M^{-1}_{li}
\end{eqnarray}

For the Laplacian, we have,
\begin{equation}
\nabla_n\cdot\nabla_n\left(\frac{\psi^\alpha}{\psi}\right) =
(\nabla_n^2\phi_{ij}^\alpha)M^{-1}_{ji} +
2(\nabla_n\phi_{ij})\cdot(\nabla_n M^{-1}_{ji}) +
\phi^\alpha_{ij}(\nabla^2 M^{-1}_{ji})
\end{equation} 
Now, the last term is the novel one:
\begin{eqnarray}
\nabla^2 M^{-1} & = & -\nabla\cdot [M^{-1}(\nabla M) M^{-1}] \\
& = & M^{-1}(\nabla M)M^{-1}(\nabla M) M^{-1} - M^{-1}(\nabla^2)M^{-1}
+ M^{-1}(\nabla M) M^{-1} (\nabla M) M^{-1} \\
& = & -M^{-1} (\nabla^2) M^{-1} + 2 M^{-1}(\nabla M) M^{-1}(\nabla M) M^{-1}
\end{eqnarray}



\end{document}