\chapter{Hamiltonian and Observables}
\label{chap:hamiltobs}



\begin{table}[h]
\begin{center}
\begin{tabularx}{\textwidth}{l l l l l l }
\hline
\multicolumn{6}{l}{\texttt{generic} element} \\
\hline
\multicolumn{2}{l}{parent elements:} & \multicolumn{4}{l}{\texttt{parent1 parent2}}\\
\multicolumn{2}{l}{child  elements:} & \multicolumn{4}{l}{\texttt{child1 child2 child3 ...}}\\
\multicolumn{2}{l}{attributes}  & \multicolumn{4}{l}{}\\
   &   \bfseries name     & \bfseries datatype & \bfseries values & \bfseries default   & \bfseries description \\
   &   \texttt{attr1}     &  text              &                  &                     &                       \\
   &   \texttt{attr2}     &  integer           &                  &                     &                       \\
   &   \texttt{attr3}     &  real              &                  &                     &                       \\
   &   \texttt{attr4}     &  boolean           &                  &                     &                       \\
   &   \texttt{attr4}     &  text array        &                  &                     &                       \\
   &   \texttt{attr4}     &  integer array     &                  &                     &                       \\
   &   \texttt{attr4}     &  real array        &                  &                     &                       \\
   &   \texttt{attr4}     &  boolean array     &                  &                     &                       \\
\multicolumn{2}{l}{parameters}  & \multicolumn{4}{l}{}\\
   &   \bfseries name     & \bfseries datatype & \bfseries values & \bfseries default   & \bfseries description \\
   &   \texttt{attr1}     &  text              &                  &                     &                       \\
   &   \texttt{attr2}     &  integer           &                  &                     &                       \\
   &   \texttt{attr3}     &  real              &                  &                     &                       \\
   &   \texttt{attr4}     &  boolean           &                  &                     &                       \\
   &   \texttt{attr4}     &  text array        &                  &                     &                       \\
   &   \texttt{attr4}     &  integer array     &                  &                     &                       \\
   &   \texttt{attr4}     &  real array        &                  &                     &                       \\
   &   \texttt{attr4}     &  boolean array     &                  &                     &                       \\
  \hline
\end{tabularx}
\end{center}
\end{table}




\begin{table}[h]
\begin{center}
\begin{tabularx}{\textwidth}{l l l l l l }
\hline
\multicolumn{6}{l}{\texttt{generic} factory element} \\
\hline
\multicolumn{2}{l}{parent elements:} & \multicolumn{4}{l}{\texttt{parent1 parent2}}\\
\multicolumn{2}{l}{child  elements:} & \multicolumn{4}{l}{\texttt{child1 child2 child3 ...}}\\
\multicolumn{2}{l}{type   selector:} & \multicolumn{4}{l}{\texttt{some} attribute}\\
\multicolumn{2}{l}{type   options :} & \multicolumn{4}{l}{Selection1}\\
\multicolumn{2}{l}{                } & \multicolumn{4}{l}{Selection2}\\
\multicolumn{2}{l}{                } & \multicolumn{4}{l}{Selection3}\\
\multicolumn{2}{l}{                } & \multicolumn{4}{l}{...}\\
\multicolumn{2}{l}{shared attributes:} & \multicolumn{4}{l}{}\\
   &   \bfseries name     & \bfseries datatype & \bfseries values & \bfseries default   & \bfseries description \\
   &   \texttt{attr1}     &  text              &                  &                     &                       \\
   &   \texttt{attr2}     &  integer           &                  &                     &                       \\
   &   ...                &                    &                  &                     &                       \\
  \hline
\end{tabularx}
\end{center}
\end{table}




\section{The Hamiltonian}

% hamiltonian element
%  dev notes
%    Hamiltonian element read
%      HamiltonianPool::put
%        reads attributes: id name role target 
%          id/name is passed to QMCHamiltonian
%          role selects the primary hamiltonian
%          target associates to quantum particleset
%      HamiltonianFactory::build
%        reads attributes: type source default
%    HamiltonianFactory cloning may be flawed for non-electron systems
%      see HamiltonianFactory::clone
%        aCopy->renameProperty(``e'',qp->getName());
%        aCopy->renameProperty(psiName,psi->getName());
%      the renaming may not work if dynamic particleset.name!=''e''
%   lots of xml inputs are simply ignored if do not explicitly match (fix! here and elsewhere in the build tree)

\FloatBarrier
\begin{table}[h]
\begin{center}
\begin{tabularx}{\textwidth}{l l l l l l }
\hline
\multicolumn{6}{l}{\texttt{hamiltonian} element} \\
\hline
\multicolumn{2}{l}{parent elements:} & \multicolumn{4}{l}{\texttt{qmcsystem}}\\
\multicolumn{2}{l}{child  elements:} & \multicolumn{4}{l}{\texttt{pairpot extpot estimator constant}(deprecated)}\\
\multicolumn{2}{l}{attributes}  & \multicolumn{4}{l}{}\\
   &   \bfseries name     & \bfseries datatype & \bfseries values & \bfseries default   & \bfseries description \\
   &   \texttt{name/id}$^o$ &  text              & \textit{anything}& h0                  & Unique id for this Hamiltonian instance                      \\
   &   \texttt{type}$^o$    &  text              &                  & generic             & \textit{No current function}                      \\
   &   \texttt{role}$^o$    &  text              & primary/extra    & extra               & Designate as primary Hamiltonian or not                      \\
   &   \texttt{source}$^o$  &  text              & \texttt{particleset.name} & i          & Identify classical particleset                      \\
   &   \texttt{target}$^r$  &  text              & \texttt{particleset.name} & e          & Identify quantum particlset                      \\
   &   \texttt{default}$^o$ &  boolean           & yes/no           & yes                 & Include kinetic energy term implicitly                      \\
  \hline
\end{tabularx}
\end{center}
\end{table}
\FloatBarrier

% All-electron hamiltonian element
\begin{lstlisting}[caption=All electron Hamiltonian XML element.]
<hamiltonian target="e">
  <pairpot name="ElecElec" type="coulomb" source="e" target="e"/>
  <pairpot name="ElecIon"  type="coulomb" source="i" target="e"/>
  <pairpot name="IonIon"   type="coulomb" source="i" target="i"/>
</hamiltonian>
\end{lstlisting}


% Pseudopotential hamiltonian element
\begin{lstlisting}[caption=Pseudopotential Hamiltonian XML element.]
<hamiltonian target="e">
  <pairpot name="ElecElec"  type="coulomb" source="e" target="e"/>
  <pairpot name="PseudoPot" type="pseudo"  source="i" wavefunction="psi0" format="xml">
    <pseudo elementType="Li" href="Li.xml"/>
    <pseudo elementType="H" href="H.xml"/>
  </pairpot>
  <pairpot name="IonIon"    type="coulomb" source="i" target="i"/>
</hamiltonian>
\end{lstlisting}


\section{Pair potentials}

% pairpot element
\FloatBarrier
\begin{table}[h]
\begin{center}
\begin{tabularx}{\textwidth}{l l l l l l }
\hline
\multicolumn{6}{l}{\texttt{pairpot} factory element} \\
\hline
\multicolumn{2}{l}{parent elements:} & \multicolumn{4}{l}{\texttt{hamiltonian}}\\
\multicolumn{2}{l}{type   selector:} & \multicolumn{4}{l}{\texttt{type} attribute}\\
\multicolumn{2}{l}{type   options: } & coulomb            & \multicolumn{3}{l}{Coulomb/Ewald potential}\\
\multicolumn{2}{l}{                } & pseudo             & \multicolumn{3}{l}{Semilocal pseudopotential}\\
\multicolumn{2}{l}{                } & mpc                & \multicolumn{3}{l}{Modified Periodic Coulomb interaction/correction}\\
\multicolumn{2}{l}{                } & cpp                & \multicolumn{3}{l}{Core polarization potential}\\
\multicolumn{2}{l}{                } & numerical/*num*    & \multicolumn{3}{l}{Numerical radial potential}\\
\multicolumn{2}{l}{                } & skpot              & \multicolumn{3}{l}{\textit{Unknown}}\\
\multicolumn{2}{l}{                } & vhxc               & \multicolumn{3}{l}{Exchange correlation potential (external)}\\
\multicolumn{2}{l}{                } & jellium            & \multicolumn{3}{l}{Atom-centered spherical jellium potential}\\
\multicolumn{2}{l}{                } & hardsphere         & \multicolumn{3}{l}{Hard sphere potential}\\
\multicolumn{2}{l}{                } & gaussian           & \multicolumn{3}{l}{Gaussian potential}\\
\multicolumn{2}{l}{                } & modpostel          & \multicolumn{3}{l}{Modified Poschl-Teller potential}\\
\multicolumn{2}{l}{                } & huse               & \multicolumn{3}{l}{Huse quintic potential}\\
\multicolumn{2}{l}{                } & modInsKE           & \multicolumn{3}{l}{Model insulator kinetic energy}\\
\multicolumn{2}{l}{                } & oscillatory        & \multicolumn{3}{l}{\textit{Unknown}}\\
\multicolumn{2}{l}{                } & LJP\_smoothed      & \multicolumn{3}{l}{Helium pair potential}\\
\multicolumn{2}{l}{                } & HeSAPT\_smoothed   & \multicolumn{3}{l}{Helium pair potential}\\
\multicolumn{2}{l}{                } & HFDHE2\_Moroni1995 & \multicolumn{3}{l}{Helium pair potential}\\
\multicolumn{2}{l}{                } & HFDHE2             & \multicolumn{3}{l}{Helium pair potential}\\
\multicolumn{2}{l}{                } & eHe                & \multicolumn{3}{l}{Helium-electron pair potential}\\
\multicolumn{2}{l}{shared attributes:} & \multicolumn{4}{l}{}\\
   &   \bfseries name     & \bfseries datatype & \bfseries values & \bfseries default   & \bfseries description \\
   &   \texttt{type}$^r$      &  text              & \textit{See above}        & 0                   & Select pairpot type         \\
   &   \texttt{name}$^r$      &  text              & \textit{anything}         & any                 & Unique name for this pairpot\\
   &   \texttt{source}$^r$    &  text              & \texttt{particleset.name} &\texttt{hamiltonian.target}& Identify interacting particles\\
   &   \texttt{target}$^r$    &  text              & \texttt{particleset.name} &\texttt{hamiltonian.target}& Identify interacting particles  \\
   &   \texttt{units}$^o$     &  text              &                           & hartree             & \textit{No current function}  \\
\hline
\end{tabularx}
\end{center}
\end{table}
\FloatBarrier

Remarks
\begin{itemize}
  \item{If an interaction is between classical (e.g. ions) and quantum (e.g. electrons), \texttt{source}/\texttt{target} should be the name of the classical/quantum particleset.}
  \item{Only \texttt{coulomb, pseudo, mpc} are described in detail below.  The older or less used types (\texttt{cpp, numerical, jellium, hardsphere, gaussian, huse, modpostel, oscillatory, skpot, vhxc, modInsKE, LJP\_smoothed, HeSAPT\_smoothed, HFDHE2\_Moroni1995, eHe, HFDHE2}) are not covered.}
  \item{Available only if \texttt{QMC\_BUILD\_LEVEL>2} and \texttt{QMC\_CUDA} is not defined: \texttt{hardsphere, gaussian, huse, modpostel, oscillatory, skpot}.}
  \item{Available only if \texttt{OHMMS\_DIM==3}: \texttt{mpc, vhxc, pseudo}.}
  \item{Available only if \texttt{OHMMS\_DIM==3} and \texttt{QMC\_BUILD\_LEVEL>2} and \texttt{QMC\_CUDA} is not defined: \texttt{cpp, LJP\_smoothed, HeSAPT\_smoothed, HFDHE2\_Moroni1995, eHe, jellium, HFDHE2, modInsKE}.}
\end{itemize}


% physical read by coulomb potentials
% potential is only for pressure estimator



% pairpot instances

%   do coulomb, pseudo, mpc

\subsection{Coulomb potentials}

\FloatBarrier
\begin{table}[h]
\begin{center}
\begin{tabularx}{\textwidth}{l l l l l l }
\hline
\multicolumn{6}{l}{\texttt{pairpot type=coulomb} element} \\
\hline
\multicolumn{2}{l}{parent elements:} & \multicolumn{4}{l}{\texttt{hamiltonian}}\\
\multicolumn{2}{l}{child  elements:} & \multicolumn{4}{l}{\textit{None}}\\
\multicolumn{2}{l}{attributes}  & \multicolumn{4}{l}{}\\
   &   \bfseries name     & \bfseries datatype & \bfseries values & \bfseries default   & \bfseries description \\
   & \texttt{type}$^r$    &  text              & \textbf{coulomb} &                     & Must be coulomb     \\
   & \texttt{name/id}$^r$ &  text              & \textit{anything}&  ElecElec           & Unique name for interaction\\
   & \texttt{source}$^r$  &  text              & \texttt{particleset.name} &\texttt{hamiltonian.target}& Identify interacting particles\\
   & \texttt{target}$^r$  &  text              & \texttt{particleset.name} &\texttt{hamiltonian.target}& Identify interacting particles\\
   & \texttt{pbc}$^o$     &  boolean           & yes/no           & yes$^*$             & Use Ewald summation  \\
   & \texttt{physical}$^o$&  boolean           & yes/no           & yes                 & Hamiltonian(yes)/observable(no) \\
   & \texttt{forces}      &  boolean           & yes/no           & no                  & \textit{Deprecated}             \\
  \hline
\end{tabularx}
\end{center}
\end{table}
\FloatBarrier

Remarks
\begin{itemize}
  \item{Ewald summation will not be performed if \texttt{simulationcell.bconds== n n n}, regardless of the value of \texttt{pbc}.  Similarly, the \texttt{pbc} attribute can only be used to turn off Ewald summation if \texttt{simulationcell.bconds!= n n n}.}
  \item{Note: traditional names for electron-electron, electron-ion, and ion-ion terms are \texttt{ElecElec}, \texttt{ElecIon}, and \texttt{IonIon}, respectively.  While any choice can be used, the data analysis tools expect to find columns in \texttt{*.scalar.dat} with these names.}
\end{itemize}


\subsection{Pseudopotentials}

% pseudopotential element
%   dev notes
%     attributes name, source, wavefunction, format are read in CoulombFactory.cpp  HamiltonianFactory::addPseudoPotential
%     format==''old'' refers to an old table format that is no longer supported
%     read continues in ECPotentialBuilder::put()
%       if format!=xml/old (i.e. table) qmcpack will attempt to read from *.psf files
%         in this case, <pairpot type=''pseudo'' format=''table''/>, ie there are no elements
%         if particlset groups are Li H (in order), then it looks for Li.psf and H.psf
%         what is the psf format?
%       if format==xml, normal read continues, i.e. <pseudo/> child elements are expected
%         read is not sensitive to particleset group/species ordering
%         child elements not named <pseudo/> are simply ignored (FIX!)
\FloatBarrier
\begin{table}[h]
\begin{center}
\begin{tabularx}{\textwidth}{l l l l l l }
\hline
\multicolumn{6}{l}{\texttt{pairpot type=pseudo} element} \\
\hline
\multicolumn{2}{l}{parent elements:} & \multicolumn{4}{l}{\texttt{hamiltonian}}\\
\multicolumn{2}{l}{child  elements:} & \multicolumn{4}{l}{\texttt{pseudo}}\\
\multicolumn{2}{l}{attributes}  & \multicolumn{4}{l}{}\\
   &   \bfseries name     & \bfseries datatype & \bfseries values & \bfseries default   & \bfseries description \\
   & \texttt{type}$^r$    &  text              & \textbf{pseudo} &                      & Must be pseudo         \\
   & \texttt{name/id}$^r$ &  text              & \textit{anything}&  PseudoPot          & \textit{No current function}\\
   & \texttt{source}$^r$  &  text              & \texttt{particleset.name} &  i                  & Ion particleset name\\
   & \texttt{target}$^r$  &  text              & \texttt{particleset.name} &\texttt{hamiltonian.target}& Electron particleset name  \\
   & \texttt{pbc}$^o$     &  boolean           & yes/no           & yes$^*$             & Use Ewald summation  \\
   & \texttt{forces}      &  boolean           & yes/no           & no                  & \textit{Deprecated}             \\
   &\texttt{wavefunction}$^r$ &  text          & \texttt{wavefunction.name}& invalid    & Identify wavefunction \\
   &   \texttt{format}$^r$    &  text          & xml/table        & table               & Select file format   \\
  \hline
\end{tabularx}
\end{center}
\end{table}
\FloatBarrier

Remarks
\begin{itemize}
  \item{Ewald summation will not be performed if \texttt{simulationcell.bconds== n n n}, regardless of the value of \texttt{pbc}.  Similarly, the \texttt{pbc} attribute can only be used to turn off Ewald summation if \texttt{simulationcell.bconds!= n n n}.}
  \item{Additional fields will be present in \texttt{*scalar.dat} output files when pseudopotentials are used.  The field \texttt{LocalECP} refers to the local part of the pseudopotential.  If non-local channels are present, a \texttt{NonLocalECP} field will be added that contains the non-local energy summed over all angular momentum channels.}
  \item{If \texttt{format}==table}, QMCPACK looks for \texttt{*.psf} files containing pseudopotential data in a tabular format.  The files must be named after the ionic species provided in \texttt{particleset} (\emph{e.g.} \texttt{Li.psf} and \texttt{H.psf}). 
  \item{If \texttt{format}==xml, additional \texttt{pseudo} child XML elements must be provided (see below).  These elements specify individual file names and formats (both the FSAtom XML and CASINO tabular data formats are supported).}
\end{itemize}



% pseudo element
%   dev notes
%     initial read of href elementType/symbol attributes at ECPotentialBuilder::useXmlFormat()
%     read continues in ECPComponentBuilder
%       format==xml and href==none (not provided) => ECPComponentBuilder::put(cur)
%       format==xml and href==a file => ECPComponentBuilder::parse(href,cur)
%       format==casino => ECPComponentBuilder::parseCasino(href,cur)
%         this reader is tucked away in ECPComponentBuilder.2.cpp
%         nice demonstration of OhmmsAsciiParser here
%         maximum cutoff defined by a 1.e-5 (Ha?) spread in the nonlocal potentials
%     quadrature rules (1-7) set as in J. Chem. Phys. 95 (3467) (1991), see below
%       Rule     # points     lexact
%        1           1          0
%        2           4          2
%        3           6          3
%        4          12          5
%        5          18          5
%        6          26          7
%        7          50         11
%     looks like channels only go from s-g (see ECPComponentBuilder constructor)
%       perhaps not, quadrature rules really do go up to 7 (lexact==11), see SetQuadratureRule()
\FloatBarrier
\begin{table}[h]
\begin{center}
\begin{tabularx}{\textwidth}{l l l l l l }
\hline
\multicolumn{6}{l}{\texttt{pseudo} element} \\
\hline
\multicolumn{2}{l}{parent elements:} & \multicolumn{4}{l}{\texttt{pairpot type=pseudo}}\\
\multicolumn{2}{l}{child  elements:} & \multicolumn{4}{l}{\texttt{header local grid}}\\
\multicolumn{2}{l}{attributes}  & \multicolumn{4}{l}{}\\
   &   \bfseries name     & \bfseries datatype & \bfseries values & \bfseries default   & \bfseries description \\
   & \texttt{elementType/symbol}$^r$&  text    &\texttt{group.name}& none               & Identify ionic species   \\
   & \texttt{href}$^r$    &  text              & \textit{filepath}& none                & Pseudopotential file path\\
   & \texttt{format}$^r$  &  text              & xml/casino       & xml                 & Specify file format\\
   & \texttt{cutoff}$^o$  &  real              &                  &                     & Non-local cutoff radius  \\
   & \texttt{lmax}$^o$    &  integer           &                  &                     & Largest angular momentum  \\
   & \texttt{nrule}$^o$   &  integer           &                  &                     & Integration grid order             \\
  \hline
\end{tabularx}
\end{center}
\end{table}
\FloatBarrier



\subsection{Modified periodic coulomb interaction/correction}

% mpc element
%  dev notes
%    most attributes are read in CoulombPotentialFactory.cpp  HamiltonianFactory::addMPCPotential()
%    user input for the name attribute is ignored and the name is always MPC
%    density G-vectors are stored in ParticleSet: Density_G and DensityReducedGvecs members
%    check the Linear Extrap and Quadratic Extrap output in some real examples (see MPC::init_f_G())
%      what are acceptable values for the discrepancies?
%      check that these decrease as cutoff is increased 
%    commented out code for MPC.dat creation in MPC::initBreakup()
%    short range part is 1/r, MPC::evalSR()
%    long range part is on a spline (VlongSpline), MPC::evalLR()
\FloatBarrier
\begin{table}[h]
\begin{center}
\begin{tabularx}{\textwidth}{l l l l l l }
\hline
\multicolumn{6}{l}{\texttt{pairpot type=mpc} element} \\
\hline
\multicolumn{2}{l}{parent elements:} & \multicolumn{4}{l}{\texttt{hamiltonian}}\\
\multicolumn{2}{l}{child  elements:} & \multicolumn{4}{l}{\textit{None}}\\
\multicolumn{2}{l}{attributes}  & \multicolumn{4}{l}{}\\
   &   \bfseries name     & \bfseries datatype & \bfseries values & \bfseries default   & \bfseries description \\
   & \texttt{type}$^r$    &  text              & \textbf{mpc}     &                     & Must be mpc         \\
   & \texttt{name/id}$^r$ &  text              & \textit{anything}&  MPC                & Unique name for interaction \\
   & \texttt{source}$^r$  &  text              & \texttt{particleset.name} &\texttt{hamiltonian.target}& Identify interacting particles\\
   & \texttt{target}$^r$  &  text              & \texttt{particleset.name} &\texttt{hamiltonian.target}& Identify interacting particles  \\
   & \texttt{physical}$^o$&  boolean           & yes/no           & no                  & Hamiltonian(yes)/observable(no) \\
   &  \texttt{cutoff}     &  real              & \textit{positive real}& 30.0           & Kinetic energy cutoff \\
  \hline
\end{tabularx}
\end{center}
\end{table}
\FloatBarrier
Remarks
\begin{itemize}
  \item{Note the meaning of the \texttt{physical} attribute.  Typically it is set to \texttt{no}, meaning the standard Ewald interaction will be used during sampling and MPC will be measured as an observable for finite-size post correction.  If \texttt{physical==yes}, the MPC interaction will be used during sampling.  In this case an electron-electron Coulomb \texttt{pairpot} element should not be supplied.}
  \item{Developer note: Currently the \texttt{name} attribute for the mpc interaction is ignored.  The name is always \texttt{MPC}.}
\end{itemize}

% MPC correction
\begin{lstlisting}[caption=Modified periodic coulomb for finite size post-correction.]
  <pairpot type="MPC" name="MPC" source="e" target="e" ecut="60.0" physical="no"/>
\end{lstlisting}



% estimator element
\section{General estimators}

\FloatBarrier
\begin{table}[h]
\begin{center}
\begin{tabularx}{\textwidth}{l l l l l l }
\hline
\multicolumn{6}{l}{\texttt{estimator} factory element} \\
\hline
\multicolumn{2}{l}{parent elements:} & \multicolumn{4}{l}{\texttt{hamiltonian, qmc}}\\
\multicolumn{2}{l}{type   selector:} & \multicolumn{4}{l}{\texttt{type} attribute}\\
\multicolumn{2}{l}{type   options: } & density            & \multicolumn{3}{l}{Density on a grid}\\
\multicolumn{2}{l}{                } & spindensity        & \multicolumn{3}{l}{Spin density on a grid}\\
\multicolumn{2}{l}{                } & gofr               & \multicolumn{3}{l}{Pair correlation function (quantum species)}\\
\multicolumn{2}{l}{                } & sk                 & \multicolumn{3}{l}{Static structure factor}\\
\multicolumn{2}{l}{                } & structurefactor    & \multicolumn{3}{l}{Species resolved structure factor}\\
\multicolumn{2}{l}{                } & momentum           & \multicolumn{3}{l}{Momentum distribution}\\
\multicolumn{2}{l}{                } & energydensity      & \multicolumn{3}{l}{Energy density on uniform or Voronoi grid}\\
\multicolumn{2}{l}{                } & dm1b               & \multicolumn{3}{l}{One body density matrix in arbitrary basis}\\
\multicolumn{2}{l}{                } & chiesa             & \multicolumn{3}{l}{Chiesa-Ceperley-Martin-Holzmann kinetic energy correction}\\
\multicolumn{2}{l}{                } & Force              & \multicolumn{3}{l}{Family of ``force'' estimators (see \ref{sec:force_est})}\\
\multicolumn{2}{l}{                } & ForwardWalking     & \multicolumn{3}{l}{Forward walking values for existing estimators}\\
\multicolumn{2}{l}{                } & orbitalimages      & \multicolumn{3}{l}{Create image files for orbitals, then exit}\\
\multicolumn{2}{l}{                } & flux               & \multicolumn{3}{l}{Checks sampling of kinetic energy}\\
\multicolumn{2}{l}{                } & localmoment        & \multicolumn{3}{l}{Atomic spin polarization within cutoff radius}\\
\multicolumn{2}{l}{                } & numberfluctuations & \multicolumn{3}{l}{Spatial number fluctuations}\\
\multicolumn{2}{l}{                } & HFDHE2             & \multicolumn{3}{l}{Helium pressure}\\
\multicolumn{2}{l}{                } & NearestNeighbors   & \multicolumn{3}{l}{Trace nearest neighbor indices}\\
\multicolumn{2}{l}{                } & Kinetic            & \multicolumn{3}{l}{\textit{No current function}}\\
\multicolumn{2}{l}{                } & Pressure           & \multicolumn{3}{l}{\textit{No current function}}\\
\multicolumn{2}{l}{                } & ZeroVarObs         & \multicolumn{3}{l}{\textit{No current function}}\\
\multicolumn{2}{l}{                } & DMCCorrection      & \multicolumn{3}{l}{\textit{No current function}}\\
\multicolumn{2}{l}{shared attributes:} & \multicolumn{4}{l}{}\\
   &   \bfseries name     & \bfseries datatype & \bfseries values & \bfseries default   & \bfseries description \\
   &   \texttt{type}$^r$      &  text              & \textit{See above}        & 0                   & Select estimator type         \\
   &   \texttt{name}$^r$      &  text              & \textit{anything}         & any                 & Unique name for this estimator\\
   %&   \texttt{source}$^r$    &  text              & \texttt{particleset.name} &\texttt{hamiltonian.target}& Identify interacting particles\\
   %&   \texttt{target}$^r$    &  text              & \texttt{particleset.name} &\texttt{hamiltonian.target}& Identify interacting particles  \\
   %&   \texttt{units}$^o$     &  text              &                           & hartree             & \textit{No current function}  \\
\hline
\end{tabularx}
\end{center}
\end{table}
\FloatBarrier

Remarks
\begin{itemize}
  \item{When an \texttt{<estimator/>} element appears in \texttt{<hamiltonian/>}, it is evaluated for all applicable chained QMC runs (\emph{e.g.} VMC$\rightarrow$DMC$\rightarrow$DMC).  When provided in a particular \texttt{<qmc/>} element, an estimator is only evaluated for that specific sub-run.}
  \item{Only \texttt{density, spindensity, gofr, sk, energydensity, dm1b, chiesa, Force} are described in detail below.  Other types (\texttt{structurefactor, momentum, ForwardWalking, orbitalimages, flux, localmoment, numberfluctuations, HFDHE2, NearestNeighbors}) are not yet covered.}
\end{itemize}


%  <estimator type="structurefactor" name="StructureFactor" report="yes"/>
%  <estimator type="nofk" name="nofk" wavefunction="psi0"/>



\FloatBarrier
\begin{table}[h]
\begin{center}
\begin{tabularx}{\textwidth}{l l l l l l }
\hline
\multicolumn{6}{l}{\texttt{estimator type=X} element} \\
\hline
\multicolumn{2}{l}{parent elements:} & \multicolumn{4}{l}{\texttt{hamiltonian, qmc}}\\
\multicolumn{2}{l}{child  elements:} & \multicolumn{4}{l}{\textit{None}}\\
\multicolumn{2}{l}{attributes}  & \multicolumn{4}{l}{}\\
   &   \bfseries name     & \bfseries datatype & \bfseries values & \bfseries default   & \bfseries description \\
   & \texttt{type}$^r$    &  text              & \textbf{X}     &                     & Must be X         \\
   & \texttt{name}$^r$    &  text              & \textit{anything}&                  & Unique name for estimator \\
   & \texttt{source}$^o$  &  text              & \texttt{particleset.name} &\texttt{hamiltonian.target}& Identify particles\\
   & \texttt{target}$^o$  &  text              & \texttt{particleset.name} &\texttt{hamiltonian.target}& Identify particles  \\
  \hline
\end{tabularx}
\end{center}
\end{table}
\FloatBarrier


\subsection{Chiesa-Ceperley-Martin-Holzmann kinetic energy correction}


\FloatBarrier
\begin{table}[h]
\begin{center}
\begin{tabularx}{\textwidth}{l l l l l l }
\hline
\multicolumn{6}{l}{\texttt{estimator type=chiesa} element} \\
\hline
\multicolumn{2}{l}{parent elements:} & \multicolumn{4}{l}{\texttt{hamiltonian, qmc}}\\
\multicolumn{2}{l}{child  elements:} & \multicolumn{4}{l}{\textit{None}}\\
\multicolumn{2}{l}{attributes}  & \multicolumn{4}{l}{}\\
   &   \bfseries name     & \bfseries datatype & \bfseries values & \bfseries default   & \bfseries description \\
   & \texttt{type}$^r$    &  text              & \textbf{chiesa}            &        & Must be chiesa         \\
   & \texttt{name}$^o$    &  text              & \textit{anything}          & KEcorr & Always reset to KEcorr \\
   & \texttt{source}$^o$  &  text              & \texttt{particleset.name}  & e      & Identify quantum particles\\
   & \texttt{psi}$^o$     &  text              & \texttt{wavefunction.name} & psi0   & Identify wavefunction  \\
  \hline
\end{tabularx}
\end{center}
\end{table}
\FloatBarrier

% kinetic energy correction
\begin{lstlisting}[caption=``Chiesa'' kinetic energy finite size post-correction.]
   <estimator name="KEcorr" type="chiesa" source="e" psi="psi0"/>
\end{lstlisting}



\subsection{Density estimator}


\FloatBarrier
\begin{table}[h]
\begin{center}
\begin{tabularx}{\textwidth}{l l l l l l }
\hline
\multicolumn{6}{l}{\texttt{estimator type=density} element} \\
\hline
\multicolumn{2}{l}{parent elements:} & \multicolumn{4}{l}{\texttt{hamiltonian, qmc}}\\
\multicolumn{2}{l}{child  elements:} & \multicolumn{4}{l}{\textit{None}}\\
\multicolumn{2}{l}{attributes}  & \multicolumn{4}{l}{}\\
   &   \bfseries name     & \bfseries datatype & \bfseries values  & \bfseries default   & \bfseries description \\
   & \texttt{type}$^r$      &  text              & \textbf{density}      &                     & Must be density         \\
   & \texttt{name}$^r$      &  text              & \textit{anything}     &                     & Unique name for estimator \\
   & \texttt{delta}$^o$     &  real array(3)     & $0\le \Delta_i \le 1$ & $\Delta_i=0.1$      & Grid cell spacing, unit coords\\
   & \texttt{x\_min}$^o$    &  real              & $>0$                  & 0                   & \\
   & \texttt{x\_max}$^o$    &  real              & $>0$                  & ${lattice_0}$  & \\
   & \texttt{y\_min}$^o$    &  real              & $>0$                  & 0                   & \\
   & \texttt{y\_max}$^o$    &  real              & $>0$                  & ${lattice_1}$  & \\
   & \texttt{z\_min}$^o$    &  real              & $>0$                  & 0                   & \\
   & \texttt{z\_max}$^o$    &  real              & $>0$                  & ${lattice_2}$  & \\
   & \texttt{potential}$^o$ &  boolean           & yes/no                & no                  & \\
   & \texttt{debug}$^o$     &  boolean           & yes/no                & no                  & \\
  \hline
\end{tabularx}
\end{center}
\end{table}
\FloatBarrier


Remarks
\begin{itemize}
  \item{Use of \texttt{x\_min, x\_max, y\_min, y\_max, z\_min, z\_max} is only appropriate for orthorhombic simulation cells with open boundary conditions.}
\end{itemize}


%  <estimator name="Density" type="density" delta="0.25 0.5 0.5"/>


\subsection{Spin density estimator}

%  <estimator type="spindensity" name="SpinDensity" report="yes">
%    <parameter name="grid"> 40 40 40 </parameter>
%  </estimator>
%   
%  <estimator type="spindensity" name="SpinDensity" report="yes">
%    <parameter name="grid">
%      20 20 20
%    </parameter>
%    <parameter name="center">
%      0.0 0.0 0.0
%    </parameter>
%    <parameter name="cell">
%      10.0  0.0  0.0
%       0.0 10.0  0.0
%       0.0  0.0 10.0
%    </parameter>
%  </estimator>


\subsection{Pair correlation function, $g(r)$}
%  <estimator type="gofr" name="gofr" num_bin="200" rmax="3.0" target="e" />
%  <estimator type="gofr" name="gofr" num_bin="200" rmax="3.0" target="e" source="ion0" />


\subsection{Static structure factor, $S(k)$}

%  <estimator type="sk" hdf5="yes"/>


\subsection{Energy density estimator}

%  <estimator type="EnergyDensity" name="EDcell" dynamic="e" static="ion0">
%    <reference_points coord="cartesian">
%      r1 1 0 0 
%      r2 0 1 0
%      r3 0 0 1
%    </reference_points>
%    <spacegrid coord="cartesian">
%      <origin p1="zero"/>
%      <axis p1="a1" scale=".5" label="x" grid="-1 (.05) 1"/>
%      <axis p1="a2" scale=".5" label="y" grid="-1 (.1) 1"/>
%      <axis p1="a3" scale=".5" label="z" grid="-1 (.1) 1"/>
%    </spacegrid>
%  </estimator>
%
%  <estimator type="EnergyDensity" name="EDatom" dynamic="e" static="ion0">
%    <reference_points coord="cartesian">
%      r1 1 0 0 
%      r2 0 1 0
%      r3 0 0 1
%    </reference_points>
%    <spacegrid coord="spherical">
%      <origin p1="ion01"/>
%      <axis p1="r1" scale="6.9" label="r"     grid="0 1"/>
%      <axis p1="r2" scale="6.9" label="phi"   grid="0 1"/>
%      <axis p1="r3" scale="6.9" label="theta" grid="0 1"/>
%    </spacegrid>
%    <spacegrid coord="spherical">
%      <origin p1="ion02"/>
%      <axis p1="r1" scale="6.9" label="r"     grid="0 1"/>
%      <axis p1="r2" scale="6.9" label="phi"   grid="0 1"/>
%      <axis p1="r3" scale="6.9" label="theta" grid="0 1"/>
%    </spacegrid>
%  </estimator>
%
%  <estimator type="EnergyDensity" name="EDvoronoi" dynamic="e" static="ion0">
%    <spacegrid coord="voronoi"/>
%  </estimator>



\subsection{One body density matrix}

%  <estimator type="dm1b" name="DensityMatrices">
%     <parameter name="energy_matrix"       >    yes                   </parameter>
%     <parameter name="integrator"          >    uniform_grid            </parameter>
%     <parameter name="points"              >    6                     </parameter>
%     <parameter name="scale"               >    1.0                   </parameter>
%     <parameter name="basis"               >
%        spo_dm
%     </parameter>
%     <parameter name="evaluator"           >    matrix                </parameter>
%     <parameter name="center">
%        0 0 0
%     </parameter>
%     <parameter name="check_overlap"       >    no                    </parameter>
%  </estimator>
%
%  <sposet_builder type="bspline" href="./dft/pwscf_output/pwscf.pwscf.h5" tilematrix="1 0 0 0 1 0 0 0 1" twistnum="0" meshfactor="1.0" gpu="no" precision="single" sort="0">
%    <sposet type="bspline" name="spo_u" size="4" spindataset="0"/>
%    <sposet type="bspline" name="spo_d" size="2" spindataset="1"/>
%    <sposet type="bspline" name="dm_basis" size="50" spindataset="0"/>
%  </sposet_builder>
%  <estimator type="dm1b" name="DensityMatrices">
%    <parameter name="energy_matrix"       >    yes                   </parameter>
%    <parameter name="integrator"          >    uniform_grid          </parameter>
%    <parameter name="points"              >    10                    </parameter>
%    <parameter name="scale"               >    1.0                   </parameter>
%    <parameter name="basis"               >    dm_basis              </parameter>
%    <parameter name="normalized"          >    no                    </parameter>
%    <parameter name="evaluator"           >    matrix                </parameter>
%    <parameter name="center"              >    0 0 0                 </parameter>
%    <parameter name="check_overlap"       >    no                    </parameter>
%    <parameter name="rstats"              >    no                    </parameter>
%  </estimator>
%
%
% found at /psi2/home/development/qmcpack/energy_density_matrix/tests/r6080_edm/02_atoms/runs/O/qmc/vmc.in.xml

%
%  <sposet_builder type="bspline" href="../dft/pwscf_output/pwscf.pwscf.h5" tilematrix="1 0 0 0 1 0 0 0 1" twistnum="0" meshfactor="1.0" gpu="no" precision="single">
%    <sposet type="bspline" name="spo_u"  group="0" size="4"/>
%    <sposet type="bspline" name="spo_d"  group="0" size="2"/>
%    <sposet type="bspline" name="spo_uv" group="0" index_min="4" index_max="10"/>
%  </sposet_builder>
%  <estimator type="dm1b" name="DensityMatrices">
%    <parameter name="basis"        >  spo_u spo_uv  </parameter>
%    <parameter name="energy_matrix">  yes           </parameter>
%    <parameter name="evaluator"    >  matrix        </parameter>
%    <parameter name="center"       >  0 0 0         </parameter>
%    <parameter name="rstats"           >  no        </parameter>
%    <parameter name="acceptance_ratio" >  no        </parameter>
%    <parameter name="check_overlap"    >  no        </parameter>
%    <parameter name="check_derivatives">  no        </parameter>
%    
%    <parameter name="integrator"   >  uniform_grid  </parameter>
%    <parameter name="points"       >  20            </parameter>
%    <parameter name="scale"        >  1.0           </parameter>
% 
%    <!--
%    <parameter name="integrator"   >  uniform       </parameter>
%    <parameter name="samples"      >  14          </parameter>
%    <parameter name="scale"        >  1.0           </parameter>
%    -->
%    
%    <!--
%    <parameter name="integrator"   >  density       </parameter>
%    <parameter name="timestep"     >  1.0           </parameter>
%    <parameter name="use_drift"    >  no            </parameter>
%    <parameter name="samples"      >  1000          </parameter>
%    -->
%    
%    <!--
%    <parameter name="integrator"   >  density       </parameter>
%    <parameter name="timestep"     >  1.0           </parameter>
%    <parameter name="use_drift"    >  yes           </parameter>
%    <parameter name="samples"      >  1000          </parameter>
%    -->
%  </estimator>







\section{``Force'' estimators} \label{sec:force_est}

% Force estimators added in CoulombPotentialFactory.cpp, HamiltonianFactory::addForceHam

\FloatBarrier
\begin{table}[h]
\begin{center}
\begin{tabularx}{\textwidth}{l l l l l l }
\hline
\multicolumn{6}{l}{\texttt{estimator type=Force} factory element} \\
\hline
\multicolumn{2}{l}{parent elements:} & \multicolumn{4}{l}{\texttt{hamiltonian, qmc}}\\
\multicolumn{2}{l}{type   selector:} & \multicolumn{4}{l}{\texttt{mode} attribute}\\
\multicolumn{2}{l}{type   options: } & bare           & \multicolumn{3}{l}{Bare force estimator}\\
\multicolumn{2}{l}{                } & cep            & \multicolumn{3}{l}{Ceperley force estimator}\\
\multicolumn{2}{l}{                } & pulay          & \multicolumn{3}{l}{Pulay force estimator}\\
\multicolumn{2}{l}{                } & zero\_variance & \multicolumn{3}{l}{Zero variance force estimator}\\
\multicolumn{2}{l}{                } & stress         & \multicolumn{3}{l}{Stress estimator}\\
\multicolumn{2}{l}{shared attributes:} & \multicolumn{4}{l}{}\\
   &   \bfseries name     & \bfseries datatype & \bfseries values & \bfseries default   & \bfseries description \\
   &   \texttt{mode}$^o$      &  text              & \textit{See above}        & bare          & Select estimator type\\
   &   \texttt{type}$^r$      &  text              & \textbf{Force}            &               & Must be Force         \\
   &   \texttt{name}$^o$      &  text              & \textit{anything}         & ForceBase     & Unique name for this estimator\\
   &   \texttt{source}$^o$    &  text              & \texttt{particleset.name} & ion0          & Identify classical particleset\\
   &   \texttt{target}$^o$    &  text              & \texttt{particleset.name} & e             & Identify quantum particleset\\
   &   \texttt{psi}$^o$       &  text              & \texttt{wavefunction.name}& psi0          & Identify wavefunction\\
   &   \texttt{pbc}$^o$       &  boolean           & yes/no                    & yes           & Using periodic BC's\\
\hline
\end{tabularx}
\end{center}
\end{table}
\FloatBarrier
