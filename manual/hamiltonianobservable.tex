\chapter{Hamiltonian and Observables}
\label{chap:hamiltobs}



\begin{table}[h]
\begin{center}
\begin{tabularx}{\textwidth}{l l l l l l }
\hline
\multicolumn{6}{l}{\texttt{generic} element} \\
\hline
\multicolumn{2}{l}{parent elements:} & \multicolumn{4}{l}{\texttt{parent1 parent2}}\\
\multicolumn{2}{l}{child  elements:} & \multicolumn{4}{l}{\texttt{child1 child2 child3 ...}}\\
\multicolumn{2}{l}{attributes}  & \multicolumn{4}{l}{}\\
   &   \bfseries name     & \bfseries datatype & \bfseries values & \bfseries default   & \bfseries description \\
   &   \texttt{attr1}     &  text              &                  &                     &                       \\
   &   \texttt{attr2}     &  integer           &                  &                     &                       \\
   &   \texttt{attr3}     &  real              &                  &                     &                       \\
   &   \texttt{attr4}     &  boolean           &                  &                     &                       \\
   &   \texttt{attr4}     &  text array        &                  &                     &                       \\
   &   \texttt{attr4}     &  integer array     &                  &                     &                       \\
   &   \texttt{attr4}     &  real array        &                  &                     &                       \\
   &   \texttt{attr4}     &  boolean array     &                  &                     &                       \\
\multicolumn{2}{l}{parameters}  & \multicolumn{4}{l}{}\\
   &   \bfseries name     & \bfseries datatype & \bfseries values & \bfseries default   & \bfseries description \\
   &   \texttt{attr1}     &  text              &                  &                     &                       \\
   &   \texttt{attr2}     &  integer           &                  &                     &                       \\
   &   \texttt{attr3}     &  real              &                  &                     &                       \\
   &   \texttt{attr4}     &  boolean           &                  &                     &                       \\
   &   \texttt{attr4}     &  text array        &                  &                     &                       \\
   &   \texttt{attr4}     &  integer array     &                  &                     &                       \\
   &   \texttt{attr4}     &  real array        &                  &                     &                       \\
   &   \texttt{attr4}     &  boolean array     &                  &                     &                       \\
  \hline
\end{tabularx}
\end{center}
\end{table}




\begin{table}[h]
\begin{center}
\begin{tabularx}{\textwidth}{l l l l l l }
\hline
\multicolumn{6}{l}{\texttt{generic} factory element} \\
\hline
\multicolumn{2}{l}{parent elements:} & \multicolumn{4}{l}{\texttt{parent1 parent2}}\\
\multicolumn{2}{l}{child  elements:} & \multicolumn{4}{l}{\texttt{child1 child2 child3 ...}}\\
\multicolumn{2}{l}{type   selector:} & \multicolumn{4}{l}{\texttt{some} attribute}\\
\multicolumn{2}{l}{type   options :} & \multicolumn{4}{l}{Selection1}\\
\multicolumn{2}{l}{                } & \multicolumn{4}{l}{Selection2}\\
\multicolumn{2}{l}{                } & \multicolumn{4}{l}{Selection3}\\
\multicolumn{2}{l}{                } & \multicolumn{4}{l}{...}\\
\multicolumn{2}{l}{shared attributes:} & \multicolumn{4}{l}{}\\
   &   \bfseries name     & \bfseries datatype & \bfseries values & \bfseries default   & \bfseries description \\
   &   \texttt{attr1}     &  text              &                  &                     &                       \\
   &   \texttt{attr2}     &  integer           &                  &                     &                       \\
   &   ...                &                    &                  &                     &                       \\
  \hline
\end{tabularx}
\end{center}
\end{table}




\subsection{Input Specification}


% hamiltonian element
%  dev notes
%    Hamiltonian element read
%      HamiltonianPool::put
%        reads attributes: id name role target 
%          id/name is passed to QMCHamiltonian
%          role selects the primary hamiltonian
%          target associates to quantum particleset
%      HamiltonianFactory::build
%        reads attributes: type source default
%    HamiltonianFactory cloning may be flawed for non-electron systems
%      see HamiltonianFactory::clone
%        aCopy->renameProperty(``e'',qp->getName());
%        aCopy->renameProperty(psiName,psi->getName());
%      the renaming may not work if dynamic particleset.name!=''e''

\FloatBarrier
\begin{table}[h]
\begin{center}
\begin{tabularx}{\textwidth}{l l l l l l }
\hline
\multicolumn{6}{l}{\texttt{hamiltonian} element} \\
\hline
\multicolumn{2}{l}{parent elements:} & \multicolumn{4}{l}{\texttt{qmcsystem}}\\
\multicolumn{2}{l}{child  elements:} & \multicolumn{4}{l}{\texttt{pairpot extpot estimator constant}(deprecated)}\\
\multicolumn{2}{l}{attributes}  & \multicolumn{4}{l}{}\\
   &   \bfseries name     & \bfseries datatype & \bfseries values & \bfseries default   & \bfseries description \\
   &   \texttt{name/id}$^o$ &  text              & \textit{anything}& h0                  & A unique id for this Hamiltonian instance.                      \\
   &   \texttt{type}$^o$    &  text              &                  & generic             & \textit{No current function.}                      \\
   &   \texttt{role}$^o$    &  text              & primary/extra    & extra               & Designate as primary Hamiltonian or not.                      \\
   &   \texttt{source}$^o$  &  text              & \texttt{particleset.name} & i          & Identify classical particleset.                      \\
   &   \texttt{target}$^r$  &  text              & \texttt{particleset.name} & e          & Identify quantum particlset.                      \\
   &   \texttt{default}$^o$ &  boolean           & yes/no           & yes                 & Include kinetic energy term implicitly.                      \\
  \hline
\end{tabularx}
\end{center}
\end{table}
\FloatBarrier

% All-electron hamiltonian element
\begin{lstlisting}[caption=All electron Hamiltonian XML element.]
<hamiltonian target="e">
  <pairpot name="ElecElec" type="coulomb" source="e" target="e"/>
  <pairpot name="ElecIon"  type="coulomb" source="i" target="e"/>
  <pairpot name="IonIon"   type="coulomb" source="i" target="i"/>
</hamiltonian>
\end{lstlisting}


% Pseudopotential hamiltonian element
\begin{lstlisting}[caption=Pseudopotential Hamiltonian XML element.]
<hamiltonian target="e">
  <pairpot name="ElecElec"  type="coulomb" source="e" target="e"/>
  <pairpot name="PseudoPot" type="pseudo"  source="i" wavefunction="psi0" format="xml">
    <pseudo elementType="Li" href="Li.xml"/>
    <pseudo elementType="H" href="H.xml"/>
  </pairpot>
  <pairpot name="IonIon"    type="coulomb" source="i" target="i"/>
</hamiltonian>
\end{lstlisting}


% pairpot element
\FloatBarrier
\begin{table}[h]
\begin{center}
\begin{tabularx}{\textwidth}{l l l l l l }
\hline
\multicolumn{6}{l}{\texttt{pairpot} factory element} \\
\hline
\multicolumn{2}{l}{parent elements:} & \multicolumn{4}{l}{\texttt{hamiltonian}}\\
\multicolumn{2}{l}{type   selector:} & \multicolumn{4}{l}{\texttt{type} attribute}\\
\multicolumn{2}{l}{type   options: } & \multicolumn{4}{l}{coulomb}\\
\multicolumn{2}{l}{                } & \multicolumn{4}{l}{pseudo}\\
\multicolumn{2}{l}{                } & \multicolumn{4}{l}{mpc}\\
\multicolumn{2}{l}{                } & \multicolumn{4}{l}{cpp}\\
\multicolumn{2}{l}{                } & \multicolumn{4}{l}{numerical/*num*}\\
\multicolumn{2}{l}{                } & \multicolumn{4}{l}{jellium}\\
\multicolumn{2}{l}{                } & \multicolumn{4}{l}{hardsphere}\\
\multicolumn{2}{l}{                } & \multicolumn{4}{l}{gaussian}\\
\multicolumn{2}{l}{                } & \multicolumn{4}{l}{huse}\\
\multicolumn{2}{l}{                } & \multicolumn{4}{l}{modpostel}\\
\multicolumn{2}{l}{                } & \multicolumn{4}{l}{oscillatory}\\
\multicolumn{2}{l}{                } & \multicolumn{4}{l}{skpot}\\
\multicolumn{2}{l}{                } & \multicolumn{4}{l}{vhxc}\\
\multicolumn{2}{l}{                } & \multicolumn{4}{l}{modInsKE}\\
\multicolumn{2}{l}{                } & \multicolumn{4}{l}{LJP\_smoothed}\\
\multicolumn{2}{l}{                } & \multicolumn{4}{l}{HeSAPT\_smoothed}\\
\multicolumn{2}{l}{                } & \multicolumn{4}{l}{HFDHE2\_Moroni1995}\\
\multicolumn{2}{l}{                } & \multicolumn{4}{l}{eHe}\\
\multicolumn{2}{l}{                } & \multicolumn{4}{l}{HFDHE2}\\
\multicolumn{2}{l}{shared attributes:} & \multicolumn{4}{l}{}\\
   &   \bfseries name     & \bfseries datatype & \bfseries values & \bfseries default   & \bfseries description \\
   &   \texttt{type}$^r$      &  text              & \textit{See above} & 0                   & Select pairpot type.         \\
   &   \texttt{name}$^r$      &  text              & \textit{anything}                 & any                 & Unique name for this pairpot.\\
   &   \texttt{source}$^r$    &  text              & \texttt{particleset.name} &\texttt{hamiltonian.target}& Identify interacting particles.\\
   &   \texttt{target}$^r$    &  text              & \texttt{particleset.name} &\texttt{hamiltonian.target}& Identify interacting particles.  \\
   &   \texttt{units}$^o$     &  text              &                  & hartree             & \textit{No current function.}  \\
\hline
\end{tabularx}
\end{center}
\end{table}
\FloatBarrier

Remarks
\begin{itemize}
  \item{If an interaction is between classical (e.g. ions) and quantum (e.g. electrons), \texttt{source}/\texttt{target} should be the name of the classical/quantum particleset.}
  \item{Only \texttt{coulomb, pseudo, mpc} are described in detail below.  The older or less used types (\texttt{cpp, numerical, jellium, hardsphere, gaussian, huse, modpostel, oscillatory, skpot, vhxc, modInsKE, LJP\_smoothed, HeSAPT\_smoothed, HFDHE2\_Moroni1995, eHe, HFDHE2}) are not covered.}
  \item{Available only if \texttt{QMC\_BUILD\_LEVEL>2} and \texttt{QMC\_CUDA} is not defined: \texttt{hardsphere, gaussian, huse, modpostel, oscillatory, skpot}.}
  \item{Available only if \texttt{OHMMS\_DIM==3}: \texttt{mpc, vhxc, pseudo}.}
  \item{Available only if \texttt{OHMMS\_DIM==3} and \texttt{QMC\_BUILD\_LEVEL>2} and \texttt{QMC\_CUDA} is not defined: \texttt{cpp, LJP\_smoothed, HeSAPT\_smoothed, HFDHE2\_Moroni1995, eHe, jellium, HFDHE2, modInsKE}.}
\end{itemize}


% physical read by coulomb potentials
% potential is only for pressure estimator



% pairpot instances

%   do coulomb, pseudo, mpc

\FloatBarrier
\begin{table}[h]
\begin{center}
\begin{tabularx}{\textwidth}{l l l l l l }
\hline
\multicolumn{6}{l}{\texttt{pairpot type=coulomb} element} \\
\hline
\multicolumn{2}{l}{parent elements:} & \multicolumn{4}{l}{\texttt{hamiltonian}}\\
\multicolumn{2}{l}{child  elements:} & \multicolumn{4}{l}{\textit{None}}\\
\multicolumn{2}{l}{attributes}  & \multicolumn{4}{l}{}\\
   &   \bfseries name     & \bfseries datatype & \bfseries values & \bfseries default   & \bfseries description \\
   & \texttt{type}$^r$    &  text              & \textbf{coulomb} &                     & Must be coulomb.         \\
   & \texttt{name/id}$^r$ &  text              & \textit{anything}&  ElecElec           & Unique name for interaction. \\
   & \texttt{source}$^r$  &  text              & \texttt{particleset.name} &\texttt{hamiltonian.target}& Identify interacting particles.\\
   & \texttt{target}$^r$  &  text              & \texttt{particleset.name} &\texttt{hamiltonian.target}& Identify interacting particles.  \\
   & \texttt{pbc}$^o$     &  boolean           & yes/no           & yes$^*$             & Use Ewald summation.  \\
   & \texttt{physical}$^o$&  boolean           & yes/no           & yes                 & Hamiltonian(yes)/observable(no). \\
   & \texttt{forces}      &  boolean           & yes/no           & no                  & \textit{Deprecated.}             \\
  \hline
\end{tabularx}
\end{center}
\end{table}
\FloatBarrier

Remarks
\begin{itemize}
  \item{Ewald summation will not be performed if \texttt{simulationcell.bconds== n n n}, regardless of the value of \texttt{pbc}.  Similarly, the \texttt{pbc} attribute can only be used to turn off Ewald summation if \texttt{simulationcell.bconds!= n n n}.}
  \item{Note: traditional names for electron-electron, electron-ion, and ion-ion terms are \texttt{ElecElec}, \texttt{ElecIon}, and \texttt{IonIon}, respectively.  While any choice can be used, the data analysis tools expect to find columns in \texttt{*.scalar.dat} with these names.}
\end{itemize}


% pseudopotential element
%   dev notes
%     attributes name, source, wavefunction, format are read in CoulombFactory.cpp  HamiltonianFactory::addPseudoPotential
%     format==''old'' refers to an old table format that is no longer supported
%     read continues in ECPotentialBuilder::put()
%       if format!=xml/old (i.e. table) qmcpack will attempt to read from *.psf files
%         in this case, <pairpot type=''pseudo'' format=''table''/>, ie there are no elements
%         if particlset groups are Li H (in order), then it looks for Li.psf and H.psf
%         what is the psf format?
%       if format==xml, normal read continues, i.e. <pseudo/> child elements are expected
%         read is not sensitive to particleset group/species ordering
%         child elements not names <pseudo/> are simply ignored (FIX!)
\FloatBarrier
\begin{table}[h]
\begin{center}
\begin{tabularx}{\textwidth}{l l l l l l }
\hline
\multicolumn{6}{l}{\texttt{pairpot type=pseudo} element} \\
\hline
\multicolumn{2}{l}{parent elements:} & \multicolumn{4}{l}{\texttt{hamiltonian}}\\
\multicolumn{2}{l}{child  elements:} & \multicolumn{4}{l}{\texttt{pseudo}}\\
\multicolumn{2}{l}{attributes}  & \multicolumn{4}{l}{}\\
   &   \bfseries name     & \bfseries datatype & \bfseries values & \bfseries default   & \bfseries description \\
   & \texttt{type}$^r$    &  text              & \textbf{pseudo} &                      & Must be pseudo.         \\
   & \texttt{name/id}$^r$ &  text              & \textit{anything}&  PseudoPot          & \textit{No current function.}\\
   & \texttt{source}$^r$  &  text              & \texttt{particleset.name} &  i                  & Ion particleset name.\\
   & \texttt{target}$^r$  &  text              & \texttt{particleset.name} &\texttt{hamiltonian.target}& Electron particleset name.  \\
   & \texttt{pbc}$^o$     &  boolean           & yes/no           & yes$^*$             & Use Ewald summation.  \\
   & \texttt{forces}      &  boolean           & yes/no           & no                  & \textit{Deprecated.}             \\
   &\texttt{wavefunction}$^r$ &  text          & \texttt{wavefunction.name}& invalid    & Identify wavefunction. \\
   &   \texttt{format}$^r$    &  text          & xml/table        & table               & Select file format.   \\
  \hline
\end{tabularx}
\end{center}
\end{table}
\FloatBarrier

Remarks
\begin{itemize}
  \item{Ewald summation will not be performed if \texttt{simulationcell.bconds== n n n}, regardless of the value of \texttt{pbc}.  Similarly, the \texttt{pbc} attribute can only be used to turn off Ewald summation if \texttt{simulationcell.bconds!= n n n}.}
  \item{Additional fields will be present in \texttt{*scalar.dat} output files when pseudopotentials are used.  The field \texttt{LocalECP} refers to the local part of the pseudopotential.  If non-local channels are present, a \texttt{NonLocalECP} field will be added that contains the non-local energy summed over all angular momentum channels.}
  \item{If \texttt{format}==table}, QMCPACK looks for \texttt{*.psf} files containing pseudopotential data in a tabular format.  The files must be named after the ionic species provided in \texttt{particleset} (\emph{e.g.} \texttt{Li.psf} and \texttt{H.psf}). 
  \item{If \texttt{format}==xml, additional \texttt{pseudo} child XML elements must be provided (see below).  These elements specify individual file names and formats (both the FSAtom XML and CASINO tabular data formats are supported).}
\end{itemize}



% pseudo element
%   dev notes
%     initial read of href elementType/symbol attributes at ECPotentialBuilder::useXmlFormat()
%     read continues in ECPComponentBuilder
%       format==xml and href==none (not provided) => ECPComponentBuilder::put(cur)
%       format==xml and href==a file => ECPComponentBuilder::parse(href,cur)
%       format==casino => ECPComponentBuilder::parseCasino(href,cur)
%         this reader is tucked away in ECPComponentBuilder.2.cpp
%         nice demonstration of OhmmsAsciiParser here
%         maximum cutoff defined by a 1.e-5 (Ha?) spread in the nonlocal potentials
%     quadrature rules (1-7) set as in J. Chem. Phys. 95 (3467) (1991), see below
%       Rule     # points     lexact
%        1           1          0
%        2           4          2
%        3           6          3
%        4          12          5
%        5          18          5
%        6          26          7
%        7          50         11
%     looks like channels only go from s-g (see ECPComponentBuilder constructor)
%       perhaps not, quadrature rules really do go up to 7 (lexact==11), see SetQuadratureRule()
\FloatBarrier
\begin{table}[h]
\begin{center}
\begin{tabularx}{\textwidth}{l l l l l l }
\hline
\multicolumn{6}{l}{\texttt{pseudo} element} \\
\hline
\multicolumn{2}{l}{parent elements:} & \multicolumn{4}{l}{\texttt{pairpot type=pseudo}}\\
\multicolumn{2}{l}{child  elements:} & \multicolumn{4}{l}{\texttt{header local grid}}\\
\multicolumn{2}{l}{attributes}  & \multicolumn{4}{l}{}\\
   &   \bfseries name     & \bfseries datatype & \bfseries values & \bfseries default   & \bfseries description \\
   & \texttt{elementType/symbol}$^r$&  text           &\texttt{group.name}& none               & Identify ionic species.         \\
   & \texttt{href}$^r$    &  text              & \textit{filepath}& none                & \textit{Pseudopotential file path.}\\
   & \texttt{format}$^r$  &  text              & xml/casino       & xml                 & Specify file format.\\
   & \texttt{cutoff}$^o$  &  real              &                  &                     & Non-local cutoff radius.  \\
   & \texttt{lmax}$^o$    &  integer           &                  &                     & Largest angular momentum.  \\
   & \texttt{nrule}$^o$   &  integer           &                  &                     & Integration grid order.             \\
  \hline
\end{tabularx}
\end{center}
\end{table}
\FloatBarrier




% mpc element
%  dev notes
%    most attributes are read in CoulombPotentialFactory.cpp  HamiltonianFactory::addMPCPotential()
%    user input for the name attribute is ignored and the name is always MPC
%    density G-vectors are stored in ParticleSet: Density_G and DensityReducedGvecs members
%    check the Linear Extrap and Quadratic Extrap output in some real examples (see MPC::init_f_G())
%      what are acceptable values for the discrepancies?
%      check that these decrease as cutoff is increased 
%    commented out code for MPC.dat creation in MPC::initBreakup()
%    short range part is 1/r, MPC::evalSR()
%    long range part is on a spline (VlongSpline), MPC::evalLR()
\FloatBarrier
\begin{table}[h]
\begin{center}
\begin{tabularx}{\textwidth}{l l l l l l }
\hline
\multicolumn{6}{l}{\texttt{pairpot type=mpc} element} \\
\hline
\multicolumn{2}{l}{parent elements:} & \multicolumn{4}{l}{\texttt{hamiltonian}}\\
\multicolumn{2}{l}{child  elements:} & \multicolumn{4}{l}{\textit{None}}\\
\multicolumn{2}{l}{attributes}  & \multicolumn{4}{l}{}\\
   &   \bfseries name     & \bfseries datatype & \bfseries values & \bfseries default   & \bfseries description \\
   & \texttt{type}$^r$    &  text              & \textbf{mpc}     &                     & Must be mpc.         \\
   & \texttt{name/id}$^r$ &  text              & \textit{anything}&  MPC                & Unique name for interaction. \\
   & \texttt{source}$^r$  &  text              & particleset.name &\texttt{hamiltonian.target}& Identify interacting particles.\\
   & \texttt{target}$^r$  &  text              & particleset.name &\texttt{hamiltonian.target}& Identify interacting particles.  \\
   & \texttt{physical}$^o$&  boolean           & yes/no           & no                  & Hamiltonian(yes)/observable(no). \\
   &   \texttt{cutoff}    &  real              & \textit{positive real}& 30.0           & Kinetic energy cutoff. \\
  \hline
\end{tabularx}
\end{center}
\end{table}
\FloatBarrier
Remarks
\begin{itemize}
  \item{Note the meaning of the \texttt{physical} attribute.  Typically it is set to \texttt{no}, meaning the standard Ewald interaction will be used during sampling and MPC will be measured as an observable for finite-size post correction.  If \texttt{physical==yes}, the MPC interaction will be used during sampling.  In this case an electron-electron Coulomb \texttt{pairpot} element should not be supplied.}
  \item{Developer note: Currently the \texttt{name} attribute for the mpc interaction is ignored.  The name is always \texttt{MPC}.}
\end{itemize}





% estimator element

\begin{table}[h]
\begin{center}
\begin{tabularx}{\textwidth}{l l l l l l }
\hline
\multicolumn{6}{l}{\texttt{estimator} element} \\
\hline
\multicolumn{2}{l}{parent elements:} & \multicolumn{4}{l}{\texttt{parent1 parent2}}\\
\multicolumn{2}{l}{child  elements:} & \multicolumn{4}{l}{\texttt{child1 child2 child3 ...}}\\
\multicolumn{2}{l}{type   selector:} & \multicolumn{4}{l}{\texttt{type} attribute}\\
\multicolumn{2}{l}{type   options :} & \multicolumn{4}{l}{}\\
  \hline
\end{tabularx}
\end{center}
\end{table}



