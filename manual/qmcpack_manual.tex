\documentclass[11pt,letterpaper]{report}
\usepackage{qmcpack_manual}
\usepackage{bibtopic}
\bibliographystyle{ieeetr}
\usepackage{amsmath}
\usepackage{amssymb}
\usepackage{delarray}
\usepackage{algorithmic}
\usepackage{algorithm}
\usepackage{makeidx}
\usepackage{fancyhdr}
\usepackage{xcolor}
\usepackage[colorlinks=true,linkcolor=blue,urlcolor=blue]{hyperref} %for urls
\usepackage{tabularx}
\usepackage{placeins}
\usepackage{caption}
\usepackage{graphicx}

% making listing behave properly
%   with setting below, listings now render correctly
%   copy/paste from pdf is still messed up (is this even possible to fix?)
%     -indentation whitespace is not preserved (needed for Python)
%     -copy/paste can result in mangled text
%     -mangling depends on pdf viewer (it is different for Acrobat and evince)
%     -verbatim suffers from this also

\usepackage{upquote}  % render ' properly
\usepackage{qmcpack_listings}

% set margins for whole document, lots of wasted space at top and bottom originally
\usepackage[left=1.0in,right=1.0in,top=1.0in,bottom=1.0in]{geometry}




\newcommand{\HRule}{\rule{\linewidth}{0.5mm}}
%% \newcommand{\courier}[1]{{\fontfamily{pcr}\selectfont #1}}

% for markup, as needed
\newcommand{\red}[1]{{\color{red} #1}}
\newcommand{\blue}[1]{{\color{blue} #1}}

% hide or show text relevant to developers
\newcommand{\dev}[1]{#1}
%\newcommand{\dev}[1]{}

% efficiently comment out/hide blocks of text for any purpose
\newcommand{\hide}[1]{}


% control display of instructions in the labs
%   normally one only wants to show the 'workstation' way of running the labs
\newif\ifws
\wstrue
%   for the pdf used during the labs, one wants to show the host supercomputer way
%\wsfalse
%  command for switching inline text (do not wrap verbatim environments with this!)
\ifws
\newcommand{\labsw}[2]{#1}
\else
\newcommand{\labsw}[2]{#2}
\fi


\oddsidemargin 0cm
\evensidemargin 0cm
\textwidth 6.5in


% proper rendering of qmcpack
\newcommand{\qmcpack}{{QMCPACK} } % apparently the trailing whitespace is significant

% mathematics convenience commands
\newcommand{\abs}[1]{\lvert #1 \rvert}
\newcommand{\norm}[1]{\lVert #1 \rVert}
\newcommand{\pnorm}[2]{\lVert #1 \rVert_{#2}}
\newcommand{\mean}[1]{\langle #1 \rangle}
\newcommand{\ket}[1]{\lvert #1 \rangle}
\newcommand{\bra}[1]{\langle #1 \rvert}
\newcommand{\expval}[3]{\bra{#1}#2\ket{#3}}
\newcommand{\expvalh}[3]{\bra{#1}\hat{#2}\ket{#3}}
\newcommand{\overlap}[2]{\langle #1 \lvert #2 \rangle}
\newcommand{\operator}[3]{\ket{#1} #2 \bra{#3}}
\newcommand{\idop}{\hat{\mathbb{1}}}
\newcommand{\bs}{\boldsymbol}
\newcommand{\tr}{\text{tr}} % trace
\newcommand{\grad}{\nabla}
\newcommand{\lap}{\nabla^2}  % laplacian

% urls are were too large
% hyperref gives us an easy way to control that
\renewcommand{\UrlFont}{\ttfamily\small}

% latex itself doesn't give us an easy way to deal with \texttt's font size
% so we need to define this command
\usepackage{letltxmacro}
% https://tex.stackexchange.com/q/88001/5764
\LetLtxMacro\oldttfamily\ttfamily
\DeclareRobustCommand{\ttfamily}{\oldttfamily\csname ttsize\endcsname}
\newcommand{\setttsize}[1]{\def\ttsize{#1}}%

% while \texttt use should be sparing (see contributing.tex) its necessary for the
% QMCPACK input XML spec tables, but those are also nearly too big for the page
\setttsize{\footnotesize}

% We have a huge number of overfull boxes, this adds another pass to
% typesetting a paragraph properly
\setlength{\emergencystretch}{3em}


\begin{document}


  \begin{center}
%\HRule\\[0.5 cm]
%{\huge \bfseries QMCPACK \\[0.5cm]}
%{\large User's Guide and Developer's Manual\\[1cm]}
%\HRule
\includegraphics[width=10cm]{figures/QMCPACK_logo.pdf}\\
{\huge User's Guide and Developer's Manual Preview\\}
{
\huge %v1.0\\ 
\today}

  \end{center}

\newpage
\tableofcontents
\newpage

\begin{btUnit}

\chapter{Introduction}

QMCPACK is an open-source, high-performance electronic structure code that implements numerous Quantum Monte Carlo algorithms.

Test of the bibliography\cite{CeperleyAlderPRL1980}.

\section{Support}
\section{Performance}
\section{Open source license}
\section{Contributing to QMCPACK}

\chapter{Features of QMCPACK}
\label{chap:features}
\section{Features in production}
The following lists the main production level features of QMCPACK. If
you do not see a specific feature that you are interested in listed,
see the remainder of this manual and ask to see if specific feature is
available or can be made full production level quickly.

\begin{itemize}
\item Variational Monte Carlo
\item Diffusion Monte Carlo
\item Reptation Monte Carlo
\item Single and multi-determinant Slater Jastrow wavefunctions
\item Wavefunction updates using optimized multi-determinant algorithm of Clark et al.
\item Backflow wavefunctions
\item One, two, and three-body Jastrow factors
\item Excited state calculations via flexible occupancy assignment of Slater determinants
\item All electron and non-local pseudopotential calculations
\item Casula T-moves for variational evaluation of non-local
  pseudopotentials (non-size consistent)
\item Wavefunction optimization using the ``linear method'' of Umrigar
  and co-workers, with arbitrary mix of variance and energy in the
  objective function
\item Blocked, low memory adaptive shift optimizer of Zhao and Neuscamman. 
\item Gaussian, Slater, plane-wave and real-space spline basis sets for orbitals
\item Interface and conversion utilities for plane-wave wavefunctions from Quantum Espresso (PWSCF)
\item Interface and conversion utilities for Gaussian-basis wavefunctions from GAMESS
\item Easy extension and interfacing to other electronic structure codes via standardized XML and HDF5 inputs
\item MPI parallelism
\item Fully threaded using OpenMP
\item GPU (NVIDIA CUDA) implementation (limited functionality)
\item HDF5 input/output for large data
\item Nexus: advanced workflow tool to automate all aspects of QMC calculation from initial DFT calculations through to final analysis
\item Analysis tools for minimal environments (perl only) through to python-based with graphs produced via matplotlib.
\end{itemize}

\subsection{Supported GPU features}

The GPU implementation supports multiple GPUs per node, with MPI tasks assigned
in a round-robin order to the GPUs. Currently, for large runs, 1 MPI task should
be used per GPU per node. For smaller calculations, use of multiple
MPI tasks per GPU may yield improved performance.

\begin{itemize}
  \item VMC, wavefunction optimization, DMC.
  \item Periodic and open boundary conditions. Mixed boundary conditions are not yet supported.
  \item Wavefunctions:
    \begin{enumerate}
        \item Single Slater determinants with 3D B-spline orbitals. Twist-averaged boundary conditions and complex wavefunctions are fully supported. Gaussian type orbitals are not supported yet.
        \item Hybrid Mixed basis representation in which orbitals are represented as 1D splines times spherical harmonics in spherical regions (muffin tins) around atoms, and 3D B-splines in the interstitial region.
        \item One-body and two-body Jastrows represented as 1D
          B-splines are supported. Threee-body jastrow functions are
          not yet supported.
    \end{enumerate}
  \item Semilocal (nonlocal and local) pseudopotentials, Coulomb interaction (electron-electron, electron-ion) and Model periodic Coulomb (MPC) interaction.
\end{itemize}

\section{Beta test teatures}

This section describes developmental features in QMCPACK that might be
ready for production, but require additional testing, features or
documentation to be ready for general use. We describe them here
because they offer significant benefits and are well tested in
specific cases.

\subsection{High performance CPU SoA optimizations}
The Structure-of-Arrays (SoA) optimizations \cite{IPCC_SC17} are a set
of improved data layouts facilitating vectorization on modern CPUs
with wide SIMD units. \textbf{For many calculations and architectures, the SoA
  implementation at least doubles the speed of the code.}  Meanwhile,
the memory footprint is dramatically reduced by better algorithms. For full benefit, modern
compilers supporting OpenMP 4.0 SIMD are recommended, e.g. GCC version
$\ge$4.9 and Intel compiler versions $\ge 15.0$. \textbf{Users should
  check what is currently tested on cdash/ctest and make specific
  requests to expand the functionality that they need.}

As described in \ref{sec:cmakeoptions}, to enable the SoA
implementation, QMCPACK should be configured with \texttt{-DENABLE\_SOA=1}.

SoA code path currently does NOT support:
\begin{itemize}
  \item Backflow wavefunctions
  \item All electron calculations requiring cusp corrected orbitals
  \item Many observables
  %\item Orbital optimization via rotation
\end{itemize}

If using any non-core feature with the SoA code, please check carefully vs the conventional AoS (non-SOA enabled) code.

\subsection{Auxiliary-Field Quantum Monte Carlo}

The orbital-space Auxiliary-Field Quantum Monte Carlo (AFMQC) method is now available in QMCPACK. The main input for the code are the matrix elements of the Hamiltonian in a given single particle basis set, which must be produced from a mean-field calculations like Hartree-Fock or Density Functional Theory. The code is under active development and many features are currently in development. Check the latest version of QMCPACK for an up-to-date description of the available features. A partial list of the current capabilities of the code follows. For a detailed description of the currently available features, see chapter \ref{chap:afqmc}.
 
\begin{itemize}
    \item Phaseless AFQMC algorithm of Zhang, et. al. (S. Zhang, H. Krakauer, "Quantum Monte Carlo Method using Phase-Free Random Walks with Slater Determinants", PRL 90, 136401 (2003)).
    \item "Hybrid" and "local energy" propagation schemes.
    \item Hamiltonian matrix elements from 1) Molpro's FCIDUMP format (which can be produced by Molpro, PySCF and VASP) and from 2) internal HDF5 format produced by PySCF (see AFQMC section below).
    \item AFQMC calculations with RHF (closed shell doubly occupied), ROHF (open shell doubly occupied) and UHF (spin polarized broken symmetry) symmetry. 
    \item Single and Multi-determinant trial wave-functions. Multi-determinant expansions with either orthogonal or non-orthogonal determinants allowed. 
    \item Fast update scheme for orthogonal multi-determinant expansions.
    \item Distributed propagation algorithms for large systems. Enables calculations where the data structures do not fit on a single node.
    \item Complex implementation for PBC calculations with complex integrals.
    \item Sparse representation of large matrices for reduced memory usage.
\end{itemize}

\chapter{Obtaining, installing and validating QMCPACK}
\label{chap:obtaininginstalling}

This chapter describes how to obtain, build and validate QMCPACK. This process is designed to be as simple as
possible and should be no harder than building a modern plane-wave density
functional theory code such as Quantum Espresso, QBox, or
VASP. Parallel builds enable a complete
compilation in under 2 minutes on a fast multicore sysyem, If you
are unfamiliar with building codes we suggest working with your system
administrator to install QMCPACK.

\section{Installation steps}
To install QMCPACK, follow the steps listed below. Full details of
each step are given in the referenced sections.
\begin{enumerate}
\item Download the source code, Sections \ref{sec:obrelease} or \ref{sec:obdevelopment}.
\item Verify that you have the required compilers, libraries and tools
  installed, Section \ref{sec:prerequisites}.
\item Run the cmake configure step and build with make, Section
  \ref{sec:cmake} and \ref{sec:cmakequick}.
\item Run the tests to verify QMCPACK, Section \ref{sec:testing}.
\item Build the ppconvert utility in QMCPACK, Section \ref{sec:buildppconvert}.
\item Download and patch Quantum Espresso. This patch adds the
  pw2qmcpack utility, Section \ref{sec:buildqe}.
\end{enumerate}

Installation instructions for many common systems are given in Section
\ref{sec:installexamples}. Troubleshooting suggestions are in Section \ref{sec:troubleshoot}.

Note that there are two different QMCPACK executables that can be
produced: the general one, which is the default, and the ``complex''
version which support periodic calculations at arbitrary twist angles and
k-points. This second version is enabled via a cmake configuration
parameter, see Section \ref{sec:cmakeoptions}. The general version
only supports wavefunctions that can be made real. If you run a
calculation that needs the complex version, QMCPACK will stop and inform you.

\section{Obtaining the latest release version}
\label{sec:obrelease} 
Major releases of QMCPACK are distributed from
\url{http://www.qmcpack.org}. These releases undergo the most testing. Unless there are
specific reasons we encourage all production calculations to use the
latest release versions.

Releases are usually compressed tar files indicating the version
number, date, and often the source code revision control number
corresponding to the release.

\begin{itemize}
\item Download the latest QMCPACK distribution from \url{http://www.qmcpack.org}.
\item Untar the archive, e.g., \texttt{tar xvf qmcpack\_v1.3.tar.gz}
\end{itemize}

\section{Obtaining the latest development version}
\label{sec:obdevelopment}
The most recent development version of QMCPACK can be obtained anonymously via 
\begin{verbatim}
svn checkout https://svn.qmcpack.org/svn/trunk 
\end{verbatim}
Once checked-out,
updates can be made via the standard \texttt{svn update}.

The subversion repository contains the day-to-day development source
with the latest updates, bugfixes etc. This may be useful
for updates to the build system to support new machines, for support
of the latest versions of Quantum Espresso, or for updates to the
documentation.  Note that the development version may not be fully
consistent with the online documentation.  We attempt to keep
the development version fully working. However, please be sure to run the tests and
compare with previous release versions before using for any serious
calculations. We try to keep bugs out, but occasionally they crawl
in! Reports of any breakages are appreciated.

\section{Prerequisites}
\label{sec:prerequisites}
The following are required to build QMCPACK. For workstations, these are available via the standard
package manager. On shared supercomputers this software is usually
installed by default and is often
access via a modules environment - check your system
documentation.

\textbf{Use of the latest versions of all compilers and libraries is
strongly encouraged}, but not absolutely essential. Generally newer versions are faster - see
Section \ref{sec:buildperformance} for performance suggestions.

\begin{itemize}
\item C/C++ compilers such as GCC, Intel, IBM XLC. CLANG-based compilers
  are not yet supported by the build system, but the source code is ready.
\item MPI libary such at OpenMPI \url{http://open-mpi.org}
\item BLAS/LAPACK, numerical and linear algebra libraries. Use
  platform-optimized libraries where available, such as Intel MKL.
  ATLAS or other optimized open-source libraries may also be used
  \url{http://math-atlas.sourceforge.net}
\item CMake, build utility, \url{http://www.cmake.org}
\item Libxml2, XML parser, \url{http://xmlsoft.org}
\item HDF5, portable I/O library, \url{http://www.hdfgroup.org/HDF5/}
\item BOOST, peer-reviewed portable C++ source libraries, \url{http://www.boost.org}
\item FFTW, FFT library, \url{http://www.fftw.org/}
\end{itemize}

To build the GPU accelerated version of QMCPACK an installation of
NVIDIA CUDA development tools is required. Ensure that this is
compatible with the C and C++ compiler versions you plan to
use. Supported versions are included in the NVIDIA release notes.

Many of the utilities provided with QMCPACK use python (v2). The numpy
and matplotlib libraries are required for full functionality.

Note that the standalone einspline library used by previous versions of QMCPACK
is no longer required. A more optimized version is included
inside. The standalone version should \emph{not} be on any standard
search paths because conflicts between the old and new include files
can result.

\section{Building with CMake}
\label{sec:cmake}
The build system for QMCPACK is based on CMake.  It will autoconfigure
based on the detected compilers and libraries. The most recent
version of CMake has the best detection for the greatest variety of
systems - at the time of writing this means CMake 3.4.3. The much
older CMake 2.8 is known to work, but might not work optimally on your system.

Previously QMCPACK made extensive use of toolchains, but the build system
has since been updated to eliminate the use of toolchain files for
most cases.  The build system is verified to work with GNU, Intel, and IBM XLC
compilers.  Specific compile options can be specified either through
specific environmental or CMake variables.  When the libraries are
installed in standard locations, e.g., /usr, /usr/local, there is no
need to set environmental or cmake variables for the packages.

\subsection{Quick build}
\label{sec:cmakequick}

If you are feeling lucky and are on a standard UNIX-like system such
as a Linux workstation, the following might work to quickly givin a
working QMCPACK:

The safest quick build option is to specify the C and C++ compilers
through their MPI wrappers. Here we use Intel MPI and Intel
compilers. Move to the build directory, run cmake and make
\begin{verbatim}
cd build
cmake -DCMAKE_C_COMPILER=mpiicc -DCMAKE_CXX_COMPILER=mpiicpc ..
make -j 8
\end{verbatim}
You can increase the \"8\" to the number of cores on your system for
faster builds. Substitute mpicc and mpicxx or other wrapped compiler names to suit
  your system. e.g. With OpenMPI use
\begin{verbatim}
cd build
cmake -DCMAKE_C_COMPILER=mpicc -DCMAKE_CXX_COMPILER=mpicxx ..
make -j 8
\end{verbatim}

If you are feeling particularly lucky, you can skip the compiler specification:
\begin{verbatim}
cd build
cmake ..
make -j 8
\end{verbatim}

The complexities of modern computer hardware and software systems are
such that you should check that the autoconfiguration system has made
good choices and picked optimized libraries and compiler settings
before doing significant production. i.e. Check the details below.

\subsection{Environment variables}
A number of enviornmental variables affect the build.  In particular
they can control the default paths for libraries, the default
compilers, etc.  The list of enviornmental variables is given below:
\begin{verbatim}
CXX              C++ compiler
CC               C Compiler
MKL_HOME         Path for MKL
LIBXML2_HOME     Path for libxml2
HDF5_ROOT        Path for HDF5
BOOST_ROOT       Path for Boost
FFTW_HOME        Path for FFTW
\end{verbatim}

\subsection{CMake Options}
\label{sec:cmakeoptions}
In addition to reading the enviornmental variables, CMake provides a
number of optional variables that can be set to control the build and
configure steps.  When passed to CMake, these variables will take
precident over the enviornmental and default variables.  To set them
add -D FLAG=VALUE to the configure line between the cmake command and
the path to the source directory.

\begin{itemize}
\item  Key QMCPACK build options
\begin{verbatim}
QMC_CUDA            Enable CUDA and GPU acceleration (1:yes, 0:no)
QMC_COMPLEX         Build the complex (general twist/k-point) version (1:yes, 0:no)
\end{verbatim}
  \item General build options
\begin{verbatim}
CMAKE_BUILD_TYPE    A variable which controls the type of build (defaults to Release).  Possible values are:
                   None (Do not set debug/optmize flags, use CMAKE_C_FLAGS or CMAKE_CXX_FLAGS)
                   Debug (create a debug build)
                   Release (create a release/optimized build)
                   RelWithDebInfo (create a release/optimized build with debug info)
                   MinSizeRel (create an executable optimized for size)
CMAKE_C_COMPILER    Set the C compiler
CMAKE_CXX_COMPILER  Set the C++ compiler
CMAKE_C_FLAGS       Set the C flags.  Note: to prevent default debug/release flags from being used, set the CMAKE_BUILD_TYPE=None
                   Also supported: CMAKE_C_FLAGS_DEBUG, CMAKE_C_FLAGS_RELEASE, CMAKE_C_FLAGS_RELWITHDEBINFO
CMAKE_CXX_FLAGS     Set the C++ flags.  Note: to prevent default debug/release flags from being used, set the CMAKE_BUILD_TYPE=None
                   Also supported: CMAKE_CXX_FLAGS_DEBUG, CMAKE_CXX_FLAGS_RELEASE, CMAKE_CXX_FLAGS_RELWITHDEBINFO
\end{verbatim}
\item Additional QMCPACK build options
\begin{verbatim}
QMC_DATA            Specify data directory for QMCPACK (currently unused, but likely to be used for performance tests)
QMC_INCLUDE         Add extra include paths
QMC_EXTRA_LIBS      Add extra link libraries
QMC_BUILD_STATIC    Add -static flags to build
\end{verbatim}
\item libxml related
\begin{verbatim}
Libxml2_INCLUDE_DIRS  Specify include directories for libxml2
Libxml2_LIBRARY_DIRS  Specify library directories for libxml2
\end{verbatim}
 \item FFTW related
\begin{verbatim}
FFTW_INCLUDE_DIRS   Specify include directories for FFTW
FFTW_LIBRARY_DIRS   Specify library directories for FFTW
\end{verbatim}
\end{itemize}

\subsection{Configure and build}
 Move to build directory, run cmake and make
\begin{verbatim}
cd build
cmake ..
make -j 8
\end{verbatim}
As you will have gathered, cmake encourages ``out of source'' builds,
where all the files for a specific build configuration reside in their
own directory separate from the source files. This allows multiple
builds to be created from the same source files which is very useful
where the filesystem is shared between different systems. You can also
build versions with different settings (e.g. QMC\_COMPLEX) and
different compiler settings. The build directory does not have to be
called build - use something descriptive such as build\_machinename or build\_complex.

\subsection{Example configure and build}
\begin{itemize}
\item Set the environments (the examples below assume bash, Intel compilers and MKL library)
\begin{verbatim}
export CXX=icpc
export CC=icc
export MKL_HOME=/usr/local/intel/mkl/10.0.3.020
export LIBXML2_HOME=/usr/local
export HDF5_ROOT=/usr/local
export BOOST_ROOT=/usr/local/boost
export FFTW_HOME=/usr/local/fftw
\end{verbatim}

\item Move to build directory, run cmake and make
\begin{verbatim}
cd build
cmake -D CMAKE_BUILD_TYPE=Release ..
make -j 8
\end{verbatim}
\end{itemize}

\subsection{Build scripts}
It is recommended to create a helper script that contains the
configure line for CMake.  This is particularly useful when avoiding
enviornmental variables, packages are installed in custom locations,
or if the configure line is long or complex.  In this case it is also
recommended to add "rm -rf CMake*" before the configure line to remove
existing CMake configure files to ensure a fresh configure each time
that the script is called. Deleting all the files in the build
directory is also acceptable. If you do so we recommend to add some sanity
checks in case the script is run from the wrong directory, e.g.,
checking for the existence of some QMCPACK files.

Some build script examples for different systems are given in the
config directory. For example, on Cray systems these scripts might
load the appropriate modules to set the appropriate programming
environment, specific library versions etc.

An example script build.sh is given below:
\begin{verbatim}
export CXX=mpic++
export CC=mpicc
export ACML_HOME=/opt/acml-5.3.1/gfortran64
export HDF5_ROOT=/opt/hdf5
export BOOST_ROOT=/opt/boost

rm -rf CMake*

cmake                                                \
  -D CMAKE_BUILD_TYPE=Debug                         \
  -D Libxml2_INCLUDE_DIRS=/usr/include/libxml2      \
  -D Libxml2_LIBRARY_DIRS=/usr/lib/x86_64-linux-gnu \
  -D FFTW_INCLUDE_DIRS=/usr/include                 \
  -D FFTW_LIBRARY_DIRS=/usr/lib/x86_64-linux-gnu    \
  -D QMC_EXTRA_LIBS="-ldl ${ACML_HOME}/lib/libacml.a -lgfortran" \
  -D QMC_DATA=/projects/QMCPACK/qmc-data            \
  ..
\end{verbatim}

\section{Installation instructions for common workstations and
  supercomputers}
\label{sec:installexamples}
This section describes how to build QMCPACK on various common systems
including multiple Linux distributions, Apple OS X, and various
supercomputers. Note that updates to operating systems may require
small modifications to these recipes.
\subsection{Installing on Ubuntu Linux}

All the required packages are available in the
default repositories making for a quick installation. Note that for
convenience we use a generic BLAS. For production a platform optimized BLAS should be used.

\begin{verbatim}
apt-get subversion cmake g++ openmpi-bin libopenmpi-dev libboost-dev
apt-get libatlas-base-dev liblapack-dev libhdf5-dev libxml2-dev fftw3-dev
export CXX=mpiCC
cd build
cmake ..
make -j 8
ls -l bin/qmcapp
\end{verbatim}

For qmca and other tools to function, we install some python libraries:
\begin{verbatim}
sudo apt-get install python-numpy python-matplotlib
\end{verbatim}

\subsection{Installing on CentOS Linux}

Currently this version of CentOS (Red Hat compatible) is using gcc
4.8.2. The installation is only complicated by the need to install
another repository to obtain HDF5 packages. Note that for convenience
we use a generic BLAS. For production a platform optimized BLAS should
be used.

\begin{verbatim}
sudo yum install make cmake gcc gcc-c++ subversion openmpi  openmpi-devel fftw fftw-devel boost boost-devel libxml2 libxml2-devel
sudo yum install blas-devel lapack-devel atlas-devel
module load mpi 

\end{verbatim}

To setup repoforge as a source for the HDF5 package, go to http://repoforge.org/use . Install the appropriate up to date release package for your OS. By default the CentOS Firefox will offer to run the installer. The CentOS 6.5 settings were usable for HDF5 on CentOS 7 in July 2014, but use CentOS 7 versions when they become available.
\begin{verbatim}
sudo yum install hdf5 hdf5-devel 
\end{verbatim}

To build QMCPACK
\begin{verbatim}
module load mpi/openmpi-x86_64
which mpirun
# Sanity check; should print something like   /usr/lib64/openmpi/bin/mpirun
export CXX=mpiCC
cd build
cmake ..
make -j 8
ls -l bin/qmcapp
\end{verbatim}

\subsection{Installing on Mac OS X using Macports}
These instructions assume a fresh installation of macports
and for consistency with current Linux distributions, use the gcc 4.8.2
compiler. It is vital to ensure matching compilers/options for all
packages and to force use of what is installed in /opt/local

Note that we utilize the Apple provided Accelerate framework for optimized BLAS.

Follow the Macports install instructions \url{https://www.macports.org/}

\begin{itemize}
\item Install Xcode and the Xcode Command Line Tools
\item Agree to Xcode license in Terminal: sudo xcodebuild -license
\item Install MacPorts for your version of OS X
\end{itemize}


Install the required tools:

\begin{verbatim} 
sudo port install gcc48
sudo port select gcc mp-gcc48  # Set default

sudo port install openmpi-devel-gcc48
sudo port select —set mpi openmpi-devel-gcc48-fortran  # Set default

# Sanity check
mpiCXX -v 
#should return … “gcc version 4.8.2 (MacPorts gcc48 4.8.2_2)” or similar.

sudo port install fftw-3 +gcc48
sudo port install cmake    # already cmake 3 as of 2014/7/29

sudo port install boost +gcc48
sudo port install libxml2
sudo port install hdf5-18 +gcc48

sudo port select —set python python27
sudo port install py27-matplotlib  # For qmca
\end{verbatim}

QMCPACK build:
\begin{verbatim}
export CXX=mpiCXX
export CC=/opt/local/bin/gcc
export LIBXML2_HOME=/opt/local/
export HDF5_HOME=/opt/local
export BOOST_HOME=/opt/local
export FFTW_HOME=/opt/local
cd build
cmake ..
make -j 6 # Adjust for available core count
\end{verbatim}

\subsection{Installing on ANL ALCF Mira IBM BGQ}
\subsection{Installing on ORNL OLCF Titan Cray XK7 (NVIDIA GPU  accelerated)}
\subsection{Installing on ORNL OLCF Titan Cray XK7 (CPU version)}
\subsection{Installing on ORNL OLCF Eos Cray XC30}
\subsection{Installing on NERSC Cori Cray XC40}
\subsection{Installing on NERSC Edison Cray XC30}

\section{Testing and validation of QMCPACK}
\label{sec:testing}
\subsection{Systems and software versions that QMCPACK is tested on}

\begin{figure}
  \centering
  \includegraphics[width=10cm]{figures/QMCPACK_CDash_Ctest_Results_20160129.png}
  \caption{Example test results for QMCPACK, showing data for a
    workstation (Intel, GCC, both CPU and GPU builds) and for two ORNL
    supercomputers. 4 errors were found.}
  \label{fig:cdash}
\end{figure}

\section{Building ppconvert, the pseudpotential format converter
  utility}
\label{sec:buildppconvert}
\section{Installing and patching Quantum Espresso}
\label{sec:buildqe}
\section{How to build the fastest executable version of QMCPACK}
\label{sec:buildperformance}

Most recent C++ compiler
Optimized BLAS. Vector math library.

\section{Troubleshooting the installation}
\label{sec:troubleshoot}
Build on a workstation you control
Out of date software
Update CMake

\chapter{Running QMCPACK}
\label{chap:running}

QMCPACK requires at least one xml input file, and is invoked via:

{\texttt{qmcpack [command line options] <XML input file(s)>}}

\section{Command line options}
\label{sec:commandline}
QMCPACK offers several command line options which affect how calculations
are performed. If the flag is absent, then the corresponding
option is disabled.

\begin{description}
\item[\texttt{-{}-dryrun}]{ Validate the input file without performing the simulation.
  This is a good way to ensure that QMCPACK will do what you think it will. }
\item[\texttt{-{}-enable-timers=none|coarse|medium|fine}]{ Control the timer granularity
  when the build option \texttt{ENABLE\_TIMER} is enabled. }
\item[\texttt{-{}-help}]{ Print version information as well as a list of optional
  command-line arguments. }
\item[\texttt{-{}-noprint}]{ Do not print extra information on Jastrow or pseudopotential.
  If this flag is not present, QMCPACK will create several \texttt{.dat} files
  that contain information about pseudopotentials (one file per PP), and jastrow
  factors (one per jastrow factor). These file may be useful for visual inspection
  of the jastrow, for example. }
\item[\texttt{-{}-save\_wfs}]{ Write a \texttt{.h5} file containing the real-space B-spline
  coefficients of the single particle wave functions. See the manual
  \ref{sec:spo_spline} for more information.}
\item[\texttt{-{}-vacuum X}]{Removed, use `vacuum' input tag described in \ref{chap:simulationcell}. }
\item[\texttt{-{}-verbosity=low|high|debug}]{ Control the output verbosity. The default low verbosity is concise and, e.g., does not include all electron or atomic positions for large systems to reduce output size. Use 'high' to see this information and more details of initialization, allocations, QMC method settings etc.. }
\item[\texttt{-{}-version}]{ Print version information and optional arguments.
  Same as \texttt{-{}-help}. }
\end{description}


\section{Input files}
\label{sec:inputs}
The input is one or more XML file(s), documented in chapter~\ref{chap:input_overview}.

\section{Output files}
QMCPACK generates multiple files, documented in chapter~\ref{chap:output_overview}.

\section{Running in parallel}
\label{sec:parallelrunning}

%considerations for mpi, threads, gpu.

\subsection{MPI}
QMCPACK is fully parallelized with MPI. When performing an ensemble job, all
the MPI ranks are first equally divided into groups which perform individual
QMC calculations. Within one calculation, all the walkers are fully distributed
across all the MPI ranks in the group. Since MPI requires distributed memory,
there must be at least one MPI per node. To maximize the efficiency, more facts
should be taken into account. When using MPI+threads on compute nodes with more
than one NUMA domain (e.g., AMD Interlagos CPU on Titan or a node with multiple
CPU sockets), it is recommended to place as many MPI ranks as the number of
NUMA domains if the memory is sufficient. On clusters with more than just one
GPU per node (NVIDIA Tesla K80), it requires to use the same number of MPI
ranks as the number of GPUs per node in order to let each MPI rank take one GPU.

\subsection{Use of OpenMP threads}
\label{sec:openmprunning}
Modern processors integrate multiple identical cores even with hardware threads
on a single die to increase the total performance and maintain a reasonable
power draw. QMCPACK takes advantage of all that compute capability on a
processor by using threads via OpenMP programming model as well as threaded linear algebra libraries. By default, QMCPACK is always built with OpenMP enabled. When launching calculations, users should instruct QMCPACK to create the right number of threads per MPI rank by specifying environment variable OMP\_NUM\_THREADS. Even in the GPU accelerated version, using threads significantly reduces the time spent on the calculations performed by the CPU.

\subsubsection{Nested OpenMP threads}
Nested threading is a advanced feature requiring experienced users to finely tune runtime parameters in order to reach the best performance.  

For small to medium problem sizes, using one thread per walker or for multiple walkers is most efficient. This is the default in QMCPACK and achieves the shortest time to solution.

For large problems of at least 1000 electrons, use of nested OpenMP threading can be enabled to reduce the time to solution further, although at some loss of efficiency. In this scheme multiple threads are used in the computations of each walker. This capability is implemented for some of the key computational kernels: the 3D spline orbital evaluation, certain portions of the distance tables, and implicitly the BLAS calls in the determinant update. Use of the batched non-local pseudopotential evaluation is also recommended.

Nested threading is enabled by setting \texttt{OMP\_NUM\_THREADS=AA,BB}, \texttt{OMP\_MAX\_ACTIVE\_LEVELS=2} and \texttt{OMP\_NESTED=TRUE} where the additional \texttt{BB} is the number of second level threads.  Choosing the thread affinity is critical to the performance.  QMCPACK provides a tool qmc-check-affinity (source file src/QMCTools/check-affinity.cpp for details) which may help users to investigate the affinity. Knowledge of how the operating system logical CPU cores (/prco/cpuinfo) are bound to the hardware is also needed.

For example, on Blue Gene/Q with Clang compiler, the best way to fully use the 16 cores each with 4 hardware threads is
\begin{verbatim}
OMP_NESTED=TRUE
OMP_NUM_THREADS=16,4
MAX_ACTIVE_LEVELS=2
OMP_PLACES=threads
OMP_PROC_BIND=spread,close
\end{verbatim}

On Intel Xeon Phi KNL with Intel compiler, to use 64 cores without using hardware threads
\begin{verbatim}
OMP_NESTED=TRUE
OMP_NUM_THREADS=16,4
MAX_ACTIVE_LEVELS=2
OMP_PLACES=cores
OMP_PROC_BIND=spread,close
KMP_HOT_TEAMS_MODE=1
KMP_HOT_TEAMS_MAX_LEVEL=2
\end{verbatim}

Most BLAS libraries do not spawn threads by default when called from a threaded region.
This results in the use of only a single thread in each second level thread team for BLAS operations. In order to fully use the second level threads,
users should check if turning on the nested threading in BLAS is beneficial or not
because it depends on the implementation of BLAS libraries.
Intel MKL library supports nested threading by setting
\begin{verbatim}
MKL_DYNAMIC=FALSE
MKL_NUM_THREADS=4
\end{verbatim}

\subsubsection{Performance considerations}
\label{sec:cpu:performance}
As walkers are the basic units of workload in QMC algorithms, they are loosely coupled and distributed across all the threads. For this reason, the best strategy to run QMCPACK efficiently is to feed enough walkers to the available threads.

In a VMC calculation, the code automatically raises the actual number of walkers per MPI rank to the number of available threads if the user-specified number of walkers is smaller, see ``walkers/mpi=XXX'' in the VMC output.  In a DMC calculation, the target number of walkers should be chosen to be slightly smaller than a multiple of the total number of available threads across all the MPI ranks belongs to this calculation. Since the number of walkers varies from generation to generation, its dynamical value should be slightly smaller or equal to that multiple most of the time.

To achieve better performance, mixed precision version (experimental) has been introduced to the CPU code. The mixed precision CPU code is currently more aggressive than the GPU version in using single precision (SP) operations. Current implementation utilizes SP on most calculations, except for matrix inversions and reductions where double precision is required to retain high accuracy. All the constant spline data in wavefunction, pseudopotentials and Coulomb potentials are initialized in double precision and later stored in single precision. The mixed precision code is as accurate as the double precision code up to a certain system size. Cross checking and verification of accuracy is always required, but particularly important above approximately 1500 electrons. The mixed precision code is currently tested on solids with real and complex wavefunctions with VMC, VMC using drift and DMC runs with wavefunction including single Slater determinant and one- and two-body Jastrow factors. Wavefunction optimization is currently not supported.

\subsubsection{Memory considerations}
When using threads, some memory objects shared by all the threads. Usually these memory are read-only when the walkers are evolving, for instance the ionic distance table and wavefunction coefficients.
If a wavefunction is represented by B-splines, the whole table is shared by all the threads. It usually takes a large chunk of memory when a large primitive cell was used in the simulation. Its actual size is reported as ``MEMORY increase XXX MB BsplineSetReader'' in the output file.
See details about how to reduce it in section~\ref{sec:spo_spline}.

The other memory objects which are distinct for each walker during random walk need to be associated with individual walkers and can not be shared. This part of memory grows linearly as the number of walkers per MPI rank. Those objects include wavefunction values (Slater determinants) at given electronic configurations and electron related distance tables (electron-electron distance table). Those matrices dominate the $N^2$ scaling of the memory usage per walker.

\subsection{Running on GPU machines}
\label{sec:gpurunning}

The GPU version on the NVIDIA CUDA platform is fully incorporated into
the main source code. Commonly used functionalities for
solid-state and molecular systems using B-spline single-particle
orbitals are supported. Use of Gaussian basis sets and three-body
Jastrow functions are not yet supported. A detailed description of the GPU
implementation can be found in Ref. \cite{EslerKimCeperleyShulenburger2012}.

The current GPU implementation assumes one MPI process per GPU. To use
nodes with multiple GPUs, use multiple MPI processes per node.
Vectorization is achieved over walkers, that is, all walkers are
propagated in parallel. In each GPU kernel, loops over electrons,
atomic cores or orbitals are further vectorized to exploit an
additional level of parallelism and to allow coalesced memory access.

%----------------------------------------------------------------------------%

\subsubsection{Performance consideration}
\label{sec:gpu:performance}

The relative speedup of the GPU implementation increases with both the number of electrons and the number of walkers running on a GPU. Typically, 128-256 walkers per GPU utilize sufficient number of threads to operate the GPU efficiently and to hide memory-access latency.

To achieve better performance, current implementation utilizes single precision operations on most GPU calculations, except for matrix inversions and Coulomb interaction where double precision is required to retain high accuracy. The mixed precision GPU code is as accurate as the double precision CPU code up to a certain system size. Cross checking and verification of accuracy are encouraged for systems with more than approximately 1500 electrons.

%------------------------------------------------------------------------------%

\subsubsection{Memory consideration}

In the GPU implementation, each walker has an anonymous buffer on the GPU's global memory to store temporary data associated with the wavefunctions. Therefore, the amount of memory available on a GPU limits the number of walkers and eventually the system size that it can process.

If the GPU memory is exhausted, reduce the number of walkers per GPU.
Coarsening the grids of the B-splines representation (by decreasing the value of meshfactor in the input file) can also lower the memory usage,
at the expense (risk) of obtaining inaccurate results. Proceed with caution if this option has to be considered.
It is also possible to distribute the B-spline coefficients table between the host and GPU memory, see option Spline\_Size\_Limit\_MB in Sec.~\ref{sec:spo_spline}.


\chapter{Units used in QMCPACK}
\label{sec:units}

Internally, QMCPACK uses atomic units throughout. Unless stated, all inputs and outputs are also in atomic units. For convenience the analysis tools offer conversions to eV, Ry, Angstrom, Bohr etc.



\chapter{Input file overview}
\label{chap:input_overview}

This chapter introduces XML as it is used in QMCPACK's input file.  The focus is on the XML file format itself and the general structure of the input file rather than an exhaustive discussion of all keywords and structure elements.  

QMCPACK uses XML to represent structured data in its input file.  Instead of text blocks like

\begin{shade}
begin project
  id     = vmc
  series = 0
end project

begin vmc
  move     = pbyp
  blocks   = 200
  steps    =  10
  timestep = 0.4
end vmc
\end{shade} 
QMCPACK input looks like
\begin{shade}
   <project id="vmc" series="0">
   </project>

   <qmc method="vmc" move="pbyp">
      <parameter name="blocks"  >  200 </parameter>
      <parameter name="steps"   >   10 </parameter>
      <parameter name="timestep">  0.4 </parameter>
   </qmc>
\end{shade}
XML elements start with \texttt{<element\_name>}, end with \texttt{</element\_name>}, and can be nested within each other to denote substructure (the trial wavefunction is composed of a Slater determinant and a Jastrow factor, which are each further composed of \ldots).  \texttt{id} and \texttt{series} are attributes of the \texttt{<project/>} element.  XML attributes are generally used to represent simple values, like names, integers, or real values.  Similar functionality is also commonly provided by \texttt{<parameter/>} elements like those shown above.

The overall structure of the input file reflects different aspects of the QMC simulation: the simulation cell, particles, trial wavefunction, Hamiltonian, and QMC run parameters.  A condensed version of the actual input file is shown below:
\begin{shade}
<?xml version="1.0"?>
<simulation>

  <project id="vmc" series="0">
    ...
  </project>

  <qmcsystem>

    <simulationcell>
      ...
    </simulationcell>

    <particleset name="e">
      ...
    </particleset>

    <particleset name="ion0">
      ...
    </particleset>

    <wavefunction name="psi0" ... >
      ...
      <determinantset>
        <slaterdeterminant>
          ..
        </slaterdeterminant>
      </determinantset>
      <jastrow type="One-Body" ... >
         ...
      </jastrow>
      <jastrow type="Two-Body" ... >
        ...
      </jastrow>
    </wavefunction>

    <hamiltonian name="h0" ... >
      <pairpot type="coulomb" name="ElecElec" ... />
      <pairpot type="coulomb" name="IonIon"   ... />
      <pairpot type="pseudo" name="PseudoPot" ... >
        ...
      </pairpot>
    </hamiltonian>

   </qmcsystem>

   <qmc method="vmc" move="pbyp">
     <parameter name="warmupSteps">   20 </parameter>
     <parameter name="blocks"     >  200 </parameter>
     <parameter name="steps"      >   10 </parameter>
     <parameter name="timestep"   >  0.4 </parameter>
   </qmc>

</simulation>
\end{shade}
The omitted portions (\texttt{...}) are more fine-grained inputs such as the axes of the simulation cell, the number of up and down electrons, positions of atomic species, external orbital files, starting Jastrow parameters, and external pseudopotential files.  


\section{Project}
The \texttt{<project>} tag uses the \texttt{id} and \texttt{series} attributes.
The value of \texttt{id} is the first part of the prefix for output file names.

Output file names also contain the series number, starting at the value given by the
\texttt{series} tag.  After every \texttt{<qmc>} section the series value will increment, giving each section a unique prefix.

For the input file shown previously, the output files will start with \texttt{vmc.s000}, for example \texttt{vmc.s000.scalar.dat}.
If there were another \texttt{<qmc>} section in the input file, the corresponding output files would use the prefix \texttt{vmc.s001}.



\section{Random number initialization}

The random number generator state is initialized from the \texttt{random} element using the \texttt{seed} attribute.
\begin{shade}
<random seed="1000"/>
\end{shade}

If the random element is not present, or the seed value is negative, the seed will be generated from the current time.

In order to initialize the many independent random number generators (one per thread and MPI process), the seed value is used (modulo 1024) as a starting index into a list of prime numbers.
Entries in this offset list of prime numbers are then used as the seed for the random generator on each thread and process.

If checkpointing is enabled, the random number state is written to an HDF file at the end of each block (suffix: \texttt{.random.h5}).
This file will be read if the \texttt{mcwalkerset} tag is present to perform a restart.
For more information, see the \texttt{checkpoint} element in the QMC methods chapter (\ref{chap:qmcmethods}) and the section on checkpoint and restart files in \ref{sec:checkpoint_files}.


\chapter{Specifying the system to be simulated}
\section{Specifying the simulation cell}
\label{sec:simulationcell}


\section{Specifying the particle set}
\label{sec:particleset}


The \texttt{particleset} blocks specify the particles in the QMC simulations: their types, attributes (mass, charge, valence), and positions.   

\subsection{Input specification}
\begin{table}[h]
\begin{center}
\begin{tabularx}{\textwidth}{l l l l l l }
\hline
\multicolumn{6}{l}{\texttt{particleset} element} \\
\hline
\multicolumn{2}{l}{parent elements:} & \multicolumn{4}{l}{\texttt{simulation}}\\
\multicolumn{2}{l}{child  elements:} & \multicolumn{4}{l}{\texttt{group, attrib}}\\
\multicolumn{2}{l}{attribute      :} & \multicolumn{4}{l}{}\\
   &   \bfseries name            & \bfseries datatype & \bfseries values & \bfseries default   & \bfseries description \\
   &   \texttt{name}/\texttt{id}   &  text              &  \textit{any}    &  e                & Name of particle set  \\
   &   \texttt{size}$^o$           &  integer           &  \textit{any}    &  0                & Number of particles in set \\
   &   \texttt{random}$^o$         &  text              &  yes/no          &  no               & Randomize starting positions \\
   &   \texttt{randomsrc}/         &  text     & \texttt{particleset.name} & \textit{none}     & Particle set to randomize  \\
   &   \texttt{random\_source}$^o$ &                    &                  &                   &                       \\
%   &   \texttt{role}     &  text              &  MC/none         &  none               & (obsolete)                       \\
  \hline
\end{tabularx}
\end{center}
\end{table}

\begin{table}[h]
\begin{center}
\begin{tabularx}{\textwidth}{l l l l l l }
\hline
\multicolumn{6}{l}{\texttt{group} element} \\
\hline
\multicolumn{2}{l}{parent elements:} & \multicolumn{4}{l}{\texttt{particleset}}\\
\multicolumn{2}{l}{child  elements:} & \multicolumn{4}{l}{\texttt{parameter, attrib}}\\
\multicolumn{2}{l}{attribute      :} & \multicolumn{4}{l}{}\\
   &   \bfseries name            & \bfseries datatype & \bfseries values & \bfseries default   & \bfseries description \\
   &   \texttt{name}               &  text              &  \textit{any}    &  e                & Name of particle set  \\
   &   \texttt{size}$^o$           &  integer           &  \textit{any}    &  0                & Number of particles in set \\
   &   \texttt{mass}$^o$           &  real              &  \textit{any}    &  1                & Mass of particles in set \\
   &   \texttt{unit}$^o$          &  text              &  au/amu          &  au               & Units for mass of particles \\
\multicolumn{2}{l}{parameters}  & \multicolumn{4}{l}{}\\
   &   \bfseries name     & \bfseries datatype & \bfseries values & \bfseries default   & \bfseries description \\
   &   \texttt{charge}    &  real              &  \textit{any}    &  0                  & Charge of particles in set \\
   &   \texttt{valence}   &  real              &  \textit{any}    &  0                  & Valence charge of particles in set \\
   &   \texttt{atomicnumber} &  integer        &  \textit{any}    &  0                  & Atomic number of particles in set \\
  \hline
  \hline
\end{tabularx}
\end{center}
\end{table}

\begin{table}[h]
\begin{center}
\begin{tabularx}{\textwidth}{l l l l l l }
\hline
\multicolumn{6}{l}{\texttt{attrib} element} \\
\hline
\multicolumn{2}{l}{parent elements:} & \multicolumn{4}{l}{\texttt{particleset,group}}\\
\multicolumn{2}{l}{attribute      :} & \multicolumn{4}{l}{}\\
   &   \bfseries name            & \bfseries datatype & \bfseries values & \bfseries default   & \bfseries description \\
   &   \texttt{name}             &  string            &  \textit{any}    &  \textit{none}    & Name of attrib              \\
   &   \texttt{datatype}         &  string            &  intArray, realArray, &  \textit{none} & Type of data in attrib \\
   &                             &                    &  posArray, stringArray &             &                        \\
   &   \texttt{size}$^o$         &  string            &  \textit{any}    &  \textit{none}    & Size of data in attrib \\
  \hline
  \hline
\end{tabularx}
\end{center}
\end{table}

\subsection{Detailed attribute description}

\subsubsection{particleset required attributes}

\begin{itemize}
\item \texttt{name}/\texttt{id} \\
Unique name for the particle set. Default is ``e" for electrons. ``i" or ``ion0" is typically used for ions. 
\end{itemize}
% Line 192 in ParticleIO/XMLParticleIO.cpp
% Lines 144-145 in QMCApp/ParticleSetPool.cpp

\subsubsection{particleset optional attributes}

\begin{itemize}
\item \texttt{size} \\
Number of particles in set
\end{itemize}
% Line 191 in ParticleIO/XMLParticleIO.cpp

%\begin{itemize}
%\item \texttt{role} \\
%What the particles do in the simulation
%\end{itemize}
% Line 146 in QMCApp/ParticleSetPool.cpp

\begin{itemize}
\item \texttt{random} \\
Randomize starting positions of particles. Each component of each particle's position is randomized independently in the range of the simulation cell in that component's direction. 
\end{itemize}
% Line 190 in ParticleIO/XMLParticleIO.cpp
% Line 147 in QMCApp/ParticleSetPool.cpp

\begin{itemize}
\item \texttt{randomsrc}/\texttt{random\_source} \\
Specify source particle set around which to randomize the initial positions of this particle set.
\end{itemize}
% Lines 148-149 in QMCApp/ParticleSetPool.cpp

\subsubsection{name required attributes}

\begin{itemize}
\item \texttt{name}/\texttt{id} \\
Unique name for the particle set group. Typically, element symbols are used for ions and ``u" or ``d" for spin-up and spin-down electron groups, respectively. 
\end{itemize}
% Line 192 in ParticleIO/XMLParticleIO.cpp
% Lines 144-145 in QMCApp/ParticleSetPool.cpp

\subsubsection{group optional attributes}

\begin{itemize}
\item \texttt{mass} \\
Mass of particles in set.
\end{itemize}
% Line 190 in Particle/ParticleSet.cpp

\begin{itemize}
\item \texttt{unit} \\
Units for mass of particles in set (au[$m_e$ = 1] or amu[$\frac{1}{12}m_{\rm ^{12}C}$ = 1]).
\end{itemize}
% Line 66 in ParticleIO/XMLParticleIO.cpp


%condition appears to be future functionality for different unit types on the position array
%condition must be an integer
% Line 407 in ParticleIO/XMLParticleIO.cpp (reads condition in)
% Line 402 in ParticleIO/XMLParticleIO.cpp (declares utype integer)

\subsection{Example use cases}
\begin{minipage}{\linewidth}
\begin{lstlisting}[caption=particleset elements for ions and electrons randomizing electron start positions.]
  <particleset name="i" size="2">
    <group name="Li">
      <parameter name="charge">3.000000</parameter>
      <parameter name="valence">3.000000</parameter>
      <parameter name="atomicnumber">3.000000</parameter>
    </group>
    <group name="H">
      <parameter name="charge">1.000000</parameter>
      <parameter name="valence">1.000000</parameter>
      <parameter name="atomicnumber">1.000000</parameter>
    </group>
    <attrib name="position" datatype="posArray" condition="1">
    0.0   0.0   0.0
    0.5   0.5   0.5
    </attrib>
    <attrib name="ionid" datatype="stringArray">
       Li H
    </attrib>
  </particleset>
  <particleset name="e" random="yes" randomsrc="i">
    <group name="u" size="2">
      <parameter name="charge">-1</parameter>
    </group>
    <group name="d" size="2">
      <parameter name="charge">-1</parameter>
    </group>
  </particleset>                 
\end{lstlisting}
\end{minipage}

\begin{minipage}{\linewidth}
\begin{lstlisting}[caption=particleset elements for ions and electrons specifying electron start positions]
  <particleset name="e">
    <group name="u" size="4">
      <parameter name="charge">-1</parameter>
      <attrib name="position" datatype="posArray">
	2.9151687332e-01 -6.5123272502e-01 -1.2188463918e-01
	5.8423636048e-01  4.2730406357e-01 -4.5964306231e-03
	3.5228575807e-01 -3.5027014639e-01  5.2644808295e-01
       -5.1686250912e-01 -1.6648002292e+00  6.5837023441e-01
      </attrib>
    </group>
    <group name="d" size="4">
      <parameter name="charge">-1</parameter>
      <attrib name="position" datatype="posArray">
	3.1443445436e-01  6.5068682609e-01 -4.0983449009e-02
       -3.8686061749e-01 -9.3744432997e-02 -6.0456005388e-01
	2.4978241724e-02 -3.2862514649e-02 -7.2266047173e-01
       -4.0352404772e-01  1.1927734805e+00  5.5610824921e-01
      </attrib>
    </group>
  </particleset>
  <particleset name="ion0" size="3">
    <group name="O">
      <parameter name="charge">6</parameter>
      <parameter name="valence">4</parameter>
      <parameter name="atomicnumber">8</parameter>
    </group>
    <group name="H">
      <parameter name="charge">1</parameter>
      <parameter name="valence">1</parameter>
      <parameter name="atomicnumber">1</parameter>
    </group>
    <attrib name="position" datatype="posArray">
      0.0000000000e+00  0.0000000000e+00  0.0000000000e+00
      0.0000000000e+00 -1.4308249289e+00  1.1078707576e+00
      0.0000000000e+00  1.4308249289e+00  1.1078707576e+00
    </attrib>
    <attrib name="ionid" datatype="stringArray">
      O H H 
    </attrib>
  </particleset>
\end{lstlisting}
\end{minipage}

\begin{minipage}{\linewidth}
\begin{lstlisting}[caption=particleset elements for ions specifying positions by ion type]
  <particleset name="ion0">
    <group name="O" size="1">
      <parameter name="charge">6</parameter>
      <parameter name="valence">4</parameter>
      <parameter name="atomicnumber">8</parameter>
      <attrib name="position" datatype="posArray">
        0.0000000000e+00  0.0000000000e+00  0.0000000000e+00
      </attrib>
    </group>
    <group name="H" size="2">
      <parameter name="charge">1</parameter>
      <parameter name="valence">1</parameter>
      <parameter name="atomicnumber">1</parameter>
      <attrib name="position" datatype="posArray">
        0.0000000000e+00 -1.4308249289e+00  1.1078707576e+00
        0.0000000000e+00  1.4308249289e+00  1.1078707576e+00
      </attrib>
    </group>
  </particleset>
\end{lstlisting}
\end{minipage}


\chapter{Trial wavefunction specification}
\section{Introduction}
\label{sec:intro_wavefunction}

This section describes the input blocks associated with the specification of the trial wavefunction in a QMCPACK calculation. These sections are contained within the $<wavefunction > ...  </wavefunction>$ xml blocks. \textbf{Users are expected to rely on converters to generate the input blocks described in this section.} The converters and the workflows are designed such that input blocks require minimum modifications from users. Unless the workflow requires modification of wavefunction blocks (e.g. setting the cutoff in a multi determinant calculation), only expert users should directly alter them.
  
The trial wavefunction in QMCPACK has a general product form:
\begin{equation}
\Psi_T(\vec{r}) = \prod_k \Theta_k(\vec{r}),
\end{equation}
where each $\Theta_k(\vec{r})$ is a function of the electron coordinates (and possibly ionic coordinates and variational parameters). For problems involving electrons, the overall trial wavefunction must be antisymmetric with respect to electron exchange, so at least one of the functions in the product must be antisymmetric. Notice that, while QMCPACK allows for the construction of arbitrary trial wavefunctions based on the functions implemented in the code (e.g. slater determinants, jastrow functions, etc), the user must make sure that a correct wavefunction is used for the problem at hand. From here on, we assume a standard trial wavefunction for an electronic structure problem, 
\begin{equation}
\Psi_T(\vec{r}) =  \textit{A}(\vec{r}) \prod_k \textit{J}_k(\vec{r}),
\end{equation}
where $\textit{A}(\vec{r})$ is one of the antisymmetric functions: 1) slater determinant, 2) multi slater determinant, or 3) pfaffian, and $\textit{J}_k$ is any of the jastrow functions (described in section \ref{sec:jastrow}).  The antisymmetric functions are built from a set of single particle orbitals (\texttt{sposet}). QMCPACK implements 4 different types of \texttt{sposet}, described in the section below. Each \texttt{sposet} is designed for a different type of calculation, so their definition and generation varies accordingly. 

\section{Single determinant wavefunctions}
\label{sec:singledeterminant}
Placing a single determinant for each spin is the most used ansatz for the antisymmetric part of a trial wavefunction.
The input xml block for \texttt{slaterdeterminant} is give in Listing~\ref{listing:singledet}. A list of options is given in
Table~\ref{table:singledet}

\begin{table}[h]
\begin{center}
\begin{tabularx}{\textwidth}{l l l l l l }
\hline
\multicolumn{6}{l}{\texttt{slaterdeterminant} element} \\
\hline
\multicolumn{2}{l}{parent elements:} & \multicolumn{4}{l}{\texttt{determinantset}}\\
\multicolumn{2}{l}{child  elements:} & \multicolumn{4}{l}{\texttt{determinant}}\\
\multicolumn{2}{l}{attribute      :} & \multicolumn{4}{l}{}\\
   &   \bfseries name       & \bfseries datatype & \bfseries values & \bfseries default & \bfseries description \\
   &   \texttt{delay\_rank} &  integer           &  >0              & 1           &  The number of delayed updates. \\
   &   \texttt{optimize}    &  text              &  yes/no          & yes         &  Enable orbital optimization. \\
  \hline
\end{tabularx}
\end{center}
\caption{Options for the \texttt{slaterdeterminant} xml-block.}
\label{table:singledet}
\end{table}

\begin{lstlisting}[caption=slaterdeterminant set XML element.\label{listing:singledet}]
  <slaterdeterminant delay_rank="32">
    <determinant id="updet" size="208">
      <occupation mode="ground" spindataset="0">
      </occupation>
    </determinant>
    <determinant id="downdet" size="208">
      <occupation mode="ground" spindataset="0">
      </occupation>
    </determinant>
  </slaterdeterminant>
\end{lstlisting}

Additional information:
\begin{itemize}
\item \texttt{delay\_rank}. This option enables the delayed updates of Slater matrix inverse when particle-by-particle move is used.
By default, \texttt{delay\_rank=1} uses the Fahy's variant of the Sherman-Morrison rank-1 update which is mostly using memory bandwidth bound BLAS-2 calls.
With \texttt{delay\_rank>1}, the delayed update algorithm turns most of the computation to compute bound BLAS-3 calls.
Tuning this parameter is highly recommended to gain the best performance on medium to large problem sizes ($>200$ electrons).
We have seen up to an order of magnitude speed-up on large problem sizes.
When studying the performance of QMCPACK, a scan of this parameter is required and we recommend to start from 32.
The best \texttt{delay\_rank} giving the maximal speed-up depends the problem size.
Usually the larger \texttt{delay\_rank} corresponds to a larger problem size.
On CPUs, \texttt{delay\_rank} must be chosen a multiple of SIMD vector length. The best \texttt{delay\_rank} depends on the processor micro architecture.
The GPU support is currently under development.
\end{itemize}


\section{Single-particle orbitals}
\label{sec:spo}

\subsection{Spline basis sets}
\label{sec:spo_spline}
In this section we describe the use of spline basis sets to expand the \texttt{sposet}.
Spline basis sets are designed to work seamless with plane wave DFT code, e.g.\ Quantum ESPRESSO as a trial wavefunction generator.

In QMC algorithms, all the SPOs $\{\phi(\vec{r})\}$ need to be updated every time a single electron moves.
Evaluating SPOs takes very large portion of computation time.
In principle, PW basis set can be used to express SPOs directly in QMC like in DFT.
but it introduces an unfavorable scaling due to the fact 
that the basis set size increases linearly as the system size.
For this reason, it is efficient to use a localized basis with compact
support and a good transferability from plane wave basis. 

In particular, 3D tricubic B-splines provide a basis in which only
64 elements are nonzero at any given point in space~\cite{blips4QMC}.
The one-dimensional cubic B-spline is given by,
\begin{equation}
f(x) = \sum_{i'=i-1}^{i+2} b^{i'\!,3}(x)\,\,  p_{i'},
\label{eq:SplineFunc}
\end{equation}
where $b^{i}(x)$ are the piecewise cubic polynomial basis functions
and $i = \text{floor}(\Delta^{-1} x)$ is the index of
the first grid point $\le x$.  Constructing a tensor product in each Cartesian
direction, we can represent a 3D orbital as
\begin{equation}
  \phi_n(x,y,z) = 
  \!\!\!\!\sum_{i'=i-1}^{i+2} \!\! b_x^{i'\!,3}(x) 
  \!\!\!\!\sum_{j'=j-1}^{j+2} \!\! b_y^{j'\!,3}(y) 
  \!\!\!\!\sum_{k'=k-1}^{k+2} \!\! b_z^{k'\!,3}(z) \,\, p_{i', j', k',n}.
\label{eq:TricubicValue}
\end{equation}
This allows the rapid evaluation of each orbital in constant time.
Furthermore, this basis is systematically improvable with a single spacing
parameter, so that accuracy is not compromised compared with plane wave basis.

The use of 3D tricubic B-splines greatly improves the computational efficiency.
The gain in computation time from plane wave basis set to an equivalent B-spline basis set 
becomes increasingly large as the system size grows.
On the downside, this computational efficiency comes at
the expense of increased memory use, which is easily overcome by the large
aggregate memory available per node through OpenMP/MPI hybrid QMC.

The input xml block for the spline SPOs is give in Listing~\ref{listing:splineSPOs}. A list of options is given in 
Table~\ref{table:splineSPOs}. \texttt{QMCPACK} has a very useful command line option \texttt{--save\_wfs} which allows to dump 
the real space B-spline coefficient table into a h5 file on the disk.
When the orbital transformation from k space to B-spline requires more than available amount of scratch memory on the compute nodes, 
users can perform this step on fat nodes and transfer back the h5 file for QMC calculations.

\begin{table}[h]
\begin{center}
\begin{tabularx}{\textwidth}{l l l l l l }
\hline
\multicolumn{6}{l}{\texttt{determinantset} element} \\
\hline
\multicolumn{2}{l}{parent elements:} & \multicolumn{4}{l}{\texttt{wavefunction}}\\
\multicolumn{2}{l}{child  elements:} & \multicolumn{4}{l}{\texttt{slaterdeterminant}}\\
\multicolumn{2}{l}{attribute      :} & \multicolumn{4}{l}{}\\
   &   \bfseries name              & \bfseries datatype & \bfseries values & \bfseries default   & \bfseries description \\
   &   \texttt{type}                    &  text               &   bspline   &               &  Type of \texttt{sposet}. \\
   &   \texttt{href}                    &  text               &             &               &  Path to the h5 file generated by pw2qmcpack.x. \\
   &   \texttt{tilematrix}              &  9 integers         &             &               &  Tiling matrix used to expand supercell. \\
   &   \texttt{twistnum}                &  integer            &             &               &  Index of the super twist. \\
   &   \texttt{twist}                   &  3 floats         &             &               &  Super twist. \\
   &   \texttt{meshfactor}              &  float              &  $\le 1.0$      &               &  Grid spacing ratio. \\
   &   \texttt{precision}               &  text               &  single/double  &               &  Precision of spline coefficients. \\
   &   \texttt{gpu}                    &  text               &  yes/no      &               &  GPU switch. \\
   &   \texttt{Spline\_Size\_Limit\_MB}  &  integer            &                 &             &  Limit the size of B-spline coefficient table on GPU. \\
   &   \texttt{source}               &  text               &   \textit{any}    &  ion0        & Particle set with the position of atom centers. \\
  \hline
\end{tabularx}
\end{center}
\caption{Options for the \texttt{determinantset} xml-block associated with B-spline single particle orbital sets.}
\label{table:splineSPOs}
\end{table}

%%\begin{lstlisting}[caption=.]
%%<sposet_builder type="bspline" href="pwscf.h5" tilematrix="2 0 0 0 2 0 0 0 2" twistnum="0"
%%                 source="i" meshfactor="1.0" precision="float" truncate="no">
%%   <sposet type="bspline" name="spo_ud" size="208" spindataset="0"/>
%%</sposet_builder>
%%<determinantset>
%%   <slaterdeterminant>
%%      <determinant id="updet" group="u" sposet="spo_ud" size="208"/>
%%      <determinant id="downdet" group="d" sposet="spo_ud" size="208"/>
%%   </slaterdeterminant>
%%</determinantset>
%%\end{lstlisting}

\begin{lstlisting}[caption=All electron Hamiltonian XML element.\label{listing:splineSPOs}]
<determinantset type="bspline" source="i" href="pwscf.h5"
                tilematrix="1 1 3 1 2 -1 -2 1 0" twistnum="-1" gpu="yes" meshfactor="0.8"
                twist="0  0  0" precision="double">
  <slaterdeterminant>
    <determinant id="updet" size="208">
      <occupation mode="ground" spindataset="0">
      </occupation>
    </determinant>
    <determinant id="downdet" size="208">
      <occupation mode="ground" spindataset="0">
      </occupation>
    </determinant>
  </slaterdeterminant>
</determinantset>
\end{lstlisting}

Additional information:
\begin{itemize}
\item \texttt{precision}. Only effective on CPU version without mixed precision, `single' is always imposed with mixed precision. Using single precision not only saves memory usage but also speeds up the B-spline evaluation. It is recommended to use single precision since we saw little chance of really compromising the accuracy of calculation.
\item \texttt{meshfactor}. It is the ratio of actual grid spacing of B-splines used in QMC calculation with respect to the original one calculated from h5. Smaller meshfactor saves memory usage but reduces accuracy. The effects are similar to reducing plane wave cutoff in DFT calculation. Use with caution! 
\item \texttt{twistnum}. If positive, it is the index. It is recommended not to take this way since the indexing may show some uncertainty. If negative, the super twist is referred by \texttt{twist}.
\item \texttt{Spline\_Size\_Limit\_MB}. Allows to distribute the B-spline coefficient table between the host and GPU memory. The compute kernels access host memory via zero-copy. Though the performance penaty introduced by it is significant but allows large calculations to go.
\end{itemize}
\label{sec:splinebasis}

\subsection{Gaussian basis sets}
\label{sec:gaussianbasis}

In this section we describe the use of localized basis sets to expand the \ixml{sposet}. The general form of a single particle orbital in this case is given by:
\begin{equation}
\phi_i(\vec{r}) = \sum_k C_{i,k} \ \eta_k(\vec{r}),
\end{equation}
where $\{\eta_k(\vec{r})\}$ is a set of M atom-centered basis
functions and $C_{i,k}$ is a coefficient matrix. This \ixml{sposet}
should be used in calculations of finite systems employing an
atom-centered basis set and is typically generated by the
\textit{convert4qmc} converter.  Examples include calculations of
molecules using Gaussian basis sets or Slater-type basis
functions. Initial support for periodic systems is described in Section
\ref{chap:LCAO}. Even though this section is called "Gaussian basis
sets" (by far the most common atom-centered basis set), QMCPACK works
with any atom-centered basis set based on either spherical
harmonic angular functions or Cartesian angular expansions. The radial
functions in the basis set can be expanded in either Gaussian
functions, Slater-type functions, or numerical radial functions.

In this section we describe the input sections for the atom-centered basis set and the \ixml{sposet} for a single Slater determinant trial wavefunction. The input sections for multideterminant trial wavefunctions are described in Section~\ref{sec:multideterminants}. The basic structure for the input block of a single Slater determinant is given in Listing~\ref{listing:lcaosposet}.
A list of options for \ixml{determinantset} associated with this \ixml{sposet} is given in Table~\ref{table:determinantset}.

\begin{minipage}{\linewidth}
\begin{lstlisting}[style=QMCPXML,caption=Basic input block for a single determinant trial wavefunction using a \ixml{sposet} expanded on an atom-centered basis set. \label{listing:lcaosposet}]
<wavefunction id="psi0" target="e">
    <determinantset>
      <basisset>
        ...
      </basisset>
      <slaterdeterminant>
        ...
      </slaterdeterminant>
    </determinantset>    
</wavefunction>
\end{lstlisting}
\end{minipage}

\begin{table}[h]
\begin{center}
\begin{tabularx}{\textwidth}{l l l l l X }
\hline
\multicolumn{6}{l}{\texttt{determinantset} element} \\
\hline
\multicolumn{2}{l}{Parent elements:} & \multicolumn{4}{l}{\texttt{wavefunction}}\\
\multicolumn{2}{l}{Child  elements:} & \multicolumn{4}{l}{\texttt{basisset,slaterdeterminant,sposet,multideterminant}}\\
\multicolumn{2}{l}{Attribute:} & \multicolumn{4}{l}{}\\
   &   \bfseries Name              & \bfseries Datatype & \bfseries Values & \bfseries Default   & \bfseries Description \\
   &   \texttt{name}/\texttt{id}   &  Text              &  \textit{Any}    &  ""             & Name of determinant set \\
   &   \texttt{type}                    &  Text               &   See below   &   ""            &  Type of \texttt{sposet} \\
   &   \texttt{keyword}             &  Text               &   NMO,GTO,STO   &  NMO        & Type of orbital set generated \\  
   &   \texttt{transform}           &  Text               &   Yes/no          &  Yes         &  Transform to numerical radial functions?  \\
   &   \texttt{source}               &  Text               &   \textit{Any}    &  Ion0        & Particle set with the position of atom centers \\
   &   \texttt{cuspCorrection}  &  Text               &   Yes/no          &  No         & Apply cusp correction scheme to \texttt{sposet}? \\
%   &   \texttt{cuspInfo}               &  text               &   \textit{any}    &  ""             & File with saved cusp data. \\       
  \hline
\end{tabularx}
\end{center}
\caption{Options for the \ixml{determinantset} xml-block associated with atom-centered single particle orbital sets.}
\label{table:determinantset}
\end{table}

The definition of the set of atom-centered basis functions is given by the \ixml{basisset} block, and the \ixml{sposet} is defined within \ixml{slaterdeterminant}. The \ixml{basisset} input block is composed from a collection of \ixml{atomicBasisSet} input blocks, one for each atomic species in the simulation where basis functions are centered. The general structure for \ixml{basisset} and \ixml{atomicBasisSet} are given in Listing~\ref{listing:basisset}, and the corresponding lists of options are given in Tables~\ref{table:basisset} and~\ref{table:atomicBasisSet}.

\begin{minipage}{\linewidth}
\begin{lstlisting}[style=QMCPXML,caption=Basic input block for \ixml{basisset}.\label{listing:basisset}]
      <basisset name="LCAOBSet">
        <atomicBasisSet name="Gaussian-G2" angular="cartesian" elementType="C" normalized="no">
          <grid type="log" ri="1.e-6" rf="1.e2" npts="1001"/>
          <basisGroup rid="C00" n="0" l="0" type="Gaussian">
            <radfunc exponent="5.134400000000e-02" contraction="1.399098787100e-02"/>
            ...
          </basisGroup>
          ...              
        </atomicBasisSet>
        <atomicBasisSet name="Gaussian-G2" angular="cartesian" type="Gaussian" elementType="C" normalized="no">
          ...              
        </atomicBasisSet>
        ...
      </basisset>
\end{lstlisting}
\end{minipage}

\begin{table}[h]
\begin{center}
\begin{tabularx}{\textwidth}{l l l l l X }
\hline
\multicolumn{6}{l}{\texttt{basisset} element} \\
\hline
\multicolumn{2}{l}{Parent elements:} & \multicolumn{4}{l}{\texttt{determinantset}}\\
\multicolumn{2}{l}{Child  elements:} & \multicolumn{4}{l}{\texttt{atomicBasisSet}}\\
\multicolumn{2}{l}{Attribute:} & \multicolumn{4}{l}{}\\
   &   \bfseries Name              & \bfseries Datatype & \bfseries Values & \bfseries Default   & \bfseries Description \\
   &   \texttt{name}/\texttt{id}   &  Text              &  \textit{Any}    &  ""                & Name of atom-centered basis set \\
  \hline
\end{tabularx}
\end{center}
\caption{Options for the \texttt{basisset} xml-block associated with atom-centered single particle orbital sets.}
\label{table:basisset}
\end{table}

\begin{table}[h]
\begin{center}
\begin{tabularx}{\textwidth}{l l l l l X }
\hline
\multicolumn{6}{l}{\texttt{atomicBasisSet} element} \\
\hline
\multicolumn{2}{l}{Parent elements:} & \multicolumn{4}{l}{\texttt{basisset}}\\
\multicolumn{2}{l}{Child  elements:} & \multicolumn{4}{l}{\texttt{grid,basisGroup}}\\
\multicolumn{2}{l}{Attribute:} & \multicolumn{4}{l}{}\\
   &   \bfseries Name              & \bfseries Datatype & \bfseries Values & \bfseries Default   & \bfseries Description \\
   &   \texttt{name}/\texttt{id}   &  Text              &  \textit{Any}    &  ""                & Name of atomic basis set \\
   &   \texttt{angular}               &  Text              &  See below    &  Default       & Type of angular functions  \\
%   &   \texttt{type}                    &  text              &  see below    &  ""                &  Type of input radial function. \\
   &   \texttt{expandYlm}         &  Text              &  See below   &  Yes                &  Expand Ylm shells? \\  
   &   \texttt{expM}                  &  Text              &  See below   &  Yes                &  Add sign for $(-1)^{m}$? \\  
   &   \texttt{elementType/species}   &  Text  &  \textit{Any}    &  e                &  Atomic species where functions are centered \\
   &   \texttt{normalized}         &  Text              &  Yes/no   &  Yes                &  Are single particle functions normalized? \\   
  \hline
\end{tabularx}
\end{center}
\caption{Options for the \texttt{atomicBasisSet} xml-block.}
\label{table:atomicBasisSet}
\end{table}

\begin{table}[h]
\begin{center}
\begin{tabularx}{\textwidth}{l l l l l X }
\hline
\multicolumn{6}{l}{\texttt{basisGroup} element} \\
\hline
\multicolumn{2}{l}{Parent elements:} & \multicolumn{4}{l}{\texttt{atomicBasisSet}}\\
\multicolumn{2}{l}{Child  elements:} & \multicolumn{4}{l}{\texttt{radfunc}}\\
\multicolumn{2}{l}{Attribute:} & \multicolumn{4}{l}{}\\
   &   \bfseries Name              & \bfseries Datatype & \bfseries Values & \bfseries Default   & \bfseries Description \\
   &   \texttt{rid}/\texttt{id}   &  Text              &  \textit{Any}    &  ""                & Name of the basisGroup \\
   &   \texttt{type}                    &  Text            &  \textit{Any}    &  ""                & Type of basisGroup \\
   &   \texttt{n/l/m/s}                 &  Integer           &  \textit{Any}    &  0                & Quantum numbers of basisGroup \\
  \hline
\end{tabularx}
\end{center}
\caption{Options for the \texttt{basisGroup} xml-block.}
\label{table:basisGroup}
\end{table}

\begin{minipage}{\linewidth}
\begin{lstlisting}[style=QMCPXML,caption=Basic input block for \texttt{slaterdeterminant} with an atom-centered \texttt{sposet}.\label{listing:slaterdeterminant}]
      <slaterdeterminant>
      </slaterdeterminant>
\end{lstlisting}
\end{minipage}

\begin{table}[h]
\begin{center}
\begin{tabularx}{\textwidth}{l l l l l X }
\hline
\multicolumn{6}{l}{\texttt{} element} \\
\hline
\multicolumn{2}{l}{Parent elements:} & \multicolumn{4}{l}{\texttt{}}\\
\multicolumn{2}{l}{Child  elements:} & \multicolumn{4}{l}{\texttt{}}\\
\multicolumn{2}{l}{Attribute:} & \multicolumn{4}{l}{}\\
   &   \bfseries Name              & \bfseries Datatype & \bfseries Values & \bfseries Default   & \bfseries Description \\
   &   \texttt{name}/\texttt{id}   &  Text              &  \textit{Any}    &  ""                & Name of determinant set \\
   &   \texttt{}                    &  Text              &  \textit{Any}    &  ""                &  \\
  \hline
\end{tabularx}
\end{center}
\end{table}

\subsubsection{Detailed description of attributes:}

In the following, we give a more detailed description of all the options presented in the various xml-blocks described in this section. Only nontrivial attributes are described. Those with simple yes/no options and whose previous description is enough to explain the intended behavior are not included. 

\hspace{1mm} \\

\ixml{determinantset} attributes:

\begin{itemize}
\item \ixml{type} \\
Type of \ixml{sposet}. For atom-centered based \ixml{sposets}, use type=``MolecularOrbital" or type=``MO." Other options described elsewhere in this manual are ``spline," ``composite," ``pw," ``heg," ``linearopt," etc.
\item \ixml{keyword}/\ixml{key} \\
Type of basis set generated, which does not necessarily match the type of basis set on the input block. The three possible options are: NMO (numerical molecular orbitals), GTO (Gaussian-type orbitals), and STO (Slater-type orbitals). The default option is NMO. By default, QMCPACK will generate numerical orbitals from both GTO and STO types and use cubic or quintic spline interpolation to evaluate the radial functions. This is typically more efficient than evaluating the radial functions in the native basis (Gaussians or exponents) and allows for arbitrarily large contractions without any additional cost. To force use of the native expansion (not recommended), use GTO or STO for each type of input basis set.
\item \ixml{transform}\\
Request (or avoid) a transformation of the radial functions to NMO type. The default and recommended behavior is to transform to numerical radial functions. If \ixml{transform} is set to \textit{``yes,"} the option \ixml{keyword} is ignored.  
\item \ixml{cuspCorrection}\\
Enable (disable) use of the cusp correction algorithm (CASINO REFERENCE) for a \ixml{basisset} built with GTO functions. The algorithm is implemented as described in (CASINO REFERENCE) and works only with transform=``yes" and an input GTO basis set. No further input is needed. 
\end{itemize}

\ixml{atomicBasisSet} attributes:

\begin{itemize}
\item \ixml{name/id}\\
Name of the basis set. Names should be unique.
\item \ixml{angular}\\
Type of angular functions used in the expansion. In general, two angular basis functions are allowed: ``spherical" (for spherical Ylm functions) and ``Cartesian" (for functions of the type $x^{n}y^{m}z^{l}$).  
%\item \ixml{type}\\ Type of input radial functions. Options are: "Numerical" (for radial functions on a numerical radial grid), "Gaussian" (for an expansion in gaussian functions) and "" 
\item \ixml{expandYlm}\\
Determines whether each basis group is expanded across the corresponding shell of m values (for spherical type) or consistent powers (for Cartesian functions). Options:
\begin{itemize}
\item ``No": Do not expand angular functions across corresponding angular shell.
\item ``Gaussian": Expand according to Gaussian03 format. This function is compatible only with angular=``spherical." For a given input (l,m), the resulting order of the angular functions becomes (1,-1,0) for l=1 and (0,1,-1,2,-2,...,l,-l) for general l.
\item ``Natural": Expand angular functions according to (-l,-l+1,...,l-1,l). 
\item ``Gamess": Expand according to Gamess' format for Cartesian functions. Notice that this option is compatible only with angular=``Cartesian." If angular=``Cartesian" is used, this option is not necessary.
\end{itemize}
\item \ixml{expM}\\ 
Determines whether the sign of the spherical Ylm function associated with m ($-1^{m}$) is included in the coefficient matrix or not.
\item \ixml{elementType/species}\\
Name of the species where basis functions are centered. Only one \ixml{atomicBasisSet} block is allowed per species. Additional blocks are ignored. The corresponding species must exist in the \ixml{particleset} given as the \ixml{source} option to \ixml{determinantset}. Basis functions for all the atoms of the corresponding species are included in the basis set, based on the order of atoms in the \ixml{particleset}.
\end{itemize}

\ixml{basisGroup} attributes:

\begin{itemize}
\item \ixml{type}\\
  Type of input basis radial function. Note that this refers to the type of radial function in the input xml-block, which might not match the radial function generated internally and used in the calculation (if \ixml{transform} is set to ``yes"). Also note that different \ixml{basisGroup} blocks within a given \ixml{atomicBasisSet} can have different \ixml{types}.
\item \ixml{n/l/m/s}\\
  Quantum numbers of the basis function. Note that if
\ixml{expandYlm} is set to \textit{``yes"} in \ixml{atomicBasisSet}, a
full shell of basis functions with the appropriate values of
\textit{``m"} will be defined for the corresponding value of
\textit{``l."} Otherwise a single basis function will be given for the
specific combination of \textit{``(l,m)."}
\end{itemize}

\ixml{radfunc} attributes for \ixml{type}=\textit{``Gaussian"}:

\begin{itemize}
\item \ixml{TBDoc}\\
\end{itemize}a

\ixml{slaterdeterminant} attributes:

\begin{itemize}
\item \ixml{TBDoc}\\
\end{itemize}




\subsection{Plane-wave basis sets}

\subsection{Homogeneous electron gas}
\label{sec:hegbasis}

The interacting Fermi Liquid has its own special determinantset for filling up a
Fermi surface.  The shell number can be specified seperately for both spin up and spin down.
This determines how many electrons to include of each time, only closed shells are currently
implemented.  The shells are filled according to the rules of a square box, if other lattice
vectors are used, the electrons may not fill up a complete shell.

This following example can also be used for Helium simulations too, by specifying the
proper pair interaction in the Hamiltonian section. 

\begin{lstlisting}[caption=2D Fermi Liquid example: particle specification ]
<qmcsystem>
<simulationcell name="global">
<parameter name="rs" pol="0" condition="74">6.5</parameter>
<parameter name="bconds">p p p</parameter>
<parameter name="LR_dim_cutoff">15</parameter>
</simulationcell>
<particleset name="e" random="yes">
<group name="u" size="37">
<parameter name="charge">-1</parameter>
<parameter name="mass">1</parameter>
</group>
<group name="d" size="37">
<parameter name="charge">-1</parameter>
<parameter name="mass">1</parameter>
</group>
</particleset>
</qmcsystem>
\end{lstlisting}

\begin{lstlisting}[caption=2D Fermi Liquid example (Slater Jastrow wave function) ]
<qmcsystem>
  <wavefunction name="psi0" target="e">
      <determinantset type="electron-gas" shell="7" shell2="7" randomize="true">
  </determinantset>
      <jastrow name="J2" type="Two-Body" function="Bspline" print="no">
        <correlation speciesA="u" speciesB="u" size="8" cusp="0">
          <coefficients id="uu" type="Array" optimize="yes"> 
        </correlation>
        <correlation speciesA="u" speciesB="d" size="8" cusp="0">
          <coefficients id="ud" type="Array" optimize="yes"> 
        </correlation>
      </jastrow>
\end{lstlisting}



\section{Jastrow Factors}
\label{sec:jastrow}

Jastrow factors are among the simplest and most effective ways of including
dynamical correlation in the trial many body wavefunction.  The resulting many body
wavefunction is expressed as the product of an antisymmetric (in the case
of Fermions) or symmetric (for Bosons) part and a correlating jastrow factor
like so:
\begin{equation}
\Psi(\vec{R}) = \mathcal{A}(\vec{R}) \exp\left[J(\vec{R})\right]
\end{equation}

In this section we will detail the types and forms of Jastrow factor used 
in QMCPACK. 

\section{One-body Jastrow functions}
\label{sec:onebodyjastrow}
The one-body Jastrow factor is a form that allows for the direct inclusion
of correlations between particles that are included in the wavefunction with
particles that are not explicitly part of it.  The most common example of
this are correlations between electrons and ions.  

The jastrow function is specified within a \texttt{wavefunction} element
and must contain one of more \texttt{correlation} elements specifying
additional parameters as well as the actual coefficients. Section
\ref{sec:1bjsplineexamples} gives examples of the typical nesting of
\texttt{jastrow}, \texttt{correlation}, and \texttt{coefficient} elements.

\subsection{Input Specification}

\begin{table}[h]
\begin{center}
\begin{tabular}{l c c c l }
\hline
\multicolumn{5}{l}{Jastrow element} \\
\hline
\bfseries name & \bfseries datatype & \bfseries values & \bfseries defaults  & \bfseries description \\
\hline
name & text &    & (required) & Unique name for this Jastrow function \\
type & text & One-body & (required) & Define a one-body function \\ 
function & text & Bspline & (required) & BSpline Jastrow \\
             & text & Pade & & Pade form \\
             & text & \ldots & & \ldots \\
source & text & name & (required) & name of attribute of classical particle set \\ 
print & text & yes / no & yes & should jastrow factor be printed to an external file\\
  \hline
\multicolumn{5}{l}{elements}\\ \hline
& Correlation & & & \\ \hline
\multicolumn{5}{l}{Contents}\\ \hline
& (None)  & & &  \\ \hline
\end{tabular}
%\end{tabular*}
\end{center}
\end{table}

To be more concrete, the one-body jastrow factors used to describe correlations
between electrons and ions take the form below
\begin{equation}
J1=\sum_I^{ion0}\sum_i^e u_{ab}(|r_i-R_I|)
\end{equation}
where I runs over all of the ions in the calculation, i runs over the electrons
and $u_{ab}$ describes the functional form of the correlation between them.
Many different forms of $u_{ab}$ are implemented in QMCPACK.  We will detail 
two of the most common ones below.

\subsubsection{Spline form}
\label{sec:onebodyjastrowspline}

The one-body spline Jastrow function is the most commonly used one-body Jastrow for solids. This form 
was first described and used in \cite{EslerKimCeperleyShulenburger2012}.  
Here $u_{ab}$ is an interpolating 1D Bspline (tricublc spline on a linear grid) between zero distance and $r_{cut}$. In 3D periodic systems 
the default cutoff distance is the Wigner Seitz cell radius. For other periodicities including isolated 
molecules the $r_{cut}$ must be specified. The cusp can be set.   $r_i$ 
and $R_I$ are most commonly the electron and ion positions, but any particlesets that can provide the 
needed centers can be used.

\paragraph{Input Specification}
\begin{table}[h]
\begin{center}
\begin{tabular}{l c c c l }
\hline
\multicolumn{5}{l}{Correlation element} \\
\hline
\bfseries name & \bfseries datatype & \bfseries values & \bfseries defaults & \bfseries description \\
\hline
elementType & text & name & see below & Classical particle target  \\
speciesA & text & name & see below & Classical particle target \\
speciesB & text & name & see below & Quantum species target \\
size & integer & $> 0$ & (required) & number of coefficients \\
rcut & real & $> 0$ & see below & distance at which the correlation goes to 0 \\
cusp & real & $\ge 0$ & 0 & value for use in Kato cusp condition \\
spin & text & yes or no & no & spin dependent jastrow factor \\
\hline
\multicolumn{5}{l}{elements}\\ \hline
& Coefficients & & & \\ \hline
\multicolumn{5}{l}{Contents}\\ \hline
& (None)  & & &  \\ \hline
\end{tabular}
%\end{tabular*}
\end{center}
\end{table}

Additional information:

 \begin{itemize}
 \item \texttt{elementType, speciesA, speciesB, spin}.  For a spin independent Jastrow factor (spin = ``no'')
elementType should be the name of the group of ions in the classical particleset to which the quantum
particles should be correlated.  For a spin dependent Jastrow factor (spin = ``yes'') set speciesA to the
group name in the classical particleset and speciesB to the group name in the quantum particleset.
 \item \texttt{rcut}. The cutoff distance for the function in atomic units (bohr). 
For 3D fully periodic systems this parameter is optional and a default of the Wigner 
Seitz cell radius is used. Otherwise this parameter is required.
 \item \texttt{cusp}. The one body jastrow factor can be used to make the wavefunction
satisfy the electron-ion cusp condition\cite{kato}.  In this case, the derivative of the jastrow
factor as the electron approaches the nucleus will be given by:
\begin{equation}
\left(\frac{\partial J}{\partial r_{iI}}\right)_{r_{iI} = 0} = -Z
\end{equation}
Note that if the antisymmetric part of the wavefunction satisfies the electron-ion cusp
condition (for instance by using single particle orbitals that respect the cusp condition)
or if a non-divergent pseudopotential is used that the Jastrow should be cuspless at the 
nucleus and this value should be kept at its default of 0.
 \end{itemize}


\begin{table}[h]
\begin{center}
\begin{tabular}{l c c c l }
\hline
\multicolumn{5}{l}{Coefficients element} \\
\hline
\bfseries name & \bfseries datatype & \bfseries values & \bfseries defaults & \bfseries description \\
\hline
id & text & & (required) & Unique identifier \\
type & text & Array & (required) & \\
optimize & text & yes or no & yes & if no, values are fixed in optimizations \\
\hline
\multicolumn{5}{l}{elements}\\ \hline
(None) & & & \\ \hline
\multicolumn{5}{l}{Contents}\\ \hline
 (no name) & real array & & zeros & Jastrow coefficients \\ \hline
\end{tabular}
%\end{tabular*}
\end{center}
\end{table}


\paragraph{Example use cases}
\label{sec:1bjsplineexamples}

Specify a spin-independent function with four parameters. Because rcut  is not 
specified, the default cutoff of the Wigner Seitz cell radius is used; this 
Jastrow must be used with a 3D periodic system such as a bulk solid. The name of 
the particleset holding the ionic positions is "i".
\begin{lstlisting}[language=xml]
<jastrow name="J1" type="One-Body" function="Bspline" print="yes" source="i">
 <correlation elementType="C" cusp="0.0" size="4">
   <coefficients id="C" type="Array"> 0  0  0  0  </coefficients>
 </correlation>
</jastrow>
\end{lstlisting}

Specify a spin-dependent function with seven upspin and seven downspin parameters. 
The cutoff distance is set to 6 atomic units.  Note here that the particleset holding
the ions is labeled as ion0 rather than ``i'' in the other example.  Also in this case
the ion is Lithium with a coulomb potential, so the cusp condition is satisfied by 
setting cusp="d".
\begin{lstlisting}[language=xml]
<jastrow name="J1" type="One-Body" function="Bspline" source="ion0" spin="yes">
  <correlation speciesA="Li" speciesB="u" size="7" rcut="6">
    <coefficients id="eLiu" cusp="3.0" type="Array"> 
    0.0 0.0 0.0 0.0 0.0 0.0 0.0
    </coefficients>
  </correlation>
  <correlation speciesA="C" speciesB="d" size="7" rcut="6">
    <coefficients id="eLid" cusp="3.0" type="Array"> 
    0.0 0.0 0.0 0.0 0.0 0.0 0.0
    </coefficients>
  </correlation>
</jastrow>
\end{lstlisting}



\section{Multideterminant wavefunctions}
\label{sec:multideterminants}
Multiple schemes to generate a multideterminant wavefunction are
possible, from CASSF to full CI or selected CI. The \qmcpack converter can
convert MCSCF multideterminant wavefunctions from
GAMESS\cite{schmidt93} and CIPSI\cite{Caffarel2013} wavefunctions from
Quantum Package\cite{QP} (QP). Full details of how to run a CIPSI
calculation and convert the wavefunction for QMCPACK are given in 
Section~\ref{sec:cipsi}.

The script \ishell{utils/determinants_tools.py} can be used to generate
useful information about the multideterminant wavefunction. This script takes, as a required argument, the path of an h5 file corresponding to the wavefunction. Used without optional arguments, it prints the number of determinants, the number of CSFs, and a histogram of the excitation degree.

\begin{lstlisting}[style=SHELL]
> determinants_tools.py ./tests/molecules/C2_pp/C2.h5
Summary:
excitation degree 0 count: 1
excitation degree 1 count: 6
excitation degree 2 count: 148
excitation degree 3 count: 27
excitation degree 4 count: 20

n_det 202
n_csf 104
\end{lstlisting}

If the \ishell{--verbose} argument is used, the script will print each determinant,
the associated CSF, and the excitation degree relative to the first determinant.
\begin{lstlisting}[style=SHELL]
> determinants_tools.py -v ./tests/molecules/C2_pp/C2.h5 | head
1
alpha  1111000000000000000000000000000000000000000000000000000000
beta   1111000000000000000000000000000000000000000000000000000000
scf    2222000000000000000000000000000000000000000000000000000000
excitation degree  0

2
alpha  1011100000000000000000000000000000000000000000000000000000
beta   1011100000000000000000000000000000000000000000000000000000
scf    2022200000000000000000000000000000000000000000000000000000
excitation degree  2
\end{lstlisting}

\section{Backflow wavefunctions}
\label{sec:backflow}

\section{Finite-difference linear response wave functions}

\label{sec:fdlr}

The finite-difference linear response wave function (FDLR) is an
experimental wave function type, described in detail in
Ref.\cite{blunt_charge-transfer_2017}. In this method, the wave
function is formed as the linear response of some existing trial wave
function in QMCPACK. This derivatives of this linear response are
approximated by a simple finite-difference.

Forming a wave function within the linear response space of an existing ansatz can be very powerful. For example, a configuration interaction singles (CIS) wave function can be formed as a linear combination of the first derivatives of a Slater determinant (with respect to its orbital rotation parameters). Thus, in this sense, CIS is the linear response of Hartree--Fock theory.

Forming a CIS wave function as the linear response of an optimizable Slater determinant is where all testing of this wave function has been performed. In theory the implementation is flexible and it can be used with other trial wave functions in QMCPACK, but this has not been tested. The FDLR trial wave function is experimental.

Mathematically, the FDLR wave function has the form:
\begin{equation}
\Psi_{\textrm{FDLR}} (\mathbf{\mu}, \mathbf{X}) = \Psi (\mathbf{X} + \mathbf{\mu}) - \Psi (\mathbf{X} - \mathbf{\mu}),
\end{equation}
where $\Psi(\mathbf{P})$ is some trial wave function in QMCPACK, and $\mathbf{P}$ are its optimizable parameters. $\mathbf{X}$ are the ``base'' parameters about which the finite difference is performed (for example, an overall orbital rotation). $\mathbf{\mu}$ are the ``finite difference'' parameters, which define the direction of the derivative, and whose magnitude determines the magnitude of the finite-difference. In the limit that the magnitude of $\mathbf{mu}$ goes to $0$, the $\Psi_{\textrm{FDLR}}$ object defined above becomes equivalent to
\begin{equation}
\Psi_{\textrm{FDLR}} (\mathbf{\mu}, \mathbf{X}) = \sum_{pq} \mu_{pq} \: \frac{\partial \Psi_{\textrm{det}} (\mathbf{X}) }{\partial X_{pq}},
\end{equation}
which is the desired linear response wave function that we are approximating. In the case that $\Psi(\mathbf{P})$ is a determinant with orbital rotation parameters $\mathbf{P}$, the above is a CIS wave function with CIS expansion coefficients $\mathbf{\mu}$ and orbital rotation $\mathbf{X}$.

\subsection{Input Specifications}
An FDLR wave function is specified within a \texttt{<fdlr> ... </fdlr>} block.

To fully specify an FDLR wave function as above, we require the initial parameters for both $\mathbf{X}$ and $\mathbf{\mu}$ to be input. This therefore requires two trial wave functions to be provided on input. Each of these is best specified in its own XML file. The names of these two files are provided in an \texttt{<include>} tag via \texttt{<include wfn\_x\_href=`` ... '' wfn\_d\_href=`` ... ''>}. \texttt{wfn\_x\_href} specifies the file that will hold the $\mathbf{X}$ parameters. \texttt{wfn\_d\_href} specifies the file that will hold the $\mathbf{\mu}$ parameters.

Other options inside the \texttt{<include>} tag are \texttt{opt\_x} and \texttt{opt\_d}, which specify whether or not $\mathbf{X}$ and $\mathbf{\mu}$ parameters are optimizable, respectively.

\subsection{Example Use Case}

\begin{lstlisting}
<fdlrwfn name="FDLR">
  <include wfn_x_href="h2.wfn_x.xml" wfn_d_href="h2.wfn_d.xml" opt_x="yes" opt_d="yes"/>
</fdlrwfn>
\end{lstlisting}

with the \texttt{h2.wfn\_x.xml} file containing one of the wave functions and corresponding set of $\mathbf{X}$ parameters, such as:

\begin{lstlisting}
<?xml version="1.0"?>
<wfn_x>
    <determinantset name="LCAOBSet" type="MolecularOrbital" transform="yes" source="ion0">
      <basisset name="LCAOBSet">
        <atomicBasisSet name="Gaussian-G2" angular="cartesian" type="Gaussian" elementType="H" normalized="no">
          <grid type="log" ri="1.e-6" rf="1.e2" npts="1001"/>
          <basisGroup rid="H00" n="0" l="0" type="Gaussian">
            <radfunc exponent="1.923840000000e+01" contraction="3.282799101900e-02"/>
            <radfunc exponent="2.898720000000e+00" contraction="2.312039367510e-01"/>
            <radfunc exponent="6.534720000000e-01" contraction="8.172257764360e-01"/>
          </basisGroup>
          <basisGroup rid="H10" n="1" l="0" type="Gaussian">
            <radfunc exponent="1.630642000000e-01" contraction="1.000000000000e+00"/>
          </basisGroup>
        </atomicBasisSet>
      </basisset>

    <slaterdeterminant optimize="yes">
      <determinant id="det_up" sposet="spo-up">
        <opt_vars size="3">
          0.0 0.0 0.0
        </opt_vars>
      </determinant>

      <determinant id="det_down" sposet="spo-dn">
        <opt_vars size="3">
          0.0 0.0 0.0
        </opt_vars>
      </determinant>
    </slaterdeterminant>

      <sposet basisset="LCAOBSet" name="spo-up" size="4" optimize="yes">
        <occupation mode="ground"/>
        <coefficient size="4" id="updetC">
  2.83630000000000e-01  3.35683000000000e-01  2.83630000000000e-01  3.35683000000000e-01
  1.66206000000000e-01  1.22367400000000e+00 -1.66206000000000e-01 -1.22367400000000e+00
  8.68279000000000e-01 -6.95081000000000e-01  8.68279000000000e-01 -6.95081000000000e-01
 -9.77898000000000e-01  1.19682400000000e+00  9.77898000000000e-01 -1.19682400000000e+00
</coefficient>
      </sposet>
      <sposet basisset="LCAOBSet" name="spo-dn" size="4" optimize="yes">
        <occupation mode="ground"/>
        <coefficient size="4" id="downdetC">
  2.83630000000000e-01  3.35683000000000e-01  2.83630000000000e-01  3.35683000000000e-01
  1.66206000000000e-01  1.22367400000000e+00 -1.66206000000000e-01 -1.22367400000000e+00
  8.68279000000000e-01 -6.95081000000000e-01  8.68279000000000e-01 -6.95081000000000e-01
 -9.77898000000000e-01  1.19682400000000e+00  9.77898000000000e-01 -1.19682400000000e+00
</coefficient>
      </sposet>

    </determinantset>
</wfn_x>
\end{lstlisting}
and similarly for the \texttt{h2.wfn\_d.xml} file, which will hold the initial $\mathbf{\mu}$ parameters.

The above is a wave function file for an optimizable determinant wave function for H$_2$, in a double zeta valence basis set. Thus, the FDLR wave function here would perform CIS on H$_2$ in a double zeta basis set.


\section{Gaussian Product Wavefunction}
\label{sec:ionwf}

The Gaussian Product wavefunction implements Equation~\ref{eq:gauss_prod_wf}
\begin{equation}
\Psi(\vec{R}) = \prod_{i=1}^N \exp\left[ -\frac{(\vec{R}_i-\vec{R}_i^o)^2}{2\sigma_i^2} \right]
\label{eq:gauss_prod_wf},
\end{equation}
where $\vec{R}_i$ is the position of the $i^{\text{th}}$ quantum particle and $\vec{R}_i^o$ is its center. $\sigma_i$ is the width of the Gaussian orbital around center $i$.

This variational wavefunction enhances single-particle density at chosen spatial locations with adjustable strengths. It is useful whenever such localization is physically relevant yet not captured by other parts of the trial wavefunction. For example, in an electron-ion simulation of a solid, the ions are localized around their crystal lattice sites. This single-particle localization is not captured by the ion-ion Jastrow. Therefore, the addition of this localization term will improve the wavefunction. The simplest use case of this wavefunction is perhaps the quantum harmonic oscillator (please see the ``tests/models/sho'' folder for examples).

\subsubsection{Input Specification}

\begin{table}[h]
\begin{center}
\begin{tabular}{l c c c l }
\hline
\multicolumn{5}{l}{Gaussian Product Wavefunction (ionwf)} \\
\hline
\bfseries Name & \bfseries Datatype & \bfseries Values & \bfseries Defaults  & \bfseries Description \\
\hline
Name & Text & ionwf & (Required) & Unique name for this wavefunction \\
Width & Floats & 1.0 -1 & (Required) & Widths of Gaussian orbitals\\ 
Source & Text & ion0 & (Required) & Name of classical particle set\\ 
\hline
\end{tabular}
\end{center}
\end{table}

\FloatBarrier

Additional information:
\begin{itemize}
\item \texttt{width} There must be one width provided for each quantum particle. If a negative width is given, then its corresponding Gaussian orbital is removed. Negative width is useful if one wants to use Gaussian wavefunction for a subset of the quantum particles.
\item \texttt{source} The Gaussian centers must be specified in the form of a classical particle set. This classical particle set is likely the ion positions ``ion0,'' hence the name ``ionwf.'' However, arbitrary centers can be defined using a different particle set. Please refer to the examples in ``tests/models/sho.''
\end{itemize}

\subsection{Example Use Case}
\begin{lstlisting}[style=QMCPXML]
  <qmcsystem>
    <simulationcell>
      <parameter name="bconds">
            n n n
      </parameter>
    </simulationcell>
    <particleset name="e">
      <group name="u" size="1">
        <parameter name="mass">5.0</parameter>
        <attrib name="position" datatype="posArray" condition="0">
          0.0001 -0.0001 0.0002
        </attrib>
      </group>
    </particleset>
    <particleset name="ion0" size="1">
      <group name="H">
        <attrib name="position" datatype="posArray" condition="0">
          0 0 0
        </attrib>
      </group>
    </particleset>
    <wavefunction target="e" id="psi0">
      <ionwf name="iwf" source="ion0" width="0.8165"/>
    </wavefunction>
    <hamiltonian name="h0" type="generic" target="e">
      <extpot type="HarmonicExt" mass="5.0" energy="0.3"/>
      <estimator type="latticedeviation" name="latdev" 
        target="e"    tgroup="u" 
        source="ion0" sgroup="H"/>
    </hamiltonian>
  </qmcsystem>
\end{lstlisting}



\chapter{Hamiltonian and Observables}



\begin{table}[h]
\begin{center}
\begin{tabularx}{\textwidth}{l l l l l l }
\hline
\multicolumn{6}{l}{\texttt{generic} element} \\
\hline
\multicolumn{2}{l}{parent elements:} & \multicolumn{4}{l}{\texttt{parent1 parent2}}\\
\multicolumn{2}{l}{child  elements:} & \multicolumn{4}{l}{\texttt{child1 child2 child3 ...}}\\
\multicolumn{2}{l}{attributes}  & \multicolumn{4}{l}{}\\
   &   \bfseries name     & \bfseries datatype & \bfseries values & \bfseries default   & \bfseries description \\
   &   \texttt{attr1}     &  text              &                  &                     &                       \\
   &   \texttt{attr2}     &  integer           &                  &                     &                       \\
   &   \texttt{attr3}     &  real              &                  &                     &                       \\
   &   \texttt{attr4}     &  boolean           &                  &                     &                       \\
   &   \texttt{attr4}     &  text array        &                  &                     &                       \\
   &   \texttt{attr4}     &  integer array     &                  &                     &                       \\
   &   \texttt{attr4}     &  real array        &                  &                     &                       \\
   &   \texttt{attr4}     &  boolean array     &                  &                     &                       \\
\multicolumn{2}{l}{parameters}  & \multicolumn{4}{l}{}\\
   &   \bfseries name     & \bfseries datatype & \bfseries values & \bfseries default   & \bfseries description \\
   &   \texttt{attr1}     &  text              &                  &                     &                       \\
   &   \texttt{attr2}     &  integer           &                  &                     &                       \\
   &   \texttt{attr3}     &  real              &                  &                     &                       \\
   &   \texttt{attr4}     &  boolean           &                  &                     &                       \\
   &   \texttt{attr4}     &  text array        &                  &                     &                       \\
   &   \texttt{attr4}     &  integer array     &                  &                     &                       \\
   &   \texttt{attr4}     &  real array        &                  &                     &                       \\
   &   \texttt{attr4}     &  boolean array     &                  &                     &                       \\
  \hline
\end{tabularx}
\end{center}
\end{table}




\begin{table}[h]
\begin{center}
\begin{tabularx}{\textwidth}{l l l l l l }
\hline
\multicolumn{6}{l}{\texttt{factory} element} \\
\hline
\multicolumn{2}{l}{parent elements:} & \multicolumn{4}{l}{\texttt{parent1 parent2}}\\
\multicolumn{2}{l}{child  elements:} & \multicolumn{4}{l}{\texttt{child1 child2 child3 ...}}\\
\multicolumn{2}{l}{type   selector:} & \multicolumn{4}{l}{\texttt{some} attribute}\\
\multicolumn{2}{l}{type   options :} & \multicolumn{4}{l}{Selection1}\\
\multicolumn{2}{l}{                } & \multicolumn{4}{l}{Selection2}\\
\multicolumn{2}{l}{                } & \multicolumn{4}{l}{Selection3}\\
\multicolumn{2}{l}{                } & \multicolumn{4}{l}{...}\\
  \hline
\end{tabularx}
\end{center}
\end{table}




\subsection{Input Specification}


%  Hamiltonian element read
%    HamiltonianPool::put
%      reads attributes: id name role target 
%        id/name is passed to QMCHamiltonian
%        role selects the primary hamiltonian
%        target associates to quantum particleset
%    HamiltonianFactory::build
%      reads attributes: type source default

\begin{table}[h]
\begin{center}
\begin{tabularx}{\textwidth}{l l l l l l }
\hline
\multicolumn{6}{l}{\texttt{hamiltonian} element} \\
\hline
\multicolumn{2}{l}{parent elements:} & \multicolumn{4}{l}{\texttt{qmcsystem}}\\
\multicolumn{2}{l}{child  elements:} & \multicolumn{4}{l}{\texttt{pairpot estimator}}\\
\multicolumn{2}{l}{attributes}  & \multicolumn{4}{l}{}\\
   &   \bfseries name     & \bfseries datatype & \bfseries values & \bfseries default   & \bfseries description \\
   &   \texttt{name/id}   &  text              & \textit{anything}& h0                  & A unique id for this Hamiltonian instance.                      \\
   &   \texttt{type   }   &  text              &                  & generic             & \textit{No current function.}                      \\
   &   \texttt{role   }   &  text              & primary/extra    & extra               & Designate as primary Hamiltonian or not.                      \\
   &   \texttt{source }   &  text              & \texttt{particleset.name} & i          & Identify classical particleset.                      \\
   &   \texttt{target }   &  text              & \texttt{particleset.name} & e          & Identify quantum particlset.                      \\
   &   \texttt{default}   &  boolean           & yes/no           & yes                 & Include kinetic energy term implicitly.                      \\
  \hline
\end{tabularx}
\end{center}
\end{table}




\begin{table}[h]
\begin{center}
\begin{tabularx}{\textwidth}{l l l l l l }
\hline
\multicolumn{6}{l}{\texttt{estimator} element} \\
\hline
\multicolumn{2}{l}{parent elements:} & \multicolumn{4}{l}{\texttt{parent1 parent2}}\\
\multicolumn{2}{l}{child  elements:} & \multicolumn{4}{l}{\texttt{child1 child2 child3 ...}}\\
\multicolumn{2}{l}{type   selector:} & \multicolumn{4}{l}{\texttt{type} attribute}\\
\multicolumn{2}{l}{type   options :} & \multicolumn{4}{l}{}\\
  \hline
\end{tabularx}
\end{center}
\end{table}





\chapter{Quantum Monte Carlo Methods}
\label{chap:qmcmethods}

\begin{table}[h]
\begin{center}
\begin{tabularx}{\textwidth}{l l l l l l }
\hline
\multicolumn{6}{l}{\texttt{qmc} factory element} \\
\hline
\multicolumn{2}{l}{parent elements:} & \multicolumn{4}{l}{\texttt{simulation, loop}}\\
\multicolumn{2}{l}{type   selector:} & \multicolumn{4}{l}{\texttt{method} attribute}\\
\multicolumn{2}{l}{type   options: } & vmc           & \multicolumn{3}{l}{Variational Monte Carlo}\\
%\multicolumn{2}{l}{                } & opt           & \multicolumn{3}{l}{}\\
\multicolumn{2}{l}{                } & linear        & \multicolumn{3}{l}{Wavefunction optimization with linear method}\\
%\multicolumn{2}{l}{                } & cslinear      & \multicolumn{3}{l}{}\\
\multicolumn{2}{l}{                } & dmc           & \multicolumn{3}{l}{Diffusion Monte Carlo}\\
\multicolumn{2}{l}{                } & rmc           & \multicolumn{3}{l}{Reptation Monte Carlo}\\
%\multicolumn{2}{l}{                } & ptcl          & \multicolumn{3}{l}{}\\
%\multicolumn{2}{l}{                } & mul           & \multicolumn{3}{l}{}\\
%\multicolumn{2}{l}{                } & warp          & \multicolumn{3}{l}{}\\
\multicolumn{2}{l}{shared attributes:} & \multicolumn{4}{l}{}\\
   &   \bfseries name         & \bfseries datatype & \bfseries values & \bfseries default & \bfseries description \\
   &   \texttt{method}        &  text              &   listed above   & invalid           & QMC driver            \\
   &   \texttt{move}          &  text              &   pbyp, alle     & pbyp              & method used to move electrons \\
   &   \texttt{gpu}           &  text              &   yes, no        & dep.              & use the GPU\\
   &   \texttt{trace}         &  text              &                  & no                & ???                      \\
   &   \texttt{checkpoints}   &  integer           &                  & -1                & checkpoint frequency \\
   &   \texttt{target}        &  text              &                  &                   & ???  \\
   &   \texttt{completed}     &  text              &                  &                   & ???  \\
   &   \texttt{append}        &  text              &   yes, no        & yes               & ???  \\
%   &   \texttt{multiple}      &  text              &   yes, no        & no                & ???  \\
%   &   \texttt{warp}          &  text              &   yes, no        & no                & ???  \\
\hline

\end{tabularx}
\end{center}
\end{table}

Additional information:
\begin{itemize}
\item \texttt{move}. There are two ways implemented to move electrons. The more used method is the particle-by-particle move. In this method, only one electron is moved for acception or rejection. The other method is the all-electron move, namely all the electrons are moved once for testing acception or rejection.

\item \texttt{gpu}. When the executable is compiled with CUDA, the target computing device can be chosen by this switch. With a regular CPU only compilation, this option is not effective.

\end{itemize}

\section{Variational Monte Carlo}
\label{sec:vmc}

\begin{table}[h]
\begin{tabularx}{\textwidth}{l l l l l X }
\hline
\multicolumn{6}{l}{\texttt{vmc} method} \\
\hline
\multicolumn{2}{l}{parameters}  & \multicolumn{4}{l}{}\\
   &   \bfseries name     & \bfseries datatype & \bfseries values & \bfseries default   & \bfseries description \\
   &   \texttt{walkers             } &  integer  & $> 0$   & dep.& Number of walkers per MPI task  \\
   &   \texttt{blocks              } &  integer  & $\ge 0$ & 1   & Number of blocks            \\
   &   \texttt{steps               } &  integer  & $\ge 0$ & 1   & Number of steps per block   \\
   &   \texttt{warmupsteps         } &  integer  & $\ge 0$ & 0   & Number of steps for warming up\\
   &   \texttt{substeps            } &  integer  & $\ge 0$ & 1   & Number of substeps per step \\
   &   \texttt{usedrift            } &  text     & yes, no & yes  & Use the algorithm with drift\\
   &   \texttt{timestep            } &  real     & $> 0$   & 0.1 & Time step for each electron move \\
   &   \texttt{samples             } &  integer  & $\ge 0$ & 0   & Number of walker samples for DMC/optimization\\
   &   \texttt{stepsbetweensamples } &  integer  & $> 0$   & 1   & Period of sample accumulation\\
   &   \texttt{samplesperthread    } &  integer  & $\ge 0$ & 0   & Number of samples per thread  \\
   &   \texttt{storeconfigs        } &  integer  & all values & 0   & Store configurations o  \\
   &   \texttt{blocks\_between\_recompute} &  integer  & $\ge 0$ & dep.  & Wavefunction recompute frequency  \\
  \hline
\end{tabularx}
\end{table}

Additional information:
\begin{itemize}
\item \ixml{walkers}: The number of walkers per MPI task. The initial default number of \ixml{walkers} is one per OpenMP thread or per MPI task if threading is disabled. The number is rounded down to a multiple of the number of threads with a minimum of one per thread to ensure perfect load balancing. One walker per thread is created in the event fewer \ixml{walkers} than threads are requested. 

\item \ixml{blocks}: This parameter is universal for all the QMC
  methods. The MC processes are divided into a number of
  \ixml{blocks}, each containing a number of steps. At the end of each block,
  the statistics accumulated in the block are dumped into files,
  e.g., \ixml{scalar.dat}. Typically, each block should have a sufficient number of steps that the I/O at the end of each block is negligible
  compared with the computational cost. Each block should not take so
  long that monitoring its progress is difficult. There should be a
  sufficient number of \ixml{blocks} to perform statistical analysis.

\item \ixml{warmupsteps}: \ixml{warmupsteps} are used only for
  equilibration. Property measurements are not performed during
  warm-up steps.

\item \ixml{steps}: \ixml{steps} are the number of energy and other property measurements to perform per block.
  
\item \ixml{substeps}: For each substep, an attempt is made to move each of the electrons once only by either particle-by-particle or an
  all-electron move.  Because the local energy is evaluated only at
  each full step and not each substep, \ixml{substeps} are computationally cheaper
  and can be used to reduce the correlation between property measurements
  at a lower cost.
  
\item \ixml{usedrift}: The VMC is implemented in two algorithms with
  or without drift. In the no-drift algorithm, the move of each
  electron is proposed with a Gaussian distribution. The standard
  deviation is chosen as the time step input. In the drift algorithm,
  electrons are moved by Langevin dynamics.

\item \ixml{timestep}: The meaning of time step depends on whether or not
  the drift is used. In general, larger time steps reduce the
  time correlation but might also reduce the acceptance ratio,
  reducing overall statistical efficiency. For VMC, typically the
  acceptance ratio should be close to 50\% for an efficient
  simulation.

\item \ixml{samples}: Seperate from conventional energy and other
  property measurements, samples refers to storing whole electron
  configurations in memory (``walker samples'') as would be needed by subsequent
  wavefunction optimization or DMC steps. \textit{A standard VMC run to
  measure the energy does not need samples to be set.}

\[
\texttt{samples}=
\frac{\texttt{blocks}\cdot\texttt{steps}\cdot\texttt{walkers}}{\texttt{stepsbetweensamples}}\cdot\texttt{number of MPI tasks}
\]

\item \ixml{samplesperthread}: This is an alternative way to set the target amount of samples and can be useful when preparing a stored population for a subsequent DMC calculation.
\[
\texttt{samplesperthread}=
\frac{\texttt{blocks}\cdot\texttt{steps}}{\texttt{stepsbetweensamples}}
\]

\item \ixml{stepsbetweensamples}: Because samples generated by consecutive steps are correlated, having \ixml{stepsbetweensamples} larger than 1 can be used to reduces that correlation. In practice, using larger substeps is cheaper than using \ixml{stepsbetweensamples} to decorrelate samples. 

\item \ixml{storeconfigs}: If \ixml{storeconfigs} is set to a nonzero value, then electron configurations during the VMC run are saved to files.

\item \ixml{blocks_between_recompute}: Recompute the accuracy critical determinant part of the wavefunction
  from scratch: =1 by default when using mixed precision. =0 (no
  recompute) by default when not using mixed precision. Recomputing
  introduces a performance penalty dependent on system size.
\end{itemize}

An example VMC section for a simple VMC run:
\begin{lstlisting}[style=QMCPXML]
  <qmc method="vmc" move="pbyp">
    <estimator name="LocalEnergy" hdf5="no"/>
    <parameter name="walkers">    256 </parameter>
    <parameter name="warmupSteps">  100 </parameter>
    <parameter name="substeps">  5 </parameter>
    <parameter name="blocks">  20 </parameter>
    <parameter name="steps">  100 </parameter>
    <parameter name="timestep">  1.0 </parameter>
    <parameter name="usedrift">   yes </parameter>
  </qmc>
\end{lstlisting}
Here we set 256 \ixml{walkers} per MPI, have a brief initial equilibration of 100 \ixml{steps}, and then have 20 \ixml{blocks} of 100 \ixml{steps} with 5 \ixml{substeps} each.

The following is an example of VMC section storing configurations (walker samples) for optimization.
\begin{lstlisting}[style=QMCPXML]
  <qmc method="vmc" move="pbyp" gpu="yes">
    <estimator name="LocalEnergy" hdf5="no"/>
    <parameter name="walkers">    256 </parameter>
    <parameter name="samples">    2867200 </parameter>
    <parameter name="stepsbetweensamples">    1 </parameter>
    <parameter name="substeps">  5 </parameter>
    <parameter name="warmupSteps">  5 </parameter>
    <parameter name="blocks">  70 </parameter>
    <parameter name="timestep">  1.0 </parameter>
    <parameter name="usedrift">   no </parameter>
  </qmc>
\end{lstlisting}




\section{Wavefunction Optimization}
\label{sec:optimization}


\section{Diffusion Monte Carlo}
\label{sec:dmc}

\section{Reptation Monte Carlo}
\label{sec:rmc}
Like diffusion monte carlo, reptation monte carlo (RMC) is a projector based method, allowing us the ability to sample the fixed-node wavefunciton.  However, by exploiting the path-integral formulation of Schr\"{o}dinger's equation, the RMC algorithm can offer some advantages over traditional DMC, such as sampling both the mixed and pure fixed-node distributions in polynomial time, as well as not having population fluctuations and biases.  The current implementation does not work with T-moves.

There are two adjustable parameters that affect the quality of the RMC projection:  imaginary projection time $\beta$ of the sampling path (commonly called a ``reptile"), and the Trotter time step $\tau$.  $\beta$ must be chosen to be large enough such that $e^{-\beta \hat{H}}|\Psi_T\rangle \approx |\Phi_0\rangle$ for mixed observables, and $e^{-\frac{\beta}{2} \hat{H}}|\Psi_T\rangle \approx |\Phi_0\rangle$ for pure observables.  The reptile is discretized into $M=\beta/\tau$ beads at the cost of an $\mathcal{O}(\tau)$ time-step error for observables arising from the Trotter-Suzuki breakup of the short-time propagator.  

The following table lists some of the more practical 
\begin{table}[h]
\begin{center}
\begin{tabularx}{\textwidth}{l l l l l l }
\hline
\multicolumn{6}{l}{\texttt{vmc} method} \\
\hline
\multicolumn{2}{l}{parameters}  & \multicolumn{4}{l}{}\\
   &   \bfseries name     & \bfseries datatype & \bfseries values & \bfseries default   & \bfseries description \\
   &   \texttt{beta            } &  real  & $> 0$ & dep.   & reptile projection time $\beta$  \\
   &   \texttt{timestep            } &  real     & $> 0$ & 0.1 & Trotter time step $\tau$ for each electron move \\
   &   \texttt{beads           } &  int     & $> 0$ & 1 & Number of reptile beads $M=\beta/\tau$ \\
   &   \texttt{blocks              } &  integer  & $\ge 0$ & 1   & number of blocks            \\
   &   \texttt{steps               } &  integer  & $\ge 0$ & 1   & number of steps per block   \\
   &   \texttt{vmcpresteps        } &  integer  & $\ge 0$ & 0   & propagates reptile using VMC for given number of steps\\
   &   \texttt{warmupsteps         } &  integer  & $\ge 0$ & 0   & number of steps for warming up\\
   &   \texttt{MaxAge              }   & integer & $\ge 0 $   & 0   & force accept for stuck reptile if age exceeds MaxAge. \\
  \hline
\end{tabularx}
\end{center}
\end{table}

Additional information:

Because of the sampling differences between DMC ensembles of walkers and RMC reptiles, the RMC block should contain the following estimator declaration to ensure correct sampling:  \texttt{ <estimator name="RMC" hdf5="no">}. 
  
\begin{itemize}
\item \texttt{beta} or \texttt{beads}?  One can specify one or the other, and from the Trotter time-step, the code will construct an appropriately sized reptile.  If both are given, \texttt{beta} overrides \texttt{beads}.  

\item \textbf{Mixed vs. Pure observables?}  Configurations sampled by the endpoints of the reptile are distributed according to the mixed distribution $f(\mathbf{R})=\Psi_T(\mathbf{R})\Phi_0(\mathbf{R})$.  Any observable that is computable within DMC and is dumped to the scalar.dat file will likewise be found in the scalar.dat file generated by RMC, except there will be an appended \texttt{\_m} to alert the user that the observable was computed on the mixed distribution.  For pure observables, care must be taken in the interpretation.  If the observable is diagonal in the position basis (in layman's terms, if it is entirely computable from a single electron configuration $\mathbf{R}$, like the potential energy), and if the observable does not have an explicit dependence on the trial wavefunction (for example, the local energy has an explicit dependence on the trial wavefunction from the kinetic energy term), then pure estimates will be correctly computed.  These observables will be found in either the scalar.dat file, where they will be appended with a \texttt{\_p} suffix, or in the stat.h5 file.  No mixed estimators will be dumped to the h5 file. 

\item \textbf{Sampling}.  For pure estimators, one should check the traces of both pure and mixed estimates.  Ergodicity is a known problem in RMC.  Because we use the bounce algorithm, it is possible for the reptile to bounce back and forth without changing the electron coordinates of the central beads.  This might not easily show up with mixed estimators, since these are accumulated at constantly regrown ends, but pure estimates are accumulated on these central beads, and so can exhibit strong autocorrelations in pure estimate traces.  

\item \textbf{Propagator}:  Our implementation of RMC uses Moroni's DMC link action (symmetrized), with Umrigar's scaled drift near nodes.  In this regard, the propagator is identical to the one QMCPACK uses in DMC.  

\item \textbf{Sampling}:  We use Ceperley's bounce algorithm.  MaxAge is used in case the reptile gets stuck, at which point the code forces move acceptance, stops accumulating statistics, and requilibrates the reptile.  Very rarely will this be required.  For move proposals, we use particle-by-particle VMC a total of $N_e$ times to generate a new all-electron configuration, at which point the action is computed and the move is either accepted or rejected.  
\end{itemize}






\chapter{Output overview}
\label{chap:output_overview}

%% Detail contents of output files.
QMCPACK writes several output files which report information about the simulation (e.g. the physical properties such as the energy), as well as information about the computational aspects of the simulation, checkpoints, and restarts.
The types of output files generated depend on the details of a calculation. The list below is not meant to be exhaustive, but rather to highlight some salient features of the more common filetypes. Further detail can be found in the description of the estimator one is interested in computing.


\section{The .scalar.dat file}
\label{sec:scalardat_file}
The most important output file is the \texttt{.scalar.dat} file. This file contains the output of block averaged properties of the system such as the local energy and other estimators.
Each line corresponds to an average over $N_{walkers}*N_{steps}$ samples.
By default, the quantities reported in the \texttt{.scalar.dat} file include:

\begin{description}
\item[LocalEnergy] The local energy.
\item[LocalEnergy\_sq] The local energy squared.
\item[LocalPotential] The local potential energy.
\item[Kinetic] The kinetic energy.
\item[ElecElec] The electron-electron potential energy.
\item[IonIon] The ion-ion potential energy.
\item[LocalECP] The energy due to the pseudopotential/effective core potential.
\item[NonLocalECP] The non-local energy due to the pseudopotential/effective core potential.
\item[MPC] The modified periodic coulomb potential energy.
\item[BlockWeight] The number of MC samples in the block.
\item[BlockCPU] The number of seconds to compute the block.
\item[AcceptRatio] The acceptance ratio.
\end{description}

QMCPACK includes a python utility, \texttt{qmca}, which can be used to process these files. Details and examples are given in chapter~\ref{chap:analyzing}.
\section{The .opt.xml file}
\label{sec:optxml_file}
This file is generated after a VMC wave function optimization, and contains the part of the input file which lists the optimized optimized jastrow factors.
Conveniently, this file is already formatted such it can easily be incorporated into a DMC input file.

\section{The .qmc.xml file}
\label{sec:qmc_file}
This file contains information about the computational aspects of the simulation, for example, which parts of the code are being executed when. This file is only generated in an ensemble run in which qmcpack runs multiple input files.

\section{The .dmc.dat file}
\label{sec:dmc_file}
This file contains information similar to the \texttt{.scalar.dat} file, but also includes extra information about the details of a DMC calculation. For example, information about the walker population.

\begin{description}
\item[Index] The block number.
\item[LocalEnergy] The local energy.
\item[Variance] The variance.
\item[Weight] The number of samples in the block.
\item[NumOfWalkers] The number of walkers times the number of steps.
\item[AvgSentWalkers] The average number of walkers sent. During a DMC simulation walkers may be created or destroyed. At every step, QMCPACK will do some load balancing to ensure that the walkers are evenly distributed across nodes.
\item[TrialEnergy] The trial energy. See \ref{sec:dmc} for an explanation of the trial energy.
\item[DiffEff] The diffusion efficiency.
\item[LivingFraction] The fraction of the walker population from the previous step that survived to the current step.
\end{description}


\section{The .bandinfo.dat file}
\label{sec:bandinfo_file}
This file contains information from the trial wavefunction about the band structure of the system,
including the available $k$-points. This can
be helpful in constructing trial wavefunctions.


\section{Checkpoint and restart files}
\label{sec:checkpoint_files}
\subsection{The .cont.xml file}
This file enables continuation of the run.  It is mostly a copy of the input XML file with the series number incremented, and the \texttt{mcwalkerset} element added to read the walkers from a config file.   The \texttt{.cont.xml} file is always created, but other files it depends on are only present if checkpointing is enabled.

\subsection{The .config.h5 file}
This file contains stored walker configurations.

\subsection{The .random.h5 file}
This file contains the state of the random number generator to allow restarts.
(Older versions used an XML file with a suffix of \texttt{.random.xml}).


\chapter{Analysing QMCPACK data}
\label{chap:analysing}

\section{Using the qmca tool}
\label{sec:qma}

\section{Densities and spin-densities}
\label{sec:densities}

\section{Energy densities}
\label{sec:energydensities}



\chapter{Periodic LCAO for solids}
\label{chap:LCAO}

\section{Introduction}

QMCPACK implements linear combination of atomic orbitals (LCAO) and Gaussian
basis sets in periodic boundary conditions. This method uses orders of
magnitude less memory than the real space spline wavefunction. While
the spline scheme enables very fast evaluation of the wavefunction, it may
require too much on-node memory for a large complex cell. The periodic
Gaussian evaluation provides a fallback that will definitely fit in
available memory, but at significantly increased computational
expense. Well designed Gaussian basis sets should be used to accurately
represent the wavefunction, typically
including both diffuse and high angular momentum functions.

The initial implementation is limited to the $\Gamma$-point using
trial wavefunctions generated by PySCF\cite{Sun2018}, but other codes such as
Crystal can be interfaced on request.

%\subsection{Single Particle Orbitals}
%
%In QMC the many-body trial wavefunction is expressed as the product of an antisymmetric part and a correlating Jastrow factor:
% \begin{equation}
%\Psi_T(\vec{R}) = \mathcal{A}(\vec{R}) \exp\left[\sum_i J_i(\vec{R})\right]
%\end{equation}
%
%Where $\Psi_T(\vec{R})$ is the trial wave function, $\vec{R}$ is a space spin coordinates, $J(\vec{R})$ the jastrow function and $\mathcal{A}(\vec{R})$  the antisymmetric wavefunction. $\mathcal{A}(\vec{R})$  is traditionally obtained from methods such as DFT, Hartree Fock, MCSCF or CI expansion.  Many trial-wavefunctions forms have been explored, but the most popular and effective general form remains the Slater Jastrow form
% \begin{equation}
%\Psi_T(\vec{R}) = \exp\left[\sum_i J_i(\vec{R})\right]\sum_k^M C_kD_k^{\uparrow}(\varphi)D_k^{\downarrow}(\varphi)
%\end{equation}
%Where $D_k^{\downarrow}(\varphi)$ is a slater determinant expressed in terms of single particle orbitals (SPO) $\varphi_i=\sum^{N_b}_l C_l ^i \Phi_l$ . The choice of SPO representation is crucial for QMC as the cost of computing $\Phi_l$ scales linearly with the number of basis functions evaluation.  The scaling grows with the system size and the total evaluation of the N SPOs scales as  $ \mathcal{O}(N)^3$ per Monte Carlo step. In the QMCPACK parallelization scheme, SPOs are stored in read only memory replicated on each node or GPU, limiting the size of the systems to the available memory per node. 
%
%In real space QMC methods it is standard to use a real space b-spline scheme or a closely related method, due to the considerable speedup over plane-waves while retaining simple convergence properties. Use of atomic orbitals and Gaussians that include more physics or chemistry results in much more efficient basis sets, but gives up easy convergence properties. 
%
%\subsubsection{B-splines}
%  3D tricubic B-splines provide a basis in which only
%64 elements are nonzero at any given point in space.
%The one-dimensional cubic B-spline is given by,
%\begin{equation}
%f(x) = \sum_{i'=i-1}^{i+2} b^{i'\!,3}(x)\,\,  p_{i'},
%\label{eq:SplineFunc}
%\end{equation}
%where $b^{i}(x)$ are $p_i$ the piecewise cubic polynomial basis functions
%and $i = \text{floor}(\Delta^{-1} x)$ is the index of
%the first grid point $\le x$.  Constructing a tensor product in each Cartesian
%direction, we can represent a 3D orbital as
%\begin{equation}
%  \phi_n(x,y,z) =
%  \!\!\!\!\sum_{i'=i-1}^{i+2} \!\! b_x^{i'\!,3}(x)
%  \!\!\!\!\sum_{j'=j-1}^{j+2} \!\! b_y^{j'\!,3}(y)
%  \!\!\!\!\sum_{k'=k-1}^{k+2} \!\! b_z^{k'\!,3}(z) \,\, p_{i', j', k',n}.
%\label{eq:TricubicValue}
%\end{equation}
%This allows the rapid evaluation of each orbital in constant time.
%Furthermore, this basis is systematically improvable with a single spacing
%parameter, so that accuracy is not compromised and convergence checks are simple.
%
%Trial wavefunctions for materials are commonly produced using plane wave codes such as Quantum Espresso. The conversion to real space b-splines is straightforward. Compared to directly evaluating Fourier series, b-splines are approximately one order of magnitude faster, with the speedup increasing with system size.
%
%\subsubsection{Linear Combination of Atomic Orbitals (LCAO)}

LCAO schemes use physical considerations to construct a highly
efficient basis set compared to plane waves. Typically only a few tens
of basis functions per atom are required compared to thousands of
plane-waves. Many forms of LCAO schemes exist and are being
implemented in QMCPACK. The details of the already implemented methods
will be described in the following section of the manual.

\noindent \textbf{Gaussian Trial Orbitals (GTOs):}
 The Gaussian basis functions follow a radial-angular decomposition
\begin{equation}
     \phi (\mathbf{r} )=R_{l}(r)Y_{lm}(\theta ,\phi )
\end{equation}
where $ Y_{{lm}}(\theta ,\phi )$ is a spherical harmonic, $l$ and $m$
are the angular momentum and its $z$ component, and $r,\theta ,\phi $
are spherical coordinates. In practice they are atom centered and the
$l$ expansion typically includes 1-3 additional channels compared to
the formally occupied states of the atom. e.g. 4-6 for a nickel atom with
occupied $s$, $p$, and $d$ electron shells.

The evaluation of GTOs within PBC differs slightly from evaluating
GTOs in Open Boundary Conditions (OBC).  The orbitals are evaluated at
a distance $r$ in the primitive cell (similar to OBC) and then the
contributions of the periodic images are added by evaluating the
orbital at a distance $r+T$ where T is a translation of the cell
lattice vector. This requires loops over the periodic images until the
contributions are orbitals $\Phi$. In the current implementation, the
number of periodic images is an input parameter named
\textit{PBCimages}, which takes three integers corresponding to the
number of periodic images along the supercell axes (X, Y and Z axes
for a cubic cell). By default these parameters are set to
\textit{PBCimages= 5 5 5} but they \textbf{require manual convergence
  checks}. Convergence checks can be performed by checking the total
energy convergence with respect to \textit{PBCimages}, similar to checks
performed for plane wave cutoff energy and b-spline grids. Use of
diffuse Gaussians may require these parameters to be increased, while
sharply localized Gaussians may permit a decrease. The cost of
evaluating the wavefunction increases sharply as \textit{PBCimages} is
increased. This input parameter will be replaced by a tolerance
factor and numerical screening in future.

\section{Generating and using periodic gaussian trial wavefunctions
  using PySCF}

Similar to any QMC calculation, using periodic GTOs requires the
generation of a periodic trial wavefunction. QMCPACK is currently
interfaced to PySCF which is a
multipurpose electronic structure written mainly in Python with key
numerical functionality implemented via optimized C and C++
libraries\cite{Sun2018}. Such a wavefunction can be generated
following the example
for a 2x1x1 supercell, below. Note that the current
implementation and example covers only the use of the Gamma ($\Gamma$)
point. More general and multiple k-points (real and complex) will be
supported in future releases.

\begin{lstlisting}[caption=Example PySCF input for single k-point calculation for a 2x1x1 Carbon supercell.]
#!/usr/bin/env python

import numpy
from pyscf.pbc import gto, scf, dft
from mpi4pyscf.pbc import df 
from pyscf.pbc.tools.pbc import super_cell

nmp = [2, 1, 1]

cell = gto.Cell()

cell.a = '''
         3.37316115       3.37316115       0.00000000
         0.00000000       3.37316115       3.37316115
         3.37316115       0.00000000       3.37316115'''
cell.atom = '''  
   C        0.00000000       0.00000000       0.00000000
   C        1.686580575      1.686580575      1.686580575 
            '''
cell.basis='bfd-vtz'
cell.ecp = 'bfd'

cell.unit='B'
cell.drop_exponent=0.1

cell.verbose = 5
cell.build()


supcell = super_cell(cell, nmp)
mydf = df.FFTDF(supcell)
mydf.auxbasis = 'weigend'

mf = dft.RKS(supcell)
mf.xc = 'lda'

mf.exxdiv = 'ewald'
mf.with_df = mydf

e_scf=mf.kernel()


print 'e_scf',e_scf

kpts=[]
title="C_Diamond"
from PyscfToQmcpack import savetoqmcpack
savetoqmcpack(supcell,mf,title=title,kpts=kpts)

\end{lstlisting}

Note that the last 4 lines of the file
\begin{lstlisting}
kpts=[]
title="C_Diamond"
from PyscfToQmcpack import savetoqmcpack
savetoqmcpack(supcell,mf,title=title,kpts=kpts)
\end{lstlisting}

contains an empty list of k-points (since this is a gamma point
calculation) and a title. The title variable will be the name of the
HDF5 file where all the data needed by QMCPACK will be stored.  The
function \textit{savetoqmcpack} will be called at the end of the
calculation and will generate the HDF5 similarly to the non-periodic
PySCF calculation in section~\ref{sec:convert4qmc} (convert4qmc). The
function is distributed with QMCPACK and located in the
qmcpack/src/QMCTools directory under the name
\textit{PyscfToQmcpack.py}. In order for the script to work, you need
to specify the path to the file in your PYTHONPATH such as

\begin{lstlisting}
export PYTHONPATH=QMCPACK_PATH/src/QMCTools:$PYTHONPATH
\end{lstlisting}


In order to generate QMCPACK input files, you will need to run \textit{convert4qmc} exactly as specified in section ~\ref{sec:convert4qmc};
\begin{lstlisting}
convert4qmc -pyscf C_Diamond.h5
\end{lstlisting}

This tool can be used with any option described in convert4qmc. Since the HDF5 contains all the information needed for a QMC run, there is no need to specify any other specific tag for periodicity.
Running such a command will generate 3 input files;\\
\begin{lstlisting}[caption=CDiamond.structure.xml. This file contains the geometry of the system.]
<?xml version="1.0"?>
<qmcsystem>
  <simulationcell>
    <parameter name="lattice">
  6.74632230000000e+00  6.74632230000000e+00  0.00000000000000e+00
  0.00000000000000e+00  3.37316115000000e+00  3.37316115000000e+00
  3.37316115000000e+00  0.00000000000000e+00  3.37316115000000e+00
</parameter>
    <parameter name="bconds">p p p</parameter>
    <parameter name="LR_dim_cutoff">15</parameter>
  </simulationcell>
  <particleset name="ion0" size="4">
    <group name="C">
      <parameter name="charge">4</parameter>
      <parameter name="valence">4</parameter>
      <parameter name="atomicnumber">6</parameter>
    </group>
    <attrib name="position" datatype="posArray">
  0.0000000000e+00  0.0000000000e+00  0.0000000000e+00
  1.6865805750e+00  1.6865805750e+00  1.6865805750e+00
  3.3731611500e+00  3.3731611500e+00  0.0000000000e+00
  5.0597417250e+00  5.0597417250e+00  1.6865805750e+00
</attrib>
    <attrib name="ionid" datatype="stringArray">
 C C C C
</attrib>
  </particleset>
  <particleset name="e" random="yes" randomsrc="ion0">
    <group name="u" size="8">
      <parameter name="charge">-1</parameter>
    </group>
    <group name="d" size="8">
      <parameter name="charge">-1</parameter>
    </group>
  </particleset>
</qmcsystem>
  \end{lstlisting}

  As one can see, the cell has been extended to contain 4 atoms in a 2x1x1 Carbon Cell.
\begin{lstlisting}[caption=CDiamond.wfj-Twist0.xml. This file contains the trial wavefunction.]
<?xml version="1.0"?>
<qmcsystem>
  <wavefunction name="psi0" target="e">
    <determinantset type="MolecularOrbital" name="LCAOBSet" source="ion0" transform="yes" twist="0  0  0" href="C_Diamond.h5" PBCimages="5  5  5">
      <slaterdeterminant>
        <determinant id="updet" size="8">
          <occupation mode="ground"/>
          <coefficient size="116" spindataset="0"/>
        </determinant>
        <determinant id="downdet" size="8">
          <occupation mode="ground"/>
          <coefficient size="116" spindataset="0"/>
        </determinant>
      </slaterdeterminant>
    </determinantset>
    <jastrow name="J2" type="Two-Body" function="Bspline" print="yes">
      <correlation size="10" speciesA="u" speciesB="u">
        <coefficients id="uu" type="Array"> 0 0 0 0 0 0 0 0 0 0</coefficients>
      </correlation>
      <correlation size="10" speciesA="u" speciesB="d">
        <coefficients id="ud" type="Array"> 0 0 0 0 0 0 0 0 0 0</coefficients>
      </correlation>
    </jastrow>
    <jastrow name="J1" type="One-Body" function="Bspline" source="ion0" print="yes">
      <correlation size="10" cusp="0" elementType="C">
        <coefficients id="eC" type="Array"> 0 0 0 0 0 0 0 0 0 0</coefficients>
      </correlation>
    </jastrow>
  </wavefunction>
</qmcsystem>
 \end{lstlisting}
This files contains information related to the trial wavefunction. It is identical to the input file from an Open Boundary Conditions calculation to the exception of the following tags:\\
\begin{table}[h]
\begin{center}
\begin{tabularx}{\textwidth}{l l l l l }
\hline
\multicolumn{5}{l}{*.wfj.xml specific tags} \\
\hline
%\multicolumn{2}{l}{Outputfiles}  & \multicolumn{3}{l}{}\\
   &   \bfseries tag     & \bfseries tag type & \bfseries default   & \bfseries description \\
   &   \texttt{twist             } &  3 doubles  & Gamma ( 0 0 0)& coordinate of the twist to compute\\
   &   \texttt{href             } &  string  & default& name of the HDF5 file generated by\\ 
   &                              &          &        &  PySCF and used for convert4qmc\\  
   &   \texttt{PBCimages            } &  3 Integer   & 5 5 5  & Number of periodic images to evaluate the orbitals\\
    \hline
    \end{tabularx}
\end{center}
\end{table}

Other files containing QMC methods (such as optimization, VMC and DMC blocks) will be generated and will behave in a similar fashion regardless of the type of SPO in the trial wavefunction. 




 


\chapter{Selected configuration interaction}
\label{chap:sCI}
A direct path towards improving the accuracy of a QMC calculation is
through a better trial wavefunction.  Although using a multireference
wavefunction can be straightforward in theory, in actual practice
methods such as CASSCF are not always intuitive and often require
being an expert in either the method or the code generating the
wavefunction.  An alternative is to use a selected configuration of
interaction method (selected CI) such as CIPSI (configuration
interaction using a perturbative selection done iteratively). This
provides a direct route to systematically improving the wavefunction.

\section{Theoretical background}

The principle behind selected CI is rather simple and was first published in 1955 by R. K. Nesbet\cite{Nesbet1955}.
The first calculations on atoms were performed by Diner, Malrieu, and Claverie\cite{Diner1967} in 1967 and became computationally viable for larger molecules in 2013 by Caffarel et al.\cite{Caffarel2013}  
%  \textbf{To Paul: I do not recall who added this section (maybe me?)
% but it is word for word the paper by caffarel in ref
% \cite{Caffarel2013}. It needs either to be removed or strongly
% rewritten. see bellow for attempt to simplify in my own words; }\\

As described by Caffarel et al. in Ref.~\cite{Caffarel2013},
multideterminantal expansions of the ground-state wavefunction
$\Psi_T$ are written as a linear combination of Slater determinants
\begin{equation}
\sum_k c_k \sum_q d_{k,q}D_{k,q\uparrow } (r^{\uparrow})D_{k,q\downarrow}(r^{\downarrow})\:, %$\ket{D_i}$
\end{equation}
  where each determinant corresponds to a given occupation by the $N_{\alpha}$ and $N_{\beta}$ electrons of $N=N_{\alpha}+N_{\beta}$ orbitals among a set of M spin-orbitals $\{\phi_1,.,\phi_M\}$ (restricted case). When no symmetries are
considered, the maximum number of such determinants is
\begin{eqnarray}
\label{eqn:Det-Permutations}
\left(
\begin{array}{c} M \hspace{1.5mm} \\ N_{\alpha}  \end{array}
\right).
\left(
\begin{array}{c} M \hspace{1.5mm} \\ N_{\beta}  \end{array}
\right),
\end{eqnarray}
a number that grows factorially with M and N. The best representation of the exact wavefunction in the determinantal basis is the full configuration interaction (FCI) wavefunction written as 
\begin{equation}
\ket{\Psi_0}=\sum_{i} c_{i} \ket{D_i}\:,
\end{equation}
where $c_i$ are the ground-state coefficients obtained by
diagonalizing the matrix, $H_{ij}=\bra{D_i}H\ket{D_j}$, within the
full orthonormalized set $\bra{D_i}\ket{D_j}=\delta_{ij}$ of
determinants $\ket{D_i}$. CIPSI provides a convenient method to build up to this full wavefunction with a single criteria.


A CIPSI wavefunction is built iteratively starting from a reference
wavefunction, usually Hartree-Fock or CASSCF, by adding all single and
double excitations and then iteratively selecting relevant
determinants according to some criteria. Detailed iterative steps can
be found in the reference by Caffarel et al. and references
within\cite{Caffarel2013, Scemama2016,Scemama2018,Garniron2017-2} and
are summarized as follows:

\begin{itemize}
\item Step 1: Define a reference wavefunction:

    \begin{equation}
     \begin{gathered}
       \begin{aligned}
         \ket{\Psi}&=\sum_{i\in D} c_i\ket{i} \,         \,
         &E_{var}&= \frac{\bra{\Psi}\hat{H}\ket{\Psi}}{\bra{\Psi}\ket{\Psi}}. 
       \end{aligned} 
     \end{gathered}
   \end{equation}
   
 \item Step 2: Generate external determinants $\ket{\alpha}$:\\
New determinants are added by generating all single and double excitations from determinants $i \in D$ such as:\\ 
\begin{equation}
\bra{\Psi_0^{(n)}}H\ket{D_{i_c}}\neq 0\:.
\end{equation}

\item Step 3: Evaluate the second-order perturbative contribution to each determinant $\ket{\alpha}$:
\begin{equation}
\Delta E=\frac{\bra{\Psi}\hat{H}\ket{\alpha}\bra{\alpha}\hat{H}\ket{\Psi}}{E_{var}-\bra{\alpha}\hat{H}\ket{\alpha}}\:.
\end{equation}

\item Step 4: Select the determinants with the largest contributions and add them to the Hamiltonian.
\item Step 5: Diagonalize the Hamiltonian within the new added determinants and update the wavefunction and the the value of $E_{var}$.
\item Step 6: Iterate until reaching convergence.
\end{itemize}

Repeating this process leads to a multireference trial wavefunction of high quality that can be used in QMC. 

\begin{equation}
\Psi_T(r)=e^{J(r)}\sum_k c_k \sum_q d_{k,q}D_{k,q\uparrow } (r^{\uparrow})D_{k,q\downarrow}(r^{\downarrow})\:.
\end{equation}
The linear coefficients $c_k$ are then optimized with the presence of the Jastrow function. 

Note the following:
\begin{itemize}
\item When all determinants $\ket{\alpha}$ are selected, the full configuration interaction result is obtained.
\item CIPSI can be seen as a deterministic counterpart of FCIQMC. 
\item In practice, any wavefunction method can be made multireference with CIPSI. For instance, a multireference coupled cluster (MRCC) with CIPSI is implemented in QP.\cite{Garniron2017-1}
\item At any time, with CIPSI selection, $E_{PT_2}=\sum_\alpha \Delta E_\alpha$ estimates the distance to the FCI solution.
\end{itemize}


\subsection{CIPSI wavefunction interface}
\label{sec:cipsi}


\begin{figure}
\begin{center}
\includegraphics[trim = 0mm 0mm 0mm 0mm, clip,width=0.3\columnwidth]{./figures/Reactant.jpg}
\end{center}
\caption{$C_2O_2H_3N$ molecule.}
\protect{\label{fig:C2O2H3N}}
\end{figure}
The CIPSI method
%\cite{XXXrecentCIPSIpaper}
is implemented in the QP code\cite{QP} developed by the Caffarel group. Once the trial wavefunction is generated, QP is able to produce output readable by the QMCPACK converter as described in Section~\ref{sec:convert4qmc}. QP can be installed with multiple plugins for different levels of theory in quantum chemistry. When installing the ``QMC" plugin, QP can save the wavefunction in a format readable by the QMCPACK converter. 

In the following we use the $C_2O_2H_3N$ molecule (Figure~\ref{fig:C2O2H3N}) as an example of how to run a multireference calculation with CIPSI as a trial wavefunction for \qmcpack. The choice of this molecule is motivated by its multireference nature. Although the molecule remains small enough for CCSD(T) calculations with aug-cc-pVTZ basis set, the D1 diagnostic shows a very high value for  $C_2O_2H_3N$, suggesting a multireference character.  Therefore, an accurate reference for the system is not available, and it becomes difficult to trust the quality of a single-determinant wavefunction even when using the DFT-B3LYP exchange and correlation functional. Therefore, in the following, we show an example of how to systematically improve the nodal surface by increasing the number of determinants in the trial wavefunction.

The following steps show how to run from Hartree-Fock to selected CI using QP, convert the wavefunction to a QMCPACK trial wavefunction, and analyze the result.

\begin{itemize}
\item Step 1: Generate the QP input file.\\
QP takes for input an XYZ file containing the geometry of the molecule such as:

\begin{center}
\begin{tabular}{ l c c c }
8\\
C2O2H3N\\
C &       1.067070 &  -0.370798 &   0.020324\\
C &      -1.115770 &  -0.239135 &   0.081860\\
O &      -0.537581 &   1.047619 &  -0.091020\\
N &       0.879629 &   0.882518 &   0.046830\\
H &      -1.525096 &  -0.354103 &   1.092299\\
H &      -1.868807 &  -0.416543 &  -0.683862\\
H &       2.035229 &  -0.841662 &   0.053363\\
O &      -0.025736 &  -1.160835 &  -0.084319   
\end{tabular}
\end{center}

The input file is generated through the following command line:\\

\begin{shade}
qp_create_ezfio_from_xyz C2O2H3N.xyz -b cc-pvtz 
\end{shade}

 
This means that we will be simulating the molecule in all electrons within the cc-pVTZ basis set. Other options are, of course, possible such as using ECPs, different spin multiplicities, etc. For more details, see the QP tutorial at \url{https://github.com/LCPQ/quantum_package/wiki/Tutorial}.

A directory called \texttt{C2O2H3N.ezfio} is created and contains all the relevant data to run the SCF Hartree-Fock calculation. Note that because of the large size of molecular orbitals (MOs) (220), it is preferable to run QP in parallel. QP parallelization is based on a master/slave process that allows a master node to manage the work load between multiple MPI processes through the LibZMQ library. In practice, the run is submitted to one master node and is then submitted to as many nodes as necessary to speed up the calculations. If a slave node dies before the end of its task, the master node will resubmit the workload to another available node. If more nodes are added at any time during the simulation, the master node will use them to reduce the time to solution.
\item Step 2: Run Hartree-Fock.\\
To save the integrals on disk and avoid recomputing them later, edit the \texttt{ezfio} directory with the following command:\\
\begin{shade}
qp_edit C2O2H3N.ezfio 
\end{shade}

This will generate a temporary file showing all the contents of the simulation and opens an editor to allow modification of their values. Look for \texttt{disk\_access\_ao\_integrals} and modify its value from \texttt{None} to \texttt{Write}.

To run a simulation with QP, use the binary \texttt{qp\_run} with the desired level of theory, in this case Hartree-Fock (SCF). \\
\begin{shade}
mpirun -np 1 qp_run SCF C2O2H3N.ezfio &> C2O2H3N-SCF.out 
\end{shade}

If run in serial, the evaluation of the integrals and the Hamiltonian diagonalization would take a substantial amount of computer time. We recommend adding a few more slave nodes to help speed up the calculation.\\

\begin{shade}
mpirun -np 20 qp_run -slave qp_ao_ints C2O2H3N.ezfio &> C2O2H3N-SCF-Slave.out 
\end{shade}
The total Hartree-Fock energy of the system in cc-pVTZ is \textit{$E_{HF}=-283.0992$}Ha.
\item Step 3: Freeze core electrons.\\
To avoid making excitation from the core electrons, freeze the core electrons and do only the excitations from the valence electrons.\\  
\begin{shade}
qp_set_frozen_core.py C2O2H3N.ezfio
\end{shade}
This will will automatically freeze the orbitals from 1 to 5, leaving the remaining orbitals active. \\
\item Step 4: Transform atomic orbitals (AOs) to MOs. \\
This step is the most costly, especially given that its implementation in QP is serial. We recommend completing it in a separate run and on one node.\\
\begin{shade}
qp_run four_idx_transform C2O2H3N.ezfio
\end{shade}

The MO integrals are now saved on disk, and unless the orbitals are changed, they will not be recomputed.
\item Step 5: CIPSI \\
At this point the wavefunction is ready for the selected CI. By default, QP has two convergence criteria: the number of determinants (set by default to 1M) or the value of PT2 (set by default to $1.10^{-4}$Ha). For this molecule, the total number of determinants in the FCI space is $2.07e+88$ determinants. Although this number is completely out of range of what is possible to compute, we will set the limit of determinants in QP to 5M determinants and see whether the nodal surface of the wavefunction is converged enough for the DMC. At this point it is important to remember that the main value of CIPSI compared with other selected CI methods, is that the value of PT2 is evaluated directly at each step, giving a good estimate of the error to the FCI energy. This allows us to conclude that when the E+PT2 energy is converged, the nodal surface is also probably  converged. Similar to the SCF runs, FCI runs have to be submitted in parallel with a master/slave process:\\

\begin{shade}
mpirun -np 1 qp_run fci_zmq C2O2H3N.ezfio &> C2O2H3N-FCI.out 
mpirun -np 199 qp_run -slave selection_davidson_slave C2O2H3N.ezfio\\
&> C2O2H3N-FCI-Slave.out 
\end{shade}

\item Step 6 (optional): Natural orbitals \\
Although this step is optional, it is important to note that using natural orbitals instead of Hartree-Fock orbitals will always improve the quality of the wavefunction and the nodal surface by reducing the number of needed determinants for the same accuracy. When a full convergence to the FCI limit is attainable, this step will not lead to any change in the energy but will only reduce the total number of determinants. However, if a full convergence is not possible, this step can significantly increase the accuracy of the calculation at the same number of determinants. 

\begin{shade}
qp_run save_natorb C2O2H3N.ezfio  
\end{shade}
\hfill
At this point, the orbitals are modified, a new AO$\rightarrow$MO transformation is required, and steps 3 and 4 need to be run again.

\item Step 7: Analyze the CIPSI results.\\
Figure~\ref{fig:CIPSI} shows the evolution of the variational energy and the energy corrected with PT2 as a function of the number of determinants up to 4M determinants. Although it is clear that the raw variational energy is far from being converged, the Energy + PT2 appears converged around 0.4M determinants.

\begin{figure}
\begin{center}
\includegraphics[trim = 2mm 2mm 2mm 2mm, clip,width=0.95\columnwidth]{./figures/CIPSI.jpg}
\end{center}
\caption{Evolution of the variational energy and the Energy + PT2 as a function of the number of determinants for the $C_2O_2H_3N$ molecule.}
\protect{\label{fig:CIPSI}}
\end{figure}





\begin{table}[t]
\centering
\caption{Energies of $C_2O_2H_3N$ using orbitals from Hartree-Fock, natural orbitals, and 0.4M and 4M determinants}
\label{TAB:CIPSI}
\begin{tabular}{l|c|c}
\hline \hline
Method & N\_det & Energy\\
\hline
Hartree-Fock &    1    & -281.6729\\
Natural orbitals & 1 & -281.6735\\
E\_Variational &  438,753 & -282.2951 \\
E\_Variational &  4,068,271   & -282.4882 \\
E+PT2 & 438,753& -282.6809 \\
E+PT2 & 4,068,271 & -282.6805  \\ \hline \hline
\end{tabular}
\end{table}

\item Step 8: Truncate the number of determinants.\\ Although using
  all the 4M determinants from CIPSI always guarantees that all
  important determinants are kept in the wavefunction, practically,
  such a large number of determinants would make any QMC calculation
  prohibitively expensive because the cost of evaluating a determinant in
  DMC grows as $\sqrt[]{N_{det}}$, where $N_{det}$ is the number of
  determinants in the trial wavefunction. To truncate the number of
  determinants, we follow the method described by Scemama
 et al.~\cite{Scemama2018} where the wavefunction is
  truncated by independently removing spin-up and spin-down
  determinants whose contribution to the norm of the wavefunction is
  below a user-defined threshold, $\epsilon$. For this step, we choose
  to truncate the determinants whose coefficients are below,
  $1.10^{-3}$, $1.10^{-4}$, $1.10^{-5}$, and $1.10^{-6}$, translating
  to 239, 44539, 541380, and 908128 determinants, respectively.

To  truncate the determinants in QP, edit the \texttt{ezfio} file as follows:

\begin{shade}
qp_edit C2O2H3N.ezfio  
\end{shade}

Then look for \texttt{ci\_threshold} and modify the value according to the desired threshold. Use the following run to truncate the determinants:

\begin{shade}
qp_run truncate_wf_spin C2O2H3N.ezfio  
\end{shade}

\item Step 9: Save the wavefunction for \qmcpack. \\
The wavefunction in QP is now ready to be converted to \qmcpack format.
Save the wavefunction into \qmcpack format and then convert the wavefunction using the \texttt{convert4qmc} tool.\\

\begin{shade}
qp_run save_for_qmcpack C2O2H3N.ezfio &> C2O2H3N.dump  
convert4qmc -QP C2O2H3N.dump -addCusp -production
\end{shade}

Since we are running all-electron calculations, orbitals in QMC need
to be corrected for the electron-nuclearcusp condition.  This is done
by adding the option \texttt{-addCusp} to \texttt{convert4qmc}, which
adds a tag forcing \qmcpack to run the correction or read them from a
file if pre-computed. When running multiple DMC runs with different
truncation thresholds, only the number of determinants is varied and
the orbitals remain unchanged from one calculation to another and the
cusp correction needs to be run only once.

\item Step 10: Run \qmcpack. \\
At this point, running a multideterminant DMC becomes identical to running a regular DMC with \qmcpack; 
After correcting the orbitals for the cusp, optimize the Jastrow functions and then run the DMC. It is important, however, to note a few items:\\

(1) \qmcpack allows reoptimization of the coefficients of the
determinants during the Jastrow optimization step. Although this has
proven to lower the energy significantly when the number of
determinants is below 10k, a large number of determinants from CIPSI
is often too large to optimize conveniently. Keeping the coefficients
of the determinants from CIPSI unoptimized is an alternative strategy.\\

(2) The large determinant expansion and the Jastrows are both trying
to recover the missing correlations from the system. When optimizing
the Jastrows, we recommend first optimizing J1 and J2 without the J3,
and then with the added J3. Trying to initially optimize J1, J2, and J3
at the same time could lead to numerical instabilities.\\

(3) The parameters of the Jastrow function will need to be optimized
for each truncation scheme and usually cannot be reused efficiently
from one truncation scheme to another.\\

\item Step 11: Analyze the DMC results from \qmcpack. \\
From Table~\ref{TAB:CIPSI-DMC}, we can see that increasing the number
of determinants from 0.5M to almost 1M keeps the energy
within error bars and does not improve the quality of the nodal
surface. We can conclude that the DMC energy is converged at 0.54M
determinants. Note that this number of determinants
also corresponds to the convergence of E+PT2 in CIPSI calculations,
confirming for this case that the convergence of the nodal surface can
follow the convergence of E+PT2 instead of the more difficult
variational energy.


\begin{table}[t]
\centering
\caption{DMC Energies and CIPSI(E+PT2) of $C_2O_2H_3N$ in function of the number of determinants in the trial wavefunction.}
\label{TAB:CIPSI-DMC}
\begin{tabular}{l|c|c}
\hline 
N\_det & DMC& CISPI\\
\hline
1 & -283.0696 (6)&-283.0063\\
239 & -283.0730 (9)&-282.9063\\
44,539 & -283.078 (1)&-282.7339\\
541,380 & -283.088 (1)&-282.6772\\
908,128& -283.089  (1)&-282.6775\end{tabular}
\end{table}

\end{itemize}

As mentioned in previous sections, DMC is variational relative to the
exact nodal surface. A nodal surface is ``better" if it lowers DMC
energy. To assess the quality of the nodal surface from CIPSI, we
compare these DMC results to other single-determinant calculations
from multiple nodal surfaces and theories. Figure~\ref{fig:CIPSI-DMC}
shows the energy of the $C_2O_2H_3N$ molecule as a function of
different single-determinant trial wavefunctions with an
aug-cc-pVTZ basis set, including Hartree-Fock, DFT-PBE, and hybrid
functionals B3LYP and PBE0. The last four points in the plot show the
systematic improvement of the nodal surface as a function of the
number of determinants. 

When the DMC-CIPSI energy is converged with respect to the number of
determinants, its nodal surface is still lower than the best SD-DMC
(B3LYP) by 6(1) mHa. When compared with CCSD(T) with the same basis set,
$E_{CCSD(T)}$ is 4 mHa higher than DMC-CIPSI and 2 mHa lower than
DMC-B3LYP. Although 6 (1) mHa can seem very small, it is important to remember that CCSD(T) cannot correctly describe multireference systems; therefore, it is impossible to assess the
correctness of the single-determinant--DMC result, making CIPSI-DMC calculations an
ideal benchmark tool for multireference systems.

\begin{figure}
\begin{center}
\includegraphics[trim = 2mm 2mm 2mm 2mm, clip,width=0.9
\columnwidth]{./figures/DMC-Multidet.jpg}
\end{center}
\caption{DMC energy of the $C_2O_2H_3N$ molecule as a function of different single-determinant trial wavefunctions with aug-ccp-VTZ basis set using nodal surfaces from Hartree-Fock, DFT-PBE, and DFT with hybrid functionals PBE0 and P3LYP. As indicated, the CIPSI trial wavefunction contains 239, 44539, 514380, and 908128 determinants (D). }
\protect{\label{fig:CIPSI-DMC}}

\end{figure}


\section{Auxiliary-Field Quantum Monte Carlo}
\label{sec:afqmc}
Unlike diffusion Monte Carlo that works in first quantized representation and in configuration space, the auxiliary-field QMC (AFQMC) methods work in second quantized representation and in an auxiliary-field space to represent the wave function or density matrix and to carry out the integrations in evaluating ground-state properties. The idea of AFQMC is to constrain the phase of the overlap of the sampled Slater determinants with a trial wave function. It eliminates the phase instability and restores low-power ($N^3$) computational scaling. AFQMC applies Trotter approximation and Hubbard-Stratonovich (HS) transformation to obtain the propagator, which converts an interacting system into non-interacting systems in fluctuating external auxiliary-fields. The sum over all configurations of auxiliary fields recovers the interaction.

The input file has six basic xml-blocks: \texttt{AFQMCInfo}, \texttt{Hamiltonian}, \texttt{Wavefunction}, \texttt{WalkerSet}, \texttt{Propagator}, and \texttt{execute}.  The first five define input structures required for various types of calculations. The \texttt{execute} block represents actual calculations and takes as input the other blocks. 
Non-execution blocks are parsed first, followed by a second pass where execution blocks are parsed (and executed) in order. All blocks contain a ``name'' argument, which is used to identify the resulting block. For example, in the previous example, multiple Hamiltonian objects with different names can be defined. The one actually used in the calculation is the one passed to ``execute'' as ham.
\begin{lstlisting}[caption=The following is an example input file in AFQMC.]
<?xml version="1.0"?>
<simulation method="afqmc">
  <project id="Carbon" series="0"/>

  <AFQMCInfo name="info0">
    <parameter name="NMO">32</parameter>
    <parameter name="NAEA">16</parameter>
    <parameter name="NAEB">16</parameter>
  </AFQMCInfo>

  <Hamiltonian name="ham0" type="SparseGeneral" info="info0">
    <parameter name="filetype">hdf5</parameter>
    <parameter name="filename">../fcidump.h5</parameter>
  </Hamiltonian>

  <Wavefunction name="wfn0" info="info0">
    <ImpSamp name="impsamp0" type="PureSD" init="ground" >
      <parameter name="filetype">none</parameter>
    </ImpSamp>
  </Wavefunction>

  <WalkerSet name="wset0" type="distributed">
  </WalkerSet>

  <Propagator name="prop0" info="info0">
  </Propagator>

  <execute wset="wset0" ham="ham0" wfn="wfn0" prop="prop0" info="info0">
    <parameter name="timestep">0.005</parameter>
    <parameter name="blocks">10000</parameter>
    <parameter name="nWalkers">20</parameter>
  </execute>

</simulation>
\end{lstlisting}

%The following table lists some of the most practical parameters in the \texttt{execute} block
%The following table lists some of the practical parameters
\begin{table}[h]
\begin{center}
\begin{tabularx}{\textwidth}{l l l l l l }
\hline
\multicolumn{6}{l}{\texttt{afqmc} method} \\
\hline
\multicolumn{6}{l}{parameters in \texttt{AFQMCInfo}} \\
   &   \bfseries name     & \bfseries datatype & \bfseries values & \bfseries default   & \bfseries description \\
%   &   \texttt{name            } &  argument    &         & no & unique name that identifies the xml-block \\
   &   \texttt{NMO             } &  integer     & $\ge 0$ & no & number of molecular orbitals \\
   &   \texttt{NAEA            } &  integer     & $\ge 0$ & no & number of active electrons of spin up \\
   &   \texttt{NAEB            } &  integer     & $\ge 0$ & no & number of active electrons of spin down \\
%   &   \texttt{NCA            } &  integer     & $\ge 0$ & 0 & number of core electrons of spin up \\
%   &   \texttt{NCB            } &  integer     & $\ge 0$ & 0 & number of core electrons of spin down \\
\multicolumn{6}{l}{parameters in \texttt{Hamiltonian}}  \\
%   &   \bfseries name     & \bfseries datatype & \bfseries values & \bfseries default   & \bfseries description \\
%   &   \texttt{name            } &  argument   &               & no   & unique name that identifies the xml-block \\
   &   \texttt{type            } &  argument   & SparseGeneral & no   & type of \texttt{Hamiltonian} \\
   &   \texttt{info            } &  argument   &               &      & name of \texttt{AFQMCInfo} block \\\\
   &   \texttt{filename        } &  string     &               & no   & name of file with the hamiltonian \\
   &   \texttt{filetype        } &  string     & fcidump       & no   & ascii based format of molpro, VASP, pyscf, etc. \\
   &   \texttt{                } &             & hdf5          &      & native hdf5-based format of QMCPACK  \\
%   &   \texttt{cutoff\_1bar            } &  real     &  & 1e-8 & cutoff applied to integrals during reading \\
%   &   \texttt{cutoff\_decomposition   } &  real     &  & 1e-6 & cutoff used to stop the iterative cycle in the generation of the Cholesky decomposition of the 2-electron integrals \\
\multicolumn{6}{l}{parameters in \texttt{Wavefunction}}\\
%   &   \bfseries name     & \bfseries datatype & \bfseries values & \bfseries default   & \bfseries description \\
%   &   \texttt{name            } &  argument   &             & no   & unique name that identifies the xml-block \\
   &   \texttt{info            } &  argument   &             &      & name of \texttt{AFQMCInfo} block \\
   &   \texttt{ImpSamp         } &  xml-block  &             & no   & defines the trial wave-function \\
%   &   \texttt{name            } &  argument   &             & no   & unique name that identifies the xml-block \\
   &   \texttt{type            } &  argument   & PureSD      & no   & single slater determinant trial wave-function \\
   &   \texttt{                } &             & MultiPureSD &      & linear combination of Slater Determinants \\
   &   \texttt{                } &             & GenSD       &      & single determinant w/o rotated 2-e integrals \\
   &   \texttt{filetype        } &  string     & ascii       & no   & ascii data file type \\
   &   \texttt{                } &             & hdf5        &      & hdf5 data file type \\
   &   \texttt{                } &             & none        &      & no input file needed, requires init=``ground'' \\
\multicolumn{6}{l}{parameters in \texttt{WalkerSet}} \\
 %  &   \bfseries name     & \bfseries datatype & \bfseries values & \bfseries default   & \bfseries description \\
%   &   \texttt{name            } &  argument   &             & no   & unique name that identifies the xml-block \\
   &   \texttt{type            } &  argument   & distributed & no   & type of walker container: standard \\
   &   \texttt{                } &             & local       &      & simple but inefficient version. \\
\multicolumn{6}{l}{parameters in \texttt{Propagator}} \\
%   &   \bfseries name     & \bfseries datatype & \bfseries values & \bfseries default   & \bfseries description \\
%   &   \texttt{name            } &  argument   &       & no    & unique name that identifies the xml-block \\
   &   \texttt{type            } &  argument   & afqmc & afqmc & type of propgator \\
   &   \texttt{                } &             &  vmc  &       & not yet functioning \\
   &   \texttt{info            } &  argument   &       &       & name of \texttt{AFQMCInfo} block \\
\multicolumn{6}{l}{parameters in \texttt{execute}} \\
%   &   \bfseries name     & \bfseries datatype & \bfseries values & \bfseries default   & \bfseries description \\
%   &   \texttt{name            } &  argument    &         & no   & unique name that identifies the xml-block \\
   &   \texttt{wset            } &  argument    &         &      &  \\
   &   \texttt{ham             } & argument     &         &      &  \\
   &   \texttt{wfn             } & argument     &         &      &  \\
   &   \texttt{prop            } & argument     &         &      &  \\
   &   \texttt{info            } &  argument    &         &      & name of \texttt{AFQMCInfo} block \\
   &   \texttt{nWalkers        } &  integer     & $\ge 0$ & 5    & initial number of walkers per task group   \\
   &   \texttt{timestep        } &  real        & $> 0$   & 0.01 & time step in 1/a.u. \\
   &   \texttt{blocks          } &  integer     & $\ge 0$ & 100  & number of blocks            \\
   &   \texttt{step            } &  integer     & $> 0$   & 1    & number of steps within a block \\
   &   \texttt{substep         } &  integer     & $> 0$   & 1    & number of substeps within a step \\
  \hline
\end{tabularx}
\end{center}
\end{table}

Additional information:\\

\texttt{AFQMCInfo}: input block that defines basic information about the calculation. It is passed to all other input blocks to propagate the basic information:
\texttt{<AFQMCInfo name="info0">}
\begin{itemize}
%\item \textbf{NMO}. Number of molecular orbitals, i.e., number of states in the single particle basis. 
%\item \textbf{NAEA}. Number of Active Electrons-Alpha, i.e., number of spin-up electrons.
%\item \textbf{NAEB}. Number of Active Electrons-Beta, i.e., number of spin-down electrons.
\item \textbf{NCA}. number of core electrons of spin up (only meaningful if reading ascii FCIDUMPs). Default: 0.
\item \textbf{NCB}. Number of core electrons of spin down (only meaningful if reading ascii FCIDUMPs). Default: 0.\\
\end{itemize}

\texttt{Hamiltonian}: controls the object that reads, stores and manages the hamiltonian. 
  \texttt{<Hamiltonian name="ham0" type="SparseGeneral" info="info0">}
\begin{itemize}
%\item \textbf{filename}. Name of file with the \texttt{Hamiltonian}. This is a required parameter.
%\item \textbf{filetype}. The type of file found in filename. Options are: fcidump, hdf5. fcidump is the ascii based format of molpro, VASP, pyscf, etc.; hdf5 is the native hdf5-based format  of QMCPACK and can also be generated by modified versions of pyscf and VASP. This is a required parameter.
\item \textbf{cutoff\_1bar}. Cutoff applied to integrals during reading. Any term in the hamiltonian smaller than this value is set to zero. (For filetype=``hdf5'', the cutoff is only applied to the 2-electron integrals). Default: 1e-8
\item \textbf{cutoff\_decomposition}. Cutoff used to stop the iterative cycle in the generation of the Cholesky decomposition of the 2-electron integrals. The generation of Cholesky vectors is stopped when the maximum error in the diagonal reaches this value. In case of an eigenvalue factorization, this becomes the cutoff applied to the eigenvalues. Only eigenvalues above this value are kept. Default: 1e-6
\item \textbf{hdf\_write\_file}. If a value is given, an hdf5 restart file for the \texttt{Hamiltonian} is generated. Default: ``'' (no file is generated)
\item \textbf{hdf\_write\_type}. What to store in the restart file. Options: ``integrals'': write directly the 1- and 2-electron integrals; ``factorized'': write the cholesky decomposition of the integrals; ``default'': Default value, write the type found in filetype.
\item \textbf{ascii\_write\_file} If a value is given, an ascii restart file for the \texttt{Hamiltonian} is generated. Default: ``'' (no file is generated)
\item \textbf{nblocks}. This parameter controls the distribution of the 2-electron integrals among processors. In the default behavior (nblocks=1), all nodes contain the entire list of integrals. If nblocks $>$ 1, the of nodes in the calculation will be split in nblocks groups. Each node in a given group contains the same subset of integrals and subsequently operates on this subset during  any further operation that requires the hamiltonian. The maximum number of groups is NMO. Currently only works for filetype=``hdf5'' and the file must contain integrals.  Not yet implemented for input hamiltonians in the form of Cholesky vectors or for ascii input. Coming soon!
    Default: No distribution
\item \textbf{printEig}. If ``yes'', prints additional information during the Cholesky decomposition.
    Default: no
\item \textbf{fix\_2eint}.  If this is set to ``yes'', orbital pairs that are found not to be positive definite are ignored in the generation of the Cholesky factorization. This is necessary if the 2-electron integrals are not positive definite due to round-off errors in their generation.
    Default: no \\
\end{itemize}

\texttt{Wavefunction}: controls the object that manages the trial wave-functions. This block expects a list of xml-blocks defining actual trial-wave functions for various roles. 
\texttt{<Wavefunction name="wfn0" info="info0">}
  \texttt{<ImpSamp name="impsamp0" type="PureSD">}
\begin{itemize}
\item \textbf{init}. Argument in ImpSamp. Selects pre-defined trial wave-functions if desired. 
Value: ground. Only for PureSD. Sets the trial wave-function to a block diagonal form. Diagonal in the occupied-occupied block and zero outside. This option corresponds to b
tate determinant (assuming orbitals ordered by increasing eigenvalue) in the orbital space that defines the hamiltonian. For a \texttt{Hamiltonian} generates based on HF orbitals, this generates a HF trial wave-function. (Only useful situation.)
\item \textbf{filename}. Name of file with wave-function information.
\item \textbf{cutoff}. cutoff applied to the terms in the calculation of the local energy. Only terms in the hamiltonian above this cutoff are included in the evaluation of the energy.
      Default: 1e-6
\item \textbf{nnodes}. Defines the parallelization of the local energy evaluation and the distribution of the \texttt{Hamiltonian} matrix (not to be confused with the list of 2-electron integrals managed by \texttt{Hamiltonian}. These are not the same.) If nnodes $>$ 1, the nodes in the simulation are split into groups of nnodes, each group works collectively in the evaluation of the local energy of their walkers. This helps distribute the effort involved in the evaluation of the local energy among the nodes in the group, but also distributes the memory associated with the wave-function among the nodes in the group.
      Default: No distribution
\item \textbf{hdf\_write\_file}. If provided, a restart file for the wave-function will be generated.
      Default: ``''
\item \textbf{runtype}. Unique to MultiPureSD with orthogonal expansions.
      Values: 0.  Use low memory algorithm. Much slower but has small memory requirements;
      1. Use high-memory algorithm. Faster but with significant memory requirements.
\item \textbf{fast}. Unique to MultiPureSD. If ``yes'', use fast-update algorithm with orthogonal expansions.
      Default: No
\item \textbf{Estimator}. Optional xml-block which defines a trial wave-function used for the calculation of energy averages.  It is used only during the collection of averages and doesn't influence the propagation. Note that this block must be set if an Estimator xml-block in the execute xml-block is specified. This must also be defined in order to calculate and print the energy in the case of hybrid propagation (see below). \\
\end{itemize}

\texttt{WalkerSet}: Controls the object that handles the set of walkers.
\texttt{<WalkerSet name="wset0" type="distributed">}
\begin{itemize}
\item \textbf{pop\_control}. Population control algorithm. Options: ``simple'': Uses a simple branching scheme with a fluctuating population. Walkers with weight above max\_weight are split into multiple walkers of weight reset\_weight. Walkers with weight below min\_weight are killed with probability (weight/min\_weight); ``pair'': Fixed-population branching algorithm, based on QWalk's branching algorithm. Pairs of walkers with weight above/below max\_weight/min\_weight are combined into 2 walkers with weights equal to $(w_1+w_2)/2$. The probability of replicating walker w1 (larger weight) occurs with probability $w_1/(w_1+w_2)$, otherwise walker w2 (lower weight) is replicated; ``comb'': Fixed-population branching algorithm based on the Comb method. Will be available in the next release. Default: ``pair''
\item \textbf{min\_weight}. Weight at which walkers are possibly killed (with probability weight/min\_weight). Default: 0.05
\item \textbf{max\_weight}. Weight at which walkers are replicated. Default: 4.0
\item \textbf{reset\_weight}. Weight to which replicated walkers are reset to. Default: 1.0
\item \textbf{extra\_spaces}. Number of empty spaces kept in the list. Useful to avoid constant reallocation. Default: 10 \\
\end{itemize}

\texttt{Propagator}: Controls the object that manages the propagators.
\texttt{<Propagator name="prop0" info="info0">}
\begin{itemize}
\item \textbf{cutoff}. Cutoff applied to Cholesky vectors. Elements of the Cholesky vectors below this value are set to zero.
    Default: 1e-6
\item \textbf{substractMF}. If ``yes'', apply mean-field substraction based on the ImpSamp trial wave-function. Must set to ``no'' to turn it off.
    Default: yes
\item \textbf{useCholesky}. If ``yes'', use iterative Cholesky decomposition to factorize 2-electron integral matrix. If ``no'', use eigenvalue decomposition (which is very slow and memory intensive).
    Default: yes
\item \textbf{vbias\_bound}. Upper bound applied to the vias potential. Components of the vias potential above this value are truncated there.
    Default: 3.0
\item \textbf{apply\_constrain}. If ``yes'', apply the phaseless constrain to the walker propagation. Currently, setting this to ``no'' produces unknown behavior, since free propagation algorithm has not been tested.
    Default: yes
\item \textbf{hybrid}. If ``yes'', use hybrid propagation algorithm. This propagation scheme doesn't use the local energy during propagation, leading to significant speed ups when its evaluation  cost is high. The local energy of the ImpSamp trial wave-function is never evaluated. To obtain energy estimates in this case, you must define an Estimator xml-block with the \texttt{Wavefunction} block. The local energy of this trial wave-function is evaluated and printed. It is possible to use a previously defined trial wave-function in the Estimator block, just set its ``name'' argument to the name of a previously defined wave-function. In this case, the same object is used for both roles.
    Default: no
\item \textbf{nnodes}. Controls the parallel propagation algorithm. If nnodes $>$ 1, the nodes in the simulation are split into groups of nnodes nodes, each group working collectively to propagate their walkers.
    Default: 1 (Serial algorithm)
\item \textbf{hdf\_read\_file}. If provided, initializes the object from the given file.
    Default: ``''
\item \textbf{hdf\_write\_file}. If provided, generates a restart file.
    Default: ``''
\item \textbf{parallel\_factorization}. If ``yes'', calculates Cholesky decomposition in parallel.
    Default: yes
\item \textbf{parallel\_propagation}. If ``yes'', uses parallel propagation algorithm even with nnodes=1. This is set to ``yes'' if nnodes $>$ 1. If ``no'', uses a serial propagation algorithm. Only possible if nnodes=1 and ncores=1 (in \texttt{execute} block).
    Default: yes
\item \textbf{dense}. If ``yes'', use a dense representation of the factorized hamiltonian. The default behavior uses a sparse representation. Which option is better depends on the sparsity of the problem.
    Default:  no \\
\end{itemize}

\texttt{execute}: Defines an execution region. 
\texttt{<execute wset="wset0" ham="ham0" wfn="wfn0" prop="prop0" info="info0">}
\begin{itemize}
\item \textbf{ortho}. number of steps between orthogonalization.
    Default: 1
\item \textbf{ncores}. Number of nodes in a task group. This number defines the number of cores on a node that share the parallel work associated with a distributed task. This number is used in the \texttt{Wavefunction} and \texttt{Propagator} task groups. The walker sets are shares by the ncores on a given node in the task group.
\item \textbf{checkpoint}. Number of blocks between checkpoint files are generated. If a value smaller than 1 is given, no file is generated. If \textbf{hdf\_write\_file} is not set, a default name is used. \textbf{Default: 0} 
\item \textbf{samplePeriod}. Number of blocks between sample collection. \textbf{Default: 0}
\item \textbf{hdf\_write\_file}. If set (and checkpoint>0), a checkpoint file with this name will be written.
\item \textbf{hdf\_read\_file}. If set, the simulation will be restarted from the given file.\\
\end{itemize}

Within the \texttt{Estimators} xml block has an argument \textbf{name}: the type of estimator we want to measure. Currently only ``basic'' or ``Basic'' or ``standard'' are allowed, , which are synonymous with one another. The basic estimator has the following optional parameters:
\begin{itemize}
\item \textbf{EstimEloc} or \textbf{estim\_eloc} or \textbf{estimeloc}. print the local energy of the ``Estimator'' wave-function. Default: false
\item \textbf{timers}. print timing information. Default: true
\item \textbf{nwalk}. print population control information ({\bf max \# branching events etc?}). Default: false \\
\end{itemize}

Recommended settings for large simulations:
\begin{itemize}
\item generate Cholesky-decomposed integrals from external code
\item use propagation hybrid algorithm: set steps/substeps to adequate values to reduce the number of energy evaluations
\item adjust cutoffs until desired accuracy is reached
\item adjust ncores to an efficient value \\
\end{itemize}

\begin{lstlisting}[caption=Below is an example of sections of an input file for a large calculation.]
...

  <Hamiltonian name="ham0" type="SparseGeneral" info="info0">
    <parameter name="filetype">hdf5</parameter>
    <parameter name="filename">fcidump.h5</parameter>
    <parameter name="cutoff_1bar">1e-6</parameter>
    <parameter name="cutoff_decomposition">1e-5</parameter>
  </Hamiltonian>

  <Wavefunction name="wfn0" info="info0">
    <ImpSamp name="impsamp0" type="PureSD" init="ground" >
      <parameter name="filetype">none</parameter>
      <parameter name="cutoff">1e-6</parameter>
    </ImpSamp>
    <Estimator name="impsamp0" />
  </Wavefunction>

  <WalkerSet name="wset0" type="distributed">
  </WalkerSet>

  <Propagator name="prop0" info="info0">
    <parameter name="hybrid">yes</parameter>
  </Propagator>

  <execute wset="wset0" ham="ham0" wfn="wfn0" prop="prop0" info="info0">
    <parameter name="ncores">8</parameter>
    <parameter name="timestep">0.01</parameter>
    <parameter name="blocks">10000</parameter>
    <parameter name="steps">10</parameter>
    <parameter name="substeps">5</parameter>
    <parameter name="nWalkers">8</parameter>
  </execute>
\end{lstlisting}

\section{Using PySCF to generate integrals for AFQMC}
\label{sec:pyscf}

PySCF (https://github.com/sunqm/pyscf) is a collection of electronic structure programs powered by Python. It is the recommended program for the generation of input for AFQMC calculations in QMCPACK. We refer the reader to the documentaion of the code (http://sunqm.github.io/pyscf/) for a detailed description of the features and the functionality of the code. While the notes below are not meant to replace a detailed study of the PySCF documentation, these notes describe useful knowledge and tips in the use of pyscf for the generation of input for QMCPACK. 

For molecular systems or periodic calculations at the Gamma point, PySCF provides a routine that generates the integral file in Molpro's FCIDUMP format, which contains all the information needed to run AFQMC with a single determinant trial wave-function. Below is an example using this routine to generate the FCIDUMP file for an 8-atom unit cell of carbon in the diamond structure with HF orbitals. For a detailed description, see PySCF's documentation.  
\begin{lstlisting}[caption=Simple example showing how to generate FCIDUMP files with PySCF]
import numpy
from pyscf.tools import fcidump
from pyscf.pbc import gto, scf, tools

cell = gto.Cell()
cell.a = '''
  3.5668 0 0
  0 3.5668 0
  0 0 3.5668'''
cell.atom = '''
  C 0. 0. 0. 
  C 0.8917 0.8917 0.8917
  C 1.7834 1.7834 0. 
  C 2.6751 2.6751 0.8917
  C 1.7834 0. 1.7834
  C 2.6751 0.8917 2.6751
  C 0. 1.7834 1.7834
  C 0.8917 2.6751 2.6751'''
cell.basis = 'gth-szv'
cell.pseudo = 'gth-pade'
cell.gs = [10]*3 # for testing purposes, must be increased for converged results 
cell.verbose = 4
cell.build()

mf = scf.RHF(cell)
ehf = mf.kernel()
print("HF energy (per unit cell) = %.17g" % ehf) 

c = mf.mo_coeff
h1e = reduce(numpy.dot, (c.T, mf.get_hcore(), c))
eri = mf.with_df.ao2mo(c,compact=True)

# nuclear energy + electronic ewald 
e0 = cell.energy_nuc() + tools.pbc.madelung(cell, numpy.zeros(3))*cell.nelectron * -.5
fcidump.from_integrals('fcidump.dat', h1e, eri, c.shape[1],cell.nelectron, ms=0, tol=1e-8, nuc=e0)
\end{lstlisting}

%hcore = mf.get_hcore(kpt=kpt)            # obtain and store core hamiltonian
%with h5py.File(mf.chkfile) as fh5:
%  fh5['scf/hcore'] = hcore
%\end{lstlisting}

%\item {For a calculation with k-points:
%Run a standard pyscf calculation, e.g. a HF or DFT calculation. Make sure you preserve the chkfile and make sure you store the core hamiltonian on the chkfile. An example of how to do this for a single k-point calculation is found below.}
%
%\begin{lstlisting}[caption=The following is an example PySCF input file for calculations with k-points.]
%import numpy
%import h5py
%from mpi4py import MPI
%from pyscf.pbc import gto, scf
%
%cell = gto.Cell()
%cell.a = '''
%3.5668 0 0
%0 3.5668 0
%0 0 3.5668'''
%cell.atom = '''C 0. 0. 0. 
%C 0.8917 0.8917 0.8917
%C 1.7834 1.7834 0. 
%C 2.6751 2.6751 0.8917
%C 1.7834 0. 1.7834
%C 2.6751 0.8917 2.6751
%C 0. 1.7834 1.7834
%C 0.8917 2.6751 2.6751'''
%cell.basis = 'gth-szv'
%cell.pseudo = 'gth-pade'
%cell.gs = [10]*3 # 10 grids on postive x direction, => 21^3 grids in total
%cell.verbose = 4
%cell.build()
%
%nk = [2,2,2]
%kpts = cell.make_kpts(nk) 
%
%mf = scf.KRHF(cell,kpts,exxdiv=0)
%mf.chkfile = "scf.dump"                         # store checkpoint file in scf.dump
%ehf = mf.kernel()
%print("HF energy (per unit cell) = %.17g" % ehf)
%
%hcore = mf.get_hcore(kpts=kpts)          # obtain and store core hamiltonian
%fock = (hcore + mf.get_veff(kpts=kpts))  # store fock matrix (required with orthoAO=True)
%with h5py.File(mf.chkfile) as fh5:
%  fh5['scf/hcore'] = hcore
%  fh5['scf/fock'] = fock
%\end{lstlisting}
%\end{itemize}
%
%Once the checkpoint file has been created, it is now possible to generate the integral file. The recommended approach is:
%
%\begin{lstlisting}[caption=The following is an example input file for calculating the integrals.]
%from mpi4py import MPI
%from qmctools import integrals_from_chkfile
%
%comm = MPI.COMM_WORLD
%rank = comm.Get_rank()
%nproc = comm.Get_size()
%
%integrals_from_chkfile.eri_to_h5("fcidump", rank, nproc, "scf.dump")    
%
%comm.Barrier()
%
%if rank==0:
%    integrals_from_chkfile.combine_eri_h5("fcidump", nproc)
%\end{lstlisting}
%
%It is also possible to generated the Cholesky decomposed integrals in pyscf directly. This is typically faster and more appropriate. 
%
%For calculations with kpoints (those generated with K...), use integrals\_from\_chkfile.eri\_to\_h5\_kpts(...).
%Additional arguments to eri\_to\_h5 are:
%
%\begin{itemize}
%\item \textbf{cholesky}. Determines whether 2-electron integrals or their cholesky factorization is calculated.
%  Default: False
%\item \textbf{orthoAO}. If True, generates the integrals in the orthogonalized AO basis. If False, generates the integrals in the MO basis found on the checkpoint file. For UHF calculations, only orthoAO=True is allowed. If set to False, the fock matrix must be stored in the scf dump file.
%  Default: False
%\item \textbf{LINDEP\_CUTOFF}.  Cutoff used to define linearly dependent basis functions.
%  Default: 1e-9
%\item \textbf{gtol}. Cutoff applied during writing for 2-electron integrals. If cholesky=True, then this is the cutoff used in the iterative cholesky factorization. In this case, the resulting factorized hamiltonian will have a residual error smaller than the requested cutoff.
%  Default: 1e-6
%\item \textbf{wfnName}. Name of the file with the wavefuntion. This is only generated  when orthoAO=True.
%  Default: ``wfn.dat''
%\item \textbf{wfnPHF}. Name of file with initial guess for phfmol code.
%  Default: None (no file is generated)
%\item \textbf{MaxIntgs}. Maximum number of integrals  (or terms in cholesky matrix) per block in the hdf5 file. This controls the size of the hdf5 data sets.
%  Default: 2000000
%\item \textbf{maxvecs}. Represents the maximum number of Cholesky vectors allowed. The actual maximum number of cholesky vectors in the calculation is set to maxvecsnmo. So a value of 10 leads to an actual cutoff in the number of vectors of 10*nmo, where nmo is the total number of molecular orbitals in the calculation. The calculation will stop at the requested number of vectors even if the tolerance is not reached.
%  Default: 20
%\end{itemize}




\chapter{Examples}
\label{chap:examples}

\textbf{WARNING: THESE EXAMPLES ARE NOT CONVERGED! YOU MUST CONVERGE PARAMETERS (SIMULATION CELL SIZE, JASTROW PARAMETER NUMBER/CUTOFF, TWIST NUMBER, DMC TIME STEP, DFT PLANE WAVE CUTOFF, DFT K-POINT MESH, ETC.) FOR REAL CALCUATIONS!}

The following examples should run in serial on a modern workstation in a few hours.

\section{Using Nexus}

\subsection{H$_2$O Molecule with Quantum ESPRESSO Orbitals}
With BFD pseudopotentials (see Section \ref{subsec:BFD}) for O and H in a subdirectory named \texttt{pseudopotentials} (both UPF and FSAtom xml formats, named \texttt{O.BFD.upf}, \texttt{O.BFD.xml}, \texttt{H.BFD.upf}, and \texttt{H.BFD.xml}, respectively) and the following XYZ file (named \texttt{H2O.xyz})
\begin{lstlisting}
3

O 0.0000000000e+00  0.0000000000e+00  0.0000000000e+00
H 0.0000000000e+00 -1.4308249289e+00  1.1078707576e+00
H 0.0000000000e+00  1.4308249289e+00  1.1078707576e+00
\end{lstlisting}
in the working directory, a Python script using Nexus to generate the orbitals using Quantum ESPRESSO, then run QMCPACK to optimize the Jastrow and then do DMC for H$_2$O in a box is:
%\begin{minipage}{\linewidth}
\begin{lstlisting}[caption=Nexus example for H$_2$O using Quantum ESPRESSO orbitals and BFD pseudopotentials]
#! /usr/bin/env python

from project import settings,Job,run_project
from project import Structure,PhysicalSystem
from project import generate_pwscf
from project import generate_pw2qmcpack
from project import generate_qmcpack,vmc,loop,linear,dmc

# General Settings (Directories For I/O, Machine Type, etc.)
settings(
    pseudo_dir      = 'pseudopotentials',
    runs            = 'runs',
    results         = 'results',
    sleep           = 3,
    generate_only   = 0,
    status_only     = 0,
    machine         = 'ws1',
    )

# Executables (Indicate Path If Needed)
pwscf               = 'pw.x'
pw2qmcpack          = 'pw2qmcpack.x'
qmcpack             = 'qmcapp'

# Pseudopotentials
dft_pps             = ['O.BFD.upf','H.BFD.upf']
qmc_pps             = ['O.BFD.xml','H.BFD.xml']

# Job Definitions (MPI Tasks, MP Threading, PBS Queue, Time, etc.)
scf_job             = Job(app=pwscf,serial=True)
p2q_job             = Job(app=pw2qmcpack,serial=True)
opt_job             = Job(threads=4,app=qmcpack,serial=True)
dmc_job             = Job(threads=4,app=qmcpack,serial=True)

# System To Be Simulated
structure = Structure()
structure.read_xyz('H2O.xyz')
structure.bounding_box(
    box             = 'cubic',
    scale           = 1.5
    )
structure.add_kmesh(
    kgrid           = (1,1,1),
    kshift          = (0,0,0)
)
H2O_molecule = PhysicalSystem(
    structure       = structure,
    net_charge      = 0,
    net_spin        = 0,
    O               = 6,
    H               = 1,
    )

sims = []

# DFT SCF To Generate Converged Density
scf = generate_pwscf(
    identifier      = 'scf',
    path            = '.',
    job             = scf_job,
    input_type      = 'scf',
    system          = H2O_molecule,
    pseudos         = dft_pps,
    ecut            = 50,
    ecutrho         = 400,
    conv_thr        = 1.0e-5,
    mixing_beta     = 0.7,
    mixing_mode     = 'local-TF',
    degauss         = 0.001
    )
sims.append(scf)

# Convert DFT Wavefunction Into HDF5 File For QMCPACK
p2q = generate_pw2qmcpack(
    identifier      = 'p2q',
    path            = '.',
    job             = p2q_job,
    write_psir      = False,
    dependencies    = (scf,'orbitals')
    )
sims.append(p2q)

# QMC Optimization Parameters - Coarse Sampling Set
linopt1 = linear(
    energy               = 0.0,
    unreweightedvariance = 1.0,
    reweightedvariance   = 0.0,
    timestep             = 0.4,
    samples              = 8192,
    warmupsteps          = 50,
    blocks               = 64,
    substeps             = 4,
    nonlocalpp           = True,
    usebuffer            = True,
    walkers              = 1,
    minwalkers           = 0.5,
    maxweight            = 1e9,
    usedrift             = True,
    minmethod            = 'quartic',
    beta                 = 0.0,
    exp0                 = -16,
    bigchange            = 15.0,
    alloweddifference    = 1e-4,
    stepsize             = 0.2,
    stabilizerscale      = 1.0,
    nstabilizers         = 3
    )

# QMC Optimization Parameters - Finer Sampling Set
linopt2 = linopt1.copy()
linopt2.samples = 16384

# QMC Optimization
opt = generate_qmcpack(
    identifier      = 'opt',
    path            = '.',
    job             = opt_job,
    input_type      = 'basic',
    system          = H2O_molecule,
    twistnum        = 0,
    bconds          = 'nnn',
    pseudos         = qmc_pps,
    jastrows        = [('J1','bspline',8,4),
                       ('J2','bspline',8,4)],
    calculations    = [loop(max=4,qmc=linopt1),
                       loop(max=4,qmc=linopt2)],
    dependencies    = (p2q,'orbitals')
    )
sims.append(opt)

# QMC VMC/DMC With Optimized Jastrow Parameters
qmc = generate_qmcpack(
    identifier      = 'dmc',
    path            = '.',
    job             = dmc_job,
    input_type      = 'basic',
    system          = H2O_molecule,
    pseudos         = qmc_pps,
    bconds          = 'nnn',
    jastrows        = [],
    calculations    = [
        vmc(
            walkers              = 1,
            samplesperthread     = 64,
            stepsbetweensamples  = 1,
            substeps             = 5,
            warmupsteps          = 100,
            blocks               = 1,
            timestep             = 1.0,
            usedrift             = False
           ),
        dmc(
            minimumtargetwalkers = 128,
            reconfiguration      = 'no',
            warmupsteps          = 100,
            timestep             = 0.005,
            steps                = 10,
            blocks               = 200,
            nonlocalmoves        = True
           )
        ],
    dependencies   = [(p2q,'orbitals'),(opt,'jastrow')]
    )
sims.append(qmc)

run_project(sims)

\end{lstlisting}
%\end{minipage}

\subsection{LiH Crystal with Quantum ESPRESSO Orbitals}
With CASINO-formatted Trail-Needs pseudopotentials (see Section \ref{subsec:CASINO}) for O and H in a subdirectory named \texttt{pseudopotentials} (both UPF and CASINO  formats, named \texttt{O.TN-DF.upf}, \texttt{O.pp.data}, \texttt{H.TN-DF.upf}, and \texttt{H.pp.data}, respectively), a Python script using Nexus to generate the orbitals using Quantum ESPRESSO, then run QMCPACK to optimize the Jastrow and then do DMC for LiH with periodic boundary conditions is:
\begin{lstlisting}[caption=Nexus example for bulk LiH using Quantum ESPRESSO orbitals and CASINO pseudopotentials]
#! /usr/bin/env python

from project import settings,Job,run_project
from project import generate_physical_system
from project import generate_pwscf
from project import generate_pw2qmcpack
from project import generate_qmcpack,vmc,loop,linear,dmc

# General Settings (Directories For I/O, Machine Type, etc.)
settings(
    pseudo_dir      = 'pseudopotentials',
    runs            = 'runs',
    results         = 'results',
    sleep           = 3,
    generate_only   = 1,
    status_only     = 0,
    machine         = 'ws1',
    )

# Executables (Indicate Path If Needed)
pwscf               = 'pw.x'
pw2qmcpack          = 'pw2qmcpack.x'
qmcpack             = 'qmcapp'

# Pseudopotentials
dft_pps             = ['Li.TN-DF.upf','H.TN-DF.upf']
qmc_pps             = ['Li.pp.data','H.pp.data']

# Job Definitions (MPI Tasks, MP Threading, PBS Queue, Time, etc.)
scf_job             = Job(app=pwscf,serial=True)
nscf_job            = Job(app=pwscf,serial=True)
p2q_job             = Job(app=pw2qmcpack,serial=True)
opt_job             = Job(threads=4,app=qmcpack,serial=True)
dmc_job             = Job(threads=4,app=qmcpack,serial=True)

# System To Be Simulated
rocksalt_LiH = generate_physical_system(
    lattice         = 'cubic',
    cell            = 'primitive',
    centering       = 'F',
    atoms           = ('Li','H'),
    basis           = [[0.0,0.0,0.0],
                       [0.5,0.5,0.5]],
    basis_vectors   = 'conventional',
    constants       = 7.1,
    units           = 'B',
    kgrid           = (17,17,17),
    kshift          = (1,1,1),
    net_charge      = 0,
    net_spin        = 0,
    Li              = 1,
    H               = 1,
    )

sims = []

# DFT SCF To Generate Converged Density
scf = generate_pwscf(
    identifier      = 'scf',
    path            = '.',
    job             = scf_job,
    input_type      = 'scf',
    system          = rocksalt_LiH,
    pseudos         = dft_pps,
    ecut            = 450,
    ecutrho         = 1800,
    conv_thr        = 1.0e-10,
    mixing_beta     = 0.7,
    )
sims.append(scf)

# DFT NSCF To Generate Wave Function At Specified K-points
nscf = generate_pwscf(
    identifier      = 'nscf',
    path            = '.',
    job             = nscf_job,
    input_type      = 'nscf',
    system          = rocksalt_LiH,
    pseudos         = dft_pps,
    ecut            = 450,
    ecutrho         = 1800,
    conv_thr        = 1.0e-10,
    mixing_beta     = 0.7,
    kgrid           = (1,1,1),
    kshift          = (0,0,0),
    dependencies    = (scf,'charge-density')
    )
sims.append(nscf)

# Convert DFT Wavefunction Into HDF5 File For QMCPACK
p2q = generate_pw2qmcpack(
    identifier      = 'p2q',
    path            = '.',
    job             = p2q_job,
    write_psir      = False,
    dependencies    = (nscf,'orbitals')
    )
sims.append(p2q)

# QMC Optimization Parameters - Coarse Sampling Set
linopt1 = linear(
    energy               = 0.0,
    unreweightedvariance = 1.0,
    reweightedvariance   = 0.0,
    timestep             = 0.4,
    samples              = 8192,
    warmupsteps          = 50,
    blocks               = 64,
    substeps             = 4,
    nonlocalpp           = True,
    usebuffer            = True,
    walkers              = 1,
    minwalkers           = 0.5,
    maxweight            = 1e9,
    usedrift             = True,
    minmethod            = 'quartic',
    beta                 = 0.0,
    exp0                 = -16,
    bigchange            = 15.0,
    alloweddifference    = 1e-4,
    stepsize             = 0.2,
    stabilizerscale      = 1.0,
    nstabilizers         = 3

# QMC Optimization Parameters - Finer Sampling Set
linopt2 = linopt1.copy()
linopt2.samples = 16384

# QMC Optimization
opt = generate_qmcpack(
    identifier      = 'opt',
    path            = '.',
    job             = opt_job,
    input_type      = 'basic',
    system          = rocksalt_LiH,
    twistnum        = 0,
    bconds          = 'ppp',
    pseudos         = qmc_pps,
    jastrows        = [('J1','bspline',8),
                       ('J2','bspline',8)],
    calculations    = [loop(max=4,qmc=linopt1),
                       loop(max=4,qmc=linopt2)],
    dependencies    = (p2q,'orbitals')
    )
pp = opt.input.get('pseudos')
pp.Li.format='casino'
pp.Li['l-local']='s'
pp.Li.nrule=2
pp.Li.lmax=2
pp.Li.cutoff=2.19
pp.H.format='casino'
pp.H['l-local']='s'
pp.H.nrule=2
pp.H.lmax=2
pp.H.cutoff=0.50
sims.append(opt)

# QMC VMC/DMC With Optimized Jastrow Parameters
qmc = generate_qmcpack(
    identifier      = 'dmc',
    path            = '.',
    job             = dmc_job,
    input_type      = 'basic',
    system          = rocksalt_LiH,
    pseudos         = qmc_pps,
    bconds          = 'ppp',
    jastrows        = [],
    calculations    = [
        vmc(
            walkers              = 1,
            samplesperthread     = 64,
            stepsbetweensamples  = 1,
            substeps             = 5,
            warmupsteps          = 100,
            blocks               = 1,
            timestep             = 1.0,
            usedrift             = False
           ),
        dmc(
            minimumtargetwalkers = 128,
            reconfiguration      = 'no',
            warmupsteps          = 100,
            timestep             = 0.005,
            steps                = 10,
            blocks               = 200,
            nonlocalmoves        = True
           )
        ],
    dependencies   = [(p2q,'orbitals'),(opt,'jastrow')]
    )
pp = qmc.input.get('pseudos')
pp.Li.format='casino'
pp.Li['l-local']='s'
pp.Li.nrule=2
pp.Li.lmax=2
pp.Li.cutoff=2.37
pp.H.format='casino'
pp.H['l-local']='s'
pp
pp.H.lmax=2
pp.H.cutoff=0.50
sims.append(qmc)

run_project(sims)
\end{lstlisting}


% labs: import each as a separate chapter for now

\chapter{Lab 1: Monte Carlo Statistical Analysis}
\label{chap:lab_qmc_statistics}


\section{Topics covered in this Lab} 

This lab focuses on the basics of analyzing data from Monte Carlo (MC)
calculations.  In this lab, participants will use data from
VMC calculations of a simple one-electron system with an analytically soluble
system (the ground state of the hydrogen atom) to understand how to interpret a
MC situation.  Most of these analyses will also carry over to diffusion Monte
Carlo (DMC) simulations.  Topics covered include:
\begin{itemize}
  \item{averaging Monte Carlo variables}
  \item{the statisical error bar of mean values}
  \item{effects of autocorrelation and variance on the error bar}
  \item{the relationship between Monte Carlo timestep and autocorrelation}
  \item{the use of blocking to reduce autocorrelation}
  \item{the significance of the acceptance ratio}
  \item{the significance of the sample size}
  \item{how to determine whether a Monte Carlo run was successful}
  \item{the relationship between wavefunction quality and variance}
  \item{gauging the efficiency of Monte Carlo runs}
  \item{the cost of scaling up to larger system sizes}
\end{itemize}


\hide{
\subsection{How to get the most out of this lab}
Be sure to practice using the various flags in the qmca tool to analyze the
data.  Although some features are not yet implemented, this will get you used
to seeing how the values in the data files produce the averages, which are the
ultimate result of the MC simulations.
}

\section{Lab directories and files}

\begin{shade}
labs/lab1_qmc_statistics/

 atom                              - H atom VMC calculation
    H.s000.scalar.dat                - H atom VMC data 
    H.xml                            - H atom VMC input file

 autocorrelation                   - varying autocorrelation
    H.dat                            - data for gnuplot
    H.plt                            - gnuplot for time step vs. E_L, tau_c
    H.s000.scalar.dat                - H atom VMC data: time step = 10 
    H.s001.scalar.dat                - H atom VMC data: time step =  5 
    H.s002.scalar.dat                - H atom VMC data: time step =  2 
    H.s003.scalar.dat                - H atom VMC data: time step =  1 
    H.s004.scalar.dat                - H atom VMC data: time step =  0.5
    H.s005.scalar.dat                - H atom VMC data: time step =  0.2
    H.s006.scalar.dat                - H atom VMC data: time step =  0.1
    H.s007.scalar.dat                - H atom VMC data: time step =  0.05 
    H.s008.scalar.dat                - H atom VMC data: time step =  0.02
    H.s009.scalar.dat                - H atom VMC data: time step =  0.01
    H.s010.scalar.dat                - H atom VMC data: time step =  0.005
    H.s011.scalar.dat                - H atom VMC data: time step =  0.002
    H.s012.scalar.dat                - H atom VMC data: time step =  0.001
    H.s013.scalar.dat                - H atom VMC data: time step =  0.0005
    H.s014.scalar.dat                - H atom VMC data: time step =  0.0002
    H.s015.scalar.dat                - H atom VMC data: time step =  0.0001
    H.xml                            - H atom VMC input file

 average                            - Python scripts for average/std. dev.
    average.py                         - average five E_L from H atom VMC
    stddev2.py                         - standard deviation using (E_L)^2
    stddev.py                          - standard deviation around the mean

 basis                              - varying basis set for orbitals
    H__exact.s000.scalar.dat           - H atom VMC data using STO basis
    H_STO-2G.s000.scalar.dat           - H atom VMC data using STO-2G basis
    H_STO-3G.s000.scalar.dat           - H atom VMC data using STO-3G basis
    H_STO-6G.s000.scalar.dat           - H atom VMC data using STO-6G basis

 blocking                           - varying block/step ratio
    H.dat                              - data for gnuplot
    H.plt                              - gnuplot for N_block vs. E, tau_c
    H.s000.scalar.dat                  - H atom VMC data 50000:1 blocks:steps
    H.s001.scalar.dat                  - "  "    "    "  25000:2 blocks:steps
    H.s002.scalar.dat                  - "  "    "    "  12500:4 blocks:steps
    H.s003.scalar.dat                  - "  "    "    "  6250: 8 blocks:steps
    H.s004.scalar.dat                  - "  "    "    "  3125:16 blocks:steps
    H.s005.scalar.dat                  - "  "    "    "  2500:20 blocks:steps
    H.s006.scalar.dat                  - "  "    "    "  1250:40 blocks:steps
    H.s007.scalar.dat                  - "  "    "    "  1000:50 blocks:steps
    H.s008.scalar.dat                  - "  "    "    "  500:100 blocks:steps
    H.s009.scalar.dat                  - "  "    "    "  250:200 blocks:steps
    H.s010.scalar.dat                  - "  "    "    "  125:400 blocks:steps
    H.s011.scalar.dat                  - "  "    "    "  100:500 blocks:steps
    H.s012.scalar.dat                  - "  "    "    "  50:1000 blocks:steps
    H.s013.scalar.dat                  - "  "    "    "  40:1250 blocks:steps
    H.s014.scalar.dat                  - "  "    "    "  20:2500 blocks:steps
    H.s015.scalar.dat                  - "  "    "    "  10:5000 blocks:steps
    H.xml                             - H atom VMC input file

 blocks                             -  varying total number of blocks
    H.dat                             - data for gnuplot
    H.plt                             - gnuplot for N_block vs. E
    H.s000.scalar.dat                 - H atom VMC data    500 blocks
    H.s001.scalar.dat                 - "  "    "    "    2000 blocks
    H.s002.scalar.dat                 - "  "    "    "    8000 blocks
    H.s003.scalar.dat                 - "  "    "    "   32000 blocks
    H.s004.scalar.dat                 - "  "    "    "  128000 blocks
    H.xml                             - H atom VMC input file 

 dimer                          - comparing no and simple Jastrow factor
    H2_STO___no_jastrow.s000.scalar.dat - H dimer VMC data without Jastrow
    H2_STO_with_jastrow.s000.scalar.dat - H dimer VMC data with Jastrow

 docs                               - documentation
    Lab_1_MC_Analysis.pdf             - this document
    Lab_1_Slides.pdf                  - slides presented in the lab

 nodes                              - varying number of computing nodes
    H.dat                             - data for gnuplot
    H.plt                             - gnuplot for N_node vs. E
    H.s000.scalar.dat                 - H atom VMC data with  32 nodes
    H.s001.scalar.dat                 - H atom VMC data with 128 nodes
    H.s002.scalar.dat                 - H atom VMC data with 512 nodes

 problematic                        - problematic VMC run
    H.s000.scalar.dat                 - H atom VMC data with a problem

 size                                - scaling with number of particles
    01________H.s000.scalar.dat       - H atom VMC data
    02_______H2.s000.scalar.dat       - H dimer "   "
    06________C.s000.scalar.dat       - C atom  "   "
    10______CH4.s000.scalar.dat       - methane "   "
    12_______C2.s000.scalar.dat       - C dimer "   "
    16_____C2H4.s000.scalar.dat       - ethene  " 
    18___CH4CH4.s000.scalar.dat       - methane dimer VMC data
    32_C2H4C2H4.s000.scalar.dat       - ethene dimer   "   "
    nelectron_tcpu.dat                - data for gnuplot
    Nelectron_tCPU.plt                - gnuplot for N_elec vs. t_CPU
\end{shade}


\section{Atomic units} 

QMCPACK operates in Hartree atomic units to reduce the
number of factors in the Schr\"odinger equation.  Thus, the unit of length is
the bohr (5.291772 $\times 10^{-11}$ m = 0.529177 \AA); the unit of energy is
the hartree (4.359744 $\times 10^{-18}$ J = 27.211385 eV).  The energy of the
ground state of the hydrogen atom in these units is -0.5 hartrees.


%\section{Monte Carlo data analysis:\newline average, error bars, variance}

\section{Reviewing statistics}
\label{sec:review}

We will practice taking the average (mean) and standard deviation of some Monte
Carlo data by hand to review the basic definitions.

Enter Python's command line by typing \textbf{python [Enter]}.
You will see a prompt ``\textgreater\textgreater\textgreater''.

The mean of a data set is given by:
\begin{align}
  \overline{x} = \frac{1}{N}\sum_{i=1}^{N} x_i
\end{align}

To calculate the average of five local energies from a MC calculation of the
ground state of an electron in the hydrogen atom, input (truncate at the
thousandths place if you cannot copy and paste; script versions are also
available in the \texttt{average} directory): 

\texttt{
(
(-0.45298911858) + 
(-0.45481953564) + 
(-0.48066105923) + 
(-0.47316713469) + 
(-0.46204733302)
)/5.
} 

Then, press \textbf{[Enter]} to get:

\begin{shade}
>>> ((-0.45298911858) + (-0.45481953564) + (-0.48066105923) + 
(-0.47316713469) + (-0.4620473302))/5.  
-0.46473683566800006
\end{shade}

To understand the significance of the mean, we also need the standard deviation
around the mean of the data (also called the error bar), given by:

\begin{align}
  \sigma = \sqrt{\frac{1}{N(N-1)}\sum_{i=1}^{N} ({x_i} - \overline{x})^2}
\end{align}

To calculate the standard deviation around the mean (-0.464736835668) of these
five data points, put in: 

\texttt{
( (1./(5.*(5.-1.))) * ( 
(-0.45298911858-(-0.464736835668))**2 + \\
(-0.45481953564-(-0.464736835668))**2 + 
(-0.48066105923-(-0.464736835668))**2 + 
(-0.47316713469-(-0.464736835668))**2 + 
(-0.46204733302-(-0.464736835668))**2 ) 
)**0.5
} 

Then, press \textbf{[Enter]} to get:

\begin{shade}
>>> ( (1./(5.*(5.-1.))) * ( (-0.45298911858-(-0.464736835668))**2 +
(-0.45481953564-(-0.464736835668))**2 + (-0.48066105923-(-0.464736835668))**2 + 
(-0.47316713469-(-0.464736835668))**2 + (-0.46204733302-(-0.464736835668))**2 
) )**0.5
0.0053303187464332066
\end{shade}

Thus, we might report this data as having a value -0.465 +/- 0.005 hartrees.
This calculation of the standard deviation assumes that the average for this
data is fixed, but we may continually add Monte Carlo samples to the data so it
is better to use an estimate of the error bar that does not rely on the overall
average.  Such an estimate is given by:

\begin{align}
  \tilde{\sigma} = \sqrt{\frac{1}{N-1}\sum_{i=1}^{N} \left[{(x^2)}_i - ({x_i})^2\right]}
\end{align}

To calculate the standard deviation with this formula, input the following,
which includes the square of the local energy calculated with each
corresponding local energy:

\texttt{
( (1./(5.-1.)) * ( 
(0.60984565298-(-0.45298911858)**2) + \\
(0.61641291630-(-0.45481953564)**2) + 
(1.35860151160-(-0.48066105923)**2) + \\
(0.78720769003-(-0.47316713469)**2) + 
(0.56393677687-(-0.46204733302)**2) ) 
)**0.5
}

and press \textbf{[Enter]} to get:

\begin{shade}
>>> ((1./(5.-1.))*((0.60984565298-(-0.45298911858)**2)+ 
(0.61641291630-(-0.45481953564)**2)+(1.35860151160-(-0.48066105923)**2)+ 
(0.78720769003-(-0.47316713469)**2)+(0.56393677687-(-0.46204733302)**2))
)**0.5
0.84491636672906634
\end{shade}

This much larger standard deviation, acknowledging that the mean of this small
data set is not the average in the limit of infinite sampling more accurately,
reports the value of the local energy as -0.5 +/- 0.8 hartrees.

Type \textbf{quit()} and press \textbf{[Enter]} to exit the Python command line.

\section{Inspecting Monte Carlo data}
\label{sec:inspect_data} 

QMCPACK outputs data from MC calculations into files ending in scalar.dat.
Several quantities are calculated and written for each block of Monte Carlo
steps in successive columns to the right of the step index. 

Change directories to \texttt{atom}, and open the file ending in
scalar.dat with a text editor (e.g., \textbf{vi *.scalar.dat} or \textbf{emacs
*.scalar.dat}.  If possible, adjust the terminal so that lines do not wrap.
The data will begin as follows (broken into three groups to fit on this page):

\begin{shade}
#   index    LocalEnergy         LocalEnergy_sq      LocalPotential     ...
         0   -4.5298911858e-01    6.0984565298e-01   -1.1708693521e+00    
         1   -4.5481953564e-01    6.1641291630e-01   -1.1863425644e+00    
         2   -4.8066105923e-01    1.3586015116e+00   -1.1766446209e+00    
         3   -4.7316713469e-01    7.8720769003e-01   -1.1799481122e+00    
         4   -4.6204733302e-01    5.6393677687e-01   -1.1619244081e+00    
         5   -4.4313854290e-01    6.0831516179e-01   -1.2064503041e+00    
         6   -4.5064926960e-01    5.9891422196e-01   -1.1521370176e+00    
         7   -4.5687452611e-01    5.8139614676e-01   -1.1423627617e+00    
         8   -4.5018503739e-01    8.4147849706e-01   -1.1842075439e+00    
         9   -4.3862013841e-01    5.5477715836e-01   -1.2080979177e+00    
\end{shade}

The first line begins with a \#, indicating that this line does not contain MC
data but rather the labels of the columns.  After a blank line, the remaining
lines consist of the MC data.  The first column, labeled index, is an integer
indicating which block of MC data is on that line.  The second column contains
the quantity usually of greatest interest from the simulation, the local
energy.  Since this simulation did not use the exact ground state wave
function, it does not produce -0.5 hartrees as the local energy although the
value lies within about 10\%.  The value of the local energy fluctuates from
block to block and the closer the trial wave function is to the ground state,
the smaller these fluctuations will be.  The next column contains an important
ingredient in estimating the error in the MC average--the square of the local
energy--found by evaluating the square of the Hamiltonian.  

\begin{shade} 
...   Kinetic             Coulomb             BlockWeight        ... 
       7.1788023352e-01   -1.1708693521e+00    1.2800000000e+04   
       7.3152302871e-01   -1.1863425644e+00    1.2800000000e+04   
       6.9598356165e-01   -1.1766446209e+00    1.2800000000e+04   
       7.0678097751e-01   -1.1799481122e+00    1.2800000000e+04   
       6.9987707508e-01   -1.1619244081e+00    1.2800000000e+04   
       7.6331176120e-01   -1.2064503041e+00    1.2800000000e+04   
       7.0148774798e-01   -1.1521370176e+00    1.2800000000e+04   
       6.8548823555e-01   -1.1423627617e+00    1.2800000000e+04   
       7.3402250655e-01   -1.1842075439e+00    1.2800000000e+04   
       7.6947777925e-01   -1.2080979177e+00    1.2800000000e+04   
\end{shade}

The fourth column from the left consists of the values of the local potential
energy.  In this simulation, it is identical to the Coulomb potential
(contained in the sixth column) because the one electron in the simulation has
only the potential energy coming from its interaction with the nucleus.  In
many-electron simulations, the local potential energy contains contributions
from the electron-electron Coulomb interactions and the nuclear potential or
pseudopotential.  The fifth column contains the local kinetic energy value for
each MC block, obtained from the Laplacian of the wave function.  The sixth
column shows the local Coulomb interaction energy.  The seventh column displays
the weight each line of data has in the average (the weights are identical in
this simulation).   

\begin{shade} 
...    BlockCPU            AcceptRatio         
       6.0178991748e-03    9.8515625000e-01
       5.8323097461e-03    9.8562500000e-01
       5.8213412744e-03    9.8531250000e-01
       5.8330412549e-03    9.8828125000e-01
       5.8108362256e-03    9.8625000000e-01
       5.8254170264e-03    9.8625000000e-01
       5.8314813086e-03    9.8679687500e-01
       5.8258469971e-03    9.8726562500e-01
       5.8158433545e-03    9.8468750000e-01
       5.7959401123e-03    9.8539062500e-01
\end{shade}

The eighth column shows the CPU time (in seconds) to calculate the data in that
line.  The ninth column from the left contains the acceptance ratio (1 being
full acceptance) for Monte Carlo steps in that line's data.  Other than the
block weight, all quantities vary from line to line.

Exit the text editor (\textbf{[Esc] :q! [Enter]} in vi, \textbf{[Ctrl]-x [Ctrl]-c} in
emacs).

\section{Averaging quantities in the MC data}
\label{sec:averaging} 

QMCPACK includes the qmca Python tool to average quantities in the scalar.dat file (and
also the dmc.dat file of DMC simulations).  Without any flags, qmca will output
the average of each column with a quantity in the scalar.dat file as follows. 

Execute qmca by \textbf{qmca *.scalar.dat}, which for this data outputs:

\begin{shade}

H  series 0 
LocalEnergy           =          -0.45446 +/-          0.00057
Variance              =             0.529 +/-            0.018 
Kinetic               =            0.7366 +/-           0.0020
LocalPotential        =           -1.1910 +/-           0.0016
Coulomb               =           -1.1910 +/-           0.0016 
LocalEnergy_sq        =             0.736 +/-            0.018
BlockWeight           =    12800.00000000 +/-       0.00000000
BlockCPU              =        0.00582002 +/-       0.00000067 
AcceptRatio           =          0.985508 +/-         0.000048
Efficiency            =        0.00000000 +/-       0.00000000 
\end{shade}

After one blank, qmca prints the title of the subsequent data, gleaned from the
data file name.  In this case, H.s000.scalar.dat became ``H  series 0''.
Everything before the first ``.s'' will be interpreted as the title, and the
number between ``.s'' and the next ``.'' will be interpreted as the series
number. 

The first column under the title is the name of each quantity qmca averaged.
The column to the right of the equal signs contains the average for the
quantity of that line, and the column to the right of the plus-slash-minus is
the statistical error bar on the quantity.  All quantities calculated from MC
simulations have and must be reported with a statistical error bar!

Two new quantities not present in the scalar.dat file are computed by qmca from
the data--variance and efficiency.  We will look at these later in this lab. 

To view only one value, \textbf{qmca} takes the \textbf{-q (quantity)} flag.
For example, the output of \textbf{qmca -q LocalEnergy *.scalar.dat} in this
directory produces a single line of output:

\begin{shade} 
H  series 0  LocalEnergy = -0.454460 +/- 0.000568 
\end{shade}

Type \textbf{qmca --help} to see the list of all quantities and their
abbreviations.

\section{Evaluating MC simulation quality}

There are several aspects of a MC simulation to consider in deciding how well
it went.  Besides the deviation of the average from an expected value (if there
is one), the stability of the simulation in its sampling, the autocorrelation
between MC steps, the value of the acceptance ratio (accepted steps over total
proposed steps), and the variance in the local energy all indicate the quality
of a MC simulation.  We will look at these one by one.

\subsection{Tracing MC quantities}

Visualizing the evolution of MC quantities over the course of the simulation by
a \textit{trace} offers a quick picture of whether the random walk had expected
behavior.  qmca plots traces with the -t flag.

Type \textbf{qmca -q e -t H.s000.scalar.dat}, which produces a graph of the
trace of the local energy:

\FloatBarrier
\begin{figure}[ht!]
\begin{center}
\includegraphics[trim = 0mm 0mm 0mm 0mm, clip,width=0.75\columnwidth]{figures/lab_qmc_statistics_tracing1.png}
\end{center}
\end{figure}
\FloatBarrier

%\includegraphics[scale=0.5]{E_L_H_STO-2G.png}

The solid black line connects the values of the local energy at each MC block
(labeled ``samples'').  The average value is marked with a horizontal, solid
red line.  One standard deviation above and below the average are marked with
horizontal, dashed red lines.  

The trace of this run is largely centered around the average with no
large-scale oscillations or major shifts, indicating a good quality MC run. 

Try tracing the kinetic and potential energies, seeing that their behavior is
comparable to the total local energy.

Change to directory \texttt{problematic} and type \textbf{qmca -q e -t
H.s000.scalar.dat} to produce this graph:

\FloatBarrier
\begin{figure}[ht!]
\begin{center}
\includegraphics[trim = 0mm 0mm 0mm 0mm, clip,width=0.75\columnwidth]{figures/lab_qmc_statistics_tracing2.png}
\end{center}
\end{figure}
\FloatBarrier

%\includegraphics[scale=0.5]{E_L_H_B-splines.png}

Here, the local energy samples cluster around the expected -0.5 hartrees for the
first 150 samples or so and then begin to oscillate more wildly and increase
erratically toward 0, indicating a poor quality MC run.

Again, trace the kinetic and potential energies in this run and see how their
behavior compares to the total local energy.

\subsection{Blocking away autocorrelation}

\textit{Autocorrelation} occurs when a given MC step biases subsequent MC
steps, leading to samples that are not statistically independent.  We must take
this autocorrelation into account in order to obtain accurate statistics.  qmca
outputs autocorrelation when given the {-}{-}sac flag.

Change to directory \texttt{autocorrelation} and type \textbf{qmca -q e
{-}{-}sac H.s000.scalar.dat}.  

\begin{shade} 
H  series 0  LocalEnergy = -0.454982 +/- 0.000430    1.0 
\end{shade}

The value after the error bar on the quantity is the autocorrelation (1.0 in
this case).

Proposing too small a step in configuration space, the MC \textit{time step},
can lead to autocorrelation since the new samples will be in the neighborhood
of previous samples.  Type \textbf{grep timestep H.xml} to see the varying time
step values in this QMCPACK input file (H.xml):

\begin{shade} 
<parameter name="timestep">10</parameter>
<parameter name="timestep">5</parameter> 
<parameter name="timestep">2</parameter> 
<parameter name="timestep">1</parameter>
<parameter name="timestep">0.5</parameter> 
<parameter name="timestep">0.2</parameter> 
<parameter name="timestep">0.1</parameter>
<parameter name="timestep">0.05</parameter> 
<parameter name="timestep">0.02</parameter> 
<parameter name="timestep">0.01</parameter>
<parameter name="timestep">0.005</parameter> 
<parameter name="timestep">0.002</parameter> 
<parameter name="timestep">0.001</parameter>
<parameter name="timestep">0.0005</parameter> 
<parameter name="timestep">0.0002</parameter> 
<parameter name="timestep">0.0001</parameter> 
\end{shade}

Generally, as the time step decreases, the autocorrelation will increase
(caveat: very large time steps will also have increasing autocorrelation). To
see this, type \textbf{qmca -q e {-}{-}sac *.scalar.dat} to see the energies
and autocorrelation times, then plot with gnuplot by inputting \textbf{gnuplot
H.plt}:

\FloatBarrier
\begin{figure}[ht!]
\begin{center}
\includegraphics[trim = 0mm 0mm 0mm 0mm, clip,width=0.75\columnwidth]{figures/lab_qmc_statistics_blocking1.png}
\end{center}
\end{figure}
\FloatBarrier

%\includegraphics[scale=1.0]{timestep_vs_autocorrelation_energy_H_STO-2G.png}

The error bar also increases with the autocorrelation.  

Press \textbf{q [Enter]} to quit gnuplot.

To get around the bias of autocorrelation, we group the MC steps into blocks,
take the average of the data in the steps of each block, and then finally
average the averages in all the blocks.  QMCPACK outputs the block averages as
each line in the scalar.dat file.  (For DMC simulations, in addition to the
scalar.dat, QMCPACK outputs the quantities at each step to the dmc.dat file,
which permits reblocking the data differently from the specification in the
input file.) 

Change directories to \texttt{blocking}.  Here we look at the time step of the
last data set in the \texttt{autocorrelation} directory.  Verify this by typing
\textbf{grep timestep H.xml} to see that all values are set to 0.001.  Now to
see how we will vary the blocking, type \textbf{grep -A1 blocks H.xml}.  The
parameter ``steps'' indicates the number of steps per block, and the parameter
``blocks'' gives the number of blocks.  For this comparison, the total number
of MC steps (equal to the product of ``steps'' and ``blocks'') is fixed at
50000.  Now check the effect of blocking on autocorrelation--type \textbf{qmca
-q e {-}{-}sac *scalar.dat} to see the data and \textbf{gnuplot H.plt} to
visualize the data:

%\begin{shaded} 
%\begin{verbatim} 
%H  series 0  LocalEnergy = -0.454433 +/- 0.003970   189.2 
%H  series 1  LocalEnergy = -0.453352 +/- 0.004159   104.5 
%H  series 2  LocalEnergy = -0.449211 +/- 0.006544   114.1 
%H  series 3  LocalEnergy = -0.449491 +/- 0.014770   381.1 
%H  series 4  LocalEnergy = -0.446602 +/- 0.008809   78.2 
%H  series 5  LocalEnergy = -0.488471 +/- 0.006704   27.2 
%H  series 6  LocalEnergy = -0.427345 +/- 0.011377   50.0 
%H  series 7  LocalEnergy = -0.456044 +/- 0.014513   51.1 
%H  series 8  LocalEnergy = -0.453782 +/- 0.016594   24.1 
%H  series 9  LocalEnergy = -0.482306 +/- 0.028252   21.6 
%H  series 10  LocalEnergy = -0.405258 +/- 0.013696   22.4 
%H  series 11  LocalEnergy = -0.423111 +/- 0.003579    2.9 
%H  series 12  LocalEnergy = -0.474759 +/- 0.016879    9.6 
%H  series 13  LocalEnergy = -0.414045 +/- 0.003606    5.5 
%H  series 14  LocalEnergy = -0.432808 +/- 0.004773    3.3 
%H  series 15  LocalEnergy = -0.465723 +/- 0.004425    2.6 
%\end{verbatim}
%\end{shaded}

\FloatBarrier
\begin{figure}[ht!]
\begin{center}
\includegraphics[trim = 0mm 0mm 0mm 0mm, clip,width=0.75\columnwidth]{figures/lab_qmc_statistics_blocking2.png}
\end{center}
\end{figure}
\FloatBarrier

%\includegraphics[scale=1.0]{steps_per_block_vs_autocorrelation_energy_H_STO-2G.png}

The greatest number of steps per block produces the smallest autocorrelation
time.  The larger number of blocks over which to average at small
step-per-block number masks the corresponding increase in error bar with
increasing autocorrelation.

Press \textbf{q [Enter]} to quit gnuplot.

\subsection{Balancing autocorrelation and acceptance ratio}

Adjusting the time step value also affects the ratio of accepted steps to
proposed steps.  Stepping nearby in configuration space implies that the
probability distribution is similar and thus more likely to result in an
accepted move.  Keeping the acceptance ratio high means the algorithm is
efficiently exploring configuration space and not sticking at particular
configurations.  Return to the \texttt{autocorrelation} directory.  Refresh your
memory on the time steps in this set of simulations by \textbf{grep timestep
H.xml}. Then, type \textbf{qmca -q ar *scalar.dat} to see the acceptance ratio
as it varies with decreasing time step:

\begin{shade} 
H  series 0  AcceptRatio = 0.047646 +/- 0.000206 
H  series 1  AcceptRatio = 0.125361 +/- 0.000308 
H  series 2  AcceptRatio = 0.328590 +/- 0.000340 
H  series 3  AcceptRatio = 0.535708 +/- 0.000313 
H  series 4  AcceptRatio = 0.732537 +/- 0.000234 
H  series 5  AcceptRatio = 0.903498 +/- 0.000156 
H  series 6  AcceptRatio = 0.961506 +/- 0.000083 
H  series 7  AcceptRatio = 0.985499 +/- 0.000051 
H  series 8  AcceptRatio = 0.996251 +/- 0.000025 
H  series 9  AcceptRatio = 0.998638 +/- 0.000014 
H  series 10  AcceptRatio = 0.999515 +/- 0.000009 
H  series 11  AcceptRatio = 0.999884 +/- 0.000004 
H  series 12  AcceptRatio = 0.999958 +/- 0.000003 
H  series 13  AcceptRatio = 0.999986 +/- 0.000002 
H  series 14  AcceptRatio = 0.999995 +/- 0.000001 
H  series 15  AcceptRatio = 0.999999 +/- 0.000000 
\end{shade}

By series 8 (time step = 0.02), the acceptance ratio is in excess of 99\%.  

Considering the increase in autocorrelation and subsequent increase in error
bar as time step decreases, it is important to choose a time step that trades
off appropriately between acceptance ratio and autocorrelation.  In this
example, a time step of 0.02 occupies a spot where acceptance ratio is high
(99.6\%), and autocorrelation is not appreciably larger than the minimum value
(1.4 vs. 1.0).

\subsection{Considering variance}

Besides autocorrelation, the dominant contributor to the error bar is the
\textit{variance} in the local energy.  The variance measures the fluctuations
around the average local energy, and, as the fluctuations go to zero, the wave
function reaches an exact eigenstate of the Hamiltonian.  qmca calculates this
from the local energy and local energy squared columns of the scalar.dat. 

Type \textbf{qmca -q v H.s009.scalar.dat} to calculate the variance on the run
with time step balancing autocorrelation and acceptance ratio:

\begin{shade}
H  series 9  Variance = 0.513570 +/- 0.010589  
\end{shade}

Just as the total energy doesn't tell us much by itself, neither does the
variance.  However, comparing the ratio of the variance to the energy indicates
how the magnitude of the fluctuations compares to the energy itself.   Type
\textbf{qmca -q ev H.s009.scalar.dat} to calculate the energy and variance on
the run side by side with the ratio:

\begin{shade}
                     LocalEnergy               Variance        ratio
H  series 0  -0.454460 +/- 0.000568   0.529496 +/- 0.018445   1.1651
\end{shade}

1.1651 is a very high ratio indicating the square of the fluctuations is on
average larger than the value itself.  In the next section, we will approach
ways to improve the variance that subsequent labs will build upon.  

\section{Reducing statistical error bars}

\subsection{Increasing MC sampling}

Increasing the number of MC samples in a data set reduces the error bar as the
inverse of the square root of the number of samples.  There are two ways to
increase the number of MC samples in a simulation: running more samples in
parallel and increasing the number of blocks (with fixed number of steps per
block, this increases the total number of MC steps).

To see the effect of the running more samples in parallel, change to the
directory \texttt{nodes}.  The series here increases the number of nodes by
factors of four from 32 to 128 to 512.  Type \textbf{qmca -q ev *scalar.dat}
and note the change in the error bar on the local energy as the number of
nodes.  Visualize this with \textbf{gnuplot H.plt}:

\FloatBarrier
\begin{figure}[ht!]
\begin{center}
\includegraphics[trim = 0mm 0mm 0mm 0mm, clip,width=0.75\columnwidth]{figures/lab_qmc_statistics_nodes.png}
\end{center}
\end{figure}
\FloatBarrier

%\includegraphics[scale=1.0]{nnode_vs_energy_H_STO-2G.png}

Increasing the number of blocks, unlike running in parallel, increases the
total CPU time of the simulation.  

Press \textbf{q [Enter]} to quit gnuplot.

To see the effect of increasing the block number, change to the directory
\texttt{blocks}. To see how we will vary the number of blocks, type
\textbf{grep -A1 blocks H.xml}.  The number of steps remains fixed, thus
increasing the total number of samples.   Visualize the tradeoff by inputting
\textbf{gnuplot H.plt}: 

\FloatBarrier
\begin{figure}[ht!]
\begin{center}
\includegraphics[trim = 0mm 0mm 0mm 0mm, clip,width=0.75\columnwidth]{figures/lab_qmc_statistics_blocks.png}
\end{center}
\end{figure}
\FloatBarrier

%\includegraphics[scale=1.0]{nblock_vs_tcpu_energy_H_STO-2G.png}

Press \textbf{q [Enter]} to quit gnuplot.

\subsection{Improving the basis set}

In all of the above examples, we are using the sum of two gaussian functions
(STO-2G) to approximate what should be a simple decaying exponential (STO =
Slater-type orbital) for the wave function of the ground state of the hydrogen
atom.  The sum of multiple copies of a function varying each copy's width and
amplitude with coefficients is called a \textit{basis set}. As we add gaussians
to the basis set, the approximation improves, the variance goes toward zero and
the energy goes to -0.5 hartrees.  In nearly every other case, the exact
function is unknown, and we add basis functions until the total energy does not
change within some threshold.

Change to the directory \texttt{basis} and look at the total energy and
variance as we change the wave function by typing \textbf{qmca -q ev H\_*}:

\begin{shade}
                            LocalEnergy               Variance        ratio 
H_STO-2G  series 0  -0.454460 +/- 0.000568   0.529496 +/- 0.018445   1.1651 
H_STO-3G  series 0  -0.465386 +/- 0.000502   0.410491 +/- 0.010051   0.8820 
H_STO-6G  series 0  -0.471332 +/- 0.000491   0.213919 +/- 0.012954   0.4539 
H__exact  series 0  -0.500000 +/- 0.000000   0.000000 +/- 0.000000   -0.0000 
\end{shade}

qmca also puts out the ratio of the variance to the local energy in a column to
the right of the variance error bar.  A typical high quality value for this
ratio is lower than 0.1 or so--none of these few-gaussian wave functions
satisfy that rule of thumb.

Use qmca to plot the trace of the local energy, kinetic energy, and potential
energy of H\_\_exact--the total energy is constantly -0.5 hartree even though
the kinetic and potential energies fluctuate from configuration to
configuration.

\subsection{Adding a Jastrow factor}

Another route to reducing the variance is the introduction of a Jastrow factor to 
account for electron-electron correlation (not the statistical autocorrelation
of Monte Carlo steps but the physical avoidance that electrons have of one another).
To do this, we will switch to the hydrogen dimer with the exact ground state
wave function of the atom (STO basis)--this will not be exact for the dimer.
The ground state energy of the hydrogen dimer is -1.174 hartrees.

Change directories to \texttt{dimer} and put in \textbf{qmca -q ev *scalar.dat}
to see the result of adding a simple, one-parameter Jastrow to the STO basis
for the hydrogen dimer at experimental bond length:

\begin{shade}
                               LocalEnergy               Variance           
H2_STO___no_jastrow  series 0  -0.876548 +/- 0.005313   0.473526 +/- 0.014910
H2_STO_with_jastrow  series 0  -0.912763 +/- 0.004470   0.279651 +/- 0.016405
\end{shade}

The energy reduces by 0.044 +/- 0.006 hartrees and the variance by 0.19 +/- 0.02.
This is still 20\% above the ground state energy, and subsequent labs will cover how
to improve on this with improved forms of the wave function that capture more
of the physics.

\section{Scaling to larger numbers of electrons}

\subsection{Calculating the efficiency}

The inverse of the product of CPU time and the variance measures the
\textit{efficiency} of an MC calculation.  Use qmca to calculate efficiency by
typing \textbf{qmca -q eff *scalar.dat} to see the efficiency of these two
H$_2$ calculations:

\begin{shade}
H2_STO___no_jastrow  series 0  Efficiency = 16698.725453 +/- 0.000000 
H2_STO_with_jastrow  series 0  Efficiency = 52912.365609 +/- 0.000000 
\end{shade}

The Jastrow factor increased the efficiency in these calculations by a factor
of three, largely through the reduction in variance (check the average block
CPU time to verify this claim).

\subsection{Scaling up}

To see how MC scales with increasing particle number, change directories to
\texttt{size}.  Here are the data from runs of increasing number of electrons
for H, H$_2$, C, CH$_4$, C$_2$, C$_2$H$_4$, (CH$_4$)$_2$, and (C$_2$H$_4$)$_2$
using the STO-6G basis set for the orbitals of the Slater determinant.  The file names begin with the number of electrons simulated for those data.

Use \textbf{qmca -q bc *scalar.dat} to see that the CPU time per block
increases with number of electrons in the simulation, then plot the total CPU
time of the simulation by \textbf{gnuplot Nelectron\_tCPU.plt}:

\FloatBarrier
\begin{figure}[ht!]
\begin{center}
\includegraphics[trim = 0mm 0mm 0mm 0mm, clip,width=0.75\columnwidth]{figures/lab_qmc_statistics_scaling.png}
\end{center}
\end{figure}
\FloatBarrier

%\includegraphics[scale=1.0]{nelectron_vs_tcpu_H_C_CH_STO-6G.png}

The green pluses represent the CPU time per block at each electron number.
The red line is a quadratic fit to those data.  For a fixed basis set size, we expect the time to scale quadratically up to 1000s of electrons, at which point a cubic scaling term may become dominant.  Knowing the scaling allows you to roughly project the calculation time for a larger number of electrons.

Press \textbf{q [Enter]} to quit gnuplot.

This isn't the whole story, however.  The variance of the energy also increases
with a fixed basis set as the number of particles increases at a faster rate
than the energy decreases.  To see this, type \textbf{qmca -q ev *scalar.dat}:

\begin{shade}
                            LocalEnergy               Variance           
01________H  series 0  -0.471352 +/- 0.000493      0.213020 +/- 0.012950 
02_______H2  series 0  -0.898875 +/- 0.000998      0.545717 +/- 0.009980 
06________C  series 0  -37.608586 +/- 0.020453   184.322000 +/- 45.481193
10______CH4  series 0  -38.821513 +/- 0.022740   169.797871 +/- 24.765674
12_______C2  series 0  -72.302390 +/- 0.037691   491.416711 +/- 106.090103
16_____C2H4  series 0  -75.488701 +/- 0.042919   404.218115 +/- 60.196642
18___CH4CH4  series 0  -58.459857 +/- 0.039309   498.579645 +/- 92.480126
32_C2H4C2H4  series 0  -91.567283 +/- 0.048392   632.114026 +/- 69.637760
\end{shade}

The increase in variance is not uniform, but the general trend is upward with a
fixed wave function form and basis set.  Subsequent labs will address how to
improve the wave function in order to keep the variance manageable.

\chapter{Lab 2: QMC Basics}
\label{chap:lab_qmc_basics}



\section{Topics covered in this Lab}
This lab focuses on the basics of performing quality QMC calculations.  As an example participants test an oxygen pseudopotential within DMC by calculating atomic and dimer properties, a common step prior to production runs.  Topics covered include:
\begin{itemize}
  \item{converting pseudopotentials into QMCPACK's FSATOM format}
  \item{generating orbitals with Quantum Espresso}
  \item{converting orbitals into QMCPACK's ESHDF format with pw2qmcpack}
  \item{optimizing Jastrow factors with QMCPACK}
  \item{removing DMC timestep error via extrapolation}
  \item{automating QMC workflows with Nexus}
  \item{testing pseudopotentials for accuracy}
  \item{(optional) running QMCPACK for a general system of interest}
\end{itemize}



\newpage
\section{Lab directories and files}

\begin{shaded}
\begin{verbatim}
Lab_2_QMC_Basics/
│
├── docs                     - documentation 
│   ├── Lab_2_QMC_Basics.pdf    - this document
│   ├── Lab_2_Slides.pdf        - slides presented during the lab
│   └── Nexus.pdf               - slides on QMCPACK automation (supplementary)
│
├── oxygen_atom              - oxygen atom calculations 
│   ├── ip_conv.py              - tool to fit oxygen IP vs timestep
│   ├── O.q0.dmc.in.xml         - neutral O DMC input file
│   ├── O.q0.dmc.qsub.in        -    "    "  "  submission file
│   ├── O.q0.opt.in.xml         -    "    " optimization input file
│   ├── O.q0.opt.qsub.in        -    "    "  "  submission file         
│   ├── O.q0.pwscf.h5           -    "    "  orbitals file         
│   ├── O.q1.dmc.in.xml         - charged O DMC input file         
│   ├── O.q1.dmc.qsub.in        -    "    "  "  submission file   
│   ├── O.q1.opt.in.xml         -    "    " optimization input file
│   ├── O.q1.opt.qsub.in        -    "    "  "  submission file   
│   ├── O.q1.pwscf.h5           -    "    "  orbitals file         
│   ├── reference               - directory w/ completed runs
│   ├── submit_O_q0_dmc         - executable to submit neutral DMC
│   ├── submit_O_q0_opt         -    "       "    "       "    optimization
│   ├── submit_O_q1_dmc         -    "       "    "    charged DMC
│   └── submit_O_q1_opt         -    "       "    "       "    optimization
│
├── oxygen_dimer             - oxygen dimer calculations
│   ├── dimer_fit.py            - tool to fit dimer binding curve
│   ├── O_dimer.py              - automation script for dimer calculations
│   ├── pseudopotentials        - directory for pseudopotentials
│   └── reference               - directory w/ completed runs
│
└── your_system              - calculations with your own physical system
    ├── example.py              - generates input files for your system
    ├── pseudopotentials        - directory for pseudopotentials
    └── reference               - directory w/ completed runs
\end{verbatim}
\end{shaded}





\section{Obtaining and converting a pseudopotential for oxygen}\label{sec:lqb_pseudo}
We will use a potential from the Burkatzki-Filippi-Dolg pseudopotential database.  To obtain the pseudopotential, go to 
\href{http://www.burkatzki.com/pseudos/index.2.html}{http://www.burkatzki.com/pseudos/index.2.html}
and click on the ``Select Pseudopotential'' button.  Next click on oxygen in the 
periodic table.  Click on the empty circle next to ``V5Z'' (a large gaussian 
basis set) and click on ``Next''.  Select the Gamess format and click on 
``Retrive Potential''.  Helpful information about the pseudopotential will be 
displayed.  The desired portion is at the bottom (the last 7 lines).  Copy 
this text into the editor of your choice and save it as \texttt{O.BFD.gamess} 
(be sure to include a newline at the end of the file).  To transform the 
pseudopotential into the fsatom xml format used by QMCPACK, use the \texttt{ppconvert} 
tool:
\begin{shaded}
\begin{verbatim}
ppconvert --gamess_pot O.BFD.gamess --s_ref "1s(2)2p(4)" \
 --p_ref "1s(2)2p(4)" --d_ref "1s(2)2p(4)" --xml O.BFD.xml
\end{verbatim}
\end{shaded}
\noindent
Observe the notation used to describe the reference valence configuration for this helium-core PP: \texttt{1s(2)2p(4)}.  The \texttt{ppconvert} tool uses the following convention for the valence states: the first $s$ state is labeled \texttt{1s} (\texttt{1s}, \texttt{2s}, \texttt{3s}, \ldots), the first $p$ state is labeled \texttt{2p} (\texttt{2p}, \texttt{3p}, \ldots), the first $d$ state is labeled \texttt{3d} (\texttt{3d}, \texttt{4d}, \ldots). Copy the resulting xml file into the \texttt{oxygen\_atom} directory.

Note: the command to convert the PP into QM Espresso's UPF format is similar:
\begin{shaded}
\begin{verbatim}
ppconvert --gamess_pot O.BFD.gamess --s_ref "1s(2)2p(4)" \
 --p_ref "1s(2)2p(4)" --d_ref "1s(2)2p(4)" --log_grid --upf O.BFD.upf
\end{verbatim}
\end{shaded}

For reference, the text of \texttt{O.BFD.gamess} should be:
\begin{shaded}
\begin{verbatim}
O-QMC GEN 2 1
3
6.00000000 1 9.29793903
55.78763416 3 8.86492204
-38.81978498 2 8.62925665
1
38.41914135 2 8.71924452

\end{verbatim}
\end{shaded}
\noindent
The full QMCPACK pseudopotential is also included in \texttt{oxygen\_atom/reference/O.BFD.xml}.


\section{Optimization walkthrough with QMCPACK: neutral O atom}\label{sec:optimization_walkthrough}
The aim of this section is to obtain a trial wavefunction of reasonable quality 
for the neutral oxygen atom.  The first subsection provides background regarding 
the wavefunction for this system, including the specific form of the Jastrow 
factors used in QMCPACK.  A brief discussion of wavefunction optimization is 
also given.  The second subsection contains the actual walkthrough to follow 
for the lab. 


% background on the wavefunction should be covered elsewhere in the manual
%   perhaps replace this with just the figure and a couple of brief comments 
\hide{
\subsubsection{Background on trial wavefunction and optimization}\label{sec:opt_background}
The trial wavefunction used to describe the neutral oxygen atom is of the 
standard Slater-Jastrow form:
\begin{align}  
  \Psi_T = e^{-(J_1+J_2)}D^\uparrow(\{\phi_u^\uparrow\}_{u=1}^{N^\uparrow})D^\downarrow(\{\phi_d^\downarrow\}_{d=1}^{N^\uparrow})
\end{align}
The orbitals forming the spin-restricted Slater determinants 
($D^\uparrow/D^\downarrow$) are obtained from DFT or Hartree-Fock (\emph{e.g.} via Quantum Espresso) 
and are fixed.  The ground state of the (pseudo) oxygen atom is spin polarized 
with $N^{\uparrow}=4$ and $N^{\downarrow}=2$.  

The part of the wavefunction we will be optimizing is the Jastrow factor 
($e^{-(J_1+J_2)}$), which in this case includes one- (electron-ion) and two- 
(electron-electron) body correlation functions.  The Jastrow factor is symmetric 
under same-spin electron exchange and does not affect the DMC fixed node 
approximation.  Optimization of the Jastrow factor does, however, improve the 
efficiency of the DMC calculation and reduces additional approximations due to 
non-local pseudopotentials (locality approximation, T-moves).  


\begin{figure}
\begin{center}
\includegraphics[trim = 0mm 0mm 0mm 0mm, clip,width=0.75\columnwidth]{./figures/lab_qmc_basics_J1}
\end{center}
\caption{Optimized $U_1$ function for 1-body Jastrow factor of an oxygen atom.
\label{fig:u1_spline}
}
\end{figure}

The explicit form of the one-body Jastrow factor we will be using is
\begin{align}\label{eq:J1}
  J_1 = \sum_{e=1}^{N^\uparrow+N^\downarrow}U_1^{\uparrow/\downarrow}(|r_e-r_O|)
\end{align}
where $r_e$ refers to the electron positions and $r_O$ is 
the position of the oxygen ion.  The $U_1^{\uparrow/\downarrow}$ term is a 
one-dimensional radial function represented with piecewise continuous cubic 
polynomials (B-splines).  The adjustable parameters to be optimized are the 
``knots'' of the B-splines which are simply the values of the $U_1$ function at 
uniformly spaced grid points (See fig. \ref{fig:u1_spline} for an example of a $U_1$ 
spline function with 8 knots).  

The two-body Jastrow factor is spin resolved ($r^\uparrow/r^\downarrow$ are up/down electron positions):
\begin{align}\label{eq:J2}
  J_2 = \sum_{u<u'}U_2^{\uparrow\uparrow/\downarrow\downarrow}(|r_u^\uparrow-r_{u'}^\uparrow|) + \sum_{d<d'}U_2^{\uparrow\uparrow/\downarrow\downarrow}(|r_d^\downarrow-r_{d'}^\downarrow|) + \sum_{u,d} U_2^{\uparrow\downarrow}(|r_u^\uparrow-r_d^\downarrow|)
\end{align}
For an atom, Pad\'{e} functions are appropriate for $U_2^{\uparrow\uparrow/\downarrow\downarrow}$ and $U_2^{\uparrow\downarrow}$:
\begin{align}
  U_2(r) = \frac{Ar}{1+Br}
\end{align}
Only $B^{\uparrow\uparrow/\downarrow\downarrow}$ and $B^{\uparrow\downarrow}$ are adjustable since the $A$ parameters are fixed by the electron-electron cusp conditions.

Wavefunction optimization essentially relies on two inequalities regarding energy and variance:
\begin{align}
  E_T(P) &= \frac{\expvalh{\Psi_T(P)}{H}{\Psi_T(P)}}{\overlap{\Psi_T(P)}{\Psi_T(P)}} \ge E_0 \\
  V_T(P) &= \frac{\expval{\Psi_T(P)}{\hat{H}^2}{\Psi_T(P)}}{\overlap{\Psi_T(P)}{\Psi_T(P)}} - \left(\frac{\expval{\Psi_T(P)}{H}{\Psi_T(P)}}{\overlap{\Psi_T(P)}{\Psi_T(P)}}\right)^2 \ge 0   
\end{align}
Here $E_0$ is the ground state energy, $E_T(P)$ is the trial energy, $V_T(P)$ is the trial variance, and $P$ denotes the set of adjustable parameters in the trial wavefunction.  Equality is reached only for the true ground state wavefunction and so the trial wavefunction can be improved by attempting to minimize a chosen cost function: 
\begin{align}
  C(P) = \alpha E_T(P) + (1-\alpha) V_T(P).
\end{align}  
Iterative varational Monte Carlo methods have been developed to handle the non-linear optimization problem $\min\limits_P C(P)$.  We will be using the linearized optimization method of Umrigar, \emph{et al.} (PRL \textbf{98} 110201 (2007)).  Let us try this now with QMCPACK.
}


Enter the \texttt{oxygen\_atom} directory and copy over the oxygen pseudopotential (\texttt{O.BFD.xml}) you downloaded and converted (section \ref{sec:pseudo}).  Alternatively, the already converted pseudopotential is located in the \texttt{oxygen\_atom/reference} directory.  All files prefixed with ``\texttt{O.q0}'' relate to the neutral oxygen atom.  

Open \texttt{O.q0.opt.in.xml} with your favorite text editor.  This is a QMCPACK input file configured for wavefunction optimization with the linear method. Take a minute to familiarize yourself with the general format and contents of the input file.  The major sections are the simulation cell, description of particle species (electrons \& ions/atoms), the trial wavefunction (orbitals, Slater determinants, and Jastrow factors), the Hamiltonian, and finally inputs describing the quantum Monte Carlo process (linear optimization in this case).  Portions marked with ``\texttt{<!-- ... -->}'' are comments describing these sections.  XML is not the easiest to read, but this can be helped by using an editor with color highlighting such as \texttt{emacs} or \texttt{vi}.

The most important parts to focus on for the purposes of this exercise are the Jastrow factors and the inputs to the linear optimization method.  Input specifying the one-body electron-ion Jastrow factor corresponding to eq. \ref{eq:J1} is 
\begin{shaded}
\begin{verbatim}
<jastrow type="One-Body" name="J1" function="bspline" source="ion0" print="yes">  
  <correlation elementType="O" size="8" rcut="4.5" cusp="0.0">
    <coefficients id="eO" type="Array">                  
      0 0 0 0 0 0 0 0    
    </coefficients>
  </correlation>
</jastrow>
\end{verbatim}
\end{shaded}
\noindent
The XML describes $U_1^{\uparrow/\downarrow}(r)$ as a B-spline with 8 knots, no cusp at the origin (the oxygen pseudopotential is finite at $r=0$), and vanishing beyond 4.5 Bohr.  The initial guess of zero for each of the 8 knot parameters corresponds to $U_1^{\uparrow/\downarrow}(r)=0$.  The input for the two-body electron-electron Jastrow is similar:
\begin{shaded}
\begin{verbatim}
<jastrow type="Two-Body" name="J2" function="pade" print="yes">
  <correlation speciesA="u" speciesB="u">
    <var id="uu_b" name="B">   0.6   </var>
  </correlation>
  <correlation speciesA="u" speciesB="d">
    <var id="ud_b" name="B">   1.0   </var>  
  </correlation>
</jastrow>
\end{verbatim}
\end{shaded}
\noindent
The XML describes $U_2^{\uparrow\uparrow/\downarrow\downarrow}(r)$ and  $U_2^{\uparrow\downarrow}(r)$ from eq. \ref{eq:J2} as Pad\'{e} functions with initial guesses of $B^{\uparrow\uparrow}=0.6~\textrm{Bohr}^{-1}$ and  $B^{\uparrow\downarrow}=1.0~\textrm{Bohr}^{-1}$ for the adjustable parameters.  

The relevant portion of the input describing the linear optimization process is
\begin{shaded}
\begin{verbatim}
<loop max="MAX">
  <qmc method="linear" move="pbyp" checkpoint="-1">
    <cost name="energy"              >  ECOST   </cost>
    <cost name="unreweightedvariance">  UVCOST  </cost>
    <cost name="reweightedvariance"  >  RVCOST  </cost>
    <parameter name="timestep"       >  TS      </parameter>
    <parameter name="samples"        >  SAMPLES </parameter>
    <parameter name="warmupSteps"    >  300     </parameter>
    <parameter name="blocks"         >  800     </parameter>
    <parameter name="subSteps"       >  10      </parameter>
    <parameter name="nonlocalpp"     >  yes     </parameter>
    <parameter name="useBuffer"      >  yes     </parameter>
    ...
  </qmc>
</loop>
\end{verbatim}
\end{shaded}
\noindent
An explanation of each input variable can be found below.  The remaining variables control specialized internal details of the linear optimization algorithm.  The meaning of these inputs is beyond the scope of this lab and reasonable results are often obtained keeping these values fixed. 
\begin{description}
  \item[energy] Fraction of trial energy in the cost function.
  \item[unreweightedvariance] Fraction of unreweighted trial variance in the cost function.  Neglecting the weights can be more robust.
  \item[reweightedvariance] Fraction of trial variance (including the full weights) in the cost function.  
  \item[timestep] Timestep of the VMC random walk, determines spatial distance moved by each electron during MC steps.  Should be chosen such that the acceptance ratio of MC moves is around 50\% (30-70\% is often acceptable).  Reasonable values are often between 0.2 and 0.6 $\textrm{Ha}^{-1}$.
  \item[samples] Total number of MC samples collected for optimization, determines statistical error bar of cost function.  Often efficient to start with a small number of samples (5-20k) and then increase (20-100k).  More samples may be required if the wavefunction contains a large number of variational parameters.  MUST be be a multiple of the number of threads/cores (use multiples of 512 on Vesta).
  \item[warmupSteps]  Number of MC steps discarded as a warmup or equilibration period of the random walk.  If this is too small, it will bias the optimization procedure.
  \item[blocks]  Number of average energy values written to output files.  Should be greater than 200 for meaningful statistical analysis of output data (\emph{e.g.} via \texttt{qmca}).
  \item[subSteps] Number of MC steps in between energy evaluations.  Each energy evaluation is expensive so taking a few steps to decorrelate between measurements can be more efficient.  Will be less efficient with many substeps.
  \item[nonlocalpp,useBuffer] If no, evaluate non-local pseudopotential derivatives approximately during optimization.  This saves time and often does not affect optimization results unless the non-local contribution to the energy is large.
  \item[loop max] Number of times to repeat the optimization.  Using the resulting wavefunction from the previous optimization in the next one improves the results.  Typical choices range between 4 and 20.   
\end{description}
The three components of the cost function, energy, unreweighted variance, and reweighted variance should sum to one.  Dedicating 100\% of the cost function to unreweighted variance is often a good choice.  Another common choice is to try 90/10 or 80/20 mixtures of reweighted variance and energy.  

Replace \texttt{MAX}, \texttt{EVCOST}, \texttt{UVCOST}, \texttt{RVCOST}, \texttt{TS}, and \texttt{SAMPLES} in the two \texttt{loop}'s with appropriate starting values in the \texttt{O.q0.opt.in.xml} input file.  Submit the optimization job to Vesta's queue by typing \texttt{./submit\_O\_q0\_opt}.  The job should only take a few minutes for reasonable values of loop \texttt{max} and \texttt{samples}.  

Log file output will appear in \texttt{O\_q0\_opt.output}.  The beginning of each linear optimization will be marked with text similar to
\begin{shaded}
\begin{verbatim}
=========================================================
  Start QMCFixedSampleLinearOptimize
  File Root O_q0_opt.s011 append = no 
=========================================================
\end{verbatim}
\end{shaded}
\noindent
At the end of each optimization section the change in cost function, new values for the Jastrow parameters, and elapsed wallclock time are reported:
\begin{shaded}
\begin{verbatim}
OldCost: 7.4701713964e-01 NewCost: 7.4681622535e-01 Delta Cost:-2.0091428584e-04
...
  <optVariables href="O_q0_opt.s011.opt.xml">
eO_0 -9.5623201640e-01 1 1  ON 0
eO_1 -8.4728730387e-01 1 1  ON 1
eO_2 -6.8954452383e-01 1 1  ON 2
eO_3 -4.9327199567e-01 1 1  ON 3
eO_4 -3.2560096773e-01 1 1  ON 4
eO_5 -1.9567566480e-01 1 1  ON 5
eO_6 -1.2940405487e-01 1 1  ON 6
eO_7 -9.5221474839e-02 1 1  ON 7
uu_b 4.2002038228e-01 0 1  ON 8
ud_b 6.3472757070e-01 0 1  ON 9
  </optVariables>
...
  QMC Execution time = 7.0060820112e+00 secs 
\end{verbatim}
\end{shaded}
\noindent
The cost function should decrease during each linear optimization (\texttt{Delta cost < 0}).  Try ``\texttt{grep OldCost *.output}''.  You should see something like this:
\begin{shaded}
\begin{verbatim}
 OldCost: 1.3644746067e+00 NewCost: 1.1049104640e+00 Delta Cost:-2.5956414268e-01
 OldCost: 1.0690085060e+00 NewCost: 8.3206148222e-01 Delta Cost:-2.3694702381e-01
 OldCost: 7.8558402137e-01 NewCost: 7.2478477600e-01 Delta Cost:-6.0799245374e-02
 OldCost: 7.3070322298e-01 NewCost: 7.1655770805e-01 Delta Cost:-1.4145514926e-02
 OldCost: 1.2184771084e+00 NewCost: 1.1923197177e+00 Delta Cost:-2.6157390699e-02
 OldCost: 6.8740347812e-01 NewCost: 6.8733036689e-01 Delta Cost:-7.3111228164e-05
 OldCost: 6.9683928634e-01 NewCost: 6.9681780340e-01 Delta Cost:-2.1482934426e-05
 OldCost: 6.7982953532e-01 NewCost: 6.7982948866e-01 Delta Cost:-4.6667065545e-08
 OldCost: 6.8674328187e-01 NewCost: 6.8674327833e-01 Delta Cost:-3.5391565234e-09
 OldCost: 7.5998537866e-01 NewCost: 7.5965629336e-01 Delta Cost:-3.2908530361e-04
 OldCost: 7.0771416413e-01 NewCost: 7.0765392787e-01 Delta Cost:-6.0236255172e-05
 OldCost: 7.4701713964e-01 NewCost: 7.4681622535e-01 Delta Cost:-2.0091428584e-04
\end{verbatim}
\end{shaded}

Blocked averages of energy data, including the kinetic energy and components of the potential energy, are written to \texttt{scalar.dat} files.  The first is named ``\texttt{O\_q0\_opt.s000.scalar.dat}'', with a series number of zero (\texttt{s000}).  In the end there will be \texttt{MAX1}+\texttt{MAX2} of them, one for each series. 

When the job has finished, use the \texttt{qmca} tool to assess the effectiveness of the optimization process.  To look at just the total energy and the variance, type ``\texttt{qmca -q ev O\_q0\_opt*scalar*}''.  This will print the energy, variance, and the variance/energy ratio in Hartree units:
\begin{shaded}
\begin{verbatim}
                            LocalEnergy               Variance           ratio 
O_q0_opt  series 0  -15.568764 +/- 0.003421   1.382681 +/- 0.056604   0.0888 
O_q0_opt  series 1  -15.638500 +/- 0.005014   1.067662 +/- 0.019865   0.0683 
O_q0_opt  series 2  -15.802163 +/- 0.002680   0.834521 +/- 0.007037   0.0528 
O_q0_opt  series 3  -15.840982 +/- 0.001791   0.752242 +/- 0.009477   0.0475 
O_q0_opt  series 4  -15.841584 +/- 0.003301   1.097355 +/- 0.252991   0.0693 
O_q0_opt  series 5  -15.848602 +/- 0.003280   0.728377 +/- 0.019288   0.0460 
O_q0_opt  series 6  -15.850839 +/- 0.001870   0.723159 +/- 0.008173   0.0456 
O_q0_opt  series 7  -15.848411 +/- 0.002449   0.708589 +/- 0.007225   0.0447 
...
\end{verbatim}
\end{shaded}
\noindent
Plots of the data can also be obtained with the ``\texttt{-p}'' option (``\texttt{qmca -p -q ev O\_q0\_opt*scalar*}'').

Identify which optimization series is the ``best'' according to your cost function.  It is likely that multiple series are similar in quality.  Note the \texttt{opt.xml} file corresponding to this series.  This file contains the final value of the optimized Jastrow parameters to be used in the DMC calculations of the next section of the lab.  

\vspace{1cm}
\begin{flushleft}
\textbf{\underline{Questions and Exercises}}
\end{flushleft}
\begin{enumerate}
  \item{What is the acceptance ratio of your optimization runs? (use ``\texttt{qmca --help}'' if necessary)  Do you expect the Monte Carlo sampling to be efficient?}
  \item{How do you know when the optimization process has converged?}
  \item{Why is the mean and the error of the variance sometimes large?  Consider using ``\texttt{qmca -t ...}'' to investigate.}
  \item{Optimization is sometimes sensitive to initial guesses of the parameters.  If you have time, try varying the initial parameters, including the cutoff radius (\texttt{rcut}) of the one-body Jastrow factor (remember to change \texttt{id} in the \texttt{<project/>} element).  Do you arrive at a similar set of final Jastrow parameters?  What is the lowest variance you are able to achieve?}
\end{enumerate}



\section{DMC timestep extrapolation I: neutral O atom}
The diffusion Monte Carlo (DMC) algorithm contains two biases in addition to the fixed node and pseudopotential approximations that are important to control: timestep and population control bias.  The following subsection briefly discusses the origin of timestep and population control biases in DMC and how they can be minimized or extrapolated away.  As before, the second subsection contains the lab walkthrough with QMCPACK.  By the end of the section, we will have a solid DMC estimate of the ground state energy of oxygen.

% background on timestep error should be covered elsewhere in the manual
%   perhaps replace this with a brief formula of error (order tau^2) on total energy
\hide{
\subsubsection{Background on timestep and population control bias}\label{sec:opt_background}
DMC improves over the VMC algorithm by projecting toward the true many-body electronic ground state of the system.  The projection operator is the (importance sampled) imaginary time propagator, which is also known as the thermodynamic density matrix:
\begin{align}
  \hat{\rho} = e^{-t\hat{H}}
\end{align}
The direct action of the projection operator on a trial wavefunction in position space
\begin{align}
  \expval{R}{e^{-t\hat{H}}}{\Psi_T} = \int dR' \rho(R,R';t)\Psi_T(R')
\end{align}
cannot be calculated in a straightforward fashion since the analytic form of $\rho(R,R';t)=\expval{R}{\rho}{R'}$ is unknown.  In order to make the algorithm computationally tractable, the finite time projection operator is expanded as a product of short-time projection operators
\begin{align}
  \expval{R}{e^{-t{H}}}{\Psi_T} &= \expval{R}{e^{-\tau\hat{H}}e^{-\tau\hat{H}}\cdots e^{-\tau\hat{H}}}{\Psi_T}\\
                                 &=\int dR_1dR_2\cdots dR_M \rho(R,R_1;\tau)\rho(R_1,R_2;\tau)\cdots\rho(R_{M-1},R_M;\tau)\Psi_T(R_M)
\end{align}
The advantage here is that reasonable approximations of the short time propagators are known.  Common approximations have the form
\begin{align}
  \rho(R,R';\tau) = e^{D(R,R';\tau)}e^{B(R,R';\tau)} + \mathcal{O}(\tau^2)
\end{align} 
where $D(R,R';\tau)$ and $B(R,R';\tau)$ represent drift and branching terms, respectively.  DMC results are biased for any finite timestep ($\tau$).  The bias can be eliminated by extrapolating to zero timestep.  In practice this is done by performing a series of runs with decreasing timesteps and then fitting the results.

The drift term can be sampled with standard Monte Carlo methods, while the branching term is incorporated as a weight assigned to each random walker.  Instead of accumulating the weight, it is more efficient to ``branch'' each walker according to the weight, resulting in some walkers being deleted and others copied multiple times.  If left uncontrolled, the walker population $(P)$ may vanish or diverge.  A stable algorithm is obtained by adjusting the branching weight to preserve the overall number of walkers on average.  Population control also biases the results, but usually to a lesser extent than timestep error (the bias is proportional to $1/P$).  A common rule of thumb is to use at least a couple thousand walkers.  This bias should be checked occasionally by performing runs with varying numbers of walkers.
}


In the same directory you used to perform wavefunction optimization (\texttt{oxygen\_atom}) you will find a sample DMC input file for the neutral oxygen atom named \texttt{O.q0.dmc.in.xml}.  Open this file in a text editor and note the differences from the optimization case.  The XML describing the wavefunction is no longer present.  In its place is the line
\begin{shaded}
\begin{verbatim}
   <include href="OPT_XML"/>
\end{verbatim}
\end{shaded}
\noindent
Replace ``\texttt{OPT\_XML}'' with the \texttt{opt.xml} file corresponding to the best Jastrow parameters you found in the last section.  The \texttt{include} element essentially amounts to an in-place copy and paste of the contents of the \texttt{opt.xml} file.

The QMC calculation section at the bottom is also different.  The linear optimization blocks have been replaced with XML describing a VMC run followed by DMC.  The input keywords are described below.

\begin{description}
  \item[timestep] Timestep of the VMC/DMC random walk.  In VMC choose a timestep corresponding to an acceptance ratio of about 50\%.  In DMC the acceptance ratio is often above 99\%.
  \item[warmupSteps]  Number of MC steps discarded as a warmup or equilibration period of the random walk.  
  \item[steps] Number of MC steps per block.  Physical quantities, such as the total energy, are averaged over walkers and steps.
  \item[blocks]  Number of blocks.  This is also the number of average energy values written to output files.  Should be greater than 200 for meaningful statistical analysis of output data (\emph{e.g.} via \texttt{qmca}).  The total number of MC steps each walker takes is \texttt{blocks}$\times$\texttt{steps}.
  \item[samples] VMC only. This is the number of walkers used in subsequent DMC runs.  Each DMC walker is initialized with electron positions sampled from the VMC random walk.
  \item[nonlocalmoves] DMC only.  If yes/no, use the locality approximation/T-moves for non-local pseudopotentials.  T-moves generally improve the stability of the algorithm and restore the variational principle for small systems (T-moves version 1).
\end{description}

The purpose of the VMC run is to provide initial electron positions for each DMC walker.  Setting $\texttt{walkers}=1$ in the VMC block ensures there will be only one VMC walker per execution thread.  There will be a total of 512 VMC walkers in this case (see \texttt{O.q0.dmc.qsub.in}).  We want the electron positions used to initialize the DMC walkers to be decorrelated from one another.  A VMC walker will often decorrelate from its current position after propagating for a few Ha$^{-1}$ in imaginary time (in general this is system dependent).  This leads to a rough rule of thumb for choosing \texttt{blocks} and \texttt{steps} for the VMC run ($\texttt{VWALKERS}=512$ here):
\begin{align}
  \texttt{VBLOCKS}\times\texttt{VSTENexus} \ge \frac{\texttt{DWALKERS}}{\texttt{VWALKERS}} \frac{5~\textrm{Ha}^{-1}}{\texttt{VTIMESTEP}}
\end{align}
Fill in the VMC XML block with appropriate values for these parameters.  There should be more than one DMC walker per thread and enough walkers in total to avoid population control bias (see previous subsection).

To study timestep bias, we will perform a sequence of DMC runs over a range of timesteps ($0.1$ Ha$^{-1}$ is too large and timesteps below $0.002$ Ha$^{-1}$ are probably too small).  A common approach is to select a fairly large timestep to begin with and then decrease the timestep by a factor of two in each subsequent DMC run.  The total amount of imaginary time the walker population propagates should be the same for each run.  A simple way to accomplish this is to choose input parameters in the following way
\begin{align}\label{eq:timestep_iter}
  \texttt{timestep}_{n}    &= \texttt{timestep}_{n-1}/2\nonumber\\
  \texttt{warmupSteps}_{n} &= \texttt{warmupSteps}_{n-1}\times 2\nonumber\\
  \texttt{blocks}_{n}      &= \texttt{blocks}_{n-1}\nonumber\\
  \texttt{steps}_{n}       &= \texttt{steps}_{n-1}\times 2
\end{align}
Each DMC run will require about twice as much computer time as the one preceeding it.  Note that the number of blocks is kept fixed for uniform statistical analysis.  $\texttt{blocks}\times\texttt{steps}\times\texttt{timestep}\sim 60~\mathrm{Ha}^{-1}$ is sufficient for this system.

Choose an initial DMC timestep and create a sequence of $N$ timesteps according to \ref{eq:timestep_iter}.  Make $N$ copies of the DMC XML block in the input file
\begin{shaded}
\begin{verbatim}
   <qmc method="dmc" move="pbyp">
      <parameter name="warmupSteps"         >    DWARMUP         </parameter>
      <parameter name="blocks"              >    DBLOCKS         </parameter>
      <parameter name="steps"               >    DSTENexus          </parameter>
      <parameter name="timestep"            >    DTIMESTEP       </parameter>
      <parameter name="nonlocalmoves"       >    yes             </parameter>
   </qmc>
\end{verbatim}
\end{shaded}
\noindent
Fill in \texttt{DWARMUP}, \texttt{DBLOCKS}, \texttt{DSTENexus}, and \texttt{DTIMESTEP} for each DMC run according to \ref{eq:timestep_iter}.  Submit the DMC timestep extrapolation run to the queue with \texttt{submit\_O\_q0\_dmc}.  The run should take only a few minutes to complete.

QMCPACK will create files prefixed with \texttt{O\_q0\_dmc}.  The log file is \texttt{O\_q0\_dmc.output}.  As before, block averaged data is written to \texttt{scalar.dat} files.  In addition, DMC runs produce \texttt{dmc.dat} files which contain energy data averaged only over the walker population (one line per DMC step).  The \texttt{dmc.dat} files also provide a record of the walker population at each step.

Use the \texttt{PlotTstepConv.pl} to obtain a linear fit to the timestep data (type ``\texttt{PlotTstepConv.pl O.q0.dmc.in.xml 40}'').  You should see a plot similar to fig. \ref{fig:timestep_conv}.  The tail end of the text output displays the parameters for the linear fit.  The ``\texttt{a}'' parameter is the total energy extrapolated to zero timestep in Hartree units. 

\begin{shaded}
\begin{verbatim}
...
Final set of parameters            Asymptotic Standard Error
=======================            ==========================

a               = -15.8911         +/- 0.000756     (0.004757%)
b               = -0.221687        +/- 0.03757      (16.95%)
...
\end{verbatim}
\end{shaded}

\begin{figure}
\begin{center}
\includegraphics[trim = 0mm 0mm 0mm 0mm, clip,width=0.75\columnwidth]{./figures/lab_qmc_basics_timestep_conv}
\end{center}
\caption{Linear fit to DMC timestep data from \texttt{PlotTstepConv.pl}.
\label{fig:timestep_conv}
}
\end{figure}


\vspace{1cm}
\begin{flushleft}
\textbf{\underline{Questions and Exercises}}
\end{flushleft}
\begin{enumerate}
  \item{What is the $\tau\rightarrow 0$ extrapolated value for the total energy?}
  \item{What is the maximum timestep you should use if you want to calculate the total energy to an accuracy of $0.05$ eV?  For convenience, $1~\textrm{Ha}=27.2113846~\textrm{eV}$.}
  \item{What is the acceptance ratio for this (bias$<0.05$ eV) run?  Does it follow the rule of thumb for sensible DMC (acceptance ratio $>99$\%) ?}
  \item{Check the fluctuations in the walker population (\texttt{qmca -t -q nw O\_q0\_dmc*dmc.dat --noac}).  Does the population seem to be stable?}
  \item{(Optional) Study population control bias for the oxygen atom.  Select a few population sizes (use multiples of 512 to fit cleanly on a single Vesta partition).  Copy \texttt{O.q0.dmc.in.xml} to a new file and remove all but one DMC run (select a single timestep).  Make one copy of the new file for each population, set ``\texttt{samples}'', and choose a unique \texttt{id} in \texttt{<project/>}.  Make submission files similar to \texttt{submit\_O\_q0\_dmc} and \texttt{O.q0.dmc.qsub.in} and run one job at a time to avoid crowding the lab allocation.  Use \texttt{qmca} to study the dependence of the DMC total energy on the walker population.  How large is the bias compared to timestep error?  What bias is incurred by following the ``rule of thumb'' of a couple thousand walkers?  Will population control bias generally be an issue for production runs on modern parallel machines?}
\end{enumerate}


\section{DMC timestep extrapolation II: O atom ionization potential}
In this section, we will repeat the calculations of the prior two sections (optimization, timestep extrapolation) for the $+1$ charge state of the oxygen atom.  Comparing the resulting 1st ionization potential (IP) with experimental data will complete our first test of the BFD oxygen pseudopotential.  In actual practice, higher IP's could also be tested prior to performing production runs.

Obtaining the timestep extrapolated DMC total energy for ionized oxygen should take much less (human) time than for the neutral case.  For convenience, the necessary steps are briefly summarized below.
\begin{enumerate}
  \item{Copy the linear optimization blocks you used in \texttt{O.q0.opt.in.xml} to  \texttt{O.q0.opt.in.xml}.}
  \item{Submit the optimization job to Vesta's queue with \texttt{submit\_O\_q1\_opt}}.
  \item{Identify the optimal set of parameters with \texttt{qmca}.}
  \item{Replace \texttt{OPT\_XML} in \texttt{submit\_O\_q1\_dmc} with the \texttt{opt.xml} file containing the optimal parameters.}
  \item{Copy the VMC and DMC blocks you used in \texttt{O.q0.dmc.in.xml} to \texttt{O.q1.dmc.in.xml}.}
  \item{Submit the DMC timestep job to Vesta's queue with \texttt{submit\_O\_q1\_dmc}}.
  \item{Obtain the DMC total energy extrapolated to zero timestep with \texttt{PlotTstepConv.pl}.}
\end{enumerate}
The process listed above, which excludes additional steps for orbital generation and conversion, can become tedious to perform by hand in production settings where many calculations are often required.  For this reason automation tools are introduced for calculations involving the oxygen dimer in section \ref{sec:dimer_automation} of the lab.  

\vspace{1cm}
\begin{flushleft}
\textbf{\underline{Questions and Exercises}}
\end{flushleft}
\begin{enumerate}
  \item{What is the $\tau\rightarrow 0$ extrapolated DMC value for the 1st ionization potential of oxygen?}
  \item{How does the extrapolated value compare to the experimental IP?  Go to\newline \href{http://physics.nist.gov/PhysRefData/ASD/ionEnergy.html}{http://physics.nist.gov/PhysRefData/ASD/ionEnergy.html} and enter ``\texttt{O I}'' in the box labeled ``\texttt{Spectra}'' and click on the ``\texttt{Retrieve Data}'' button.  For comparison the LDA value is $12.25$ eV.}
  \item{What can we conclude about the accuracy of the pseudopotential?  What factors complicate this assessment?}
  \item{Explore the sensitivity of the IP to the choice of timestep.  Type ``\texttt{ip\_conv.py}'' to view three timestep extrapolation plots: two for the $q=0,1$ total energies and one for the IP.  Is the IP more, less, or similarly sensitive to timestep than the total energy?}
  \item{What is the maximum timestep you should use if you want to calculate the ionization potential to an accuracy of $0.05$ eV?  What factor of cpu time is saved by assessing timestep convergence on the IP (a total energy difference) vs. a single total energy?}
  \item{Are the acceptance ratio and population fluctuations reasonable for the $q=1$ calculations?}
\end{enumerate}




\section{DMC workflow automation with Nexus}
Production QMC projects are often composed of many similar workflows.  The simplest of these is a single DMC calculation involving four different compute jobs:
\begin{enumerate}
  \item{Orbital generation via Quantum Espresso or GAMESS.}
  \item{Conversion of orbital data via \texttt{pw2qmcpack.x} or \texttt{convert4qmc}.}
  \item{Optimization of Jastrow factors via QMCPACK.}
  \item{DMC calculation via QMCPACK.}
\end{enumerate}
Simulation workflows quickly become more complex with increasing costs in terms of human time for the researcher.  Automation tools can decrease both human time and error if used well.

The set of automation tools we will be using is known as Nexus, which is distributed with QMCPACK.  Nexus is capable of generating input files, submitting and monitoring compute jobs, passing data between simulations (such as relaxed structures, orbital files, optimized Jastrow parameters, etc.), and data analysis.  The user interface to Nexus is through a set of functions defined in the Python programming language.  User scripts which execute simple workflows resemble input files and do not require programming experience.  More complex workflows require only basic programming constructs (\emph{e.g.} for loops and if statements).  Nexus input files/scripts should be easier to navigate than QMCPACK input files and more efficient than submitting all the jobs by hand.

Nexus is driven by simple user-defined scripts that resemble keyword-driven input files.  An example Nexus input file that performs a single VMC calculation is shown below.  Take a moment to read it over and especially note the comments (prefixed with ``\texttt{\#}'') explaining most of the contents.  If the input syntax is unclear you may want to consult portions of appendix \ref{app:python_basics}, which gives a condensed summary of Python constructs.  For more information about the functionality and effective use of Nexus, consult \texttt{docs/Nexus.pdf} first.  More information can be found in the user guide distributed with QMCPACK, although examples in this lab series and \texttt{Nexus.pdf} are more up to date (if \texttt{qmcpack} is the location of your QMCPACK distribution, the user guide can be found at \texttt{qmcpack/nexus/documentation/nexus\_user\_guide.pdf}).

\begin{shaded}
\begin{verbatim}
#! /usr/bin/env python

# import Nexus functions
from nexus import settings,Job,get_machine,run_project 
from nexus import generate_physical_system
from nexus import generate_qmcpack,vmc

settings(                             # Nexus settings
    pseudo_dir    = './pseudopotentials', # location of PP files
    runs          = '',                   # root directory for simulations
    results       = '',                   # root directory for simulation results
    status_only   = 0,                    # show simulation status, then exit
    generate_only = 0,                    # generate input files, then exit
    sleep         = 3,                    # seconds between checks on sim. progress
    machine       = 'vesta',              # name of local machine
    account       = 'QMC_2014_training'   # charge account for cpu time
    ) 

vesta = get_machine('vesta')          # allow max of one job at a time (lab only)
vesta.queue_size = 1

qmcjob = Job(                         # specify job parameters
    nodes   = 32,                         # use 32 Vesta nodes
    threads = 16,                         # 16 OpenMP threads per node (32 MPI tasks)
    hours   = 1,                          # wallclock limit of 1 hour
                                          # use QMCPACK executable
    app     = '/soft/applications/qmcpack/build_XL_real/bin/qmcapp'
    )

qmc_calcs = [                         # list QMC calculation methods
    vmc(                                  #   VMC
        walkers     =   1,                #     1 walker
        warmupsteps =  50,                #    50 MC steps for warmup
        blocks      = 200,                #   200 blocks
        steps       =  10,                #    10 steps per block
        timestep    =  .4                 #   0.4 1/Ha timestep
        )]

dimer = generate_physical_system(     # make a dimer system
    type       = 'dimer',                 # system type is dimer
    dimer      = ('O','O'),               # dimer is two oxygen atoms
    separation = 1.2074,                  # separated by 1.2074 Angstrom
    Lbox       = 15.0,                    # simulation box is 15 Angstrom 
    units      = 'A',                     # Angstrom is dist. unit
    net_spin   = 2,                       # nup-ndown is 2
    O          = 6                        # pseudo-oxygen has 6 valence el.
    )

qmc = generate_qmcpack(                # make a qmcpack simulation 
    identifier   = 'example',             # prefix files with 'example'
    path         = 'scale_1.0',           # run in ./scale_1.0 directory
    system       = dimer,                 # run the dimer system
    job          = qmcjob,                # set job parameters
    input_type   = 'basic',               # basic qmcpack inputs given below    
    pseudos      = ['O.BFD.xml'],         # list of PP's to use
    orbitals_h5  = 'O2.pwscf.h5',         # file with orbitals from DFT
    bconds       = 'nnn',                 # open boundary conditions
    jastrows     = [],                    # no jastrow factors
    calculations = qmc_calcs              # QMC calculations to perform
    )
                       
run_project(qmc)                       # write input file and submit job
\end{verbatim}
\end{shaded}



\section{Automated binding curve of the oxygen dimer}
\label{sec:dimer_automation}
In this we will use Nexus to calculate the DMC total energy of the oxygen dimer over a series of bond lengths.  The equilibrium bond length and binding energy of the dimer will be determined by performing a polynomial fit to the data (Morse potential fits should be preferred in production tests).  Comparing these values with correponding experimental data provides a second test of the BFD pseudopotential for oxygen.

Enter the \texttt{oxygen\_dimer} directory.  Copy your BFD pseudopotential from the atom runs into \texttt{oxygen\_dimer/pseudopotentials}.  Open \texttt{O\_dimer.py} with a text editor.  The overall format is similar to the example file shown in the last section.  The header material, including Nexus imports, settings, and the job parameters for QMC are identical.  The main difference is that optimization and DMC runs are being performed rather than a single VMC run.  

Following the job parameters, inputs for the optimization method are given.  The keywords should all be familiar from the QMCPACK XML input files you used previously:
\begin{shaded}
\begin{verbatim}
linopt1 = linear(
    energy               = 0.0,
    unreweightedvariance = 1.0,
    reweightedvariance   = 0.0,
    timestep             = 0.4,
    samples              = 5000, 
    warmupsteps          = 50,
    blocks               = 200,
    substeps             = 1,
    nonlocalpp           = True,
    usebuffer            = True,
    walkers              = 1,
    minwalkers           = 0.5,
    maxweight            = 1e9,
    usedrift             = True,
    minmethod            = 'quartic',
    beta                 = 0.025,
    exp0                 = -16,
    bigchange            = 15.0,
    alloweddifference    = 1e-4,
    stepsize             = 0.2,
    stabilizerscale      = 1.0,
    nstabilizers         = 3
    )
\end{verbatim}
\end{shaded}
\noindent
Requesting multiple loop's with different numbers of samples is more compact than in XML:
\begin{shaded}
\begin{verbatim}
linopt1 = ...

linopt2 = linopt1.copy()  
linopt2.samples = 20000 # opt w/ 20000 samples

linopt3 = linopt1.copy()
linopt3.samples = 40000 # opt w/ 40000 samples

opt_calcs = [loop(max=8,qmc=linopt1), # loops over opt's
             loop(max=6,qmc=linopt2),
             loop(max=4,qmc=linopt3)]
\end{verbatim}
\end{shaded}
\noindent
The VMC/DMC method inputs should also look familiar:
\begin{shaded}
\begin{verbatim}
qmc_calcs = [
    vmc(
        walkers     =   1,
        warmupsteps =  30,
        blocks      =  20,
        steps       =  10,
        substeps    =   2,
        timestep    =  .4,
        samples     = 2048
        ),
    dmc(
        warmupsteps   = 100, 
        blocks        = 400,
        steps         =  32,
        timestep      = 0.01,
        nonlocalmoves = True
        )
    ]
\end{verbatim}
\end{shaded}
\noindent
As in the example in the last section, the oxygen dimer is generated with the \texttt{generate\_physical\_system} function:
\begin{shaded}
\begin{verbatim}
dimer = generate_physical_system(
    type       = 'dimer',
    dimer      = ('O','O'),
    separation = 1.2074*scale,
    Lbox       = 15.0,
    units      = 'A',
    net_spin   = 2,
    O          = 6
    )
\end{verbatim}
\end{shaded}
\noindent
Similar syntax can be used to generate crystal structures or to specify systems with arbitrary atomic configurations and simulation cells.  Notice that a ``\texttt{scale}'' variable has been introduced to stretch or compress the dimer.  

Next, objects representing QMCPACK simulations are constructed with the \texttt{generate\_qmcpack} function:
\begin{shaded}
\begin{verbatim}
opt = generate_qmcpack(
    identifier   = 'opt',
    ...
    jastrows     = [('J1','bspline',8,4.5), 
                    ('J2','pade',0.5,0.5)],
    calculations = opt_calcs
    )
sims.append(opt)

qmc = generate_qmcpack(
    identifier   = 'qmc',
    ...
    jastrows     = [],            
    calculations = qmc_calcs,
    dependencies = (opt,'jastrow') 
    )
sims.append(qmc)
\end{verbatim}
\end{shaded}
\noindent
Shared details such as the run directory, job, pseudopotentials, and orbital file have been omitted (\texttt{...}).  The ``\texttt{opt}'' run will optimize a 1-body B-spline Jastrow with 8 knots having a cutoff of 4.5 Bohr and a 2-body Pad\'{e} Jastrow with up-up and up-down ``\texttt{B}'' parameters set to 0.5 1/Bohr.  The Jastrow list for the DMC run is empty and a new keyword is present: \texttt{dependencies}.  The usage of \texttt{dependencies} above indicates that the DMC run depends on the optimization run for the Jastrow factor.  Nexus will submit the ``\texttt{opt}'' run first and upon completion it will scan the output, select the optimal set of parameters, pass the Jastrow information to the ``\texttt{qmc}'' run and then submit the DMC job.  Independent job workflows are submitted in parallel when permitted (we have explicitly prevented this for this lab by setting \texttt{queue\_size=1} for Vesta).  No input files are written or job submissions made until the ``\texttt{run\_project}'' function is reached.

As written, \texttt{O\_dimer.py} will only perform calculations at the equilibrium separation distance of 1.2074 Angstrom.  Modify the file now to perform DMC calculations across a range of separation distances with each DMC run using the Jastrow factor optimized at the equilibrium separation distance.  The necessary Python \texttt{for} loop syntax should look something like this:
\begin{shaded}
\begin{verbatim}
sims = []
for scale in [1.00,0.90,0.95,1.05,1.10]:
    ...
    dimer = ...
    if scale==1.00:
        opt = ...
        ...
    #end if
    qmc = ...
    ...
#end for
run_project(sims)
\end{verbatim}
\end{shaded}
\noindent
Note that the text inside the \texttt{for} loop and the \texttt{if} block must be indented by precisely four spaces.  If you use Emacs, changes in indentation can be performed easily with \texttt{Cntrl-C >} and \texttt{Cntrl-C <} after highlighting a block of text (other editors should have similar functionality).  If you see something like  ``\texttt{SyntaxError: invalid syntax}'' print to the screen when you run \texttt{O\_dimer.py} later on, consult the completed file in \texttt{oxygen\_dimer/reference}.

The values of ``\texttt{scale}'' in the loop must be a subset of \newline \texttt{[0.90,0.925,0.95,0.975,1.00,1.025,1.05,1.075,1.10]} since orbital files have been pre-generated with \texttt{PWSCF} for only these values.  If other values are selected, the job will be submitted but \texttt{QMCPACK} will fail when it attempts to read the non-existent \texttt{O2.pwscf.h5} file (in later labs we will run \texttt{PWSCF} to generate the orbital files directly with Nexus).  Begin with the reduced set of \texttt{scale} values shown above.

Change the ``\texttt{status\_only}'' parameter in the ``\texttt{settings}'' function to \texttt{1} and type ``./O\_dimer.py'' at the command line.  This will print the status of all simulations:
\begin{shaded}
\begin{verbatim}
Project starting 
  checking for file collisions 
  loading cascade images 
    cascade 0 checking in 
  checking cascade dependencies 
    all simulation dependencies satisfied 
  cascade status 
    setup, sent_files, submitted, finished, got_output, analyzed 
    000000  opt  ./scale_1.0 
    000000  qmc  ./scale_1.0 
    000000  qmc  ./scale_0.9 
    000000  qmc  ./scale_0.95 
    000000  qmc  ./scale_1.05 
    000000  qmc  ./scale_1.1 
    setup, sent_files, submitted, finished, got_output, analyzed 
\end{verbatim}
\end{shaded}
\noindent
In this case, a single independent simulation ``cascade'' (workflow) has been identified, containing one ``\texttt{opt}'' and five dependent ``\texttt{qmc}'' runs.  The six status flags (\texttt{setup}, \texttt{sent\_files}, \texttt{submitted}, \texttt{finished}, \texttt{got\_output}, \texttt{analyzed}) each show \texttt{0}, indicating that no work has been done yet.  

Now change ``\texttt{status\_only}'' back to \texttt{0}, set ``\texttt{generate\_only}'' to \texttt{1}, and run \texttt{O\_dimer.py} again.  This will perform a dry-run of all simulations.  The dry-run should finish in about 20 seconds:
\begin{shaded}
\begin{verbatim}
Project starting 
  checking for file collisions 
  loading cascade images 
    cascade 0 checking in 
  checking cascade dependencies 
    all simulation dependencies satisfied 
  
  starting runs:
  ~~~~~~~~~~~~~~~~~~~~~~~~~~~~~~ 
  poll 0  memory 88.54 MB 
    Entering ./scale_1.0 0 
      writing input files  0 opt 
    Entering ./scale_1.0 0 
      sending required files  0 opt 
      submitting job  0 opt 
    Entering ./scale_1.0 1 
      Would have executed:  qsub --mode script --env BG_SHAREDMEMSIZE=32 opt.qsub.in 

  poll 1  memory 88.54 MB 
    Entering ./scale_1.0 0 
      copying results  0 opt 
    Entering ./scale_1.0 0 
      analyzing  0 opt 

  poll 2  memory 88.87 MB 
    Entering ./scale_1.0 1 
      writing input files  1 qmc 
    Entering ./scale_1.0 1 
      sending required files  1 qmc 
      submitting job  1 qmc 
    ...
    Entering ./scale_1.0 2 
      Would have executed:  qsub --mode script --env BG_SHAREDMEMSIZE=32 qmc.qsub.in 
    ...

Project finished
\end{verbatim}
\end{shaded}
\noindent
Nexus polls the simulation status every 3 seconds and sleeps in between.  The ``scale\_*'' directories should now contain several files:
\begin{shaded}
\begin{verbatim}
scale_1.0
├── O2.pwscf.h5
├── O.BFD.xml
├── opt.in.xml
├── opt.qsub.in
├── qmc.in.xml
├── qmc.qsub.in
├── sim_opt
│   ├── analyzer.p
│   ├── input.p
│   └── sim.p
└── sim_qmc
    ├── analyzer.p
    ├── input.p
    └── sim.p
\end{verbatim}
\end{shaded}
\noindent
Take a minute to inspect the generated input (\texttt{opt.in.xml}, \texttt{qmc.in.xml}) and submission (\texttt{opt.qsub.in}, \texttt{qmc.qsub.in}) files.  The pseudopotential file \texttt{O.BFD.xml} has been copied into each local directory. Two additional directories have been created: \texttt{sim\_opt} and \texttt{sim\_qmc}.  The \texttt{sim.p} files in each directory contain the current status of each simulation.  If you run \texttt{O\_dimer.py} again, it should not attempt to rerun any of the simulations:   
\begin{shaded}
\begin{verbatim}
Project starting 
  checking for file collisions 
  loading cascade images 
    cascade 0 checking in 
    cascade 8 checking in 
    cascade 2 checking in 
    cascade 4 checking in 
    cascade 6 checking in 
  checking cascade dependencies 
    all simulation dependencies satisfied 
  
  starting runs:
  ~~~~~~~~~~~~~~~~~~~~~~~~~~~~~~ 
  poll 0  memory 60.10 MB 
Project finished
\end{verbatim}
\end{shaded}
\noindent
This way one can continue to add to the \texttt{O\_dimer.py} file (\emph{e.g.} adding more separation distances) without worrying about duplicate job submissions.

Let's actually submit the optimization and DMC jobs now.  Reset the state of the simulations by removing the \texttt{sim.p} files (``\texttt{rm ./scale*/sim*/sim.p}''), set ``\texttt{generate\_only}'' to \texttt{0}, and rerun \texttt{O\_dimer.py}.  It should take about 15 minutes for all the jobs to complete.  You may wish to open another terminal to monitor the progress of the individual jobs while the current terminal runs \texttt{O\_dimer.py} in the foreground.  You can begin the first exercise below once the optimization job completes.

\vspace{3cm}
\begin{flushleft}
\textbf{\underline{Questions and Exercises}}
\end{flushleft}
\begin{enumerate}
  \item{Evaluate the quality of the optimization at \texttt{scale=1.0} using the \texttt{qmca} tool.  Did the optimization succeed?  How does the variance compare with the neutral oxygen atom?  Is the wavefunction of similar quality to the atomic case?}

  \item{Evaluate the traces of the local energy and the DMC walker population for each separation distance with the \texttt{qmca} tool.  Are there any anomalies in the runs?  Is the acceptance ratio reasonable?  Is the wavefunction of similar quality across all separation distances?}

  \item{Use the \texttt{dimer\_fit.py} tool located in \texttt{oxygen\_dimer} to fit the oxygen dimer binding curve.   To get the binding energy of the dimer, we will need the DMC energy of the atom.  Before performing the fit, answer: What DMC timestep should be used for the oxygen atom results?  The tool accepts three arguments (``\texttt{O\_dimer.py P N E Eerr}''}), \texttt{P} is the prefix of the DMC input files (should be ``\texttt{qmc}'' at this point), \texttt{N} is the order of the fit (use 2 to start), \texttt{E} and \texttt{Eerr} are your DMC total energy and error bar, respectively for the oxygen atom (in eV).  A plot of the dimer data will be displayed and text output will show the DMC equilibrium bond length and binding energy as well as experimental values.  How accurately does your fit to the DMC data reproduce the experimental values?  What factors affect the accuracy of your results? 

  \item{Refit your data with a fourth-order polynomial.  How do your predictions change with a fourth-order fit?  Is a fourth-order fit appropriate for the available data?}
 
  \item{Add the four remaining ``\texttt{scale}'' values to the list in \texttt{O\_dimer.py} that interpolate between the original set.  Perform the DMC calculations and redo the fits.  How accurately does your fit to the DMC data reproduce the experimental values?  Should this pseudopotential be used in production calculations?}

  \item{(Optional) Perform optimization runs at the extremal separation distances corresponding to \texttt{scale=[0.90,1.10]}}.  Are the individually optimized wavefunctions of significantly better quality than the one imported from \texttt{scale=1.00}?  Why?  What form of Jastrow factor might give an even better improvement? 
\end{enumerate}




\section{(Optional) Running your system with QMCPACK}\label{sec:your_system}
This section covers a fairly simple route to get started on QMC calculations of an arbitrary system of interest using Nexus (Nexus) automation system to setup input files and optionally perform the runs.  The example provided in this section uses QM Espresso (PWSCF) to generate the orbitals forming the Slater determinant part of the trial wavefunction.  PWSCF is a natural choice for solid state systems and it can be used for surface/slab and molecular systems as well, albeit at the price of describing additional vacuum space with plane waves.

To start out with, you will need pseudopotentials (PP's) for each element in your system in both the UPF (PWSCF) and FSATOM/XML (QMCPACK) formats.  A good place to start is the Burkatzki-Filippi-Dolg (BFD) pseudopotential database \newline (\href{http://www.burkatzki.com/pseudos/index.2.html}{http://www.burkatzki.com/pseudos/index.2.html}), which we have already used in our study of the oxygen atom.  The database does not contain PP's for the 4th and 5th row transition metals or any of the lanthanides or actinides.  If you need a PP that is not in the BFD database, you may need to generate and test one manually (\emph{e.g.} with OPIUM, \href{http://opium.sourceforge.net/}{http://opium.sourceforge.net/}).  Otherwise, use \texttt{ppconvert} as outlined in section \ref{sec:pseudo} to obtain PP's in the formats used by PWSCF and QMCPACK.  Enter the \texttt{your\_system} lab directory and place the converted PP's in \texttt{your\_system/pseudopotentials}.

Before performing production calculations (more than just the initial setup in this section) be sure to converge the plane wave energy cutoff in PWSCF as these PP's can be rather hard, sometimes requiring cutoffs in excess of 300 Ry.  Depending on the system under study, the amount of memory required to represent the orbitals (QMCPACK uses 3D B-splines) becomes prohibitive and one may be forced to search for softer PP's.

Beyond pseudopotentials, all that is required to get started are the atomic positions and the dimensions/shape of the simulation cell.  The Nexus file \texttt{example.py} illustrates how to setup PWSCF and QMCPACK input files by providing minimal information regarding the physical system (an 8-atom cubic cell of diamond in the example).  Most of the contents should be familiar from your experience with the automated calculations of the oxygen dimer binding curve in section \ref{sec:dimer_automation} (if you've skipped ahead you may want to skim that section for relevant information).  The most important change is the expanded description of the physical system:

\begin{shaded}
\begin{verbatim}
# details of your physical system (diamond conventional cell below)
my_project_name = 'diamond_vmc'   # directory to perform runs
my_dft_pps      = ['C.BFD.upf']   # pwscf pseudopotentials
my_qmc_pps      = ['C.BFD.xml']   # qmcpack pseudopotentials

#  generate your system
#    units      :  'A'/'B' for Angstrom/Bohr
#    axes       :  simulation cell axes in cartesian coordinates (a1,a2,a3)
#    elem       :  list of atoms in the system
#    pos        :  corresponding atomic positions in cartesian coordinates
#    kgrid      :  Monkhorst-Pack grid
#    kshift     :  Monkhorst-Pack shift (between 0 and 0.5)
#    net_charge :  system charge in units of e
#    net_spin   :  # of up spins - # of down spins
#    C = 4      :  (pseudo) carbon has 4 valence electrons
my_system = generate_physical_system(
    units      = 'A',
    axes       = [[ 3.57000000e+00, 0.00000000e+00, 0.00000000e+00],
                  [ 0.00000000e+00, 3.57000000e+00, 0.00000000e+00],
                  [ 0.00000000e+00, 0.00000000e+00, 3.57000000e+00]],
    elem       = ['C','C','C','C','C','C','C','C'],
    pos        = [[ 0.00000000e+00, 0.00000000e+00, 0.00000000e+00],
                  [ 8.92500000e-01, 8.92500000e-01, 8.92500000e-01],
                  [ 0.00000000e+00, 1.78500000e+00, 1.78500000e+00],
                  [ 8.92500000e-01, 2.67750000e+00, 2.67750000e+00],
                  [ 1.78500000e+00, 0.00000000e+00, 1.78500000e+00],
                  [ 2.67750000e+00, 8.92500000e-01, 2.67750000e+00],
                  [ 1.78500000e+00, 1.78500000e+00, 0.00000000e+00],
                  [ 2.67750000e+00, 2.67750000e+00, 8.92500000e-01]],
    kgrid      = (1,1,1),
    kshift     = (0,0,0),
    net_charge = 0,
    net_spin   = 0,
    C          = 4       # one line like this for each atomic species
    )

my_bconds       = 'ppp'  #  ppp/nnn for periodic/open BC's in QMC
                         #  if nnn, center atoms about (a1+a2+a3)/2
\end{verbatim}
\end{shaded}

If you have a system you would like to try with QMC, make a copy of \texttt{example.py} and fill in the relevant information about the pseudopotentials, simulation cell axes, and atomic species/positions.  Otherwise, you can proceed with \texttt{example.py} as it is.

The other new aspects are two additional compute jobs to generate the orbitals with PWSCF and convert them into the ESHDF format with \texttt{pw2qmcpack.x}:

\begin{shaded}
\begin{verbatim}
# scf run to generate orbitals
scf = generate_pwscf(
    identifier   = 'scf',
    path         = my_project_name,
    job          = Job(nodes=32,hours=2,app=pwscf),
    input_type   = 'scf',
    system       = my_system,
    pseudos      = my_dft_pps,
    input_dft    = 'lda', 
    ecut         = 200,   # PW energy cutoff in Ry
    conv_thr     = 1e-8, 
    mixing_beta  = .7,
    nosym        = True,
    wf_collect   = True
    )

# conversion step to create h5 file with orbitals
p2q = generate_pw2qmcpack(
    identifier   = 'p2q',
    path         = my_project_name,
    job          = Job(cores=1,hours=2,app=pw2qmcpack),
    write_psir   = False,
    dependencies = (scf,'orbitals')
    )
\end{verbatim}
\end{shaded}

Set ``\texttt{generate\_only}'' to \texttt{1} and type ``\texttt{./example.py}'' or similar to generate the input files.  All files will be written to ``\texttt{./diamond\_vmc}'' (``\texttt{./[my\_project\_name]}'' if you have changed ``\texttt{my\_project\_name}'' in the file).  The input files for PWSCF, pw2qmcpack, and QMCPACK are \texttt{scf.in}, \texttt{pw2qmcpack.in}, and \texttt{vmc.in.xml}, repectively.  Take some time to inspect the generated input files.  If you have questions about the file contents, or run into issues with the generation process, feel free to consult with a lab instructor.  

If desired, you can submit the runs directly with \texttt{example.py}.  To do this, first reset the Nexus simulation record by typing ``\texttt{rm ./diamond\_vmc/sim*/sim.p}'' or similar and set ``\texttt{generate\_only}'' back to \texttt{0}.  Next rerun \texttt{example.py}  (you may want to redirect the text output).  

Alternatively the runs can be submitted by hand:
\begin{shaded}
\begin{verbatim}
qsub --mode script --env BG_SHAREDMEMSIZE=32 scf.qsub.in

(wait until JOB DONE appears in scf.output)

qsub --mode script --env BG_SHAREDMEMSIZE=32 p2q.qsub.in
\end{verbatim}
\end{shaded}
Once the conversion process has finished the orbitals should be located in the file \texttt{diamond\_vmc/pwscf\_output/pwscf.pwscf.h5}.  Open \texttt{diamond\_vmc/vmc.in.xml} and replace ``\texttt{MISSING.h5}'' with ``\texttt{./pwscf\_output/pwscf.pwscf.h5}''.  Next submit the VMC run:
\begin{shaded}
\begin{verbatim}
qsub --mode script --env BG_SHAREDMEMSIZE=32 vmc.qsub.in
\end{verbatim}
\end{shaded}

Note: If your system is large, the above process may not complete within the time frame of this lab.  Working with a stripped down (but relevant) example is a good idea for exploratory runs.

Once the runs have finished, you may want to begin exploring Jastrow optimization and DMC for your system.  Example calculations are provided at the end of \texttt{example.py} in the commented out text).



%\section{(Optional) Revisiting the oxygen atom: 3-body Jastrow \& population control bias}

%\subsection{Optimization of 3-body Jastrow factors}

%\subsection{Investigation of DMC population control bias}




% cover basic python elsewhere in the manual?  refer to Nexus user guide or websites instead?
\hide{
\appendix

\section{Basic Python constructs\label{app:python_basics}}
Basic Python data types (\texttt{int}, \texttt{float}, \texttt{str}, \texttt{tuple}, \texttt{list}, \texttt{array}, \texttt{dict}, \texttt{obj}) and programming constructs (\texttt{if} statements, \texttt{for} loops, functions w/ keyword arguments) are briefly overviewed below.  All examples can be executed interactively in Python.  To do this, type ``\texttt{python}'' at the command line and paste any of the shaded text below at the ``\texttt{>>>}'' prompt.  For more information about effective use of Python, consult the detailed online documentation: \href{https://docs.python.org/2/}{https://docs.python.org/2/}.

\subsubsection{Intrinsic types: \texttt{int, float, str}}
\begin{shaded}
\begin{verbatim}
#this is a comment
i=5                     # integer
f=3.6                   # float
s='quantum/monte/carlo' # string
n=None                  # represents "nothing"

f+=1.4                  # add-assign (-,*,/ also): 5.0
2**3                    # raise to a power: 8
str(i)                  # int to string: '5'
s+'/simulations'        # joining strings: 'quantum/monte/carlo/simulations'
'i={0}'.format(i)       # format string: 'i=5'

\end{verbatim}
\end{shaded}

 
\subsubsection{Container types: \texttt{tuple, list, array, dict, obj}}
\begin{shaded}
\begin{verbatim}
from numpy import array  # get array from numpy module
from generic import obj  # get obj from generic module

t=('A',42,56,123.0)     # tuple

l=['B',3.14,196]        # list

a=array([1,2,3])        # array

d={'a':5,'b':6}         # dict

o=obj(a=5,b=6)          # obj

                        # printing
print t                 #  ('A', 42, 56, 123.0)
print l                 #  ['B', 3.1400000000000001, 196]
print a                 #  [1 2 3]
print d                 #  {'a': 5, 'b': 6}
print o                 #    a               = 5
                        #    b               = 6

len(t),len(l),len(a),len(d),len(o) #number of elements: (4, 3, 3, 2, 2)

t[0],l[0],a[0],d['a'],o.a  #element access: ('A', 'B', 1, 5, 5)

s = array([0,1,2,3,4])  # slices: works for tuple, list, array
s[:]                    #   array([0, 1, 2, 3, 4])
s[2:]                   #   array([2, 3, 4])
s[:2]                   #   array([0, 1])
s[1:4]                  #   array([1, 2, 3])
s[0:5:2]                #   array([0, 2, 4])

                        # list operations
l2 = list(l)            #   make independent copy
l.append(4)             #   add new element: ['B', 3.14, 196, 4]
l+[5,6,7]               #   addition: ['B', 3.14, 196, 4, 5, 6, 7]
3*[0,1]                 #   multiplication:  [0, 1, 0, 1, 0, 1]

b=array([5,6,7])        # array operations
a2 = a.copy()           #   make independent copy
a+b                     #   addition: array([ 6, 8, 10])
a+3                     #   addition: array([ 4, 5, 6])
a*b                     #   multiplication: array([ 5, 12, 21])
3*a                     #   multiplication: array([3, 6, 9])

                        # dict/obj operations
d2 = d.copy()           #   make independent copy
d['c'] = 7              #   add/assign element 
d.keys()                #   get element names: ['a', 'c', 'b']
d.values()              #   get element values: [5, 7, 6]

                        # obj-specific operations
o.c = 7                 #   add/assign element
o.set(c=7,d=8)          #   add/assign multiple elements

\end{verbatim}
\end{shaded}
An important feature of Python to be aware of is that assignment is most often by reference, \emph{i.e.} new values are not always created.  This point is illustrated below with an \texttt{obj} instance, but it also holds for \texttt{list}, \texttt{array}, \texttt{dict}, and others.
\begin{shaded}
\begin{verbatim}
>>> o = obj(a=5,b=6)
>>> 
>>> p=o
>>> 
>>> p.a=7
>>> 
>>> print o
  a               = 7
  b               = 6

>>> q=o.copy()
>>> 
>>> q.a=9
>>> 
>>> print o
  a               = 7
  b               = 6
\end{verbatim}
\end{shaded}
\noindent
Here \texttt{p} is just another name for \texttt{o}, while \texttt{q} is a fully independent copy of it.


\subsubsection{Conditional Statements: \texttt{if/elif/else}}
\begin{shaded}
\begin{verbatim}
a = 5
if a is None:
    print 'a is None'
elif a==4:
    print 'a is 4'
elif a<=6 and a>2:
    print 'a is in the range (2,6]'
elif a<-1 or a>26:
    print 'a is not in the range [-1,26]'
elif a!=10: 
    print 'a is not 10'
else:
    print 'a is 10'
#end if

\end{verbatim}
\end{shaded}
The ``\texttt{\#end if}'' is not part of Python syntax, but you will see text like this throughout Nexus for clear encapsulation.

\subsubsection{Iteration: \texttt{for}}
\begin{shaded}
\begin{verbatim}
from generic import obj

l = [1,2,3]              
m = [4,5,6]
s = 0
for i in range(len(l)):  # loop over list indices
    s += l[i] + m[i]
#end for

print s                  # s is 21

s = 0                    
for v in l:              # loop over list elements
    s += v
#end for

print s                  # s is 6

o = obj(a=5,b=6)
s = 0
for v in o:              # loop over obj elements
    s += v
#end for

print s                  # s is 11

d = {'a':5,'b':4}
for n,v in o.iteritems():# loop over name/value pairs in obj
    d[n] += v
#end for

print d                  # d is {'a': 10, 'b': 10}

\end{verbatim}
\end{shaded}


\subsubsection{Functions: \texttt{def}, argument syntax}
\begin{shaded}
\begin{verbatim}
def f(a,b,c=5):          # basic function, c has a default value
    print a,b,c
#end def f

f(1,b=2)                 # prints: 1 2 5


def f(*args,**kwargs):   # general function, returns nothing
    print args           #     args: tuple of positional arguments
    print kwargs         #   kwargs: dict of keyword arguments
#end def f

f('s',(1,2),a=3,b='t')   # 2 pos., 2 kw. args, prints:
                         #   ('s', (1, 2))
                         #   {'a': 3, 'b': 't'}

l = [0,1,2]
f(*l,a=6)                # pos. args from list, 1 kw. arg, prints:
                         #   (0, 1, 2)
                         #   {'a': 6}
o = obj(a=5,b=6)
f(*l,**o)                # pos./kw. args from list/obj, prints:
                         #   (0, 1, 2)
                         #   {'a': 5, 'b': 6}

f(                       # indented kw. args, prints
    blocks   = 200,      #   () 
    steps    = 10,       #   {'steps': 10, 'blocks': 200, 'timestep': 0.01}
    timestep = 0.01
    )

o = obj(                 # obj w/ indented kw. args
    blocks   = 100,
    steps    =  5,
    timestep = 0.02
    )

f(**o)                   # kw. args from obj, prints:
                         #   ()
                         #   {'timestep': 0.02, 'blocks': 100, 'steps': 5}
\end{verbatim}
\end{shaded}
}

\chapter{Lab 3: Advanced molecular calculations}
\label{chap:lab_advanced_molecules}

\section{Topics covered in this lab}
This lab covers molecular QMC calculations with wavefunctions of increasing sophistication.  All of the trial wavefunctions are initially generated with the GAMESS code.  Topics covered include:
\begin{itemize}
  \item{Generating single-determinant trial wavefunctions with GAMESS (HF and DFT)}
  \item{Generating multideterminant trial wavefunctions with GAMESS (CISD, CASCI, and SOCI)}
  \item{Optimizing wavefunctions (Jastrow factors and CSF coefficients) with QMC}
  \item{DMC time step and walker population convergence studies}
  \item{Systematic progressions of Jastrow factors in VMC}
  \item{Systematic convergence of DMC energies with multideterminant wavefunctions}
  \item{Influence of orbitals basis choice on DMC energy}
  %\item{Influence of orbitals basis choice on rate of multi-determinant DMC convergence}
\end{itemize}

\section{Lab directories and files}
\footnotesize
\begin{verbatim}
labs/lab3_advanced_molecules/exercises
│    
├── ex1_first-run-hartree-fock    - basic work flow from Hatree-Fock to DMC
│   ├── gms                        - Hatree-Fock calculation using GAMESS
│   │   ├── h2o.hf.inp               - GAMESS input
│   │   ├── h2o.hf.dat               - GAMESS punch file containing orbitals
│   │   └── h2o.hf.out               - GAMESS output with orbitals and other info
│   ├── convert                    - Convert GAMESS wavefunction to QMCPACK format
│   │   ├── h2o.hf.out               - GAMESS output
│   │   ├── h2o.ptcl.xml             - converted particle positions
│   │   └── h2o.wfs.xml              - converted wave function
│   ├── opt                        - VMC optimization
│   │   └── optm.xml                 - QMCPACK VMC optimization input
│   ├── dmc_timestep               - Check DMC timestep bias
│   │   └── dmc_ts.xml               - QMCPACK DMC input
│   └── dmc_walkers                - Check DMC population control bias
│       └── dmc_wk.xml               - QMCPACK DMC input template
│   
├── ex2_slater-jastrow-wf-options - explore jastrow and orbital options
│   ├── jastrow                    - Jastrow options
│   │   ├── 12j                      - no 3-body Jastrow
│   │   ├── 1j                       - only 1-body Jastrow
│   │   └── 2j                       - only 2-body Jastrow
│   └── orbitals                   - Orbital options
│       ├── pbe                      - PBE orbitals
│       │   └── gms                    - DFT calculation using GAMESS
│       │      └── h2o.pbe.inp          - GAMESS DFT input
│       ├── pbe0                     - PBE0  orbitals
│       ├── blyp                     - BLYP  orbitals
│       └── b3lyp                    - B3LYP orbitals
│       
├── ex3_multi-slater-jastrow
│   ├── cisd                      - CISD wave function
│   │   ├── gms                     - CISD calculation using GAMESS
│   │   │   ├── h2o.cisd.inp           - GAMESS input
│   │   │   ├── h2o.cisd.dat           - GAMESS punch file containing orbitals
│   │   │   └── h2o.cisd.out           - GAMESS output with orbitals and other info
│   │   └── convert                 - Convert GAMESS wavefunction to QMCPACK format
│   │      └── h2o.hf.out             - GAMESS output
│   ├── casci                     - CASCI wave function
│   │   └── gms                     - CASCI calculation using GAMESS
│   └── soci                      - SOCI wave function
│       ├── gms                     - SOCI calculation using GAMESS
│       ├── thres0.01               - VMC optimization with few determinants
│       └── thres0.0075             - VMC optimization with more determinants
│   
└── pseudo
    ├── H.BFD.gamess             - BFD pseudopotential for H in GAMESS format
    ├── O.BFD.CCT.gamess         - BFD pseudopotential for O in GAMESS format
    ├── H.xml                    - BFD pseudopotential for H in QMCPACK format
    └── O.xml                    - BFD pseudopotential for H in QMCPACK format
\end{verbatim}
\normalsize

\section{Exercise \#1: Basics}

The purpose of this exercise is to show how to generate wavefunctions for QMCPACK
using GAMESS and to optimize the resulting wavefunctions using VMC. This will be
followed by a study of the time step and walker population dependence of DMC energies.
The exercise will be performed on a water molecule at the equilibrium geometry.


\subsection{Generation of a Hartree-Fock wavefunction with GAMESS}

From the top directory, go to ``\texttt{ex1\_first-run-hartree-fock/gms}.'' This directory contains an input
file for a HF calculation of a water molecule using BFD ECPs and the corresponding
cc-pVTZ basis set. The input file should be named: “h2o.hf.inp.” Study the input
file. See Section~\ref{sec:lab_adv_mol_gamess}, ``Appendix A: GAMESS input" for a more detailed description of the
GAMESS input syntax. However, there will be a better time to do this soon, so we recommend
continuing with the exercise at this point. After you are done, execute
GAMESS with this input and store the standard output in a file named ``h2o.hf.output.''
Finally, in the ``convert'' folder, use \texttt{convert4qmc} to generate the QMCPACK \texttt{particleset} and \texttt{wavefunction} files. It
is always useful to rename the files generated by \texttt{convert4qmc} to something meaningful
since by default they are called \texttt{sample.Gaussian-G2.xml} and \texttt{sample.Gaussian-G2.ptcl.xml}.
In a standard computer (without cross-compilation), these tasks can be accomplished by
the following commands.
\begin{lstlisting}[style=SHELL]
cd ${TRAINING TOP}/ex1_first-run-hartree-fock/gms
jobrun_vesta rungms h2o.hf 
cd ../convert
cp ../gms/h2o.hf.output
jobrun_vesta convert4qmc -gamessAscii h2o.hf.output -add3BodyJ
mv sample.Gaussian-G2.xml h2o.wfs.xml
mv sample.Gaussian-G2.ptcl.xml h2o.ptcl.xml
\end{lstlisting}
\noindent

%Due to the particular requirements of Vesta these executions can not be performed on
%the login nodes, they must be performed in the compute nodes. In order to accomplish
%this, we must make a submission script with the appropriate commands and submit it to
%the batch system. Two submission scripts have been provided, one for the GAMESS 
%execution called submit gamess.csh and one for the submission of convert4qmc, with a similar
%descriptive name. Study these input files. (In this and all other exercises, you will need
%to make all submission scripts executables with the command chmod u+x script.csh.)
%When you are ready, start by submitting the GAMESS execution to the batch system
%using ``./script gamess.csh'' (This script calls the GAMESS run script, which itself calls
%qsub. Do not attempt to submit this specific script with qsub). 
The HF energy of the
system is -16.9600590022 Ha. To search for the energy in the output file quickly, you can
use 
\begin{shade}
grep "TOTAL ENERGY =" h2o.hf.output
\end{shade}
%When the calculation completes,
%submit the execution of the converter using ``qsub submit convert.csh'' (all QMCPACK execution 
%scripts will be submitted with qsub). This is a good time to review section B, which
%contains a description on the use of the converter.
As the job runs on VESTA, it is a good time to review Section~\ref{sec:lab_adv_mol_convert4qmc}, ``Appendix B: convert4qmc," which contains a description on the use of the converter.


\subsection{Optimize the wavefunction}
When execution of the previous steps is completed, there should be two new
files called \texttt{h2o.wfs.xml} and \texttt{h2o.ptcl.xml}. Now we will use VMC to optimize the 
Jastrow parameters in the wavefunction.  From the top directory, go to
``\texttt{ex1\_first-run-hartree-fock/opt}.'' Copy the xml files generated in the previous step
to the current directory. This directory should already contain a basic QMCPACK input
file for an optimization calculation (\texttt{optm.xml}) %and a submission script (submit.csh). 
Open \texttt{optm.xml} with your favorite text editor and modify the name of the files that contain the
\texttt{wavefunction} and \texttt{particleset} XML blocks. These files are included with the commands:
\begin{lstlisting}[style=QMCPXML]
<include href=ptcl.xml/>
<include href=wfs.xml/>
\end{lstlisting}
(the particle set must be
defined before the wavefunction). The name of the particle set and wavefunction files should now be \texttt{h2o.ptcl.xml}
and \texttt{h2o.wfs.xml}, respectively. Study both files and submit when you are ready. Notice that the
location of the ECPs has been set for you; in your own calculations you have to make
sure you obtain the ECPs from the appropriate libraries and convert them to QMCPACK
format using ppconvert. While these calculations finish is a good time to study Section~\ref{sec:lab_adv_mol_opt_appendix}, ``Appendix C: Wavefunction optimization XML block," which contains a review
of the main parameters in the optimization XML block.  The
previous steps can be accomplished by the following commands:
\begin{shade}
cd ${TRAINING TOP}/ex1_first-run-hartree-fock/opt
cp ../convert/h2o.wfs.xml ./
cp ../convert/h2o.ptcl.xml ./
# edit optm.xml to include the correct ptcl.xml and wfs.xml
jobrun_vesta qmcpack optm.xml
\end{shade}

Use the analysis tool \texttt{qmca} to analyze the results of the calculation. Obtain the VMC
energy and variance for each step in the optimization and plot it using your favorite program.
Remember that \texttt{qmca} has built-in functions to plot the analyzed data.
\begin{shade}
qmca -q e *scalar.dat -p
\end{shade}

\begin{figure}
\begin{center}
\includegraphics[trim = 0mm 0mm 0mm 0mm, clip,width=0.75\columnwidth]{./figures/lab_advanced_molecules_opt_conv.png}
\end{center}
\caption{VMC energy as a function of optimization step.}
\label{fig:lam_opt_conv}
\end{figure}

The resulting energy as a function of the optimization step should look qualitatively similar to Figure~\ref{fig:lam_opt_conv}.
The energy should decrease quickly as a function of the number of optimization steps. After 6--8 steps, the energy should be converged to $\sim$2--3 mHa. To improve convergence,
we would need to increase the number of samples used during optimization (You can
check this for yourself later.). With optimized wavefunctions, we are in a position
to perform VMC and DMC calculations. The modified wavefunction files after each step
are written in a file named \texttt{ID.sNNN.opt.xml}, where ID is the identifier of the calculation
defined in the input file (this is defined in the project XML block with parameter “id”) and
NNN is a series number that increases with every executable xml block in the input file.


\subsection{Time-step study}
Now we will study the dependence of the DMC energy with time step. From the top directory, 
go to “\texttt{ex1\_first-run-hartree-fock/dmc\_timestep}.” This folder contains a basic XML input
file (\texttt{dmc\_ts.xml}) that performs a short VMC calculation and three DMC calculations
with varying time steps (0.1, 0.05, 0.01). Link the \texttt{particleset} and the last \texttt{optimization}
file from the previous folder (the file called \texttt{jopt-h2o.sNNN.opt.xml} with the largest value of
NNN). Rename the optimized \texttt{wavefunction} file to any suitable name if you wish (for example,
\texttt{h2o.opt.xml}) and change the name of the \texttt{particleset} and \texttt{wavefunction} files in the
input file. An optimized wavefunction can be found in the reference files (same location)
in case it is needed. %Using the submission script of the previous exercise as a base, create a
%submission script for this step and submit the run. Set the number of nodes to 32 (2 places
%must be changed), the number of threads to 16 and leave the number of tasks at 1.

The main steps needed to perform this exercise are:
\begin{lstlisting}[style=SHELL]
cd \$\{TRAINING TOP\}/ex1_first-run-hartree-fock/dmc_timestep
cp ../opt/h2o.ptcl.xml ./
cp ../opt/jopt-h2o.s007.opt.xml h2o.opt.wfs.xml
# edit dmc_ts.xml to include the correct ptcl.xml and wfs.xml
jobrun_vesta qmcpack dmc_ts.xml
\end{lstlisting}
While these runs complete, go to Section\ref{sec:lab_adv_mol_vmcdmc_appendix}, ``Appendix D: VMC and DMC XML block," and review the basic VMC and DMC input
blocks. Notice that in the current DMC blocks the time step is decreased as the number of blocks is increased. Why is this?

When the simulations are finished, use \texttt{qmca} to analyze the output files and plot the
DMC energy as a function of time step. Results should be qualitatively similar to those
presented in Figure~\ref{fig:lam_dmc_timestep}; in this case we present more time steps with well converged results to
better illustrate the time step dependence. In realistic calculations, the time step must be
chosen small enough so that the resulting error is below the desired accuracy. Alternatively,
various calculations can be performed and the results extrapolated to the zero time-step
limit.


\begin{figure}
\begin{center}
\includegraphics[trim = 0mm 0mm 0mm 0mm, clip,width=0.75\columnwidth]{./figures/lab_advanced_molecules_dmc_timestep.png}
\end{center}
\caption{DMC energy as a function of time step.}
\label{fig:lam_dmc_timestep}
\end{figure}


\subsection{Walker population study}
Now we will study the dependence of the DMC energy with the number of walkers in the
simulation. Remember that, in principle, the DMC distribution is reached in the limit of
an infinite number of walkers. In practice, the energy and most properties converge to high
accuracy with $\sim$100--1,000 walkers. The actual number of walkers needed in a calculation
will depend on the accuracy of the VMC wavefunction and on the complexity and size of
the system. Also notice that using too many walkers is not a problem; at worse it will be
inefficient since it will cost more computer time than necessary. In fact, this is the strategy
used when running QMC calculations on large parallel computers since we can reduce the
statistical error bars efficiently by running with large walker populations distributed across
all processors.

From the top directory, go to ``\texttt{ex1\_first-run-hartree-fock/dmc\_walkers}.'' Copy the
optimized \texttt{wavefunction} and \texttt{particleset} files used in the previous calculations to the current
folder; these are the files generated during step 2 of this exercise. An optimized \texttt{wavefunction} file can be found in the reference files (same location) in case it is needed. The directory
contains a sample DMC input file and submission script. Create three  directories named NWx,
with x values of 120,240,480, and copy the input file to each one. Go
to ``NW120,'' and, in the input file, change the name of the \texttt{wavefunction} and \texttt{particleset}
files (in this case they will be located one directory above, so use ``\texttt{../dmc\_timestep/h2.opt.xml},'' for
example); change the PP directory so that it points to one directory above; change ``targetWalkers'' to 120; and change the number of steps to 100, the time step
to 0.01, and the number of blocks to 400. Notice that ``targetWalkers'' is one way to set the desired (average) number of walkers in a DMC calculation. One can alternatively set ``samples'' in the \ixml{<qmc method="vmc"} block to carry over de-correlated VMC configurations as DMC walkers. For your own simulations, we generally recommend setting $\sim$2*(\#threads)
walkers per node (slightly smaller than this value).

The main steps needed to perform this exercise are
\begin{shade}
cd ${TRAINING TOP}/ex1_first-run-hartree-fock/dmc_walkers
cp ../opt/h2o.ptcl.xml ./
cp ../opt/jopt-h2o.s007.opt.xml h2o.opt.wfs.xml
# edit dmc_wk.xml to include the correct ptcl.xml and wfs.xml and 
#  use the correct pseudopotential directory
mkdir NW120
cp dmc_wk.xml NW120
# edit dmc_wk.xml to use the desired number of walkers, 
#  and collect the desired amount of statistics
jobrun_vesta qmcpack dmc_wk.xml
# repeat for NW240, NW480
\end{shade}

\begin{figure}
\begin{center}
\includegraphics[trim = 0mm 0mm 0mm 0mm, clip,width=0.75\columnwidth]{./figures/lab_advanced_molecules_dmc_popcont.png}
\end{center}
\caption{DMC energy as a function of the average number of walkers.}
\label{fig:lam_dmc_popcont}
\end{figure}

Repeat the same procedure in the other folders by setting (targetWalkers=240,
steps=100, timestep=0.01, blocks=200) in NW240 and (targetWalkers=480, 
steps=100, timestep=0.01, blocks=100) in NW480. When
the simulations complete, use \texttt{qmca} to analyze and plot the energy as a function of the
number of walkers in the calculation. As always, Figure~\ref{fig:lam_dmc_popcont} 
shows representative results of the
energy dependence on the number of walkers for a single water molecule. As shown,
less than 240 walkers are needed to obtain an accuracy of 0.1 mHa.


\section{Exercise \#2 Slater-Jastrow wavefunction options}
From this point on in the tutorial we assume familiarity with the basic parameters in the
optimization, VMC, and DMC XML input blocks of QMCPACK. In addition, we assume
familiarity with the submission system. As a result, the folder structure will not contain
any prepared input or submission files, so you will need to generate them using 
input files from exercise 1. In the case of QMCPACK sample 
files, you will find \texttt{optm.xml}, \texttt{vmc dmc.xml}, and \texttt{submit.csh files}. Some of
the options in these files can be left unaltered, but many of them will need to be tailored to
the particular calculation.

In this exercise we will study the dependence of the DMC energy on the choices made
in the wavefunction ansatz. In particular, we will study the influence/dependence of the
VMC energy with the various terms in the Jastrow. We will also study the influence of
the VMC and DMC energies on the SPOs used to form the Slater determinant 
in single-determinant wavefunctions. For this we will use wavefunctions generated
with various exchange-correlation functionals in DFT. Finally, we will optimize a simple
multideterminant wavefunction and study the dependence of the energy on the number of
configurations used in the expansion. All of these exercises will be performed on the water 
molecule at equilibrium.


\subsection{Influence of Jastrow on VMC energy with HF wavefunction}
In this section we will study the dependence of the VMC energy on the various Jastrow
terms (e.g., 1-body, 2-body and 3-body. From the top directory, go to ``\texttt{ex2\_slater-jastrow-wf-options/jastrow.''} 
We will compare the single-determinant VMC energy using a 2-body 
Jastrow term, both 1- and 2-body terms, and finally 1-, 2- and 3-body
terms. Since we are interested in the influence of the Jastrow, we will use the HF orbitals
calculated in exercise \#1. Make three folders named 2j, 12j, and 123j. For both 2j and
12j, %(we have already optimized a wave-function for the 1-2-3J case, so the steps will be
%slightly different in this case)
 copy the input file \texttt{optm.xml} %and the sample submission file
from ``\texttt{ex1\_first-run-hartree-fock/opt.}'' This input file performs both wavefunction optimization 
and a VMC calculation. Remember to correct relative paths to the PP directory. Copy the un-optimized HF \texttt{wavefunction} and \texttt{particleset} files
from ``\texttt{ex1\_first-run-hartree-fock/convert}''; if you followed the instructions in exercise \#1 these should be
named \texttt{h2o.wfs.xml} and \texttt{h2o.ptcl.xml}. Otherwise, you can obtained them from the
REFERENCE files. Modify the \texttt{h2o.wfs.xml} file to remove the appropriate Jastrow
blocks. For example, for a 2-body Jastrow (only), you need to eliminate the Jastrow
blocks named \ixml{<jastrow name="J1"} and \ixml{<jastrow name="J3."} In the case of 12j, remove
only \ixml{<jastrow name="J3."} Recommended settings for the optimization run are nodes=32,
threads=16, blocks=250, samples=128000, time-step=0.5, 8 optimization loops. Recommended settings in the
VMC section are walkers=16, blocks=1000, steps=1, substeps=100. Notice that
samples should always be set to blocks*threads per node*nodes = 32*16*250=128000. Repeat 
the process in both 2j and 12j cases. For the 123j case, the wavefunction has
already been optimized in the previous exercise. Copy the optimized HF wavefunction and
the particleset from ``\texttt{ex1\_first-run-hartree-fock/opt.}'' Copy the input file from any of the previous runs and remove the optimization block from the
input, just leave the VMC step. In all three cases, modify the submission script and submit the run.

\begin{figure}
\begin{center}
\includegraphics[trim = 0mm 0mm 0mm 0mm, clip,width=0.75\columnwidth]{./figures/lab_advanced_molecules_vmc_jastrow.png}
\end{center}
\caption{VMC energy as a function of Jastrow type.}
\label{fig:lam_vmc_jastrow}
\end{figure}

Because these simulations will take several minutes to complete, this is an excellent opportunity
to go to Section~\ref{sec:lab_adv_mol_wf_appendix}, ``Appendix E: Wavefunction XML block," and review the wavefunction XML block used by QMCPACK. When the
simulations are completed, use \texttt{qmca} to analyze the output files. Using your favorite plotting
program (e.g., gnu plot), plot the energy and variance as a function of the Jastrow form.
Figure~\ref{fig:lam_vmc_jastrow} shows a typical result for this calculation. As can be seen, the VMC energy and
variance depends strongly on the form of the Jastrow. Since the DMC error bar is directly
related to the variance of the VMC energy, improving the Jastrow will always lead to a
reduction in the DMC effort. In addition, systematic approximations (time step, number of
walkers, etc.) are also reduced with improved wavefunctions.


\subsection{Generation of wavefunctions from DFT using GAMESS}
In this section we will use GAMESS to generate wavefunctions for QMCPACK from
DFT calculations. From the top folder, go to ``\texttt{ex2\_slater-jastrow-wf-options/orbitals}.'' To demonstrate
the variation in DMC energies with the choice of DFT orbitals, we will choose the following
set of exchange-correlation functionals (PBE, PBE0, BLYP, B3LYP). For each functional,
make a directory using your preferred naming convention (e.g., the name of the functional).
Go into each folder and copy a GAMESS input file from %for a ROHF calculation from
``\texttt{ex1\_first-run-hartree-fock/gms}.'' %, a file named rohf.inp should exist.
 Rename the file with your preferred naming convention; we suggest using \texttt{h2o.[dft].inp}, where [dft] is the name of
the functional used in the calculation. At this point, this input file should be identical to the
one used to generate the HF wavefunction in exercise \#1. To perform a DFT
calculation we only need to add ``DFTTYP'' to the \igamess{\$CONTRL ... \$END} section and set
it to the desired functional type, for example, ``DFTTYP=PBE'' for a PBE functional. This
variable must be set to (PBE, PBE0, BLYP, B3LYP) to obtain the appropriate functional in
GAMESS. For a complete list of implemented functionals, see the GAMESS input manual.


\subsection{Optimization and DMC calculations with DFT wavefunctions}
In this section we will optimize the wavefunction generated in the previous step and
perform DMC calculations. From the top directory, go to “\texttt{ex2\_slater-jastrow-wf-options/orbitals}.”
The steps required to achieve this are identical to those used to optimize the wavefunction
with HF orbitals. Make individual folders for each calculation and obtain the necessary files
to perform optimization, for example, VMC and DMC calculations from ``for \texttt{ex1\_first-run-hartree-fock/opt}'' and ``\texttt{ex1\_first-run-hartree-fock/dmc\_ts}.''
%A file named optm vmc dmc.xml should exist that contains all three execution blocks. 
For each functional, make the appropriate modifications to the input files and copy the \texttt{particleset} and \texttt{wavefunction} files from the appropriate directory in “\texttt{ex2\_slater-jastrow-wf-options/orbitals/[dft]}.” We
recommend the following settings: nodes=32, threads=16, (in optimization) blocks=250,
samples=128000, timestep=0.5, 8 optimization loops, (in VMC) walkers=16, blocks=100,
steps=1, substeps=100, (in DMC) blocks 400, targetWalkers=960, and timestep=0.01. Submit
the runs and analyze the results using \texttt{qmca}.

How do the energies compare against each other? How do they compare against DMC
energies with HF orbitals?
%Orbital Sets and Configurations in 
\section{Exercise \#3: Multideterminant wavefunctions}
In this exercise we will study the dependence of the DMC energy on the set of orbitals
and the type of configurations included in a multideterminant wavefunction. 

\subsection{Generation of a CISD wavefunctions using GAMESS}
In this section we will use GAMESS to generate a multideterminant wavefunction with
configuration interaction with single and double excitations (CISD). In CISD, the Schrodinger equation is solved exactly on a basis of determinants 
including the HF determinant and all its single and double excitations. 

Go to ``\texttt{ex3\_multi-slater-jastrow/cisd/gms}'' and you will see input and output files named \texttt{h2o.cisd.inp} and \texttt{h2o.cisd.out}. Because of technical problems with GAMESS in the BGQ architecture of VESTA, we are unable to use CISD properly in GAMESS. Consequently, the output of the calculation is already provided in the directory. 

%You'll see several input and output files named h2o.XXX.inp
%and h2o.XXX.out, where XXX is one of the following multi-determinant methods: CISD,
%CASSCF, CASCI, SOCI. 

There will be time in the next step to study the GAMESS input
files and the description in Section~\ref{sec:lab_adv_mol_gamess}, ``Appendix A: GAMESS input." %In the next exercise we will use the CISD output, in
%the next exercise we will use the remaining files. 
Since the output is already provided, the
only action needed is to use the converter to generate the appropriate QMCPACK files.  %Copy a submission script from GAMESS/Generic Files and execute the converter for all the output 
%files in the directory (with the exception of CASSCF, which is used to generate orbitals).
%but it doesn’t contain appropriate CI coefficients). 
\begin{shade}
jobrun_vesta convert4qmc h2o.cisd.out -ci h2o.cisd.out \
-readInitialGuess 57 -threshold 0.0075
\end{shade}

We used the PRTMO=.T. flag in the GUESS section to include orbitals in the output file. You should read these orbitals from the output (-readInitialGuess 40).
The highest occupied orbital in any determinant should be 34, so reading 40 orbitals is a safe choice. In this case, it is important to rename the XML files with meaningful names, for example, \texttt{h2o.cisd.wfs.xml}. A threshold of 0.0075 is sufficient for the calculations in the training.


\subsection{Optimization of a multideterminant wavefunction}

In this section we will optimize the wavefunction generated in the previous step. There
is no difference in the optimization steps if a single determinant and a multideterminant wavefunction.
QMCPACK will recognize the presence of a multideterminant wavefunction and will automatically 
optimize the linear coefficients by default. Go to ``\texttt{ex3\_multi-slater-jastrow/cisd}'' and make a folder called 
\texttt{thres0.01}. Copy the \texttt{particleset} and \texttt{wavefunction} files created in the previous step to the current 
directory. With your favorite text editor, open the \texttt{wavefunction} file \texttt{h2o.wfs.xml}. Look for 
the multideterminant XML block and change the ``cutoff'' parameter in detlist to 0.01. Then follow 
the same steps used in Section 9.4.3, ``Optimization and DMC calculations with DFT wavefunctions''
to optimize the wavefunction. Similar to this case, design a QMCPACK input file that performs
wavefunction optimization followed by VMC and DMC calculations. Submit the calculation.

This is a good time to review the GAMESS input file description in Section~\ref{sec:lab_adv_mol_gamess}, ``Appendix A. GAMESS input."
When the run is completed, go to the previous directory and make a new folder named
\texttt{thres0.0075}. Repeat the previous steps to optimize the wavefunction with a cutoff of 0.01, but use a cutoff of 0.0075 this time. This will increase the number of determinants used in the calculation. Notice that the ``cutoff'' parameter in the XML should be less than the ``-threshold 0.0075'' flag passed to the converted, which is further bounded by the PRTTOL flag in the GAMESS input.

After the wavefunction is generated, we are ready to optimize. Instead of starting from an un-optimized wavefunction, we can start from the optimized wavefunction from thres0.01 to speed up convergence. You will need to modify the file and change the cutoff in detlist to 0.0075 with a text editor. Repeat the optimization steps and submit the calculation.

\begin{figure}
\begin{center}
\includegraphics[trim = 0mm 0mm 0mm 0mm, clip,width=0.75\columnwidth]{./figures/lab_advanced_molecules_dmc_ci_cisd.png}
\end{center}
\caption{DMC energy as a function of the sum of the square of CI coefficients from CISD.}
\label{fig:lam_dmc_ci_cisd}
\end{figure}

When you are done, use \texttt{qmca} to analyze the results. Compare the energies at these two
coefficient cutoffs with the energies obtained with DFT orbitals. Because of the time limitations of this tutorial, it is not practical to optimize the wavefunctions with a smaller cutoff since this would require more samples and longer runs due to the larger number of optimizable parameters. Figure~\ref{fig:lam_dmc_ci_cisd} shows the results of such exercise: the DMC energy as a function of the cutoff in the wavefunction. As can be seen, a large improvement in the energy is obtained as the number of configurations is increased.


%Since the un-optimized wave-functions were generated in subsection “Generation of a CISD wavefunctions 
%using GAMESS” of exercise \#2, we can skip this section and go straight to the
%wave-function optimization. 

\subsection{CISD, CASCI, and SOCI}

Go to “\texttt{ex3\_multi-slater-jastrow}” and inspect the folders for the remaining wavefunction types: CASCI and SOCI. Follow the steps in the previous exercise and obtain the optimized wavefunctions for these determinant choices. Notice that the SOCI GAMESS output is not included because it is large. Already converted XML inputs can be found in ``\texttt{ex3\_multi-slater-jastrow/soci/thres*}.'' %The exercise has already been performed with a CISD wave-function in exercise \#2.

A CASCI wavefunction is produced from a CI calculation that includes all the determinants 
in a complete active space (CAS) calculation, in this case using the orbitals from a previous CASSCF
calculation. In this case we used a CAS(8,8) active space that includes all determinants
generated by distributing 8 electrons in the lowest 8 orbitals. A SOCI calculation is similar
to the CAS-CI calculation, but in addition to the determinants in the CAS it also includes
all single and double excitations from all of them, leading to a much larger determinant
set. Since you now have considerable experience optimizing wavefunctions and calculating
DMC energies, we will leave it to you to complete the remaining tasks on your own.
If you need help, refer to previous exercises in the tutorial. Perform optimizations for both
wavefunctions using cutoffs in the CI expansion of 0.01 an 0.0075. If you have time, try to optimize the wavefunctions with a cutoff of 0.005. Analyze the results and plot
the energy as a function of cutoff for all three cases: CISD, CAS-CI, and SOCI.

Figure  \ref{fig:lam_dmc_ci_cisd} shows the result of similar calculations using more samples and smaller cutoffs.
The results should be similar to those produced in the tutorial. For reference, the exact
energy of the water molecule with ECPs is approximately -17.276 Ha. From the results of the
tutorial, how does the selection of determinants relate to the expected DMC energy?
What about the choice in the set of orbitals?


\newpage
\section{Appendix A: GAMESS input}\label{sec:lab_adv_mol_gamess}
In this section we provide a brief description of the GAMESS input needed to produce
trial wavefunction for QMC calculations with QMCPACK. We assume basic familiarity
with GAMESS input structure, particularly regarding the input of atomic coordinates and
the definition of Gaussian basis sets. This section focuses on generation of the output
files needed by the converter tool, \texttt{convert4qmc}. For a description of the converter, see Section~\ref{sec:lab_adv_mol_convert4qmc}, ``Appendix B: convert4qmc."

Only a subset of the methods available in GAMESS can be used to generate wavefunctions 
for QMCPACK, and we restrict our description to these.
For a complete description of all the options and methods available
in GAMESS, please refer to the official documentation at ``http://www.msg.ameslab.gov/gamess/documentation.html.”

Currently, \texttt{convert4qmc} can process output for the following methods in GAMESS (in
SCFTYP): RHF, ROHF, and MCSCF. Both HF and DFT calculations (any DFT
type) can be used in combination with RHF and ROHF calculations. For MCSCF and CI
calculations, ALDET, ORMAS, and GUGA drivers can be used (details follow).


\subsection{HF input}
The following input will perform a restricted HF calculation on a closed-shell singlet 
(multiplicity=1). This will generate RHF orbitals for any molecular system defined in 
\ishell{\$DATA ... \$END}.

\begin{lstlisting}[style=GAMESS]
$CONTRL SCFTYP=RHF RUNTYP=ENERGY MULT=1
ISPHER=1 EXETYP=RUN COORD=UNIQUE MAXIT=200 $END
$SYSTEM MEMORY=150000000 $END
$GUESS GUESS=HUCKEL $END
$SCF DIRSCF=.TRUE. $END
$DATA
...
Atomic Coordinates and basis set
...
$END
\end{lstlisting}

Main options:
\begin{enumerate}
  \item{SCFTYP: Type of SCF method, options: RHF, ROHF, MCSCF, UHF and NONE.}
  \item{RUNTYP: Type of run. For QMCPACK wavefunction generation this should always be ENERGY.}
  \item{MULT: Multiplicity of the molecule.}
  \item{ISPHER: Use spherical harmonics (1) or Cartesian basis functions (-1).}
  \item{COORD: Input structure for the atomic coordinates in \$DATA.}
\end{enumerate}


\subsection{DFT calculations}
The main difference between the input for a RHF/ROHF calculation and a DFT calculation 
is the definition of the DFTTYP parameter. If this is set in the \$CONTROL
section, a DFT calculation will be performed with the appropriate functional. Notice that
although the default values are usually adequate, DFT calculations have many options involving
the integration grids and accuracy settings. Make sure you study the input manual to be
aware of these. Refer to the input manual for a list of the implemented exchange-correlation
functionals.


\subsection{MCSCF}
MCSCF calculations are performed by setting SCFTYP=MCSCF in the CONTROL
section. If this option is set, an MCSCF section must be added to the input file with the
options for the calculation. An example section for the water molecule used in the tutorial
follows.

\begin{lstlisting}[style=GAMESS]
$MCSCF CISTEP=GUGA MAXIT=1000 FULLNR=.TRUE. ACURCY=1.0D-5 $END
\end{lstlisting}

The most important parameter is CISTEP, which defines the CI package used. The only
options compatible with QMCPACK are: ALDET, GUGA, and ORMAS. Depending on the
package used, additional input sections are needed.


\subsection{CI}
Configuration interaction (full CI, truncated CI, CAS-CI, etc) calculations are performed
by setting \igamess{SCFTYP=NONE} and \igamess{CITYP=GUGA,ALDET,ORMAS}. Each one of these packages 
requires further input sections, which are typically slightly different from the input sections
needed for MCSCF runs.


\subsection{GUGA: Unitary group CI package}
The GUGA package is the only alternative if one wants CSFs with GAMESS. We subsequently provide a very brief description of the input sections needed to perform MCSCF, CASCI,
truncated CI, and SOCI with this package. For a complete description of these methods and
all the options available, please refer to the GAMESS input manual.

\subsubsection{GUGA-MCSCF}
The following input section performs a CASCI calculation with a CAS that includes 8
electrons in 8 orbitals (4 DOC and 4 VAL), for example, CAS(8,8). NMCC is the number of frozen
orbitals (doubly occupied orbitals in all determinants), NDOC is the number of double
occupied orbitals in the reference determinant, NVAL is the number of singly occupied
orbitals in the reference (for spin polarized cases), and NVAL is the number of orbitals in
the active space. Since FORS is set to .TRUE., all configurations in the active space will
be included. ISTSYM defines the symmetry of the desired state.

\begin{lstlisting}[style=GAMESS]
$MCSCF CISTEP=GUGA MAXIT=1000 FULLNR=.TRUE. ACURCY=1.0D-5 $END
$DRT GROUP=C2v NMCC=0 NDOC=4 NALP=0 NVAL=4 ISTSYM=1 MXNINT= 500000 FORS=.TRUE. $END
\end{lstlisting}

\subsubsection{GUGA-CASCI}
The following input section performs a CASCI calculation with a CAS that includes 8
electrons in 8 orbitals (4 DOC and 4 VAL), for example, CAS(8,8). NFZC is the number of frozen
orbitals (doubly occupied orbitals in all determinants). All other parameters are identical
to those in the MCSCF input section.

\begin{lstlisting}[style=GAMESS]
$CIDRT GROUP=C2v NFZC=0 NDOC=4 NALP=0 NVAL=4 NPRT=2 ISTSYM=1 FORS=.TRUE. MXNINT= 500000 $END
$GUGDIA PRTTOL=0.001 CVGTOL=1.0E-5 ITERMX=1000 $END
\end{lstlisting}

\subsubsection{GUGA-truncated CI}
The following input sections will lead to a truncated CI calculation. In this particular case
it will perform a CISD calculation since IEXCIT is set to 2. Other values in IEXCIT will lead
to different CI truncations; for example, IEXCIT=4 will lead to CISDTQ. Notice that only
the lowest 30 orbitals will be included in the generation of the excited determinants in this
case. For a full CISD calculation, NVAL should be set to the total number of virtual orbitals.

\begin{lstlisting}[style=GAMESS]
$CIDRT GROUP=C2v NFZC=0 NDOC=4 NALP=0 NVAL=30 NPRT=2 ISTSYM=1 IEXCIT=2 MXNINT= 500000 $END
$GUGDIA PRTTOL=0.001 CVGTOL=1.0E-5 ITERMX=1000 $END
\end{lstlisting}

\subsubsection{GUGA-SOCI}
The following input section performs a SOCI calculation with a CAS that includes 8
electrons in 8 orbitals (4 DOC and 4 VAL), for example, CAS(8,8). Since SOCI is set to .TRUE.,
all single and double determinants from all determinants in the CAS(8,8) will be included.

\begin{lstlisting}[style=GAMESS]
$CIDRT GROUP=C2v NFZC=0 NDOC=4 NALP=0 NVAL=4 NPRT=2 ISTSYM=1 SOCI=.TRUE. NEXT=30 MXNINT= 500000 $END
$GUGDIA PRTTOL=0.001 CVGTOL=1.0E-5 ITERMX=1000 $END
\end{lstlisting}


\subsection{ECP}
To use ECPs in GAMESS, you must define a \{\ishell{\$ECP ... \$END}\} 
block. There must be a definition of a potential for every atom in the system, including
symmetry equivalent ones. In addition, they must appear in the particular order expected
by GAMESS. The following example shows an ECP input block for a single water molecule using
BFD ECPs. To turn on the use of ECPs, the option “ECP=READ” must be added to the
CONTROL input block.

\begin{lstlisting}[style=GAMESS]
$ECP
O-QMC GEN 2 1
3
6.00000000 1 9.29793903
55.78763416 3 8.86492204
-38.81978498 2 8.62925665
1
38.41914135 2 8.71924452
H-QMC GEN 0 0
3
1.000000000000 1 25.000000000000
25.000000000000 3 10.821821902641
-8.228005709676 2 9.368618758833
H-QMC
$END
\end{lstlisting}


\newpage
\section{Appendix B: convert4qmc}\label{sec:lab_adv_mol_convert4qmc}
To generate the particleset and wavefunction XML blocks required by QMCPACK in
calculations with molecular systems, the converter \texttt{convert4qmc} must be used. The converter
will read the standard output from the appropriate quantum chemistry calculation and will
generate all the necessary input for QMCPACK. In the following, we describe the main options of the
converter for GAMESS output. In general, there are three ways to use the converter depending
on the type of calculation performed. The minimum syntax for each option is shown subsequently.
For a description of the XML files produced by the converter, see Section~\ref{sec:lab_adv_mol_wf_appendix}, ``Appendix E: Wavefunction XML block."

\begin{enumerate}
  \item{For all single-determinant calculations (HF and DFT with any DFTTYP):}
  \begin{shade}
convert4qmc -gamessAscii single det.out
  \end{shade}
  \begin{itemize}
    \item{single det.out is the standard output generated by GAMESS.}
  \end{itemize}
  \item{\textit{(This option is not recommended. Use the following option to avoid mistakes.)} For 
    multideterminant calculations where the orbitals and configurations are read from different
    files (e.g., when using orbitals from a MCSCF run and configurations from a
    subsequent CI run):}
  \begin{shade}
convert4qmc -gamessAscii orbitals multidet.out -ci cicoeff multidet.out
  \end{shade}
  \begin{itemize}
    \item{orbitals\_multidet.out is the standard output from the calculation that generates the
       orbitals. cicoeff multidet.out is the standard output from the calculation that calculates 
       the CI expansion.}
  \end{itemize}
  \item{For multideterminant calculations where the orbitals and configurations are read from
    the same file, using PRTMO=.T. in the GUESS input block:}
  \begin{shade}
convert4qmc -gamessAscii multi det.out -ci multi det.out -readInitialGuess Norb
  \end{shade}
  \begin{itemize}
    \item{multi\_det.out is the standard output from the calculation that calculates the CI expansion.}
  \end{itemize}
\end{enumerate}

Options:
\begin{itemize}
\item{\textbf{-gamessAscii file.out}: Standard output of GAMESS calculation. With the exception 
of determinant configurations and coefficients in multideterminant calculations,
everything else is read from this file including atom coordinates, basis sets, SPOs, ECPs, number of electrons, multiplicity, etc.}

\item{\textbf{-ci file.out}: In multideterminant calculations, determinant configurations and 
coefficients are read from this file. Notice that SPOs are NOT read
from this file. Recognized CI packages are ALDET, GUGA, and ORMAS. Output
produced with the GUGA package MUST have the option “NPRT=2” in the CIDRT
or DRT input blocks.}

\item{\textbf{-threshold cutoff}: Cutoff in multideterminant expansion. Only configurations with
coefficients above this value are printed.}

\item{\textbf{-zeroCI}: Sets to zero the CI coefficients of all determinants, with the exception of the
first one.}

\item{\textbf{-readInitialGuess Norb}: Reads Norb initial orbitals (“INITIAL GUESS ORBITALS”) 
from GAMESS output. These are orbitals generated by the GUESS input
block and printed with the option “PRTMO=.T.”. Notice that this is useful only in
combination with the option “GUESS=MOREAD” and in cases where the orbitals
are not modified in the GAMESS calculation, e.g. CI runs. This is the recommended
option in all CI calculations.}

\item{\textbf{-NaturalOrbitals Norb}: Read Norb “NATURAL ORBITALS” from GAMESS
output. The natural orbitals must exists in the output, otherwise the code aborts.}

\item{\textbf{-add3BodyJ}: Adds 3-body Jastrow terms (e-e-I) between electron pairs (both
same spin and opposite spin terms) and all ion species in the system. The radial
function is initialized to zero, and the default cutoff is 10.0 bohr. The converter will
add a 1- and 2-body Jastrow to the wavefunction block by default.}
\end{itemize}

\subsection{Useful notes}
\begin{itemize}
  \item{The type of SPOs read by the converter depends on the type of
calculation and on the options used. By default, when neither -readInitialGuess nor
-NaturalOrbitals are used, the following orbitals are read in each case (notice that
-readInitialGuess or -NaturalOrbitals are mutually exclusive):}
  \begin{itemize}
    \item{RHF and ROHF: “EIGENVECTORS”}
    \item{MCSCF: “MCSCF OPTIMIZED ORBITALS”}
    \item{GUGA, ALDET, ORMAS: Cannot read orbitals without -readInitialGuess or -NaturalOrbitals options.}
  \end{itemize}
  \item{The SPOs and printed CI coefficients in MCSCF calculations are
not consistent in GAMESS. The printed CI coefficients correspond to the next-to-last
iteration; they are not recalculated with the final orbitals. So to get appropriate 
CI coefficients from MCSCF calculations, a subsequent CI (no SCF) calculation
is needed to produce consistent orbitals. In principle, it is possible to read the orbitals 
from the MCSCF output and the CI coefficients and configurations from the
output of the following CI calculations. This could lead to problems in principle since
GAMESS will rotate initial orbitals by default to obtain an initial guess consistent 
with the symmetry of the molecule. This last step is done by default and can
change the orbitals reported in the MCSCF calculation before the CI is performed.
To avoid this problem, we highly recommend using the preceding option \#3 to
read all the information from the output of the CI calculation; this requires the use
of “PRTMO=.T.” in the GUESS input block. Since the orbitals are printed after any
symmetry rotation, the resulting output will always be consistent.}
\end{itemize}


\newpage
\section{Appendix C: Wavefunction optimization XML block}\label{sec:lab_adv_mol_opt_appendix}

\begin{lstlisting}[style=QMCPXML,caption=``Sample XML optimization block.",label=lst:lam_xml_opt]
  <loop max="10">
    <qmc method="linear" move="pbyp" checkpoint="-1" gpu="no">
    <parameter name="blocks">     10  </parameter>
      <parameter name="warmupsteps"> 25 </parameter>
      <parameter name="steps"> 1 </parameter>
      <parameter name="substeps"> 20 </parameter>
      <parameter name="timestep"> 0.5 </parameter>
      <parameter name="samples"> 10240  </parameter>
      <cost name="energy">                   0.95 </cost>
      <cost name="unreweightedvariance">     0.0 </cost>
      <cost name="reweightedvariance">       0.05 </cost>
      <parameter name="useDrift">  yes </parameter>
      <parameter name="bigchange">10.0</parameter>
      <estimator name="LocalEnergy" hdf5="no"/>
      <parameter name="usebuffer"> yes </parameter>
      <parameter name="nonlocalpp"> yes </parameter>
      <parameter name="MinMethod">quartic</parameter>
      <parameter name="exp0">-6</parameter>
      <parameter name="alloweddifference"> 1.0e-5 </parameter>
      <parameter name="stepsize">  0.15 </parameter>
      <parameter name="nstabilizers"> 1 </parameter>
    </qmc>
  </loop>
\end{lstlisting}

Options:
\begin{itemize}
  \item{bigchange: (default 50.0) Largest parameter change allowed}
  \item{usebuffer: (default no) Save useful information during VMC}
  \item{nonlocalpp: (default no) Include nonlocal energy on 1-D min}
  \item{MinMethod: (default quartic) Method to calculate magnitude of parameter change
quartic: fit quartic polynomial to four values of the cost function obtained using reweighting 
along chosen direction linemin: direct line minimization using reweighting rescale:
no 1-D minimization. Uses Umrigars suggestions.}
  \item{stepsize: (default 0.25) Step size in either quartic or linemin methods.}
  \item{alloweddifference: (default 1e-4) Allowed increase in energy}
  \item{exp0: (default -16.0) Initial value for stabilizer (shift to diagonal of H). Actual value
of stabilizer is 10 exp0}
  \item{nstabilizers: (default 3) Number of stabilizers to try}
  \item{stabilizaterScale: (default 2.0) Increase in value of exp0 between iterations.}
  \item{max its: (default 1) Number of inner loops with same sample}
  \item{minwalkers: (default 0.3) Minimum value allowed for the ratio of effective samples
to actual number of walkers in a reweighting step. The optimization will stop if the
effective number of walkers in any reweighting calculation drops below this value. Last
set of acceptable parameters are kept.}
  \item{maxWeight: (defaul 1e6) Maximum weight allowed in reweighting. Any weight above
this value will be reset to this value.}
\end{itemize}

Recommendations:
\begin{itemize}
  \item{Set samples to equal to (\#threads)*blocks.}
  \item{Set steps to 1. Use substeps to control correlation between samples.}
  \item{For cases where equilibration is slow, increase both substeps and warmupsteps.}
  \item{For hard cases (e.g., simultaneous optimization of long MSD and 3-Body J), set exp0
to 0 and do a single inner iteration (max its=1) per sample of configurations.}
\end{itemize}


\newpage
\section{Appendix D: VMC and DMC XML block}\label{sec:lab_adv_mol_vmcdmc_appendix}

\begin{lstlisting}[style=QMCPXML,caption=``Sample XML blocks for VMC and DMC calculations.",label=lst:lam_xml_vmc_dmc]
  <qmc method="vmc" move="pbyp" checkpoint="-1">
    <parameter name="useDrift">yes</parameter>
    <parameter name="warmupsteps">100</parameter>
    <parameter name="blocks">100</parameter>
    <parameter name="steps">1</parameter>
    <parameter name="substeps">20</parameter>
    <parameter name="walkers">30</parameter>
    <parameter name="timestep">0.3</parameter>
    <estimator name="LocalEnergy" hdf5="no"/>
  </qmc>
  <qmc method="dmc" move="pbyp" checkpoint="-1">
    <parameter name="nonlocalmoves">yes</parameter>
    <parameter name="targetWalkers">1920</parameter>
    <parameter name="blocks">100</parameter>
    <parameter name="steps">100</parameter>
    <parameter name="timestep">0.1</parameter>
    <estimator name="LocalEnergy" hdf5="no"/>
  </qmc>
\end{lstlisting}

General Options:
\begin{itemize}
\item{\textbf{move}: (default ``walker”) Type of electron move. Options: ``pbyp” and ``walker.”}
\item{\textbf{checkpoint}: (default ``-1”) (If > 0) Generate checkpoint files with given frequency.
The calculations can be restarted/continued with the produced checkpoint files.}
\item{\textbf{useDrift}: (default ``yes”) Defines the sampling mode. useDrift = ``yes” will
use Langevin acceleration to sample the VMC and DMC distributions, while
useDrift=``no” will use random displacements in a box.}
\item{\textbf{warmupSteps}: (default 0) Number of steps warmup steps at the beginning of the
calculation. No output is produced for these steps.}
\item{\textbf{blocks}: (default 1) Number of blocks (outer loop).}
\item{\textbf{steps}: (default 1) Number of steps per blocks (middle loop).}
\item{\textbf{sub steps}: (default 1) Number of substeps per step (inner loop). During substeps,
the local energy is not evaluated in VMC calculations, which leads to faster execution.
In VMC calculations, set substeps to the average autocorrelation time of the desired
quantity.}
\item{\textbf{time step}: (default 0.1) Electronic time step in bohr.}
\item{\textbf{samples}: (default 0) Number of walker configurations saved during the current 
calculation.}
\item{\textbf{walkers}: (default \#threads) In VMC, sets the number of walkers per node. The total
number of walkers in the calculation will be equal to walkers*(\# nodes).}
\end{itemize}

Options unique to DMC:
\begin{itemize}
\item{\textbf{targetWalkers}: (default \#walkers from previous calculation, e.g., VMC). Sets the
target number of walkers. The actual population of walkers will fluctuate around this
value. The walkers will be distributed across all the nodes in the calculation. On a
given node, the walkers are split across all the threads in the system.}
\item{\textbf{nonlocalmoves}: (default ``no”) Set to ``yes” to turns on the use of Casula’s T-moves.}
\end{itemize}


\newpage
\section{Appendix E: Wavefunction XML block}\label{sec:lab_adv_mol_wf_appendix}

\begin{lstlisting}[style=QMCPXML,caption=``Basic framework for a single-determinant determinantset XML block.",label=lst:lam_xml_determinantset]
<wavefunction name="psi0" target="e">
  <determinantset type="MolecularOrbital" name="LCAOBSet"
   source="ion0" transform="yes">
    <basisset name="LCAOBSet">
      <atomicBasisSet name="Gaussian-G2" angular="cartesian" type="Gaussian" elementType="O" normalized="no">
      ...
      </atomicBasisSet>
    </basisset>
    <slaterdeterminant>
      <determinant id="updet" size="4">
        <occupation mode="ground"/>
        <coefficient size="57" id="updetC">
        ...
        </coefficient>
      </determinant>
      <determinant id="downdet" size="4">
        <occupation mode="ground"/>
        <coefficient size="57" id="downdetC">
        ...
        </coefficient>
      </determinant>
    </slaterdeterminant>

  </determinantset>

  <jastrow name="J2" type="Two-Body" function="Bspline" print="yes">
  ...
  </jastrow>

</wavefunction>
\end{lstlisting}

In this section we describe the basic format of a QMCPACK wavefunction XML block.
Everything listed in this section is generated by the appropriate converter tools. Little to
no modification is needed when performing standard QMC calculations. As a result, this
section is meant mainly for illustration purposes. Only experts should attempt to modify
these files (with very few exceptions like the cutoff of CI coefficients and the cutoff in Jastrow
functions) since changes can lead to unexpected results.

A QMCPACK wavefunction XML block is a combination of a determinantset, which
contains the antisymmetric part of the wavefunction and one or more Jastrow blocks.
The syntax of the antisymmetric block depends on whether the wavefunction is a single
determinant or a multideterminant expansion. Listing~\ref{lst:lam_xml_determinantset} 
shows the general structure of the
single-determinant case. The determinantset block is composed of a basisset block, which
defines the atomic orbital basis set, and a slaterdeterminant block, which defines the SPOs and occupation numbers of the Slater determinant.
% Listing~\ref{lst:lam_xml_basisset} 
%shows a section
%of a basisset block for a gold atom. The structure of this block is rigid and should not
%be modified.
 Listing~\ref{lst:lam_xml_slaterdeterminant} shows a (piece of a) sample of a 
slaterdeterminant block. The
slaterdeterminant block consists of two determinant blocks, one for each electron spin. The
parameter ``size” in the determinant block refers to the number of SPOs
present while the ``size” parameter in the coefficient block refers to the number of atomic
basis functions per SPO.

\begin{lstlisting}[style=QMCPXML,caption=``Sample XML block for the single Slater determinant case.",label=lst:lam_xml_slaterdeterminant]
      <slaterdeterminant>
        <determinant id="updet" size="5">
          <occupation mode="ground"/>
          <coefficient size="134" id="updetC">
  9.55471000000000e-01 -3.87000000000000e-04  6.51140000000000e-02  2.17700000000000e-03
  1.43900000000000e-03  4.00000000000000e-06 -4.58000000000000e-04 -5.20000000000000e-05
 -2.40000000000000e-05  6.00000000000000e-06 -0.00000000000000e+00 -0.00000000000000e+00
 -0.00000000000000e+00 -0.00000000000000e+00 -0.00000000000000e+00 -0.00000000000000e+00
 -0.00000000000000e+00 -0.00000000000000e+00 -0.00000000000000e+00 -0.00000000000000e+00
 -0.00000000000000e+00 -0.00000000000000e+00 -0.00000000000000e+00 -0.00000000000000e+00
 -0.00000000000000e+00 -0.00000000000000e+00 -0.00000000000000e+00 -0.00000000000000e+00
 -0.00000000000000e+00 -0.00000000000000e+00 -0.00000000000000e+00 -0.00000000000000e+00
 -0.00000000000000e+00 -0.00000000000000e+00 -0.00000000000000e+00 -0.00000000000000e+00
 -0.00000000000000e+00 -5.26000000000000e-04  2.63000000000000e-04  2.63000000000000e-04
 -0.00000000000000e+00 -0.00000000000000e+00 -0.00000000000000e+00 -1.27000000000000e-04
  6.30000000000000e-05  6.30000000000000e-05 -0.00000000000000e+00 -0.00000000000000e+00
 -0.00000000000000e+00 -3.20000000000000e-05  1.60000000000000e-05  1.60000000000000e-05
 -0.00000000000000e+00 -0.00000000000000e+00 -0.00000000000000e+00  7.00000000000000e-06
\end{lstlisting}
Listing~\ref{lam_xml_multideterminant} shows the general structure of the multideterminant case. 
Similar to the
single-determinant case, the determinantset must contain a basisset block. This definition is
identical to the one described previously. In this case, the definition of the SPOs
must be done independently from the definition of the determinant configurations; the latter
is done in the sposet block, while the former is done on the multideterminant block. Notice
that two sposet sets must be defined, one for each electron spin. The name of each sposet set
is required in the definition of the multideterminant block. The determinants are defined in
terms of occupation numbers based on these orbitals.

\begin{lstlisting}[style=QMCPXML,caption=``Basic framework for a multideterminant determinantset XML block.",label=lam_xml_multideterminant]
  <wavefunction id="psi0" target="e">
    <determinantset name="LCAOBSet" type="MolecularOrbital" transform="yes" source="ion0">
      <basisset name="LCAOBSet">
        <atomicBasisSet name="Gaussian-G2" angular="cartesian" type="Gaussian" elementType="O" normalized="no">
        ...
        </atomicBasisSet>
        ...
      </basisset>
      <sposet basisset="LCAOBSet" name="spo-up" size="8">
        <occupation mode="ground"/>
        <coefficient size="40" id="updetC">
        ...
</coefficient>
      </sposet>
      <sposet basisset="LCAOBSet" name="spo-dn" size="8">
        <occupation mode="ground"/>
        <coefficient size="40" id="downdetC">
        ...
      </coefficient>
      </sposet>
      <multideterminant optimize="yes" spo_up="spo-up" spo_dn="spo-dn">
        <detlist size="97" type="CSF" nca="0" ncb="0" nea="4" neb="4" nstates="8" cutoff="0.001">
          <csf id="CSFcoeff_0" exctLvl="0" coeff="0.984378" qchem_coeff="0.984378" occ="22220000">
            <det id="csf_0-0" coeff="1" alpha="11110000" beta="11110000"/>
          </csf>
          ...
        </detlist>
      </multideterminant>
    </determinantset>
    <jastrow name="J2" type="Two-Body" function="Bspline" print="yes">
    ...
    </jastrow>
  </wavefunction>
\end{lstlisting}

There are various options in the multideterminant block that users should be aware of.
\begin{itemize}
  \item{cutoff: (IMPORTANT! ) Only configurations with (absolute value) “qchem coeff”
larger than this value will be read by QMCPACK.}
  \item{optimize: Turn on/off the optimization of linear CI coefficients.}
  \item{coeff: (in csf ) Current coefficient of given configuration. Gets updated during 
wavefunction optimization.}
  \item{qchem coeff: (in csf ) Original coefficient of given configuration from GAMESS 
calculation. This is used when applying a cutoff to the configurations read from the file.
The cutoff is applied on this parameter and not on the optimized coefficient.}
  \item{nca and nab: Number of core orbitals for up/down electrons. A core orbital is an
orbital that is doubly occupied in all determinant configurations, not to be confused
with core electrons. These are not explicitly listed on the definition of configurations.}
  \item{nea and neb: Number of up/down active electrons (those being explicitly correlated).}
  \item{nstates: Number of correlated orbitals}.
  \item{size (in detlist ): Contains the number of configurations in the list.}
\end{itemize}
The remaining part of the determinantset block is the definition of Jastrow factor. Any
number of these can be defined. Figure~\ref{fig:lam_xml_jastrow} shows a sample Jastrow 
block including 1-, 2- and 3-body terms. This is the standard block produced by 
\texttt{convert4qmc} with the option -add3BodyJ (this particular example is for a water molecule). 
Optimization of individual radial functions can be turned on/off using the “optimize” 
parameter. It can be added to any coefficients block, even though it is currently not 
present in the J1 and J2 blocks.

\begin{lstlisting}[style=QMCPXML,caption=``Sample Jastrow XML block.",label=fig:lam_xml_jastrow]
<jastrow name="J2" type="Two-Body" function="Bspline" print="yes">
      <correlation rcut="10" size="10" speciesA="u" speciesB="u">
        <coefficients id="uu" type="Array">0.0 0.0 0.0 0.0 0.0 0.0 0.0 0.0 0.0 0.0</coefficients>
      </correlation>
      <correlation rcut="10" size="10" speciesA="u" speciesB="d">
        <coefficients id="ud" type="Array">0.0 0.0 0.0 0.0 0.0 0.0 0.0 0.0 0.0 0.0</coefficients>
      </correlation>
    </jastrow>
    <jastrow name="J1" type="One-Body" function="Bspline" source="ion0" print="yes">
      <correlation rcut="10" size="10" cusp="0" elementType="O">
        <coefficients id="eO" type="Array">0.0 0.0 0.0 0.0 0.0 0.0 0.0 0.0 0.0 0.0</coefficients>
      </correlation>
      <correlation rcut="10" size="10" cusp="0" elementType="H">
        <coefficients id="eH" type="Array">0.0 0.0 0.0 0.0 0.0 0.0 0.0 0.0 0.0 0.0</coefficients>
      </correlation>
    </jastrow>
    <jastrow name="J3" type="eeI" function="polynomial" source="ion0" print="yes">
      <correlation ispecies="O" especies="u" isize="3" esize="3" rcut="10">
        <coefficients id="uuO" type="Array" optimize="yes">
        </coefficients>
      </correlation>
      <correlation ispecies="O" especies1="u" especies2="d" isize="3" esize="3" rcut="10">
        <coefficients id="udO" type="Array" optimize="yes">
        </coefficients>
      </correlation>
      <correlation ispecies="H" especies="u" isize="3" esize="3" rcut="10">
        <coefficients id="uuH" type="Array" optimize="yes">
        </coefficients>
      </correlation>
      <correlation ispecies="H" especies1="u" especies2="d" isize="3" esize="3" rcut="10">
        <coefficients id="udH" type="Array" optimize="yes">
        </coefficients>
      </correlation>
      </jastrow>
\end{lstlisting}

This training assumes basic familiarity with the UNIX operating system. In particular,
we use simple scripts written in “csh.” In addition, we assume you have obtained
all the necessary files and executables and that the training files are located
at \$\{TRAINING TOP\}.

The goal of this training is not only to familiarize you with the execution and
options in QMCPACK but also to introduce you to important concepts in QMC calculations and many-body electronic structure calculations.



%%% Local Variables:
%%% mode: latex
%%% TeX-master: "just_labs"
%%% End:

\chapter{Lab 4: Condensed Matter Calculations} % changed title to match schedule, can revert if desired
%\chapter{Lab 4: Using PWSCF and QMCPACK to perform total energy calculations of
%condensed systems}


\hide{
\begin{flushleft}
\textbf{Lab author: Luke Shulenburger}\footnote{Sandia National Laboratories is a multiprogram
laboratory managed and operated by Sandia Corporation, a wholly owned
subsidiary of Lockheed Martin Corporation, for the U.S. Department of Energy's
National Nuclear Security Administration under Contract No.
DE-AC04-94AL85000.}

\textbf{Creation date: July 17, 2014}
\end{flushleft}
}


\section{Topics covered in this Lab}
\begin{itemize}
  \item{tiling DFT primitive cells into QMC supercells}
  \item{reducing finite size errors via extrapolation}
  \item{reducing finite size erors via averaging over twisted boundary conditions}
  \item{using the B-spline mesh factor to reduce memory requirements}
  \item{using a coarsely resolved vacuum buffer region to reduce memory requirements}
  \item{calculating the DMC total energies of representative 2D and 3D extended systems}
\end{itemize}



\section{Lab directories and files}

\begin{shade}
labs/lab4_condensed_matter/
├── Be-2at-setup.py           - DFT only for prim to conv cell
├── Be-2at-qmc.py             - QMC only for prim to conv cell
├── Be-16at-qmc.py            - DFT and QMC for prim to 16 atom cell
├── graphene-setup.py         - DFT and OPT for graphene
├── graphene-loop-mesh.py     - VMC scan over orbital bspline mesh factors
├── graphene-loop-buffer.py   - VMC scan over orbital bspline buffer region size
├── graphene-final.py         - DMC for final meshfactor and buffer region
└── pseudopotentials          - pseudopotential directory
    ├── Be.ncpp                 - Be PP for Quantum ESPRESSO
    ├── Be.xml                  - Be PP for QMCPACK
    ├── C.BFD.upf               - C  PP for Quantum ESPRESSO
    └── C.BFD.xml               - C  PP for QMCPACK
\end{shade}


The goal of this lab will be to introduce you to the somewhat specialized problems involved in performing diffusion Monte Carlo calculations on condensed matter as opposed to the atoms and molecules that were the focus of earlier labs.   Calculations will be performed on two different systems.  Firstly, we will perform a series of calculations on BCC beryllium focusing on the necessary methodology to limit finite size effects.  Secondly, we will perform calculations on graphene as an example of a system where qmcpack’s ability to handle cases with mixed periodic and open boundary conditions is useful.  This example will also focus on strategies to limit memory usage for such systems.
All of the calculations performed in this lab will utilize the Nexus workflow management system that vastly simplifies the process by automating the steps of generating trial wavefunctions and performing DMC calculations.

\newcommand{\vp}{\mathbf{a}^\text{p}}
\newcommand{\vs}{\mathbf{a}^\text{s}} 
\newcommand{\Smat}{\mathbf{S}}
\section{Preliminaries}
For any DMC calculation, we must start with a trial wavefunction. As is typical for our calculations of condensed matter, we will produce this wavefunction using density functional theory.  Specifically, we will use quantum espresso to generate a slater determinant of single particle orbitals.  This is done as a three step process.  First, we calculate the converged charge density by performing a DFT calculation with a fine grid of k-points to fully sample the Brilloiun zone.  Next, a non-self consistent calculation is performed at the specific k-points needed for the supercell and twists needed in the DMC calculation (more on this later).  Finally, a wavefunction is converted from the binary representation used by quantum espresso to the portable hdf5 representation used by qmcpack.

The choice of k-points necessary to generate the wavefunctions dependes on both the supercell chosen for the DMC calculation and by the supercell twist vectors needed.  Recall that the wavefunction in a plane wave DFT calculation is written using Bloch's theorem as:
\begin{equation}
\Psi(\vec{r}) = e^{i\vec{k}\cdot\vec{r}}u(\vec{r})
\end{equation}
Where $\vec{k}$ is confined ot the first Brillouin zone of the cell chosen and $u(\vec{r})$ is periodic in this simulation cell.  A plane wave DFT calculation stores the periodic part of the wavefunction as a linear combination of plane waves for each single particle orbital at all k-points selected.  The symmetry of the system allows us to generate an arbitrary supercell of the primitive cell as follows:  Consider the set of primitive lattice vectors, $ \{ \mathbf{a}^p_1, \mathbf{a}^p_2,
\mathbf{a}^p_3\} $.  We may write these vectors in a matrix, $\mathbf{L}_p$, whose
rows are the primitive lattice vectors.  Consider a non-singular
matrix of integers, $\Smat$.  A corresponding set of supercell lattice
vectors, $\{\mathbf{a}^s_1, \mathbf{a}^s_2, \mathbf{a}^s_3\}$, can be constructed by the matrix
product 
\begin{equation}
\mathbf{a}^s_i = S_{ij} \mathbf{a}^p_j
\end{equation}
If the primitive cell contains $N_p$ atoms, the supercell will then
contain $N_s = |\det(\Smat)| N_p$ atoms.

Now, the wavefunciton at any point in this new supercell can be related to the wavefunction in the primitive cell by  finding the linear combination of primitive lattice vectors that maps this point back to the primitive cell:
\begin{equation}
\vec{r}' = \vec{r} + x \mathbf{a}^p_1 + y \mathbf{a}^p_2 + z\mathbf{a}^p_3 = \vec{r} + \vec{T}
\end{equation}
where $x, y, z$ are integers.   Now the wavefunction in the supercell at point $\vec{r}'$ can be written in terms of the wavefunction in the primitive cell at $\vec{r}'$ as:
\begin{equation}
\Psi(\vec{r}) = \Psi(\vec{r}') e^{i \vec{T} \cdot \vec{k}}
\end{equation}
where $\vec{k}$ is confined to the first Brillouin zone of the primitive cell.  We have also chosen the supercell twist vector which places a constraint on the form of the wavefunction in the supercell.  The combination of these two constraints allows us to identify family of N k-points in the primitive cell that satisfy the constraints.  Thus for a given supercell tiling matrix and twist angle, we can write the wavefunction everywhere in the supercell by knowing the wavefunction a N k-points in the primitive cell.  This means that the memory necesary to store the wavefunction in a supercell is only linear in the size of the supercell rather than the quadratic cost if symmetry were neglected.

\section{Total energy of BCC beryllium}

As was discussed in this morning’s lectures when performing calculations of periodic solids with QMC, it is essential to work with a reasonable size supercell rather than the primitive cells that are common in mean field calculations.  Specifically, all of the finite size correction schemes discussed in the morning require that the exchange-correlation hole be considerably smaller than the periodic simulation cell.  Additionally, finite size effects are lessened as the distance between the electrons in the cell and their periodic images increases, so it is advantageous to generate supercells that are as spherical as possible so as to maximize this distance.  However, there is a competing consideration in that for calculating total energies we often want to be able to extrapolate the energy per particle to the thermodynamic limit by means of the following formula in 3 dimensions:
\begin{equation}
E_{\inf} = C + E_{N}/N
\end{equation}
This formula derived assuming the shape of the supercells is consistent (more specifically that the periodic distances scale uniformly with system size), meaning we will need to do a uniform tiling, ie, 2x2x2, 3x3x3 etc.  As a 3x3x3 tiling is 27 times larger than the supercell and the practical limit of DMC is on the order of 200 atoms (depending on Z), sometimes it is advantagous to choose a less spherical supercell with fewer atoms rather than a more spherical one that is too expensive to tile.

In the case of a BCC crystal, it is possible to tile the one atom primitive cell to a cubic supercell by only doubling the number of electrons.  This is the best possible combination of a small number of atoms that can be tiled and a regular box that maximizes the distance between periodic images.  We will need to determine the tiling matrix S that generates this cubic supercell by solving the following equation for the coefficients of the S matrix:
\begin{equation}
 \left[\begin{array}{rrr}
  1 & 0 & 0 \\
  0 & 1 & 0 \\
  0 & 0 & 1 
  \end{array}\right] =  \left[\begin{array}{rrr}
  s_{11} & s_{12} & s_{13} \\
  s_{21} & s_{22} & s_{23} \\
  s_{31} & s_{32} & s_{33} 
  \end{array}\right] \cdot 
\left[\begin{array}{rrr}
  0.5 &  0.5 & -0.5 \\
 -0.5 &  0.5 &  0.5 \\
  0.5 & -0.5 &  0.5
\end{array}\right] 
\end{equation}

We will now use Nexus to generate the trial wavefunction for this BCC beryllium.

Fortunately, the Nexus will handle determination of the proper k-vectors given the tiling matrix.  All that is needed is to place the tiling matrix in the Be-2at-setup.py file.   Now the definition of the physical system is:

\begin{lstlisting}
    bcc_Be = generate_physical_system(
        lattice    = 'cubic',
        cell       = 'primitive',
        centering  = 'I',
        atoms      = 'Be',
        constants  = 3.490,
        units      = 'A',
        net_charge = 0,
        net_spin   = 0,
        Be         = 2,
        tiling     = [[a,b,c],[d,e,f],[g,h,i]],
        kgrid      = kgrid,
        kshift     = (.5,.5,.5)
        )
\end{lstlisting}
Where the tiling line should be replaced with the row major tiling matrix from above.  This script file will now perform a converged DFT calculation to generate the charge density in a directory called bcc-beryllium/scf and perform a non self consistend DFT calculation to generate single particle orbitals in the direcotry bcc-beryllium/nscf.  Fortunately, Nexus will calculate the required k-points needed to tile the wavefunction to the supercell, so all that is necessary is the granularity of the supercell twists and whether this grid is shifted from the origin.  Once this is finished, it performs the conversion from pwscf's binary format to the hdf5 format used by qmcpack.  Finally, it will optimize the coefficients of one-body and two-body jastrow factors in the supercell defined by the tiling matrix.

Run these calculations by executing the script Be-2at-setup.py.  You will notice that such small calcuations as are required to generate the wavefunction of Be in a one atom cell are rather inefficent to run on a high performance computer such as vesta in terms of the time spent doing calculations versus time waiting on the scheduler and booting compute nodes.  One of the benefits of the portable hdf format that is used by qmcpack is that you can generate data like wavefunctions on a local workstation or other convenient resource and only use high performance clusters for the more expensive QMC calculations.

In this case, the wavefunction is generated in the directory bcc-beryllium/nscf-2at\_222/pwscf\_output in a file called pwscf.pwscf.h5.  It can be useful for debugging purposes to be able to verify the contents of this file are what you expect.  For instance, you can use the tool h5ls to check the geometry of the cell where the dft calculations were performed, or number of k-points or electrons in the calculation.  This is done with the command: h5ls -d pwscf.pwscf.h5/supercell or h5ls -d pwscf.pwscf.h5/electrons.

In the course of running Be-2at-setup.py, you will get an error when attempting to perform the vmc and wavefunction optimization calculations.  This is due to the fact that the wavefunction has been generated supercell twists of the form (+/- 1/4, +/- 1/4, +/- 1/4).  In the case that the supercell twist contains only 0 or 1/2, it is possible to operate entirely with real arithmetic.  The executabe that has been indicated in Be-2at-setup.py has been compiled for this case.  Note that where this is possible, the memory usage is a factor of two less than the general case and the calculations are somewhat faster.  However, it is often necessary to perform calculations away from these special twist angles in order to reduce finite size effects.  To fix this, delete the directory bcc-beryllium/opt-2at, change the line in near the top of Be-2at-setup.py from 
\begin{lstlisting}
qmcpack    = '/soft/applications/qmcpack/Binaries/qmcpack'
\end{lstlisting}
to
\begin{lstlisting}
qmcpack    = '/soft/applications/qmcpack/Binaries/qmcpack_comp'
\end{lstlisting}
and rerun the script.

When the optimiztion calculation has finished, check that everything as proceeded correctly by looking at the output in the opt-2at directory.  Firstly, you can grep the output file for Delta to see if the cost function has indeed been decreasing during the optimization.  You should find something like:
\begin{shade}
 OldCost: 4.8789147e-02 NewCost: 4.0695360e-02 Delta Cost:-8.0937871e-03
 OldCost: 3.8507795e-02 NewCost: 3.8338486e-02 Delta Cost:-1.6930674e-04
 OldCost: 4.1079105e-02 NewCost: 4.0898345e-02 Delta Cost:-1.8076319e-04
 OldCost: 4.2681333e-02 NewCost: 4.2356598e-02 Delta Cost:-3.2473514e-04
 OldCost: 3.9168577e-02 NewCost: 3.8552883e-02 Delta Cost:-6.1569350e-04
 OldCost: 4.2176276e-02 NewCost: 4.2083371e-02 Delta Cost:-9.2903058e-05
 OldCost: 4.3977361e-02 NewCost: 4.2865751e-02 Delta Cost:-1.11161830-03
 OldCost: 4.1420944e-02 NewCost: 4.0779569e-02 Delta Cost:-6.4137501e-04
\end{shade}
Which shows that the starting wavefunction was fairly good and that most of the optimizaiton occurred in the first step.  Confirm this by using qmca to look at how the energy and variance changed over the course of the calculation with teh comand: qmca -q ev -e 10 *.scalar.dat executed in the opt-2at directory.  You should get output like the following:
\begin{shade}
                 LocalEnergy               Variance             ratio
opt  series 0  -2.159139 +/- 0.001897   0.047343 +/- 0.000758   0.0219 
opt  series 1  -2.163752 +/- 0.001305   0.039389 +/- 0.000666   0.0182 
opt  series 2  -2.160913 +/- 0.001347   0.040879 +/- 0.000682   0.0189 
opt  series 3  -2.162043 +/- 0.001223   0.041183 +/- 0.001250   0.0190 
opt  series 4  -2.162441 +/- 0.000865   0.039597 +/- 0.000342   0.0183 
opt  series 5  -2.161287 +/- 0.000732   0.039954 +/- 0.000498   0.0185 
opt  series 6  -2.163458 +/- 0.000973   0.044431 +/- 0.003583   0.0205 
opt  series 7  -2.163495 +/- 0.001027   0.040783 +/- 0.000413   0.0189 
\end{shade}

Now that the optimization has completed successfully, we can perform dmc calculations.  The first goal of the calculations will be to try to eliminate the one body finite size effects by twist averaging.  The script Be-2at-qmc.py has the necessary input.  Note on line 42 two twist grids are specified, (2,2,2) and (3,3,3).  Change the tiling matrix in this input file as in Be-2at-qmc.py and start the calculations.  Note that this workflow takes advantage of qmcpack's ability to group jobs.  If you look in the directory dmc-2at\_222 at the job submission script, (dmc.qsub.in) you will note that rather than operating on an xml input file, qmcapp is targeting a text file called dmc.in.  This file is a simple text file that contains the names of the 8 xml input files needed for this job, one for each twist.  When operated in this mode, qmcpack will use mpi groups to run multiple copies of itself within the same mpi context.  This is often useful both in terms of organizing calculations and also for taking advantage of the large job sizes that computer centers often encourage.

The dmc calculations in this case are designed to complete in a few minutes.  When they have finished running, first look at the scalar.dat files corresponding to the dmc calculations at the various twists in dmc-2at\_222.  Using a command like 'qmca -q ev -e 32 *.s001.scalar.dat' (with a suitably chosen number of blocks for the equilibration), you will see that the dmc energy in each calcuation is nearly identical within the statistical uncertainty of the calculations.  In the case of a large supercell, this is often indicative of a situation where the Brilloiun zone is so small that the one body finite size effects are nearly converged without any twist averaging.  In this case, however, this is because of the symmetry of the system.  For this cubic supercell, all of the twist angles chosen in this shifted 2x2x2 grid are equivalent by symmetry.  In the case where substantial resources are required to equilibrate the dmc calculations, it can be beneficial to avoid repeating such twists and instead simply weight them properly.  In this case however where the equilibration is inexpensive, there is no benefit to adding such complexity as the calculations can simply be averaged together and the result is equivalent to performing a single longer calcuation.

Using the command qmc -a -q ev -e 16 *.s001.scalar.dat, average the dmc energies in dmc-2at\_222 and dmc-2at\_333 to see whether the one body finite size effects are converged with a 3x3x3 grid of twists.  As beryllium as a metal, the convergence is quite poor (~0.025 Ha / Be or ~ 0.7 eV / Be).  If this were a production calculation it would be necessary to perform calculations on much larger grids of supercell twists to eliminate the one body finite size effects.

In this case there are several other calculations that would warrent a high priority.  A script Be-16at-qmc.py has been provided where you can imput the appropriate tiling matrix for a 16 atom cell and perform calculations to estimate the two body finite size effects which will also be quite large in the 2 atom calculations.  This script will take approximately 30 minutes to run to completion, so depending on interest,  you can either run it, or also work to modify the scripts to address the other technical issues that would be necessary for a production calculation such as calculating the population bias or the timestep error in the dmc calculations.  

Another useful exercise would be to attempt to validate this pseudopotential by calculating the ionization potential and electron affinity of the isolated atom and comparing to the experimental values:  IP = 9.3227 eV , EA = 2.4 eV.

\section{Handling a 2D system: graphene}
In this section we will examine a calculation of an isolated sheet of graphene.  As graphene is a two dimensional system, we will take advantage of qmcpack's ability to mix periodic and open boundary conditions to eliminate and spurious interaction of the sheet with its images in the z direction.  Run the script graphene-setup.py which will generate the wavefunction and optimize one and two body jastrow factors.  In the script, notice line 160: bconds = 'ppn' in the generate\_qmcpack function which specifies this mix of open and periodic boundary conditions.  As a consequence of this, the atoms will need to be kept away from this open boundary in the z direction as the electronic wavefunction will not be defined outside of the simulation box in this direction.  For this reason, all of the atom positions in at the beginning of the file have z coordinates 7.5.  At this point, run the script graphene-setup.py.

Aside from the change in boundary conditions, the main thing that distinguished this kind of calculation from the beryllium example above is the large amount of vacuum in the cell.  While this is a very small calculation designed to run quickly in the tutorial, in general a more converged calculation would quickly become memory limited on an architecture like BG/Q.  When the initial wavefunciton optimizaiton has completed to your satisfaction, run the scripts graphene-loop-buffer.py and graphene-loop-mesh.py.  These examine within variational Monte Carlo two approaches to reducing the memory required to store the wavefunction.  In graphene-loop-mesh.py, the spacing between the b-spline points is varied uniformly.  The mesh spacing is a prefactor to the linear spacing between the spline points, so the memory usage goes as the cube of the meshfactor.  When you run the calculations, examine the .s000.scalar.dat files with qmca to determine the lowest possible mesh spacing that preserves both the vmc energy and the variance.  Similarly, the script graphene-loop-buffer.py uses a feature which generates two spline tables for the wavefunction.  One will have half of the mesh spacing requested in the input file and will be valid everywhere.  The second one will only be defined in the smallest parallelpiped that contains all of the atoms in the simulation cell with minimum distance given by the buffer size.  Again, see what the smallest possible buffer size is that preserves the vmc energy and variance.

Finally, edit the file graphene-final.py which will perform two DMC calculations.  In the first, (qmc1) replace the following lines:
\begin{lstlisting}
    meshfactor   = xxx,
    precision    = '---',
    truncate     = False,
    buffer       = 0.0,
\end{lstlisting}
using the values you have determined to perform the calculation with as small as possible of wavefunction.  Note that we can also use single precision arithmetic to store the wavefunction by specifying precision='single'.  When you run the script, compare the output of the two DMC calculations in terms of energy and variance.  Also see if you can calculate the fraction of memory that you were able to save by using a meshfactor other than 1, a buffer table and single precision arithmetic.

\section{Conclusion}
Upon completion of this lab, you should be able to use Nexus to perform DMC calculations on periodic solids when provided with a pseudopotential.  You should also be able to reduce the size of the wavefunction in a solid state calculation in cases where memory is a limiting factor.

\hide{
\section{Acknowledgment}
 This tutorial was created with support from Sandia National Laboratories.

 Sandia National Laboratories is a multiprogram laboratory managed and operated by
 Sandia Corporation, a wholly owned subsidiary of Lockheed Martin Corporation, for
 the U.S. Department of Energy's National Nuclear Security Administration under
 Contract No. DE-AC04-94AL85000.
}

\chapter{Lab 5: Excited State Calculations}
\label{chap:excited}

\hide{
	\begin{flushleft}
		\textbf{Lab author: Kayahan Saritas}\footnote{Oak Ridge National Laboratory}
		
		\textbf{Creation date: November 29, 2018}
	\end{flushleft}
}

\section{Topics covered in this Lab}
\begin{itemize}
	\item{Tiling DFT primitive cells into optimal QMC supercells}
	\item{Fundamentals of  between neutral and charged calculations}
	\item{Calculating quasiparticle excitation energies of condensed matter systems}
	\item{Calculating optical excitation energies of condensed matter systems}
\end{itemize}

\section{Lab directories and files}

\begin{verbatim}
labs/lab5_excited_properties/
├── band.py           - Band structure calculation for Carbon Diamond
├── optical.py        - VMC optical gap calculation using the tiling matrix from band.py
├── quasiparticle.py  - VMC quasiparticle gap calculation using the tiling matrix from band.py
└── pseudopotentials      - pseudopotential directory
    ├── C.BFD.upf         - C PP for Quantum ESPRESSO
    └── C.BFD.xml         - C PP for QMCPACK
\end{verbatim}

The goal of this lab is to perform neutral and charged excitation calculations in condensed matter systems using QMCPACK. 
Throughout this lab, a working knowledge of \textit{Lab4 Condensed Matter Calculations} is assumed. 
First, we will introduce the concepts of neutral and charged excitations. 
We will briefly discuss these in relation to the specific experimental studies that must be used to benchmark DMC results. 
Secondly, we will perform charged (quasiparticle) and neutral (optical) excitations calculations on C-diamond.

\section{Basics and excited state experiments}
Although VMC and DMC methods are better suited for studying ground state properties of materials, they can still provide useful information regarding the excited states. 
Unlike the applications of band structure theory such as DFT and GW, it is more challenging to obtain the complete excitation spectra using DMC. 
However, it is relatively straightforward to calculate the band gap minimum of a condensed matter system using DMC. 

We will briefly discuss the two main ways of obtaining the band gap minimum through experiments: photoemission and absorption studies.  
The energy required to remove an electron from a neutral system is called the ionization potential (IP), which is available from direct photoemission experiments. 
In contrast, the emission energy of a negatively charged system (or the energy required to convert a negatively charged system to a neutral system) known as electron affinity (EA) and it is available from inverse photoemission experiments. 
Outline of these experiments are shown in Fig. \ref{fig:lab_ex_exp}. 

Following the explanation in the previous paragraph and Fig. \ref{fig:lab_ex_exp}, the \textit{quasiparticle} band gap of a material can be defined as:
\begin{equation}
	E_g=EA-IP=(E_{N+1}^{CBM}-E_{N}^{K'})-(E_{N}^{K'}-E_{N-1}^{VBM})=E_{N+1}^{CBM}+E_{N-1}^{VBM}-2*E_{N}^{K'}\label{eq:qp}
\end{equation}
where $N$ is the number of electrons in the neutral system and $E_{N}$ is the ground state energy of the neutral system. 
CBM and VBM stand for the conduction band minimum and valence band maximum, respectively. K' can formally be arbitrary at the infinite limit.
However, in practical calculations, a supertwist which accommodates both CBM and VBM can be more efficient in terms of computational time and systematic finite size error cancellation. 
In the literature, the quasiparticle gap is also called the electronic gap. 
The term electronic comes from the fact that in both photoemission experiments, it is assumed that the perturbed electron is non-interacting with the sample. 

\begin{figure}
	\centering
	\includegraphics[width=0.5\textwidth]{./figures/lab_excited_experiments}
	\caption{Direct and inverse photoemission experiments involve charged excitations, whereas optical absorption experiments involves excitation that are just enough to be excited to the conduction band. From ref. \cite{Onida2002a}}
	\label{fig:lab_ex_exp}
\end{figure}

Additionally, one can also perform absorption experiments where electrons are perturbed at relatively lower energies, just enough to be excited into the conduction band. 
In absorption experiments,  electrons are perturbed at lower energies. 
Therefore, they are not completely free and the system is still considered neutral. 
Since a \textit{quasihole} and \textit{quasielectron} are formed simultaneously, it creates a bound state, unlike the free electron in the quasiparticle gap as described above. 
This process is also known as \textit{optical} excitation, which is schematically shown in Fig. \ref{fig:lab_ex_exp}, under "Absorption". 
The optical gap can be formulated as follows:
\begin{equation}
E_g^{K_1 {\rightarrow} K_2}=E^{K_1 {\rightarrow} K_2}- E_{0}\label{eq:optical}
\end{equation}
where $E^{K_1 {\rightarrow} K_2}$ is the energy of the system when a valence electron at wavevector $K_1$ is promoted to the conduction band at wavevector $K_2$. 
Therefore, the $E_g^{K_1 {\rightarrow} K_2}$ is called the optical gap for promoting an electron at $K_1$ to $K_2$.
If both CBM and VBM are on the same k-vector then the material is called direct band gap, since it can directly emit photons without any external perturbation (phonons). 
However, if CBM and VBM share different k-vectors, then the photon emitting electron has to transfer some of its momenta to the crystal lattice and then decay to the ground state. 
As this process involves an intermediate step, this property is called the indirect band gap. 
Difference between the optical and electronic band gaps are called the exciton binding energy. 
Exciton binding energy is very important for optoelectronic applications such as lasers. 
Since the recombination usually occurs between free holes and free electrons, a bound electron and hole state means that the spectrum of emission energies will be narrower. 
In the examples that follow, we will investigate the optical excitations of C-diamond.

\begin{figure}
	\hfill
	\includegraphics[width=0.41\textwidth]{./figures/lab_excited_xcrysden1}
	\includegraphics[width=0.48\textwidth]{./figures/lab_excited_xcrysden2}
	\hfill
	\caption{Visualizing the Brillouin Zone using XCRYSDEN.}
	\label{fig:lab_ex_xcrysden}
\end{figure}


\section{Preparation for the excited state calculations}\label{sec:lab_ex_prep}

In this section, we will study the preparation steps to perform excited state calculations with quantum Monte Carlo. 
Here, the most basic steps are listed in the implementation order:
\begin{enumerate}
	\item Identify the high symmetry k-points of the standardized primitive cell 
	\item Perform DFT band structure calculation along high symmetry paths
	\item Find a supertwist which includes all the k-points of interest
	\item Identify the indexing of k-points in the supertwist to be used in QMCPACK
\end{enumerate}

\subsection{Identifying high-symmetry k-points}\label{sec:lab_ex_highk}
Primitive cell is the most basic, non-unique repeat unit of a crystal in the real space. 
However, the translations of the repeat unit, the Bravais lattice is unique for each crystal, and can be represented using discrete translation operations, $R_n$:
\begin{equation}
{\bf R_n} = n_1{\bf a_1} + n_2{\bf a_2} + n_3{\bf a_3}
\end{equation}
$a_n$ are the real space lattice vectors in three dimensions. Thanks to the periodicity of the Bravais lattice, a crystal can also be represented using periodic functions in the reciprocal space:
\begin{equation}
f({\bf R_n + r})= \sum_{m}f_me^{iG_m({\bf R_n+r})}\label{eqn:lab_ex_rec_real}
\end{equation}
where $G_m$ are called as the reciprocal lattice vectors. Equation \ref{eqn:lab_ex_rec_real} also satisfies the equality $G_m\cdot{R_n}=2{\pi}N$. High-symmetry structures can be represented using a subspace of the BZ, which is called as the irreducible Brillouin Zone (iBZ). If we choose series of  paths of high-symmetry k-points which encapsulates the iBZ, we can determine the band gap and electronic structure of the material. For more discussion, please refer to any solid state physics textbook. 

There are multiple practical ways to find the high-symmetry k-point path. 
For example, one can use pymatgen, \cite{Ong2013} XCRYSDEN \cite{Kokalj1999} or SeeK-path \cite{Hinuma2017}. 
Figure \ref{fig:lab_ex_xcrysden} shows the procedure for visualizing the Brillouin Zone using XCRYSDEN after the structure file is loaded. 
However, the primitive cell is not unique, and the actual shape of the BZ can depend on the structure used. 
In our example, we use the python libraries of SeeK-path, using a wrapper written in Nexus. 
SeeK-path includes routines to standardize primitive cells, which will be useful for our work. 
In the \ishell{band.py} script, identification of high symmetry k-points and band structure calculations are done within the workflow. 
In the script, where the \ishell{dia} PhysicalSystem object is used as the input structure, \ishell{dia2_structure} is the standardized primitive cell and \ishell{dia2_kpath} is the respective k-path around the iBZ. 
\ishell{dia2_kpath} has a dictionary of the k-path in various coordinate systems, please make sure you are using the right one. 

\begin{lstlisting}[style=Python]
from structure import get_primitive_cell, get_kpath
dia2_structure   = get_primitive_cell(structure=dia.structure)['structure']
dia2_kpath       = get_kpath(structure=dia2_structure)
\end{lstlisting}

\begin{figure}
	\centering
	\includegraphics[width=0.5\textwidth]{figures/lab_excited_band_si}
	\caption{Band structure calculation of C-diamond performed at DFT-LDA level. Conduction band minimum (CBM) are shown with red points, and the valence band maximum(VBM)  are shown with the green points both at $\Gamma$.  DFT-LDA calculations suggest that the material has an indirect band gap from $\Gamma\rightarrow{\Delta}$. However, $\Gamma\rightarrow{\Gamma}$ transition can also be investigated for more complete check. }
	\label{fig:lab_ex_bands}
\end{figure}

\subsection{DFT band structure calculation along high symmetry paths}
After the high-symmetry kpoints are identified, one can perform band structure calculations in DFT. 
For an insulating structure, DFT can provide VBM and CBM wavevectors which would be of interest to the DMC calculations. 
However, if available, CBM and VBM from DFT would need to be compared to the experiments.  
Basically,  \ishell{band.py} will:
\begin{enumerate}
	\item Perform an SCF calculation in QE using a high density reciprocal grid.
	\item Identifies the high-symmetry k-points on the iBZ and provides a k-path.
	\item Perform a 'band' calculation in QE explicitly writing all the k-points on the path. (Make sure to add extra unoccupied bands)
	\item Plot the band structure curves and the location of VBM/CBM if available.
\end{enumerate}
In Fig. \ref{fig:lab_ex_bands}, C-diamond is shown to have an indirect band gap between the red and green dots (CBM and VBM respectively). 
VBM is located at $\Gamma$. CBM is not located on a high symmetry k-point in this case. 
Therefore, we can use the symbol $\Delta$ to denote the CBM wavevector in the rest of this document. 
In \ishell{band.py} script, once the band structure calculation is finished, you can use the following lines to get the exact location of VBM and CBM using:
\begin{lstlisting}[style=Python]
from pwscf_analyzer import PwscfAnalyzer
p = PwscfAnalyzer(band)
p.analyze()
print "VBM: {0}".format(p.bands.vbm)
print "CBM: {0}".format(p.bands.cbm)
\end{lstlisting}
Output must be the following:
\begin{lstlisting}[style=Python]
VBM:   band_number     = 3
  energy          = 13.6851
  index           = 0
  kpoint_2pi_alat = [0. 0. 0.]
  kpoint_rel      = [0. 0. 0.]
  pol             = up

CBM:   band_number     = 4
  energy          = 17.2861
  index           = 52
  kpoint_2pi_alat = [0.        0.1117088 0.       ]
  kpoint_rel      = [0.3768116 0.        0.3768116]
  pol             = up
\end{lstlisting}
\subsection{Finding a supertwist which includes all the k-points of interest}
Using the VBM and CBM wavevectors defined in the previous section, we now construct the supertwist which will hopefully contain both VBM and CBM. In Fig. \ref{fig:lab_ex_twists}, we provide a simple example using 2D rectangular lattice. 
Let us assume that we are interested in the indirect transition, $\Gamma \rightarrow X_1$. 
In Fig. \ref{fig:lab_ex_twists}a, the first BZ of the primitive cell is shown as the square centered on $\Gamma$, which is drawn using dashed lines. Due to the periodicity of the lattice, this primitive cell BZ repeats itself with spacings equal to the reciprocal lattice vectors: (2$\pi$/a, 0) and (0, 2$\pi$/a) (or (1,0) and (0,1) in crystal coordinates). 
We are interested in the  first BZ, where $X_1$ is at (0,0.5). 
In Fig. \ref{fig:lab_ex_twists}b, the first BZ of the 2x2 supercell is the smaller square, drawn using solid lines. 
In Fig. \ref{fig:lab_ex_twists}c, the BZ of the 2x2 supercell also repeats in the space, similar to Fig. \ref{fig:lab_ex_twists}a. 
Therefore, in the 2x2 supercell, $X_1$, $X_2$ and $R$ are only the periodic images of $\Gamma$.  2x2 supercell calculation can be performed in reciprocal space using [2,2] tiling matrix. 
Therefore, individual kpoints (twists) of the primitive cell are combined in the supercell calculation, which are then called as supertwists. 
In more complex primitive cell (hence BZ), more general criteria would be constructing a set of supercell reciprocal lattice vectors which contains the $\Gamma \rightarrow X_1$ (e.g. $G_1$ in Fig. \ref{fig:lab_ex_twists}) vector within their convex hull. 
Under this constraint, Wigner-Seitz radius of the simulation cell can be maximized to in an effort to reduce finite size errors. 

\begin{figure}
	\includegraphics[width=\textwidth]{figures/lab_excited_twists}
	\caption{a) First Brillouin Zone (BZ) of the primitive cell centered on $\Gamma$. Dashed lines indicate zone boundaries. b) First BZ of the 2x2 supercell inside the first BZ of the primitive cell. First BZ boundaries of the supercell are shown using solid lines. c) Periodic translations of the first BZ of the supercell showing that $\Gamma$ and $X_1$ are periodic images of each other given the supercell BZ. }
	\label{fig:lab_ex_twists}
\end{figure}

For the case of the indirect band gap in Diamond, one may need to deal with using several approximations to generate a supertwist which corresponds to a reasonable simulation cell. 
$\Delta$ in Diamond band gap is at \ishell{[0.3695653, 0., 0.3695653]}. 
In your calculations, the $\Delta$ wavevector and the eigenvalues you find can be slightly different in value. 
Closest simple fraction to this number with the smallest denominator is 1/3. If we use $\Delta'=[1/3, 0., 1/3]$, we could use 3x1x3 supercell as the simple choice and include both $\Delta'$ and $\Gamma$ in the same supertwist exactly. 
Near  $\Delta$, the LDA band curvature is very low and using  $\Delta'$ can indeed be a good approximation. 
We can compare the eigenvalues using their index numbers:
\begin{lstlisting}[mathescape=true,style=Python]
>>> p.bands.up[51] ## CBM, $\Delta$ ##
{'eigs': array([-3.2076,  4.9221,  7.5433,  7.5433, 17.1545, 19.7598, 28.3242,
28.3242]), 'index': 51, 'occs': array([1., 1., 1., 1., 0., 0., 0., 0.]), 'kpoint_rel': array([0.3695653, 0.       , 0.3695653]), 'kpoint_2pi_alat': array([0.       , 0.1095605, 0.       ])}
>>> p.bands.up[46] ## $\Delta'$ ##
{'eigs': array([-4.0953,  6.1376,  7.9247,  7.9247, 17.1972, 20.6393, 27.3653,
27.3653]), 'index': 46, 'occs': array([1., 1., 1., 1., 0., 0., 0., 0.]), 'kpoint_rel': array([0.3333334, 0.       , 0.3333334]), 'kpoint_2pi_alat': array([0.       , 0.0988193, 0.       ])}
\end{lstlisting}
This shows that the eigenvalues of the first unoccupied bands in $\Delta$ and $\Delta'$ are 17.1545 and 17.1972 eV respectively, meaning that according to LDA, a correction of nearly -40 meV is obtained. 
After electronic transitions between $\Gamma$ and $\Delta'$ are studied using DMC, one can apply the LDA correction to extrapolate the results to $\Gamma$ and $\Delta$ transitions.

\subsection{Identifying the indexing of k-points of interest in the supertwist}
At this stage, we must have performed \textit{scf} calculation using a converged k-point grid and then an \textit{nscf} calculation using the supertwist kpoints given above. 
We will be using the orbitals from neutral DFT calculations, therefore we need to explicitly define the band and twist indexes of the excitations in QMCPACK (e.g. in order to define electron promotion).
In C-diamond, we can give an example by finding the band and twist indexes of $\Gamma$ and $\Delta'$. 
For this end, one can run a mock VMC calculation and read \ishell{einspline.tile_300010003} \ishell{.spin_0.tw_0.g0.bandinfo.dat} file. Einspline file prints out the eigenstates information from DFT calculations. 
Therefore, we can obtain the band and the state index from this file, which can later be used to define the electron promotion. 
Below, you can see an explanation of how the band and twist indexes are defined using a portion of the \ishell{einspline.tile_300010003.spin_0.tw_0.g0.bandinfo.dat} file. 
Spin\_0 in the file name suggests that we are reading the spin up eigenstates. Band, state, twistindex and bandindex numbers all start from zero. We know that we have 72 electrons in the simulation cell, where 36 of them are spin-up polarized. 
Since state number starts from 0, state number 35 must be occupied while state 36 should be unoccupied. 
States 35 and 36 have the same reciprocal crystal coordinates (K1,K2,K3) as $\Gamma$ and $\Delta'$, respectively. 
Therefore, one should promote an electron from state number 35 to 36 to study the indirect band gap here.
\begin{lstlisting}[style=SHELL]
#  Band State TwistIndex BandIndex Energy Kx Ky Kz K1 K2 K3 KmK
33 33 0  1     0.488302  0.0000  0.0000  0.0000 -0.0000 -0.0000 -0.0000      1
34 34 0  2     0.488302  0.0000  0.0000  0.0000 -0.0000 -0.0000 -0.0000      1
35 35 0  3     0.488302  0.0000  0.0000  0.0000 -0.0000 -0.0000 -0.0000      1
36 36 4  4     0.631985  0.0000 -0.6209  0.0000 -0.3333 -0.0000 -0.3333      1
37 37 8  4     0.631985  0.0000 -1.2418  0.0000 -0.6667 -0.0000 -0.6667      1
38 38 0  4     0.691907  0.0000  0.0000  0.0000 -0.0000 -0.0000 -0.0000      1
\end{lstlisting}
However, one should always check whether this is really what we want. 
It can be seen  that band \# 33, 34 and 35 are degenerate (energy eigenvalues are listed in the 5th column), but also they have the same reciprocal coordinates in (K1,K2,K3). 
This is actually expected as one can see from Fig. \ref{fig:lab_ex_bands}, in the band diagram the band structure is threefold degenerate at $\Gamma$.  
Here, we can choose the state with the largest band index: (0,3). 
Following the (twistindex, bandindex) notation, we can say that $\Gamma$ to $\Delta'$ transition can be defined as from (0,3) to (4,4). 

Alternatively, one can also read the band and twist indexes using PwscfAnalyzer and determine the band/twist indexes on the go:
\begin{lstlisting}[style=Python]
from pwscf_analyzer import PwscfAnalyzer
p = PwscfAnalyzer(nscf)
>>> p.bands.up
[{'eigs': array([-8.0883, 13.2874, 13.2874, 13.2874, 18.8277, 18.8277, 18.8277,
	25.9151]), 'index': 0, 'occs': array([1., 1., 1., 1., 0., 0., 0., 0.]), 'kpoint_rel': array([0., 0., 0.]), 'kpoint_2pi_alat': array([0., 0., 0.])}, {'eigs': array([-5.0893,  3.8761, 10.9518, 10.9518, 21.5031, 21.5031, 21.5361,
	28.2574]), 'index': 1, 'occs': array([1., 1., 1., 1., 0., 0., 0., 0.]), 'kpoint_rel': array([0.3333333, 0.       , 0.       ]), 'kpoint_2pi_alat': array([-0.0494096,  0.0494096,  0.0494096])}, {'eigs': array([-5.0893,  3.8761, 10.9518, 10.9518, 21.5031, 21.5031, 21.5361,
	28.2574]), 'index': 2, 'occs': array([1., 1., 1., 1., 0., 0., 0., 0.]), 'kpoint_rel': array([0.6666667, 0.       , 0.       ]), 'kpoint_2pi_alat': array([-0.0988193,  0.0988193,  0.0988193])}, {'eigs': array([-5.0893,  3.8761, 10.9518, 10.9518, 21.5031, 21.5031, 21.5361,
	28.2574]), 'index': 3, 'occs': array([1., 1., 1., 1., 0., 0., 0., 0.]), 'kpoint_rel': array([0.       , 0.       , 0.3333333]), 'kpoint_2pi_alat': array([ 0.0494096,  0.0494096, -0.0494096])}, {'eigs': array([-4.0953,  6.1376,  7.9247,  7.9247, 17.1972, 20.6393, 27.3652,
	27.3652]), 'index': 4, 'occs': array([1., 1., 1., 1., 0., 0., 0., 0.]), 'kpoint_rel': array([0.3333333, 0.       , 0.3333333]), 'kpoint_2pi_alat': array([0.       , 0.0988193, 0.       ])}, {'eigs': array([-0.6681,  2.3792,  3.7836,  8.5596, 19.3423, 26.2181, 26.6666,
	28.0506]), 'index': 5, 'occs': array([1., 1., 1., 1., 0., 0., 0., 0.]), 'kpoint_rel': array([0.6666667, 0.       , 0.3333333]), 'kpoint_2pi_alat': array([-0.0494096,  0.1482289,  0.0494096])}, {'eigs': array([-5.0893,  3.8761, 10.9518, 10.9518, 21.5031, 21.5031, 21.5361,
	28.2574]), 'index': 6, 'occs': array([1., 1., 1., 1., 0., 0., 0., 0.]), 'kpoint_rel': array([0.       , 0.       , 0.6666667]), 'kpoint_2pi_alat': array([ 0.0988193,  0.0988193, -0.0988193])}, {'eigs': array([-0.6681,  2.3792,  3.7836,  8.5596, 19.3423, 26.2181, 26.6666,
	28.0506]), 'index': 7, 'occs': array([1., 1., 1., 1., 0., 0., 0., 0.]), 'kpoint_rel': array([0.3333333, 0.       , 0.6666667]), 'kpoint_2pi_alat': array([ 0.0494096,  0.1482289, -0.0494096])}, {'eigs': array([-4.0953,  6.1376,  7.9247,  7.9247, 17.1972, 20.6393, 27.3652,
	27.3652]), 'index': 8, 'occs': array([1., 1., 1., 1., 0., 0., 0., 0.]), 'kpoint_rel': array([0.6666667, 0.       , 0.6666667]), 'kpoint_2pi_alat': array([0.       , 0.1976385, 0.       ])}]
\end{lstlisting}

\section{Quasiparticle (electronic) gap calculations}\label{sec:lab_ex_qp}
In quasiparticle calculations, it is essential to work with reasonably large sized supercells in order to avoid spurious "1/N effects". 
Since quasiparticle calculations involve charged cells, large simulation cells ensure that the extra charge is diluted over the simulation cell. Coulombic interactions are conditionally convergent for neutral periodic systems, but they are divergent for the charged systems. 
A typical workflow for a quasiparticle calculation includes:
\begin{enumerate}
	\item SCF calculation in a neutral charged cell with QE using a high-density reciprocal grid.
	\item Choose a tiling matrix which will at least approximately include VBM and CBM k-points. 
	\item 'nscf'/'p2q' calculations using the tiling matrix 
	\item VMC/DMC calculations for the neutral, positively and negatively charged cells in QMCPACK
	\item Check the convergence of the quasiparticle gap with respect to the simulation cell size
\end{enumerate}
\begin{lstlisting}[style=QMCPXML]
<particleset name="e" random="yes">
	<group name="u" size="36" mass="1.0"> ##Change size to 35
		<parameter name="charge"              >    -1                    </parameter>
		<parameter name="mass"                >    1.0                   </parameter>
</group>
...
...
<determinantset>
	<slaterdeterminant>
		<determinant id="updet" group="u" sposet="spo_u" size="36"> ##Change size to 35
		    <occupation mode="ground" spindataset="0"/>	
		</determinant>
		<determinant id="downdet" group="d" sposet="spo_d" size="36">
		    <occupation mode="ground" spindataset="1"/>	
		</determinant>
	</slaterdeterminant>
</determinantset>
\end{lstlisting}
Going back to equation \ref{eq:qp}, one can see that it is essential to include VBM and CBM wavevectors in the same twist for quasiparticle calculations as well. 
Therefore, the added electron will sit at CBM while the subtracted electron will be removed from VBM. 
However, for the charged cell calculations, one may need to make changes in the input files for the fourth step.  Alternatively, in \ishell{quasiparticle.py} file the changes in the qmc input are shown for negatively charged system:
\begin{lstlisting}[style=Python]
qmc.input.simulation.qmcsystem.particlesets.e.groups.u.size +=1
qmc.input.simulation.qmcsystem.wavefunction.determinantset.slaterdeterminant.determinants.updet.size += 1
\end{lstlisting}
Here, the number of up electrons are increased by one (negatively charged system), and QMCPACK is instructed to read more one orbital in the up channel from the .h5 file. 

QE uses symmetry in order to reduce the number of k-points required for the calculation. 
Therefore, all symmetry tags in QE (\ishell{nosym}, \ishell{noinv} and \ishell{nosym_evc}) must be set to false. 
An easy way to check whether this is the case is to see that all KmK values \ishell{einspline} files are equal to 1. 
Above, the input for the neutral cell is given, while the changes are denoted as comments for the positively charged cell. 
Notice that, we have used \ishell{det_format      = "old"} in the \ishell{vmc_+/-e.py} files.
\section{Optical gap calculations}
Routines for the optical gap calculations are very similar to the quasiparticle gap calculations. 
The first three items in the quasiparticle band gap calculations can be reused for the optical gap calculations. 
However, at the VMC/DMC level, one should explicitly state the electronic transitions that are performed. 
Therefore, compared to the quasiparticle calculations, only the item number 4 is different for optical gap calculations. 
Here, the modified input file is given for the $\Gamma\rightarrow\Delta'$ transition, which can be compared to the ground state input file in the previous section. 
\begin{lstlisting}[style=QMCPXML]
<determinantset>
	<slaterdeterminant>
		<determinant id="updet" group="u" sposet="spo_u" size="36">
		    <occupation mode="excited" spindataset="0" format="band" pairs="1" >
					0 3 4 4
		    </occupation>
		</determinant>
		<determinant id="downdet" group="d" sposet="spo_d" size="36">
		    <occupation mode="ground" spindataset="1"/>	
		</determinant>
	</slaterdeterminant>
</determinantset>
\end{lstlisting}
We have used the (twistindex, bandindex) notation in the annihilaion/creation order for the up spin electrons.
After resubmitting the batch job, in the output, you should be able to see the following lines in the \ishell{vmc.out} file:
\begin{lstlisting}[style=SHELL]
Sorting the bands now:
  Occupying bands based on (ti,bi) data.
removing orbital 35
adding orbital 36
We will read 36 distinct orbitals.
There are 0 core states and 36 valence states.
\end{lstlisting}
And the \ishell{einspline.tile_300010003.spin_0.tw_0.g0.bandinfo.dat} file must be changed in the following way: 
\begin{lstlisting}[style=SHELL]
#  Band State TwistIndex BandIndex Energy Kx Ky Kz K1 K2 K3 KmK
33 33 0	1 0.499956	0.0000  0.0000 0.0000  0.0000 0.0000  0.0000 1
34 34 0	2 0.500126	0.0000  0.0000 0.0000  0.0000 0.0000  0.0000 1
35 35 4	4 0.637231	0.0000 -0.6209 0.0000 -0.3333 0.0000 -0.3333 1
36 36 0	3 0.502916	0.0000  0.0000 0.0000  0.0000 0.0000  0.0000 1
37 37 8	4 0.637231	0.0000 -1.2418 0.0000 -0.6667 0.0000 -0.6667 1
38 38 0	4 0.699993	0.0000  0.0000 0.0000  0.0000 0.0000  0.0000 1
\end{lstlisting}
Alternatively, one can define the excitations within Nexus as shown in \ishell{optical.py} file:
\begin{lstlisting}[style=Python]
qmc = generate_qmcpack(
    ...
    excitation     = ['up', '0 3 4 4'], # (ti, bi) notation
    #excitation     = ['up', '-35 + 36'], # Orbital (state) index notation
    ...
    )
\end{lstlisting}



\chapter{Additional Tools}
\label{chap:additional_tools}
In addition to Nexus, QMCPACK provides a set of light-weight executables which address certain
common problems in QMC workflow and analysis.  These range from conversion utilities between 
different file formats and QMCPACK (ppconvert and convert4qmc), HDF5 extraction utilities (extract-eshdf-kvectors), postprocessing utilities (trace-density and qmcfinitesize), and others.  They are all written in C++, and are built alongside QMCPACK.  They can be found in the build/bin directory.  In this chapter, we cover the use cases, syntax, and features of all additional tools provided with QMCPACK.  
   
\section{convert4qmc}
\section{extract-eshdf-kvectors}
\section{getSupercell}
\section{MSDgenerator}
\section{ppconvert}
\section{trace-density}
\section{qmcfinitesize}



\chapter{External Tools}
\label{chap:external_tools}
This chapter provides some information on using QMCPACK with external tools.

\section{LLVM Sanitizer Libraries}\label{tool:LLVM-Sanitizer-Libraries}

Using CMake, set one of these flags for using the clang sanitizer libraries with or without lldb.

\begin{shade}
-DLLVM_SANITIZE_ADDRESS    link with the %*\href{https://clang.llvm.org/docs/AddressSanitizer.html}{Clang address sanitizer library}*
-DLLVM_SANITIZE_MEMORY     link with the %*\href{https://clang.llvm.org/docs/MemorySanitizer.html}{Clang memory sanitizer library}*
\end{shade}

These set the basic flags required to build with either of these sanitizer libraries. They require a build of clang with dynamic libraries somehow visible (i.e., through \ishell{LD_FLAGS=-L/your/path/to/llvm/lib}). You must link through clang, which is generally the default when building with it. Depending on your system and linker, this may be incompatible with the ``Release'' build, so set \ishell{-DCMAKE_BUILD_TYPE=Debug}. They have been tested with the default spack install of llvm@7.0.0 and been manually built with llvm 7.0.1. See the preceding  links for additional information on use, run time, and build options of the sanitizers.

In general, the address sanitizer libraries will catch most pointer-based errors. ASAN can also catch memory links but requires that additional options be set. MSAN will catch more subtle memory management errors but is difficult to use without a full set of MSAN-instrumented libraries.

\section{Intel VTune}

Intel's VTune profiler has an API that allows program control over collection (pause/resume) and can add information to the profile data (e.g., delineating tasks).

\subsection{VTune API}

If the variable \ishell{USE\_VTUNE\_API} is set, QMCPACK will check that the
include file (\ishell{ittnotify.h}) and the library (\ishell{libittnotify.a}) can
be found.
To provide CMake with the VTune paths, add the include path to \ishell{CMAKE\_CXX\_FLAGS} and the library path to \ishell{CMAKE\_LIBRARY\_PATH}.

An example of options to be passed to CMake:
\begin{shade}
 -DCMAKE_CXX_FLAGS=-I/opt/intel/vtune_amplifier_xe/include \
 -DCMAKE_LIBRARY_PATH=/opt/intel/vtune_amplifier_xe/lib64
\end{shade}

\section{NVIDIA Tools Extensions}

NVIDIA's Tools Extensions (NVTX) API enables programmers to annotate their source code when used with the NVIDIA profilers.

\subsection{NVTX API}

If the variable \ishell{USE_NVTX_API} is set, QMCPACK will add the library (\ishell{libnvToolsExt.so}) to the QMCPACK target. To add NVTX annotations
to a function, it is necessary to include the \ishell{nvToolsExt.h} header file and then make the appropriate calls into the NVTX API. For more information
about the NVTX API, see \url{https://docs.nvidia.com/cuda/profiler-users-guide/index.html#nvtx}. Any additional calls to the NVTX API should be guarded by
the \ishell{USE\_NVTX\_API} compiler define.

\subsection{Timers as Tasks}
To aid in connecting the timers in the code to the profile data, the start/stop of
timers will be recorded as a task if \ishell{USE_VTUNE_TASKS} is set.

In addition to compiling with \ishell{USE_VTUNE_TASKS}, an option needs to be set at run time to collect the task API data.
In the graphical user interface (GUI), select the checkbox labeled ``Analyze user tasks" when setting up the analysis type.
For the command line, set the \ishell{enable-user-tasks} knob to \ishell{true}. For example,
\begin{shade}
amplxe-cl -collect hotspots -knob enable-user-tasks=true ...
\end{shade}

Collection with the timers set at ``fine" can generate too much task data in the profile.
Collection with the timers at ``medium" collects a more reasonable amount of task data.

\section{Scitools Understand}

Scitools Understand (\url{https://scitools.com/}) is a tool for static
code analysis. The easiest configuration route is to use the JSON output
from CMake, which the Understand project importer can read directly:
\begin{enumerate}
\item Configure QMCPACK by running CMake with
  \ishell{CMAKE_EXPORT_COMPILE_COMMANDS=ON}, for example:
  \begin{lstlisting}[style=SHELL]
  cmake -DCMAKE_C_COMPILER=clang -DCMAKE_CXX_COMPILER=clang++
  -DQMC_MPI=0 -DCMAKE_EXPORT_COMPILE_COMMANDS=ON ../qmcpack/
  \end{lstlisting}
\item Run Understand and create a new C++ project. At the import files
  and settings dialog, import the \ishell{compile_commands.json} created by
  CMake in the build directory.  
\end{enumerate}


\chapter{Contributing to the Manual}
\label{chap:contrib}

This section briefly describes how to contribute to the manual.  This is primarily ``by developers, for developers''.   This section should iterate until a consistent view on style/contents is reached.

\textbf{\underline{Desirable:}}
\begin{itemize}
\item{Use the table templates below when describing XML input.}
\item{Instead of \ilatex{\\texttt} or \ilatex{\\verb} use
    \begin{itemize}
      \item{\ilatex{\\ishell} for shell text}
      \item{\ilatex{\\ixml} for xml text}
      \item{\ilatex{\\icode} for C++ text}
      \end{itemize}
     \bf{Except} within tabularx or math environments}
    \item{Instead of \ilatex{\\begin\{verbatim\}} environments use the appropriate \ilatex{\\begin\{lstlisting\}[style=<see qmcpack_listings.sty>]}}
\item{\ilatex{\\begin\{shade\}} can be used in place of \ilatex{\\begin\{lstlisting\}[style=SHELL]}}
\item{Unicode rules}
\begin{itemize}
\item Do not use characters for which well established latex idiom exists, especially dashes, quotes, and apostrophes.
\item Use math mode markup instead of unicode characters for equations.
\item Be cautious of WYSIWYG word processors, cutting and pasting can pickup characters promoted to unicode by the program.
\item Take a look at your text multibyte expanded i.e. open in (emacs and `esc-x toggle-enable-multibyte-characters`), see any unicode you didn't intend?
\end{itemize}
\item{Place unformatted text targeted at developers working on the latex in comments.  Include generously.}
\item{Encapsulate formatted text aimed at developers (like this entire chapter), in \ilatex{\\dev\{\}}.  Text encapsulated in this way will be removed from the user version of the manual by editing the definition of \ilatex{\\dev\{\}} in \ishell{qmcpack_manual.tex}.  Existing but deprecated or partially functioning features fall in this category.}
\end{itemize}

\textbf{\underline{Forbidden:}}
\begin{itemize}
\item Including images instead of using lstlisting sections for text.
\item Packages the LaTeX community considers \href{https://latex.org/forum/viewtopic.php?f=37&t=6637}{deprecated}.
\item Do not use packages, features, or fonts not included in texlive 2017 unless you insure they degrade reasonably for 2017.
\item Don't add packages unless they are bringing great value and are supported by tex4ht (unless you are willing to add the support).
\item Tex files and Bib files are UTF8 encoded, do not save them in other encodings. Some may report being ASCII encoded since they contain no unicode characters.
\end{itemize}


\textbf{\underline{Missing sections (these are opinions, not decided priorities):}}
\begin{itemize}
  \item{Description of XML input in general.  Discuss XML format, use of attributes and \texttt{<parameter/>}'s in general, case sensitivity (input is generally case sensitive), and behavior of \qmcpack when unrecognized XML elements are encountered (they are generally ignored without notification).}
  \item{Overview of the input file in general, broad structure, and at least one full example that works in isolation.}
\end{itemize}


\textbf{\underline{Information currently missing for a complete reference specification:}}
\begin{itemize}
  \item{Noting how many instances of each child element are allowed.  Examples: \texttt{simulation}--1 only, \texttt{method}--1 or more, \texttt{jastrow}--0 or more}.
\end{itemize}


Below are template tables for describing XML elements in reference fashion.  A number of examples can be found in \textit{e.g.} Chapter \ref{chap:hamiltobs}.  Preliminary style is (please weigh in with opinions): typewriter text (\ilatex{\\texttt\{\}}) for XML element, attribute, and parameter names, normal text for literal information in datatype/values/default columns, bold (\ilatex{\\textbf\{\}}) text if an attribute/parameter must take on a particular value (values column), italics (\ilatex{\\textit\{\}}) for descriptive (non-literal) information in the values column (e.g. \textit{anything}, \textit{non-zero}, etc.), required/optional attributes/parameters noted by \texttt{some\_attr$^r$}/\texttt{some\_attr$^o$} superscripts.  Valid datatypes are text, integer, real, boolean, and arrays of each.  Fixed lengh arrays can be noted, \textit{e.g.} by ``real array(3)''.


Template for a generic XML element:
\FloatBarrier
\begin{table}[h]
\begin{center}
\begin{tabularx}{\textwidth}{l l l l l X }
\hline
\multicolumn{6}{l}{\texttt{generic} element} \\
\hline
\multicolumn{2}{l}{parent elements:} & \multicolumn{4}{l}{\texttt{parent1 parent2}}\\
\multicolumn{2}{l}{child  elements:} & \multicolumn{4}{l}{\texttt{child1 child2 child3 ...}}\\
\multicolumn{2}{l}{attributes}  & \multicolumn{4}{l}{}\\
   &   \bfseries name     & \bfseries datatype & \bfseries values & \bfseries default   & \bfseries description \\
   &   \texttt{attr1}$^r$ &  text              &                  &                     &                       \\
   &   \texttt{attr2}$^r$ &  integer           &                  &                     &                       \\
   &   \texttt{attr3}$^o$ &  real              &                  &                     &                       \\
   &   \texttt{attr4}$^o$ &  boolean           &                  &                     &                       \\
   &   \texttt{attr5}$^o$ &  text array        &                  &                     &                       \\
   &   \texttt{attr6}$^o$ &  integer array     &                  &                     &                       \\
   &   \texttt{attr7}$^o$ &  real array        &                  &                     &                       \\
   &   \texttt{attr8}$^o$ &  boolean array     &                  &                     &                       \\
\multicolumn{2}{l}{parameters}  & \multicolumn{4}{l}{}\\
   &   \bfseries name     & \bfseries datatype & \bfseries values & \bfseries default   & \bfseries description \\
   &   \texttt{param1}$^r$&  text              &                  &                     &                       \\
   &   \texttt{param2}$^r$&  integer           &                  &                     &                       \\
   &   \texttt{param3}$^o$&  real              &                  &                     &                       \\
   &   \texttt{param4}$^o$&  boolean           &                  &                     &                       \\
   &   \texttt{param5}$^o$&  text array        &                  &                     &                       \\
   &   \texttt{param6}$^o$&  integer array     &                  &                     &                       \\
   &   \texttt{param7}$^o$&  real array        &                  &                     &                       \\
   &   \texttt{param8}$^o$&  boolean array     &                  &                     &                       \\
\multicolumn{2}{l}{body text}  & \multicolumn{4}{l}{}\\
   &                           & \multicolumn{4}{l}{Long form description of body text format}                   \\
  \hline
\end{tabularx}
\end{center}
\end{table}
\FloatBarrier



``Factory'' elements are XML elements that share a tag, but whose contents change based on the value an attribute (or sometimes multiple attributes take).  The attribute(s) that determine the allowed contents is referred to below as the ``type selector''  (\textit{e.g.} for \ixml{<estimator/>} elements, the type selector is usually the \ixml{type} attribute).  These types of elements are frequently encountered as they correspond (sometimes loosely, sometimes literally) to polymorphic classes in \qmcpack that are built in ``factories''.  This name is true to the underlying code, but may be obscure to the general user (is there a better name to retain the general meaning?).   

The template below should be provided each time a new ``factory'' type is encountered (like \ixml{<estimator/>}).  The table lists all types of possible elements (see ``type options'' below) and any attributes that are common to all possible related elements.  Specific ``derived'' elements are then described one at a time with the template above, noting the type selector in addition to the XML tag (\textit{e.g.} ``\ixml{estimator type=density} element'').

Template for shared information about ``factory'' elements.
\FloatBarrier
\begin{table}[h]
\begin{center}
\begin{tabularx}{\textwidth}{l l l l l X }
\hline
\multicolumn{6}{l}{\texttt{generic} factory element} \\
\hline
\multicolumn{2}{l}{parent elements:} & \multicolumn{4}{l}{\texttt{parent1 parent2}}\\
\multicolumn{2}{l}{child  elements:} & \multicolumn{4}{l}{\texttt{child1 child2 child3 ...}}\\
\multicolumn{2}{l}{type   selector:} & \multicolumn{4}{l}{\texttt{some} attribute}\\
\multicolumn{2}{l}{type   options :} & \multicolumn{4}{l}{Selection1}\\
\multicolumn{2}{l}{                } & \multicolumn{4}{l}{Selection2}\\
\multicolumn{2}{l}{                } & \multicolumn{4}{l}{Selection3}\\
\multicolumn{2}{l}{                } & \multicolumn{4}{l}{...}\\
\multicolumn{2}{l}{shared attributes:} & \multicolumn{4}{l}{}\\
   &   \bfseries name     & \bfseries datatype & \bfseries values & \bfseries default   & \bfseries description \\
   &   \texttt{attr1}     &  text              &                  &                     &                       \\
   &   \texttt{attr2}     &  integer           &                  &                     &                       \\
   &   ...                &                    &                  &                     &                       \\
  \hline
\end{tabularx}
\end{center}
\end{table}
\FloatBarrier

\chapter{Unit Testing}
\label{chap:unit_testing}

Unit testing is a standard software engineering practice to aid in ensuring a quality product. A good suite of unit tests provides confidence in refactoring and changing code, provides some documentation on how classes and functions are used, and can drive a more decoupled design.

If unit tests do not already exist for a section of code, you are encouraged to add them when modifying that section of code.  New code additions should also include unit tests.
When possible, fixes for specific bugs should also include a unit test that would have caught the bug.

\section {Unit testing framework} The Catch framework is used for unit testing.
See the project site for a tutorial and documentation: \url{https://github.com/philsquared/Catch}

Catch consists solely of header files. It is distributed as a single include file about 400KB in size.  In QMCPACK, it is stored in \texttt{external\_codes/catch}.

\section{Unit test organization}

The source for the unit tests is located in the \texttt{tests} directory under each directory in \texttt{src} (e.g. \texttt{src/QMCWavefunctions/tests}).
All of the tests in each \texttt{tests} directory get compiled into an executable.
After building the project, the individual unit test executables can be found in \texttt{build/tests/bin}.
For example, the tests in \texttt{src/QMCWavefunctions/tests} are compiled into \texttt{build/tests/bin/test\_wavefunction}.

All the unit test executables are collected under ctest with the \texttt{unit} label.
When checking the whole code, it's useful to run through cmake (\texttt{cmake -L unit}).
When working on an individual directory, it's useful to run the individual executable.

Some of the tests reference input files. The unit test CMake setup places those input files in particular locations under the \texttt{tests} directory (e.g. \texttt{tests/xml\_test}).  The individual test needs to be run from that directory to find the expected input files.

Command line options are available on the unit test executables.  Some of the more useful ones are
\begin{description}
\item[\texttt{-h}]  List command line options.
\item [\texttt{--list-tests}] List all the tests in the executable.
\end{description}

A test name can be given on the command line to execute just that test.  This is useful when iterating
on a particular test, or when running in the debugger.   Test names often contain spaces, so most command line environments require enclosing the test name in single or double quotes.



\section{Example}

The first example is one test from \texttt{src/Numerics/tests/test\_grid\_functor.cpp}

\begin{minipage}{\linewidth}
\begin{lstlisting}[language=C++,caption={Unit test example using Catch},label=CatchExample,basicstyle=\ttfamily]
TEST_CASE("double_1d_grid_functor", "[numerics]")
{
  LinearGrid<double> grid;
  OneDimGridFunctor<double> f(&grid);

  grid.set(0.0, 1.0, 3);

  REQUIRE(grid.size() == 3);
  REQUIRE(grid.rmin() == 0.0);
  REQUIRE(grid.rmax() == 1.0);
  REQUIRE(grid.dh() == Approx(0.5));
  REQUIRE(grid.dr(1) == Approx(0.5));
}
\end{lstlisting}
\end{minipage}

The test function declaration is
\texttt{TEST\_CASE("double\_1d\_grid\_functor","[numerics]")}.
The first argument is the test name, and it must be unique in the test suite.
The second argument is an optional list of tags.  Each tag is a name surrounded by brackets (\texttt{"[tag1][tag2]"}).  It can also be the empty string.

The \texttt{REQUIRE} macro accepts expressions with C++ comparison operators and records an error if the value of the expression is false.

Floating point numbers may have small differences due to roundoff, etc.   The \texttt{Approx} class adds some tolerance to the comparison.  Place it on either side of the comparison (e.g. \texttt{Approx(a) == 0.3} or \texttt{a = Approx(0.3)}).   To adjust the tolerance, use the \texttt{epsilon} and \texttt{scale} methods to \texttt{Approx} (\texttt{REQUIRE(Approx(a).epsilon(0.001) = 0.3);}.

\subsection{Expected output}

When running the test executables individually, the output of a run with no failures should look like
\begin{shade}
===============================================================================
All tests passed (26 assertions in 4 test cases)
\end{shade}

A test with failures will look like

\begin{minipage}{\linewidth}
\begin{shade}
~~~~~~~~~~~~~~~~~~~~~~~~~~~~~~~~~~~~~~~~~~~~~~~~~~~~~~~~~~~~~~~~~~~~~~~~~~~~~~~
test_numerics is a Catch v1.4.0 host application.
Run with -? for options

-------------------------------------------------------------------------------
double_1d_grid_functor
-------------------------------------------------------------------------------
/home/user/qmcpack/src/Numerics/tests/test_grid_functor.cpp:29
...............................................................................

/home/user/qmcpack/src/Numerics/tests/test_grid_functor.cpp:39: FAILED:
  REQUIRE( grid.dh() == Approx(0.6) )
with expansion:
  0.5 == Approx( 0.6 )

===============================================================================
test cases:  4 |  3 passed | 1 failed
assertions: 25 | 24 passed | 1 failed
\end{shade}
\end{minipage}


\section{Adding tests}
There are three scenarios covered here: adding a new test in an existing file, adding a new test file, or adding a new \texttt{test} directory.

\subsection{Adding a test to existing file}
Copy an existing test, or from the example shown here.  Be sure to change the test name.

\subsection{Adding a test file}
When adding a new test file,
create a file in the test directory, or copy from an existing file.  Add the file name to the \texttt{ADD\_EXECUTABLE} in the \texttt{CMakeLists.txt} file in that directory.

One (and only one) file must define the \texttt{main} function for the test executable by defining \texttt{CATCH\_CONFIG\_MAIN} before including the Catch header.  If more than one file defines this value, there will be linking errors about multiply defined values.

Some of the tests need to shut down MPI properly to avoid extraneous error messages. Those tests include \texttt{Message/catch\_mpi\_main.hpp} instead of defining \texttt{CATCH\_CONFIG\_MAIN}.


\subsection{Adding a test directory}
Copy the CMakeLists.txt file from an existing \texttt{tests} directory.
Change the \texttt{SRC\_DIR} name and the  files in the \texttt{ADD\_EXECUTABLES} line.  The libraries to link in \texttt{TARGET\_LINK\_LIBRARIES} may need to be updated.

Add the new test directory to \texttt{src/CMakeLists.txt} in the \texttt{BUILD\_UNIT\_TESTS} section near the end.


\section{Testing with random numbers}
Many algorithms and parts of the code depend on random numbers, which makes validating the results difficult.
One solution is to verify that certain properties hold for any random number.
This approach is valuable at some levels of testing, but is unsatisfying at the unit test level.

The \texttt{Utilities} directory contains a 'fake' random number generator that can be used for deterministic tests of these parts of the code.
Currently it outputs a single, fixed value every time it is called, but it could be expanded to produce more varied, but still deterministic, sequences.
See \texttt{src/QMCDrivers/test\_vmc.cpp} for an example of using the fake random number generator.


\chapter{QMCPACK Design and Feature Documentation}
\label{chap:design_features}

This section contains information on the overall design of QMCPACK.  Also included in this section are detailed explanations/derivations of major features and algorithms present in the code.


\section{QMCPACK Design}
TBD.



\newpage
\section{Feature: Optimized Long-Ranged Breakup (Ewald)}

% Written by Ken Esler as part of the Common codebase used in wfconvert
% Originally titled ``Ewald Breakup for Long-Range Potentials in PIMC''
% PIMC-specific portions have been commented out

Consider a group of particles interacting with long-ranged central
potentials, $v^{\alpha \beta}(|r^{\alpha}_i - r^{\beta}_j|)$, where the Greek superscripts
represent the particle species (eg. $\alpha=\text{electron}$,
$\beta=\text{proton}$), and Roman subscripts refer to particle number
within a species.  We can then write the total interaction energy for
the system as,
\newcommand{\vr}{\mathbf{r}}
\newcommand{\vR}{\mathbf{R}}
\newcommand{\vk}{\mathbf{k}}
\newcommand{\vq}{\mathbf{q}}
\begin{equation}
V = \sum_\alpha \left\{\sum_{i<j} v^{\alpha\alpha}(|\vr^\alpha_i - \vr^\alpha_j|) +
\sum_{\beta<\alpha} 
\sum_{i,j} v^{\alpha \beta}(|\vr^{\alpha}_i - \vr^{\beta}_j|) \right\}
\label{eq:Vperiodic}
\end{equation}
\newcommand{\va}{\mathbf{a}}
\newcommand{\vb}{\mathbf{b}}
\newcommand{\vL}{\mathbf{L}}

\subsection{The Long-Range Problem}
Consider such a system in periodic boundary conditions in a cell
defined by primitive lattice vectors $\va_1$, $\va_2$, and $\va_3$.
Let $\vL \equiv n_1 \va_1 + n_2 \va_2 + n_3\va_3$ be a direct lattice
vector.  Then the interaction energy per cell for the periodic system
is given by
\begin{equation}
\begin{split}
V = & \sum_\vL \sum_\alpha \left\{ 
\overbrace{\sum_{i<j} v^{\alpha\alpha}(|\vr^\alpha_i - \vr^\alpha_j + \vL|)}^{\text{homologous}} +
\overbrace{\sum_{\beta<\alpha} 
\sum_{i,j} v^{\alpha \beta}(|\vr^{\alpha}_i - \vr^{\beta}_j+\vL|)}^{\text{heterologous}}
\right\}  \\
& + \underbrace{\sum_{\vL \neq \mathbf{0}} \sum_\alpha N^\alpha v^{\alpha \alpha} (|\vL|)}_\text{Madelung}
\end{split}
\label{eq:direct},
\end{equation}
where $N^\alpha$ is the number particles of species $\alpha$.
If the potentials $v^{\alpha\beta}(r)$ are indeed long-range, the
summation over direct lattice vectors will not converge in this naive
form.  A solution to the problem was posited by Ewald.  We break the
central potentials into two pieces -- a short range and a long range
part define by
\begin{equation}
v^{\alpha \beta}(r) = v_s^{\alpha\beta}(r) + v_l^{\alpha \beta}(r).
\end{equation}
We will perform the summation over images for the short-range part in
real space, while performing the sum for the long-range part in
reciprocal space.  For simplicity, we choose $v^{\alpha \beta}_s(r)$
so that it is identically zero at the half the box length.  This
eliminates the need to sum over images in real space.


\subsection{Reciprocal-Space Sums}
\subsubsection{Heterologous terms}
We begin with (\ref{eq:direct}), starting with the heterologous terms,
i.e. the terms involving particles of different species.  The
short-range terms are trivial, so we neglect them here.
\begin{equation}
\text{heterologous} = \frac{1}{2} \sum_{\alpha \neq \beta} \sum_{i,j} \sum_\vL
v^{\alpha\beta}_l(\vr_i^\alpha - \vr_j^\beta + \vL)
\end{equation}
We insert the resolution of unity in real space twice,
\begin{eqnarray}
\text{heterologous} & = & \frac{1}{2}\sum_{\alpha \neq \beta} \int_\text{cell} d\vr \, d\vr' \, \sum_{i,j}
\delta(\vr_i^\alpha - \vr) \delta(\vr_j^\beta-\vr') \sum_\vL
v^{\alpha\beta}_l(|\vr - \vr' + \vL|) \\
& = & \frac{1}{2\Omega^2}\sum_{\alpha \neq \beta} \int_\text{cell} d\vr \, d\vr' \, \sum_{\vk, \vk', i, j} e^{i\vk\cdot(\vr_i^\alpha
  - \vr)} e^{i\vk'\cdot(\vr_j^\beta - \vr')} \sum_\vL
v^{\alpha\beta}_l(|\vr - \vr' + \vL|) \nonumber \\
& = & \frac{1}{2\Omega^2} \sum_{\alpha \neq \beta} \int_\text{cell} d\vr \, d\vr'\,
\sum_{\vk, \vk', \vk'', i, j} e^{i\vk\cdot(\vr_i^\alpha - \vr)}
e^{i\vk'\cdot(\vr_j^\beta-\vr')} e^{i\vk''\cdot(\vr -\vr')}
v^{\alpha\beta}_{\vk''}, \nonumber.
\end{eqnarray}
Here, the $\vk$ summations are over reciprocal lattice vectors given
by $\vk = m_1 \vb_1 + m_2\vb_2 + m_3\vb_3$, where
\begin{eqnarray}
\vb_1 & = & 2\pi \frac{\va_2 \times \va_3}{\va_1 \cdot (\va_2 \times
  \va_3)} \nonumber \\
\vb_2 & = & 2\pi \frac{\va_3 \times \va_1}{\va_1 \cdot (\va_2 \times
  \va_3)} \\
\vb_3 & = & 2\pi \frac{\va_1 \times \va_2}{\va_1 \cdot (\va_2 \times
  \va_3)} \nonumber.
\end{eqnarray}
We note that $\vk \cdot \vL = 2\pi(n_1 m_1 + n_2 m_2 + n_3 m_3)$. 

\begin{eqnarray}
v_{k''}^{\alpha \beta} & = & 
\frac{1}{\Omega} \int_{\text{cell}} d\vr'' \sum_\vL
e^{-i\vk''\cdot(|\vr''+\vL|)} v^{\alpha\beta}(|\vr''+\vL|), \\
& = & \frac{1}{\Omega} \int_\text{all space} d\tilde{\vr} \, 
    e^{-i\vk'' \cdot \tilde{\vr}} v^{\alpha\beta}(\tilde{r}), \label{eq:vk}
\end{eqnarray}
where $\Omega$ is the volume of the cell. Here we have used the fact
that summing over all cells of the integral over the cell is
equivalent to integrating over all space.
\begin{equation}
\text{hetero} = \frac{1}{2\Omega^2} \sum_{\alpha \neq \beta}
\int_\text{cell} d\vr \, d\vr' \, \sum_{\vk, \vk', \vk'', i, j}
e^{i(\vk \cdot \vr_i^\alpha + \vk' \cdot\vr_j^\beta)} e^{i(\vk''-\vk)\cdot \vr}
e^{-i(\vk'' + \vk')\cdot \vr'} v^{\alpha \beta}_{\vk''}.
\end{equation}
We have
\begin{equation}
\frac{1}{\Omega} \int d\vr \  e^{i(\vk -\vk')\cdot \vr} =
\delta_{\vk,\vk'},
\end{equation}
Then, performing the integrations we have
\begin{eqnarray}
\text{hetero} = \frac{1}{2} \sum_{\alpha \neq \beta}
\sum_{\vk, \vk', \vk'', i, j}
e^{i(\vk \cdot \vr_i^\alpha + \vk' \cdot\vr_j^\beta)} \delta_{\vk,\vk''}
\delta_{-\vk', \vk''} v^{\alpha \beta}_{\vk''}.
\end{eqnarray}
We now separate the summations, yielding
\begin{equation}
\text{hetero} = \frac{1}{2} \sum_{\alpha \neq \beta} \sum_{\vk, \vk'}
\underbrace{\left[\sum_i e^{i\vk  \cdot \vr_i^\alpha} \rule{0cm}{0.705cm}
    \right]}_{\rho_\vk^\alpha}
\underbrace{\left[\sum_j e^{i\vk' \cdot \vr_j^\beta} \right]}_{\rho_{\vk'}^\beta}
 \delta_{\vk,\vk''} \delta_{-\vk', \vk''} v^{\alpha
  \beta}_{\vk''}.
\end{equation}
Summing over $\vk$ and $\vk'$, we have
\begin{equation}
\text{hetero} = \frac{1}{2} \sum_{\alpha \neq \beta} \sum_{\vk''}
\rho_{\vk''}^\alpha \, \rho_{-\vk''}^\beta v_{k''}^{\alpha \beta}.
\end{equation}
We can simplify the calculation a bit further by rearranging the
sums over species,
\begin{eqnarray}
\text{hetero} & = & \frac{1}{2} \sum_{\alpha > \beta} \sum_{\vk}
\left(\rho^\alpha_\vk \rho^\beta_{-\vk} + \rho^\alpha_{-\vk}
\rho^\beta_\vk\right) v_{k}^{\alpha\beta} \\
& = & \sum_{\alpha > \beta} \sum_\vk \mathcal{R}e\left(\rho_\vk^\alpha
\rho_{-\vk}^\beta\right)v_k^{\alpha\beta} .
\end{eqnarray}


\subsubsection{Homologous Terms}
We now consider the terms involving particles of the same species
interacting with each other.  The algebra is very similar to that
above, with the slight difficulty of avoiding the self-interaction term.
\begin{eqnarray}
\text{homologous} & = & \sum_\alpha \sum_L \sum_{i<j} v_l^{\alpha
  \alpha}(|\vr_i^\alpha - \vr_j^\alpha + \vL|) \\
 & = & \frac{1}{2} \sum_\alpha \sum_L \sum_{i\neq j} v_l^{\alpha
  \alpha}(|\vr_i^\alpha - \vr_j^\alpha + \vL|) 
\end{eqnarray}
\begin{eqnarray}
\text{homologous} & = & \frac{1}{2} \sum_\alpha \sum_L 
\left[
-N^\alpha v_l^{\alpha \alpha}(|\vL|)  + \sum_{i,j} v^{\alpha \alpha}_l(|\vr_i^\alpha - \vr_j^\alpha + \vL|)
  \right] \\
& = & \frac{1}{2} \sum_\alpha \sum_\vk \left(|\rho_k^\alpha|^2 - N
\right) v_k^{\alpha \alpha}
\end{eqnarray}

\subsubsection{Madelung Terms}
Let us now consider the Madelung term for a single particle of species
$\alpha$.  This term corresponds to the interaction of a particle with
all of its periodic images.  
\begin{eqnarray}
v_M^{\alpha} & = & \frac{1}{2} \sum_{\vL \neq \mathbf{0}} v^{\alpha
  \alpha}(|\vL|) \\
& = & \frac{1}{2} \left[ -v_l^{\alpha \alpha}(0) + \sum_\vL v^{\alpha
  \alpha}(|\vL|) \right] \\
& = & \frac{1}{2} \left[ -v_l^{\alpha \alpha}(0) + \sum_\vk v^{\alpha
  \alpha}_\vk \right]  
\end{eqnarray}

\subsubsection{$\vk=\mathbf{0}$ terms}
Thus far, we have neglected what happens at the special point $\vk =
\mathbf{0}$.  For many long-range potentials, such as the coulomb
potential, $v_k^{\alpha \alpha}$ diverges for $k=0$.  However, we
recognize that for a charge-neutral system, the divergent part of the
terms cancel each other.  If all the potential in the system were
precisely coulomb, the $\vk=\mathbf{0}$ terms would cancel precisely,
yielding zero.  For systems involving pseudopotentials, however, it
may be the case the resulting term is finite, but nonzero.  Consider
the terms from $\vk=\mathbf{0}$,
\begin{eqnarray}
V_{k=0} & = & \sum_{\alpha>\beta} N^\alpha N^\beta v^{\alpha \beta}_{k=0}
+ \frac{1}{2} \sum_\alpha \left(N^{\alpha}\right)^2 v^{\alpha\alpha}_{k=0} \\
& = & \frac{1}{2} \sum_{\alpha,\beta} N^\alpha N^\beta v^{\alpha
  \beta}_{k=0}.
\label{eq:kzero}
\end{eqnarray}
Next, we must compute $v^{\alpha \beta}_{k=0}$.  
\begin{equation}
v^{\alpha \beta}_{k=0} = \frac{4 \pi}{\Omega} \int_0^\infty dr\ r^2
v_l^{\alpha \beta}(r)
\end{equation}
We recognize that this integral will not converge because of the
large-$r$ behavior.  However, we recognize that when we do the sum in
(\ref{eq:kzero}), the large-$r$ parts of the integrals will cancel
precisely.  Therefore, we define
\begin{equation}
\tilde{v}^{\alpha \beta}_{k=0} = \frac{4 \pi}{\Omega} 
\int_0^{r_\text{end}} dr\ r^2 v_l^{\alpha \beta}(r),
\end{equation}
where $r_\text{end}$ is some cutoff value after which the potential
tails precisely cancel.

\subsubsection{Neutralizing Background Terms}
For systems with a net charge, such as the one-component plasma
(jellium), we add a uniform background charge which makes the system
neutral.  When we do this, we must add a term which comes from the
interaction of the particle with the neutral background.  It is a
constant term, independent of the particle positions.  In general, we
have a compensating background for each species, which largely cancels
out for neutral systems.
\begin{equation}
V_\text{background} = -\frac{1}{2} \sum_\alpha \left(N^\alpha\right)^2 
v^{\alpha \alpha}_{s\mathbf{0}}
-\sum_{\alpha > \beta} N_\alpha N_\beta
v^{\alpha\beta}_{s\mathbf{0}},
\end{equation}
where $v^{\alpha \beta}_{s\mathbf{0}}$ is given by
\begin{eqnarray}
v^{\alpha \beta}_{s\mathbf{0}} & = & \frac{1}{\Omega} \int_0^{r_c} d^3 r\ 
v^{\alpha \beta}_s(r) \\
& = & \frac{4 \pi}{\Omega} \int_0^{r_c} r^2 v_s(r) \ dr \nonumber
\end{eqnarray}


\subsection{Combining Terms}
Here, we sum all of the terms we computed in the sections above,
\begin{eqnarray}
V & = & \sum_{\alpha > \beta} \left[\sum_{i,j} v_s(|\vr_i^\alpha
  -\vr_j^\beta|) + \sum_\vk \mathcal{R}e\left(\rho_\vk^\alpha
  \rho_{-\vk}^\beta\right)v^{\alpha\beta}_k  -N^\alpha N^\beta
  v^{\alpha \beta}_{s\mathbf{0}}  \right] \nonumber \\
& + & \sum_\alpha \left[ N^\alpha v_M^\alpha + \sum_{i>j} v_s(|\vr_i^\alpha -
  \vr_j^\alpha|) + \frac{1}{2} \sum_\vk \left( |\rho_\vk^\alpha|^2 -
  N\right) v^{\alpha\alpha}_\vk -\frac{1}{2}\left(N_\alpha\right)^2 v_{s\mathbf{0}}^{\alpha\alpha}\right] \nonumber \\
& = & \sum_{\alpha > \beta} \left[\sum_{i,j} v_s(|\vr_i^\alpha
  -\vr_j^\beta|) + \sum_\vk \mathcal{R}e\left(\rho_\vk^\alpha
  \rho_{-\vk}^\beta\right) v^{\alpha \beta}_k   -N^\alpha N^\beta
  v^{\alpha \beta}_{s\mathbf{0}}  +\tilde{V}_{k=0} \right] \\
& + & \sum_\alpha \left[ -\frac{N^\alpha v_l^{\alpha \alpha}(0)}{2}  + \sum_{i>j} v_s(|\vr_i^\alpha -
  \vr_j^\alpha|) + \frac{1}{2} \sum_\vk |\rho_\vk^\alpha|^2 v^{\alpha\alpha}_\vk - \frac{1}{2}\left(N_\alpha\right)^2
  v_{s\mathbf{0}}^{\alpha\alpha} +\tilde{V}_{k=0}\right]  \nonumber
\end{eqnarray}

\subsection {Computing the Reciprocal Potential}
Now we return to (\ref{eq:vk}).  Without loss of generality, we define
for convenience $\vk = k\hat{\mathbf{z}}$.
\begin{equation}
v^{\alpha \beta}_k = \frac{2\pi}{\Omega} \int_0^\infty dr \int_{-1}^1
  d\cos(\theta) \ r^2 e^{-i k r \cos(\theta)} v_l^{\alpha \beta}(r)
\end{equation}
We do the angular integral first.  By inversion symmetry, the
imaginary part of the integral vanishes, yielding
\begin{equation}
v^{\alpha \beta}_k = \frac{4\pi}{\Omega k}\int _0^\infty dr\ r \sin(kr)
v^{\alpha \beta}_l(r).
\label{eq:vkint}
\end{equation}

\subsection{The Coulomb Potential}
For the case of the Coulomb potential, the above integral is not
formally convergent if we do the integral naively. We may remedy the
situation by including a convergence factor, $e^{-k_0 r}$.  For a
potential of the form $v^\text{coul}(r) = q_1 q_2/r$, this yields
\begin{eqnarray}
v^{\text{screened coul}}_k & = & \frac{4\pi q_1 q_2}{\Omega k} \int_0^\infty dr\ \sin(kr)
e^{-k_0r} \\ 
& = & \frac{4\pi q_1 q_2}{\Omega (k^2 + k_0^2)}
\end{eqnarray}
Allowing the convergence factor to tend to zero, we have
\begin{equation}
v_k^\text{coul} = \frac{4 \pi q_1 q_2}{\Omega k^2}
\end{equation}

For more generalized potentials with a coulomb tail, we cannot
evaluate (\ref{eq:vkint}) numerically but must handle the coulomb part
analytically.  In this case, we have
\begin{equation}
v_k^{\alpha \beta} = \frac{4\pi}{\Omega} 
\left\{ \frac{q_1 q_2}{k^2} + \int_0^\infty dr \ r \sin(kr) \left[ v_l^{\alpha \beta}(r) -
  \frac{q_1 q_2}{r} \right] \right\}
\end{equation}

\subsection{Efficient calculation methods}
\subsubsection{Fast computation of $\rho_\vk$}
We wish to quickly calculate the quantity
\begin{equation}
\rho_\vk^\alpha \equiv \sum_i e^{i\vk \cdot r_i^\alpha}
\end{equation}
First, we write 
\begin{eqnarray}
\vk & = & m_1 \vb_1 + m_2 \vb_2 + m_3 \vb_3 \\
\vk \cdot \vr_i^\alpha & = &  m_1 \vb_1 \cdot \vr_i^\alpha + 
m_2 \vb_2 \cdot \vr_i^\alpha + m_3 \vb_3 \cdot \vr_i^\alpha \\
e^{i\vk \cdot r_i^\alpha} & = & 
{\underbrace{\left[e^{i \vb_1 \cdot\vr_i^\alpha}\right]}_{C^{i\alpha}_1}}^{m_1}
{\underbrace{\left[e^{i \vb_2 \cdot\vr_i^\alpha}\right]}_{C^{i\alpha}_2}}^{m_2}
{\underbrace{\left[e^{i \vb_3 \cdot\vr_i^\alpha}\right]}_{C^{i\alpha}_3}}^{m_3}
\end{eqnarray}
Now, we note that
\begin{equation}
[C^{i\alpha}_1]^{m_1} = C^{i\alpha}_1 [C^{i\alpha}]^{(m_1-1)}.
\end{equation}
This allows us to recursively build up an array of the $C^{i\alpha}$s,
and then compute $\rho_\vk$ for all $\vk$-vectors by looping over all
k-vectors, requiring only two complex multiplies per particle per
$\vk$.
\begin{algorithm}
\caption{Algorithm to quickly calculate $\rho_\vk^\alpha$.}
\begin{algorithmic}
\STATE Create list of $\vk$-vectors and corresponding $(m_1, m_2,
m_3)$ indices.
\FORALL{$\alpha \in $ species}
  \STATE Zero out $\rho_\vk^\alpha$
  \FORALL{$i \in $ particles}
    \FOR{$j \in [1\cdots3]$}
      \STATE Compute $C^{i \alpha}_j \equiv e^{i \vb_j \cdot
        \vr^{\alpha}_i}$
       \FOR{$m \in [-m_{\text{max}}\dots m_\text{max}]$}
         \STATE Compute $[C^{i \alpha}_j]^m$ and store in array
       \ENDFOR
    \ENDFOR
     \FORALL{$(m_1, m_2, m_3) \in $ index list}
       \STATE Compute $e^{i \vk \cdot r^\alpha_i} =
         [C^{i\alpha}_1]^{m_1} [C^{i\alpha}_2]^{m_2}
         [C^{i\alpha}_3]^{m_3}$ from array
    \ENDFOR
  \ENDFOR
\ENDFOR
\end{algorithmic}
\end{algorithm}

\subsection{Gaussian Charge Screening Breakup}
This original approach to the short and long-ranged breakup adds an
opposite screening charge of gaussian shape around each point charge.
It then removes the charge in the long-ranged part of the potential.
In this potential,
\begin{equation}
v_{\text{long}}(r) = \frac{q_1 q_2}{r} \text{erf}(\alpha r),
\end{equation}
where $\alpha$ is an adjustable parameter used to control how
short-ranged the potential should be.  If the box size is $L$, a
typical value for $\alpha$ might be $7/(Lq_1 q_2)$. We should note
that this form for the long-ranged potential should also work for any
general potential with a coulomb tail, e.g. pseudo-Hamiltonian
potentials.  For this form of the long-ranged potential, we have in $k$-space
\begin{equation}
v_k = \frac{4\pi q_1 q_2 \exp\left[\frac{-k^2}{4\alpha^2}\right]}{\Omega k^2}.
\end{equation}

\subsection{Optimized Breakup Method}
In this section, we undertake the task of choosing a
long-range/short-range partitioning of the potential which is optimal
in that it minimizes the error for given real and $k$-space cutoffs
$r_c$ and $k_c$.  Here, we modify slightly the method introduced
Natoli and Ceperley\cite{Natoli1995}. We choose $r_c =
\frac{1}{2}\min\{L_i\}$, so that we require the nearest image in
real space summation.  $k_c$ is then chosen so as to satisfy our
accuracy requirements.

Here we modify our notation slightly to accommodate details not
required above.  We restrict our discussion to the interaction of two
particles species (which may be the same), and drop our species
indices.  Thus we are looking for short and long-range potentials
defined by,
\renewcommand{\vs}{v^s}
\newcommand{\vl}{v^\ell}
\begin{equation}
v(r) = \vs(r) + \vl(r)
\end{equation}
Define $\vs_k$ and $\vl_k$ to be the respective Fourier transforms of
the above.  The goal is to choose $v_s(r)$ such that its value and
first two derivatives vanish at $r_c$, while making $\vl(r)$ as smooth as
possible so that $k$-space components, $\vl_k$, are very small for
$k>k_c$.  Here, we describe how to do this is an optimal way.

Define the periodic potential, $V_p$, as 
\begin{equation}
V_p(\vr) = \sum_l v(|\vr + \mathbf{l}|),
\end{equation}
where $\vr$ is the displacement between the two particles and
$\mathbf{l}$ is a lattice vector.  Let us then define our
approximation to this potential, $V_a$, as
\begin{equation}
V_a(\vr) = \vs(r) + \sum_{|\vk| < k_c} \vl_k e^{i\mathbf \vk \cdot \vr}
\end{equation}
Now, we seek to minimize the RMS error over the cell,
\begin{equation}
\chi^2 = \frac{1}{\Omega}\int_\Omega d^3 \mathbf{r} \ 
\left| V_p(\vr) - V_a(\vr)\right|^2 
\end{equation}
We may write
\begin{equation}
V_p(\vr) = \sum_{\vk} v_k e^{i \vk \cdot \vr},
\end{equation}
where 
\begin{equation}
v_k = \frac{1}{\Omega} \int d^3\vr \ e^{-i\vk\cdot\vr}v(r).
\end{equation}
We now need a basis in which to represent the broken up potential.  We
may choose to represent either $\vs(r)$ or $\vl(r)$ in a real-space
basis.  Natoli and Ceperley chose the prior in their paper.  We choose
the latter for a number of reasons.  First, singular potentials are
difficult to represent in a linear basis unless the singularity is
explicitly included.  This requires a separate basis for each type of
singularity.  The short-range potential may have an arbitrary number
of features for $r<r_c$ and still be a valid potential.  By
construction, however, we desire that $\vl(r)$ be smooth in real-space
so that its Fourier transform falls off quickly with increasing $k$.
We therefore expect that, in general, $\vl(r)$ should be
well-represented by fewer basis functions than $\vs(r)$.  Therefore,
we define,
\begin{equation}
\vl(r) \equiv
\begin{cases}
 \sum_{n=0}^{J-1} t_n h_n(r) & \text{for } r \le r_c \\
 v(r) & \text{for } r > r_c.
\end{cases}
\end{equation}
where the $h_n(r)$ are a set of $J$ basis functions.  We require that
the two cases agree on the value and first two derivatives at $r_c$.
We may then define
\begin{equation}
c_{nk} \equiv \frac{1}{\Omega} \int_0^{r_c} d^3 \vr \ e^{-i\vk\cdot\vr} h_n(r).
\end{equation}
Similarly, we define
\begin{equation}
x_k \equiv -\frac{1}{\Omega} \int_{r_c}^\infty d^3\vr \ e^{-i\vk\cdot\vr} v(r)
\end{equation}
Therefore,
\begin{equation}
\vl_k = -x_k + \sum_{n=0}^{J-1} t_n c_{nk} 
\end{equation}
Because $\vs(r)$ goes identically to zero at the box edge, inside the
cell we may write
\begin{equation}
\vs(\vr) = \sum_\vk \vs_k e^{i\vk \cdot \vr}
\end{equation}
We then write
\begin{equation}
\chi^2 = \frac{1}{\Omega} \int_\Omega d^3 \vr \ 
\left| \sum_\vk e^{i\vk \cdot \vr} \left(v_k - \vs_k \right)
-\sum_{|\vk| \le k_c} \vl_k \right|^2
\end{equation}
We see that if we define
\begin{equation}
\vs(r) \equiv v(r) - \vl(r)
\end{equation}
Then
\begin{equation}
\vl_k + \vs_k = v_k,
\end{equation}
which then cancels out all terms for $|\vk| < k_c$.  Then we have
\begin{eqnarray}
\chi^2 & = & \frac{1}{\Omega} \int_\Omega d^3 \vr \ 
\left|\sum_{|\vk|>k_c} e^{i\vk\cdot\vr} 
\left(v_k -\vs_k \right)\right|^2 \\
& = & \frac{1}{\Omega} \int_\Omega d^3 \vr \ 
\left|\sum_{|\vk|>k_c} e^{i\vk\cdot\vr} \vl_k \right|^2 \\ 
& = & 
\frac{1}{\Omega} \int_\Omega d^3 \vr
\left|\sum_{|\vk|>k_c} e^{i\vk\cdot\vr}\left( -x_k + \sum_{n=0}^{J-1} t_n
c_{nk}\right) \right|^2
\end{eqnarray}
We expand the summation,
\newcommand{\ns}{\negthickspace}
\begin{equation}
\chi^2 = \frac{1}{\Omega} \int_\Omega d^3 \vr \ns \ns \ns
\sum_{\{|\vk|,|\vk'|\}>k_c} \ns\ns\ns\ns\ns
 e^{i(\vk-\vk')\cdot \vr}
\left(x_k -\sum_{n=0}^{J-1} t_n c_{nk} \right)
\left(x_k -\sum_{m=0}^{J-1} t_{m} c_{mk'} \right)
\end{equation}
We take the derivative w.r.t. $t_{m}$,
\begin{equation}
\frac{\partial (\chi^2)}{\partial t_{m}} =
\frac{2}{\Omega}\int_\Omega d^3 \vr \ns \ns \ns
\sum_{\{|\vk|,|\vk'|\}>k_c} \ns\ns\ns\ns\ns
 e^{i(\vk-\vk')\cdot \vr}
\left(x_k -\sum_{n=0}^{J-1} t_n c_{nk} \right) c_{mk'}
\end{equation}
We integrate w.r.t. $\vr$, yielding a Kronecker $\delta$.
\begin{equation}
\frac{\partial (\chi^2)}{\partial t_{m}} =
2 \ns\ns\ns\ns\ns\ns\ns 
\sum_{\ \ \ \ \{|\vk|,|\vk'|\}>k_c} \ns\ns\ns\ns\ns\ns\ns \delta_{\vk, \vk'} 
\left(x_k -\sum_{n=0}^{J-1} t_n c_{nk} \right) c_{mk'}
\end{equation}
Summing over $\vk'$ and equating the derivative to zero, we find the
minimum of our error function is given by
\begin{equation}
\sum_{n=0}^{J-1} \sum_{|\vk|>k_c} c_{mk}c_{nk} t_n = 
\sum_{|\vk|>k_c} x_k c_{mk},
\end{equation}
which is equivalent in form to equation (19) in \cite{Natoli1995}, where
we have $x_k$, instead of $V_k$.  Thus, we see that we may optimize
the short-range or long-range potential in simply by choosing to use
$V_k$ or $x_k$ in the above equation.  We now define
\begin{eqnarray}
A_{mn} & \equiv & \sum_{|\vk|>k_c} c_{mk} c_{nk} \\
b_{m} & \equiv & \sum_{|\vk|>k_c} x_k c_{mk}
\end{eqnarray}
Thus, it becomes clear that our minimization equations can be cast in
the canonical linear form,
\newcommand{\bA}{\mathbf{A}}
\newcommand{\bU}{\mathbf{U}}
\newcommand{\bV}{\mathbf{V}}
\newcommand{\bb}{\mathbf{b}}
\newcommand{\bS}{\mathbf{S}}
\begin{equation}
\bA\mathbf{t} = \mathbf{b}.
\end{equation}

\subsubsection{Solution by SVD}
In practice, we note that the matrix $\bA$ frequently becomes singular
in practice.  For this reason, we use the singular value decomposition
to solve for $t_n$.  This factorization decomposes $A$ as
\begin{equation}
\bA = \bU \bS \bV^T,
\end{equation}
where $\bU^T\bU = \bV^T\bV = 1$ and $\bS$ is diagonal.  In this form, we have
\begin{equation}
\mathbf{t} = \sum_{i=0}^{J-1} \left( \frac{\bU_{(i)} \cdot
  \bb}{\bS_{ii}} \right) \bV_{(i)},
\end{equation}
where the parenthesized subscripts refer to columns.  The advantage of
this form is that if $\bS_{ii}$ is zero or very near zero, the
contribution of the $i^{\text{th}}$ of $\bV$, may be neglected, since
it represents a numerical instability and has little physical
meaning.  It represents the fact that the system cannot distinguish
between two linear combinations of the basis functions.  Using the SVD
in this manner is guaranteed to be stable.  This decomposition is
available in LAPACK in the DGESVD subroutine.

\subsubsection{Constraining Values}
Often, we wish to constrain the value of $t_n$ to have a fixed value
to enforce a boundary condition, for example.  To do this, we define
\begin{equation}
\bb' \equiv \vb - t_n \bA_{(n)}.
\end{equation}
We then define $\bA^*$ as $\bA$ with the $n^\text{th}$ row and column
removed, and $\bb^*$ as $\vb'$ with the $n^\text{th}$ element removed.  Then
we solve the reduced equation $\bA^* \mathbf{t}^* = \bb^*$, and
finally insert $t_n$ back into the appropriate place in $\mathbf{t}^*$
to recover the complete, constrained vector $\mathbf{t}$.  This may be
trivially generalized to an arbitrary number of constraints.
\label{sec:contraints}

\subsubsection{The LPQHI basis}
The above discussion was general and independent of the basis used to
represent $\vl(r)$.  In this section, we introduce a convenient basis
of localized interpolant functions, similar to those used for
splines, which have a number of properties which are convenient for
our purposes.  

First, we divide the region from 0 to $r_c$ into $M-1$ subregions,
bounded above and below by points we term {\em knots}, defined by $r_j
\equiv j\Delta$, where $\Delta \equiv r_c/(M-1)$.  We then define
compact basis elements, $h_{j\alpha}$ which span the region
$[r_{j-1},r_{j+1}]$, except for $j=0$ and $j=M$.  For $j=0$, only the
region $[r_0,r_1]$, while for $j=M$, only $[r_{M-1}, r_M]$.  Thus the
index $j$ identify the knot the element is centered on, while $\alpha$
is an integer from 0 to 2 indicating one of three function shapes.
The dual index can be mapped to the single index above by the
relation, $n = 3j + \alpha$.  The basis functions are then defined as
\begin{equation}
h_{j\alpha}(r) = 
\begin{cases}
\ \ \ \, \Delta^\alpha \, \, \sum_{n=0}^5 S_{\alpha n} 
\left( \frac{r-r_j}{\Delta}\right)^n,    & r_j < r \le r_{j+1} \\
(-\Delta)^\alpha \sum_{n=0}^5 S_{\alpha n} 
\left( \frac{r_j-r}{\Delta}\right)^n,    & r_{j-1} < r \le r_j \\
\quad\quad\quad\quad\quad 0, & \text{otherwise},
\end{cases}
\end{equation}
where the matrix $S_{\alpha n}$ is given by
\begin{equation}
S = 
\left[\begin{matrix}
1 & 0 & 0 & -10 & 15 & -6 \\
0 & 1 & 0 & -6  &  8 & -3 \\
0 & 0 & \frac{1}{2} & -\frac{3}{2} & \frac{3}{2} & -\frac{1}{2}
\end{matrix}\right].
\end{equation}
\begin{figure}
\begin{center}
\epsfig{figure=./figures/LPQHI.eps,width=3.5in}
\caption{Basis functions $h_{j0}$, $h_{j1}$, and $h_{j2}$ are shown.
We note at the left and right extremes, the values and first two
derivatives of the functions are zero, while at the center, $h_{j0}$
has a value of 1, $h_{j1}$ has a first derivative of 1, and $h_{j2}$
has a second derivative of 1. \label{fig:LPQHI} }
\end{center}
\end{figure}
Figure~\ref{fig:LPQHI} shows plots of these function shapes.

The basis functions have the property that at the left and right
extremes, i.e. $r_{j-1}$ and $r_{j+1}$, their values and first two
derivatives are zero.  At the center, $r_j$, we have the properties,
\begin{eqnarray}
h_{j0}(r_j)=1, & h'_{j0}(r_j)=0, & h''_{j0}(r_j)= 0 \\
h_{j1}(r_j)=0, & h'_{j1}(r_j)=1, & h''_{j1}(r_j)= 0 \\
h_{j2}(r_j)=0, & h'_{j2}(r_j)=0, & h''_{j2}(r_j)= 1 
\end{eqnarray}
These properties allow the control of the value and first two derivatives
of the represented function at any knot value simply by setting the
coefficients of the basis functions centered around that knot.  Used
in combination with the method described in
section~\ref{sec:contraints} above, boundary conditions can easily be
enforced.  In our case, we wish require that
\begin{equation}
h_{M0} = v(r_c), \ \ h_{M1} = v'(r_c), \ \ \text{and} \ \  h_{M2} = v''(r_c).
\end{equation}
This ensures that $\vs$ and its first two derivatives vanish at $r_c$.

\subsubsection*{Fourier coefficients}
We wish now to calculate the Fourier transforms of the basis
functions, defined as
\begin{equation}
c_{j\alpha k} \equiv \frac{1}{\Omega} \int_0^{r_c} d^3 \vr 
e^{-i \vk \cdot \vr} h_{j\alpha}(r)
\end{equation}
We then may write,
\begin{equation}
c_{j\alpha k} = 
\begin{cases}
\Delta^\alpha \sum_{n=0}^5 S_{\alpha n} D^+_{0 k n}, & j = 0 \\
\Delta^\alpha \sum_{n=0}^5 S_{\alpha n} (-1)^{\alpha+n} D^-_{M k n}, &
j = M \\
\Delta^\alpha \sum_{n=0}^5 S_{\alpha n} 
\left[ D^+_{j k n} + (-1)^{\alpha+n}D^-_{j k n} \right] & \text{otherwise},
\end{cases}
\end{equation}
where
\begin{equation}
D^{\pm}_{jkn} \equiv \frac{1}{\Omega} \int_{r_j}^{r_{j\pm1}} d^3\!\vr \ 
e^{-i\vk \cdot \vr} \left( \frac{r-r_j}{\Delta}\right)^n.
\end{equation}
We then further make the definition that
\renewcommand{\Im}{\text{Im}}
\begin{equation}
D^{\pm}_{jkn} = \pm \frac{4\pi}{k \Omega} 
\left[ \Delta \Im \left(E^{\pm}_{jk(n+1)}\right) + 
r_j \Im \left(E^{\pm}_{jkn}\right)\right]
\end{equation}
It can then be shown that 
\begin{equation}
E^{\pm}_{jkn} =
\begin{cases}
-\frac{i}{k} e^{ikr_j} \left( e^{\pm i k \Delta} - 1 \right) &
\text{if } n=0, \\
-\frac{i}{k} 
\left[ \left(\pm1\right)^n e^{i k (r_j \pm \Delta)} - \frac{n}{\Delta}
E^\pm_{jk(n-1)}  \right] & \text{otherwise}.
\end{cases}
\end{equation}
Note that these equations correct typographical errors present in \cite{Natoli1995}.
\subsubsection{Enumerating $k$-points}
We note that the summations over $k$ which have been ubiquitous in
this paper requires enumeration of the $k$-vectors.  In particular, we
should sum over all $|\vk| > k_c$.  In practice, we must limit our
summation to some finite cutoff value $k_c < |\vk| < k_\text{max}$,
where $k_\text{max}$ should be of order $3000/L$, where $L$ is the
minimum box dimension.  Enumerating these vectors in a naive fashion
even for this finite cutoff would prove quite prohibitive, as it
requires $\sim 10^9$ vectors.

Our first optimization come in realizing that all quantities in this
calculation require only $|\vk|$, and not $\vk$ itself.  Thus, we may
take advantage of the great degeneracy of $|\vk|$.  We create a list
of $(k,N)$ pairs, where $N$ is the number of vectors with magnitude $k$.
We make nested loops over
$n_1$, $n_2$, and $n_3$, yielding $\vk = n_1 \vb_1 + n_2 \vb_2 + n_3
\vb_3$. If $|\vk|$ is in the required range, we check to see if there
is already an entry with that magnitude on our list, incrementing the
corresponding $N$ if there is, or creating a new entry if not.  Doing
so typically saves a factor of $\sim 200$ in storage and computation.

This reduction is still not sufficient for large $k_max$, since it
requires that we still look over $10^9$ entries.  To further reduce
cost, we may pick an intermediate cutoff, $k_\text{cont}$, above which
we will approximate the degeneracy assuming a continuum of
$k$-points.  We stop our exact enumeration at $k_\text{cont}$, and
then add $\sim 1000$ points, $k_i$, uniformly spaced between $k_\text{cont}$
and $k_\text{max}$. We then approximate the degeneracy by
\begin{equation}
N_i = \frac{4 \pi}{3} \frac{\left( k_b^3 -k_a^3\right)}{(2\pi)^3/\Omega},
\end{equation}
where $k_b = (k_i + k_{i+1})/2$ and $k_a = (k_i + k_{i-1})$.  In doing
so, we typically reduce our total number of k-points to sum over $\sim
2500$ from the $10^9$ we had to start.

\subsubsection{Calculating $x_k$'s}
\subsubsection*{The coulomb potential}
For $v(r) = \frac{1}{r}$, $x_k$ is given by
\begin{equation}
x_k^{\text{coulomb}} = -\frac{4 \pi}{\Omega k^2} \cos(k r_c)
\end{equation}

\subsection*{The $1/r^2$ potential}
For $v(r) = \frac{1}{r^2}$, $x_k$ is given by
\begin{equation}
x_k^{1/r^2} = \frac{4 \pi}{\omega k} 
\left[ \text{Si}(k r_c) -\frac{\pi}{2}\right],
\end{equation}
where the {\em sin integral}, $\text{Si}(z)$, is given by
\begin{equation}
\text{Si}(z) \equiv \int_0^z \frac{\sin \ t}{t} dt.
\end{equation}

\subsection*{The $1/r^3$ potential}
For $v(r) = \frac{1}{r^3}$, $x_k$ is given by
\begin{equation}
x_k^{1/r^3} = \frac{4\pi}{\Omega k} 
\left[k\text{Ci}(k r_c) - \frac{\sin(k r_c)}{r_c} \right],
\end{equation}
where the {\em cosine integral}, $\text{Ci}(z)$, is given by
\begin{equation}
\text{Ci}(z) \equiv -\int_z^\infty \frac{\cos t}{t} dt.
\end{equation}

\subsection*{The $1/r^4$ potential}
For $v(r) = \frac{1}{r^4}$, $x_k$ is given by
\begin{equation}
x_k^{1/r^4} = -\frac{4 \pi}{\Omega k} 
\left\{
\frac{k \cos(k r_c)}{2 r_c} + \frac{\sin(k r_c)}{2r_c^2} + \frac{k^2}{2} \left[ \text{Si}(k r_c) - \frac{\pi}{2}\right]\right\}
\end{equation}


%\section{Adapting to PIMC}
%\subsection{Pair actions}
%Let us begin by summarizing what we have done so far.  We began with the many-body Hamiltonian given by 
%\begin{equation}
%\mathcal{H} = \sum_i -\lambda_i \nabla_i^2 + V,
%\end{equation}
%where $V$ is the periodic potential given by (\ref{eq:Vperiodic}), and $\lambda \equiv \frac{\hbar^2}{2m_i}$. 
%
%We approximately solved the action of this Hamiltonian by considering
%the particles pairwise, and solving for the density matrix for the
%density matrix of each pair exactly using the matrix squaring method.
%This yields the the {\em pair action}, defined by
%\begin{equation}
%\rho^{\alpha \beta}(\vr, \vr';\tau) \equiv \rho_0(\vr, \vr';\tau)
%e^{-u^{\alpha \beta}(\vr, \vr';\tau)},
%\end{equation}
%where $\rho_0$ is the {\em free particle} density matrix for species
%$\alpha$ interacting with species $\beta$.  $\rho^{\alpha \beta}$ is
%the density matrix for the pair Hamiltonian
%\begin{equation}
%H^{\alpha\beta} = -\lambda^{\alpha\beta} \nabla^2 + v^{\alpha\beta}(|\vr|),
%\end{equation}
%where $\vr \equiv \vr_i - \vr_j$ and particles $i$ and $j$ are members
%of species $\alpha$ and $\beta$, respectively, and
%$\lambda^{\alpha\beta}$ is given by
%\begin{equation}
%\lambda^{\alpha \beta} = \frac{\hbar^2}{2m_{\alpha}} +
%\frac{\hbar^2}{2m_\beta}.
%\end{equation}
%If the potential $v^{\alpha \beta}(r)$ is long range, then the action,
%$u^{\alpha \beta}(\vr, \vr';\beta)$, will also be long range.  We
%note, however, that the action is not a simple function of the scalar
%$r$, as the potential is.  Experience shows, however, that at large
%distances, the action is well-approximated by
%\begin{eqnarray}
%u^{\alpha\beta}(\vr, \vr';\tau) & \approx & 
%\frac{1}{2} \left[ u^{\alpha\beta}(\vr,\vr;\tau) +
%  u^{\alpha\beta}(\vr',\vr';\tau)\right] \\
%& = & \frac{1}{2} \left[ u^{\alpha\beta}_\text{diag}(r,\tau)+
%u^{\alpha\beta}_\text{diag}(r',\tau)\right]
%\end{eqnarray}
%This is known as the {\em diagonal approximation}.  Thus, as long as
%this approximation is valid at half the minimum box dimension, we may
%break up the diagonal action as we did the potential.  This
%effectively neglects the off-diagonal parts of the action for
%particles more than a half-box length apart, but experience has shown
%that these contributions are usually quite small.  The same
%analysis follows for the $\tau$-derivative the the action, which is
%required to compute the total energy.  Note that PIMC simulation
%requires the pair action at several values of $\tau$, so that in
%practice, we need to do several optimized breakups for each
%$u_\text{diag}^{\alpha\beta}$ and $\dot{u}_\text{diag}^{\alpha\beta}$ and a single breakup for
%each $v^{\alpha\beta}$.
%
%\subsection{Beyond the pair approximation: RPA improvements}
%Consider the limit of a dense gas of charged particles.  We know from
%solid state theory that collective density fluctuations, known as
%plasmons, contribute significantly to the energy spectrum of such a system.
%An approximation to the density matrix determined by considering only
%pairs of particles will neglect these contributions at finite $\tau$.
%As $\tau$ approaches zero, Trotter still guarantees we will approach
%the right limit.
%
%Nonetheless, it is possible to significantly reduce the finite-$\tau$
%timestep error by utilizing a different approximation for the long
%range part of the action.  We begin by defining our effective,
%long-range potential.  As noted above, we may perform an optimized
%breakup on the diagonal action, $u_\text{diag}^{\alpha\beta}(r)$.
%\begin{equation}
%u_\text{diag}^{\alpha\beta}(r) = \hat{u}^{\alpha\beta}_\text{diag}(r) +
%\bar{u}^{\alpha\beta}_\text{diag}(r),
%\end{equation}
%where the $\hat{u}$ and $\bar{u}$ refer to the short and long range
%diagonal actions, respectively, borrowing the notation for short and
%long vowels.
%We subtract the long range part form the total pair action in a
%quasi-primitive approximation by defining
%\begin{equation}
%\bar{u}^{\alpha\beta}_\text{diag}(r) \equiv \tau \bar{v}^{\alpha \beta}(r).
%\end{equation}
%Let $\bar{v}^{\alpha \beta}_k$ represent the Fourier transform the the
%effective potential, $\bar{v}^{\alpha\beta}(r)$.  Finally, let its
%short-range counterpart be defined by 
%\begin{equation}
%\hat{v}^{\alpha \beta}_k \equiv v^{\alpha\beta}_k - \bar{v}^{\alpha\beta}_k
%\end{equation}
%
%Now, we wish to reintroduce a new long range action, which we will
%calculate in $k$-space within the {\em Random Phase Approximation
%  (RPA)}.   We begin with the Bloch equation,
%\begin{equation}
%\dot{\rho} = -\mathcal{H} \rho,
%\end{equation}
%where the dot refers to differentiation w.r.t. $\tau$.  The
%Hamiltonian is given by
%\begin{equation}
%\mathcal{H} = \left[\sum_\alpha \sum_{i\in \alpha} -\lambda_\alpha
%\nabla_i^2\right] + \hat{V} + \bar{V},
%\end{equation}
%where $\hat{V}$ and $\bar{V}$ are the total short and long range
%periodic potentials, respectively.
%Let us now make the partitioning that
%\begin{equation}
%\rho(\vR, \vR';\tau) = \rho_0(\vR, \vR';\tau) e^{-\hat{U}(\vR,
%  \vR';\tau)} e^{-\bar{U}(\vR, \vR';\tau)},
%\end{equation}
%We assume that $\rho_s \equiv \rho_0 e^{-\hat{U}}$ satisfies the Bloch
%equation for the short-range Hamiltonian,
%\begin{equation}
%\mathcal{H}_s = \left[\sum_\alpha \sum_{i\in \alpha} -\lambda_\alpha
%\nabla_i^2\right] + \hat{V}.
%\end{equation}
%In fact, this is only strictly true in the limit that $\tau=0$, but
%this relation will suffice for our present analysis.
%
%
%Recall that $\nabla^2(ab) = a\nabla^2 b + b\nabla^2a +2(\nabla a)
%\cdot (\nabla b)$.  Thus, we have for our Bloch equation,
%\begin{eqnarray}
%-\left [\dot{\rho_s} -\rho_s\dot{\bar{U}}\right] e^{-\bar{U}} & = &
%\sum_{\alpha,\  i\in\alpha} -\lambda_\alpha
%\left[\rho_s \nabla^2_i e^{-\bar{U}} + e^{-\bar{U}} \nabla^2_i \rho_s
%  + 2(\nabla_i \rho_s)\cdot (\nabla_i e^{-\bar{U}}) 
%\right] \nonumber \\ & & + (\hat{V} + \bar{V}) \rho_s e^{-\bar{U}}.
%\end{eqnarray} 
%Subtracting the Bloch equation for the short range part,
%we are left with
%\begin{equation}
%\left[\dot{\bar{U}}-\bar{V}\right] \rho_s e^{-\bar{U}}  = 
%\sum_{\alpha,\  i\in\alpha} -\lambda_\alpha
%\left[\rho_s \nabla^2_i e^{-\bar{U}}
%  + 2(\nabla_i \rho_s)\cdot (\nabla_i e^{-\bar{U}}) 
%\right].
%\end{equation} 
%Recall that
%\begin{eqnarray}
%\nabla e^{-\bar{U}} & = & -\nabla\bar{U}e^{-\bar{U}} \\
%\nabla \rho_0 & = & 0 \ \ \ \ \ \ \ \ \ \ \ \ \ \ \ \ \ \ \ \ \ \ \ \
%\ \ \ \ \ \ \ \ \ 
%\text{ (for $\vR = \vR'$)} \\
%\nabla \rho_s & = & -\rho_s \nabla \hat{U} \\
%\nabla^2 e^{-\bar{U}} & = & 
%\left[(\nabla \bar{U})^2 - \nabla^2 \bar{U}\right] e^{-\bar{U}} 
%\end{eqnarray} 
%We now attempt to solve the Bloch equation under the restriction that
%$\vR = \vR'$, i.e. along the diagonal of the density matrix.  Hence
%let us define
%\begin{eqnarray} 
%\bar{U}(\vR, \vR';\tau) & \equiv &
%  \frac{1}{2}\left[\bar{\mathcal{U}}(\vR;\tau) +
%  \bar{\mathcal{U}}(\vR';\tau) \right] \\ 
% \hat{U}(\vR,\vR';\tau) & \equiv & 
%  \frac{1}{2} \left[\hat{\mathcal{U}}(\vR;\tau) +
%  \hat{\mathcal{U}}(\vR';\tau) \right] 
%\end{eqnarray}
%as the long and short range diagonal actions written as
%functions of only one spatial argument.  Then we have, along the
%diagonal,
%\begin{eqnarray}
%\nabla U   & = & \frac{1}{2} \nabla\mathcal{U} \\
%\nabla^2 U & = & \frac{1}{2} \nabla^2\mathcal{U}.
%\end{eqnarray}
%Substituting back into out Bloch equation,
%\begin{equation}
%\dot{\bar{\mathcal{U}}} = \sum_{\alpha, \ i\in \alpha} -\lambda_\alpha
%\left\{ \frac{1}{4} (\nabla_i \bar{\mathcal{U}})^2 - 
%\frac{1}{2}\nabla_i^2 \bar{\mathcal{U}} + \frac{1}{2} (\nabla_i\hat{\mathcal{U}})
%  \cdot (\nabla_i \bar{\mathcal{U}}) \right\} +\bar{V}
%\end{equation}
%
%We recall that the long range potential, $\bar{V}$, may be written as
%\begin{equation}
%\bar{V} = \sum_\vk \sum_{\alpha} \left[ 
%\frac{1}{2} \left| \rho^\alpha_\vk\right|^2 \bar{v}^{\alpha
%  \alpha}_k + 
%\sum_{\beta < \alpha} \mathcal{R}e \left( \rho^{\alpha}_\vk
%  \rho^\beta_{-\vk} \bar{v}^{\alpha\beta}_k \right)
%\right] 
%\end{equation}
%When we wrote this expression above, we did so to optimize the speed
%of computation.  For the following analysis, we will find it more
%convenient to write
%\begin{equation}
%\bar{V} = \frac{1}{2} \sum_\vk \sum_{\alpha, \beta} \rho_\vk^\alpha
%\rho_{-\vk}^\beta v^{\alpha \beta}_k.
%\end{equation}
%The sum is guaranteed to be real since for every $\vk$, we have a
%corresponding $-\vk$ in the sum.  Hence we need not be concerned by
%taking the real part. We may similarly write $\bar{\mathcal{U}}$ and $\hat{\mathcal{U}}$ in
%terms of $\bar{u}_k^{\alpha\beta}$ and $\hat{u}_k^{\alpha\beta}$.
%
%
%We now proceed to calculate gradients and laplacians.
%Recall that 
%\begin{equation}
%\rho_\vk^\alpha = \sum_{i\in\alpha} e^{i\vk \cdot \vr_i}
%\end{equation}
%\begin{eqnarray}
%\nabla_i \mathcal{U} & = & \frac{1}{2}\sum_\vk \left[ i\vk e^{i\vk \cdot \vr_i} \sum_\alpha
%\rho_{-\vk}^\alpha u^{\alpha \beta}_k + \text{c.c.} \right] \\
%& = & \frac{1}{2} \sum_\vk 2\mathcal{R}e \left[i\vk e^{i\vk \cdot \vr_i} \sum_\alpha
%\rho_{-\vk}^\alpha u^{\alpha \beta}_k\right] \\ 
%& = & \mathcal{R}e \left[ \sum_\vk i\vk e^{i\vk \cdot \vr_i} \sum_\alpha
%\rho_{-\vk}^\alpha u^{\alpha \beta}_k \right] \\
%& = & \sum_\vk i\vk e^{i\vk \cdot \vr_i} \sum_\alpha
%\rho_{-\vk}^\alpha u^{\alpha \beta}_k.
%\end{eqnarray}
%In the last line, we have again recognized that for every $\vk$
%there is a corresponding $-\vk$, so that the sum is purely real.
%
%Next, we compute the Laplacian w.r.t. the $i^{\text{th}}$ particle.
%\begin{eqnarray}
%\nabla^2_i \mathcal{U} & = & \nabla_i \cdot \nabla_i \mathcal{U} \\
%& = & \nabla_i \cdot \sum_\vk i\vk e^{i\vk \cdot \vr_i} \sum_\alpha
%\rho_{-\vk}^\alpha u^{\alpha \sigma_i}_k \\
%& = & \sum_\vk i\vk \cdot \nabla_i \left[ e^{i\vk \cdot \vr_i}
%  \sum_\alpha \rho_{-\vk}^\alpha u^{\alpha\sigma_i}_k \right] \\
%& = & \sum_{\vk} k^2 \left[ u^{\sigma_i \sigma_i}_k - e^{i\vk\cdot\vr_i}\sum_\alpha \rho_{-\vk}
%u_k^{\alpha \sigma_i}\right],
%\end{eqnarray}
%where $\sigma_i$ is the species of the $i^{\text{th}}$ particle.  Now,
%let us sum over all particles,
%\begin{eqnarray}
%\sum_i \lambda_i \nabla^2_i \mathcal{U} & = & \sum_\vk k^2 \left[\sum_\beta N_\beta u_k^{\beta
%  \beta} - \rho_{\vk}^\beta \sum_\alpha \rho_{-\vk} u_k^{\alpha \beta}
%  \right] \\
%& = & \sum_{\vk} k^2 \sum_{\alpha, \beta}
%  \lambda_\beta \left[N^{\alpha}\delta_{\alpha,\beta} -
%  \rho_{-\vk}^{\alpha}\rho_\vk^\beta \right]u_k^{\alpha \beta} 
%%\\
%%& = & \sum_{\vk} k^2 \sum_{\alpha, \beta}
%%  \left(\frac{\lambda_\alpha +\lambda_\beta}{2} \right) \left[N^{\alpha}\delta_{\alpha,\beta} -
%%  \rho_{-\vk}^{\alpha}\rho_\vk^\beta \right]u_k^{\alpha \beta}.
%\end{eqnarray}
%%In the last step, we have added half the sum with $\alpha$ and $\beta$
%%swapped so as to symmetrize the summation.
%Now, let us consider the cross term,
%\begin{eqnarray}
%(\nabla_i \hat{\mathcal{U}}) \cdot ( \nabla_i \bar{\mathcal{U}} ) 
%& = & \left[\sum_\vk i\vk e^{i\vk\cdot \vr_i} \sum_\alpha
%  \rho_{-\vk}^\alpha \hat{u}^{\sigma_i \alpha}_k \right] \cdot
%\left[\sum_\vq i\vq e^{i\vq\cdot \vr_i} \sum_\beta
%  \rho_{-\vk}^\beta \bar{u}^{\sigma_i \beta}_k \right] \nonumber \\
%& = & -\sum_{\vk,\vq} \vk \cdot \vq e^{i(\vk + \vq)\cdot \vr_i}
%\sum_{\alpha, \beta} \rho_{-\vk}^\alpha \rho_{-\vq}^\beta 
%\hat{u}^{\alpha \sigma_i}_k \bar{u}^{\beta \sigma_i}_k
%\end{eqnarray}
%Again, summing over all particles,
%\begin{equation}
%\sum_i (\nabla_i \hat{\mathcal{U}}) \cdot ( \nabla_i \bar{\mathcal{U}} ) =
%-\sum_{\vk, \vq} \vk \cdot \vq \sum_{\alpha, \beta, \gamma}
%\rho_{\vk + \vq}^\gamma \rho_{-\vk}^{\alpha} \rho_{-\vq}^\beta
%\hat{u}^{\alpha \gamma}_k \bar{u}^{\beta \gamma}_k
%\end{equation}
%Similarly,
%\begin{equation}
%\sum_i (\nabla_i \bar{\mathcal{U}})^2 = -\sum_{\vk, \vq} \vk \cdot \vq 
%\sum_{\alpha, \beta, \gamma} \rho^\gamma_{\vk+\vq} \rho^\alpha_{-\vk}
%\rho^\gamma_{-\vq} \bar{u}^{\alpha \gamma}_k \bar{u}^{\beta \gamma}_k
%\end{equation}
%
%The {\em Random Phase Approximation} (RPA) amounts to the assumption
%that $\rho^\gamma_{\vk + \vq} \approx N_\gamma \delta_{\vk + \vq}$. 
%Then we have,
%\begin{eqnarray}
%\sum_i (\nabla_i \hat{\mathcal{U}}) \cdot ( \nabla_i \bar{\mathcal{U}}
%) & \overset{\text{RPA}}{=} &
%\sum_\vk k^2 \sum_{\alpha, \beta, \gamma} N_\gamma \rho^\alpha_{-\vk}
%\rho^\beta_\vk \hat{u}^{\alpha \gamma}_k \bar{u}^{\beta \gamma}_k \\
%\sum_i (\nabla_i \bar{\mathcal{U}})^2 & \overset{\text{RPA}}{=} &
%\sum_\vk k^2 \sum_{\alpha, \beta, \gamma} N_\gamma
%\rho^{\alpha}_{-\vk} \rho^\beta_\vk \bar{u}_k^{\alpha \gamma}
%\bar{u}_k^{\beta \gamma}
%\end{eqnarray}
%We now return to the Bloch equation
%\begin{equation}
%\begin{split}
%\sum_\vk \sum_{\alpha,\beta} & \left\{ 
%\frac{1}{2} \rho_\vk^\alpha
%\rho_{-\vk}^\beta \left(\dot{\bar{u}}_k - \bar{v}_k^{\alpha \beta} \right)
%+\frac{1}{2} \lambda_\alpha k^2 \bar{u}_k^{\alpha \beta}
%\left(\rho_{-\vk}^\alpha 
%  \rho_\vk^\beta - N_\beta \delta_{\alpha,\beta} \right) 
%\right. \\
%& \left. -\sum_\gamma k^2 N_\gamma \lambda_\gamma \rho_{-\vk}^\alpha
%  \rho_\vk^\beta \left[ 
%\frac{1}{4} \hat{u}^{\alpha \gamma}_k \bar{u}^{\beta\gamma}_k +
%\frac{1}{2} \bar{u}^{\alpha \gamma}_k \bar{u}^{\beta \gamma}
%\right]\right\} = 0
%\end{split}
%\end{equation}
%Next, we symmetrize this equation w.r.t $\alpha$ and $\beta$.
%\begin{equation}
%\begin{split}
%\sum_{\vk, \alpha, \beta} & \left\{ \left( \rho^\alpha_{\vk} \rho^\beta_{-\vk} 
%+ \rho^\alpha_{-\vk} \rho^\beta_{\vk} \right) \left[
%\dot{\bar{u}}_k^{\alpha \beta} - \bar{v}_k^{\alpha \beta}
% +k^2 \left(\frac{\lambda_\alpha+\lambda_\beta}{2}\right) 
%\bar{u}_k^{\alpha \beta} \rule{0cm}{0.6cm} \right.\right. \\
%& \ \ \left.\left.+\sum_\gamma \frac{k^2}{2} N^\gamma \left(
%\bar{u}_k^{\alpha \gamma} \bar{u}_k^{\beta \gamma} +
%\hat{u}_k^{\alpha \gamma} \bar{u}_k^{\beta \gamma} +
%\bar{u}_k^{\alpha \gamma} \hat{u}_k^{\beta \gamma}  \right)
%\right]
%- k^2 N^\alpha \delta_{\alpha \beta}
%\right\} = 0
%\end{split}
%\end{equation}
%We require that this expression hold independent of the positions of
%the particles, i.e. independent of the values of $\rho^\alpha_{\vk}$ and
%$\rho^\beta_{\vk}$.  Thus, the equations separate for each value of
%$\vk$, $\alpha$, and $\beta$.  For $\vk \neq 0$,
%\begin{equation}
%\dot{\bar{u}}_k^{\alpha \beta} = \bar{v}_k^{\alpha \beta} 
%- k^2 \left(\frac{\lambda_\alpha+\lambda_\beta}{2}\right)
%\bar{u}_k^{\alpha\beta} -\frac{k^2}{2} \sum_\gamma N_\gamma
%\left(
%\bar{u}_k^{\alpha \gamma} \bar{u}_k^{\beta \gamma} +
%\hat{u}_k^{\alpha \gamma} \bar{u}_k^{\beta \gamma} +
%\bar{u}_k^{\alpha \gamma} \hat{u}_k^{\beta \gamma}
%\right)
%\end{equation}
%Next, we need an equation for the time propagation of
%$\hat{u}_k^{\alpha \beta}$.  Above, we assumed that $\hat{U}$ was the
%solution to the short-range problem.  Our Bloch equation for
%$\hat{mathcal{U}}$ is then given by
%\begin{equation}
%\dot{\hat{\mathcal{U}}} = \sum_i -\lambda_i
%\left\{ \frac{1}{4} (\nabla_i \hat{\mathcal{U}} 
%-\frac{1}{2} \nabla_i^2 \hat{\mathcal{U}} \right\} + \hat{V}
%\end{equation}
%Following the RPA procedure above, we arrive at the following
%equations for $\hat{u}_k^{\alpha\beta}$.
%\begin{equation}
%\dot{\hat{u}}^{\alpha \beta}_k = \hat{v}^{\alpha \beta}_k
%-k^2 \left( \frac{\lambda_\alpha + \lambda_\beta}{2} \right)
%\hat{u}^{\alpha \beta}_k - \frac{k^2}{2} \sum_\gamma N_\gamma
%\hat{u}^{\alpha \gamma}_k \hat{u}^{\beta \gamma}_k.
%\end{equation}
%Hence, for each value of $k$, we have a coupled set of differential
%equations we must solve.  We note that while the equations for
%$\bar{u}$ couple to $\hat{u}$, those for $\hat{u}$ do not couple to
%$\bar{u}$.
%%% \begin{equation}
%%% \begin{split}
%%% \sum_\vk \sum_{\alpha,\beta} 
%%%  & \left\{
%%% \rho_{-\vk}^\alpha \rho_\vk^\beta 
%%% \left[
%%% \dot{\bar{u}}^{\alpha \beta}_k + k^2 \sum_\gamma \lambda_\gamma
%%% N_\gamma 
%%% \left(\frac{\bar{u}^{\alpha \gamma}_k \bar{u}^{\beta \gamma}_k}{4} -
%%% \frac{\hat{u}^{\alpha \gamma}_k \bar{u}^{\beta \gamma}_k}{2}
%%% \right) 
%%% \frac{k^2}{2} \lambda_\beta \bar{u}_k^{\alpha \beta} +
%%% \bar{v}_k^{\alpha \beta}
%%% \right] \right.\\
%%%  & \left. \rule{0pt}{0.6cm}
%%% + \frac{k^2}{2}\lambda_\beta N_\beta \delta_{\alpha,\beta}
%%% \bar{u}_k^{\alpha \beta}
%%% \right\} = 0
%%% \end{split}
%%% \end{equation}
%
%
%%% While this
%%% is correct in the limit that the timestep, $\tau$, goes to zero, it
%%% may incur a substantial error for finite $\tau$.  In this section, we
%%% describe a method to reduce the timestep error of the long range part
%%% of the action by using the Bloch equation combined with the Random
%%% Phase Approximation (RPA).  
%
%%% The Bloch equation may be written,
%%% \begin{equation}
%%% \dot{\rho} = -\mathcal{H} \rho,
%%% \end{equation}
%%% where the dot indicates differentiation with respect to $\tau$.  Now,
%%% we define
%%% \begin{equation}
%%% \rho = \rho_0 e^{-U_s}e^{-U_l}.
%%% \end{equation}
%%% \begin{equation}
%%% \mathcal{H} = \left[ -\lambda \sum_i \nabla_i^2 \right] + V_s + V_l
%%% \end{equation}
%%% The Bloch equation gives us 




\newpage
\section{Feature: Optimized Long-Ranged Breakup (Ewald) 2}

% Written by Simone Chiesa for the FITPN code/tool (Ceperley)
% Originally titled ``Notes on fitnp''

\newcommand{\rv}{\mathbf{r}}
\newcommand{\kv}{\mathbf{k}}
\newcommand{\Rv}{\mathbf{R}}
\newcommand{\Lv}{\mathbf{L}}
\newcommand{\Rc}{\mathcal{R}}
\newcommand{\tV}{\widetilde{V}}
\newcommand{\tW}{\widetilde{W}}
\newcommand{\tc}{\widetilde{c}}
\newcommand{\tY}{\widetilde{Y}}
\newcommand{\Nk}{N_\text{knot}}
\newcommand{\wk}{w_\text{knot}}

Given a lattice of vectors $\Lv$, its associated reciprocal
lattice of vectors $\kv$ and a function $\psi(\rv)$ periodic
on the lattice we define its Fourier transform $\widetilde{\psi}(\kv)$ as
\begin{equation}
\widetilde{\psi}(\kv)=\frac{1}{\Omega}\int_\Omega d\rv \psi(\rv) e^{-i\kv\rv}
\end{equation}
where we indicated both the cell domain and the cell volume by $\Omega$. 
$\psi(\rv)$ can then be expressed as
\begin{equation}
\psi(\rv)=\sum_{\kv} \widetilde{\psi}(\kv)e^{i\kv\rv}
\end{equation}
The potential generated by charges sitting on the lattice positions
at a particular point $\rv$ inside the cell is given by
\begin{equation}
V(\rv)=\sum_{\Lv}v(|\rv+\Lv|)
\end{equation}
and its Fourier transform can be explicitly written as a function of $V$ or $v$
\begin{equation}
\widetilde{V}(\kv)=\frac{1}{\Omega}\int_\Omega d\rv V(\rv) e^{-i\kv\rv}=
\frac{1}{\Omega}\int_{\mathbb{R}^3} d\rv v(\rv) e^{-i\kv\rv}
\end{equation}
where $\mathbb{R}^3$ denotes the whole 3-dimensional space.
We now want to find the best (``best'' to be defined later) approximate 
potential of the form
\begin{equation}
V_a(\rv)=\sum_{k\le k_c} \widetilde{Y}(k) e^{i\kv\rv} + W(r)
\end{equation}
where $W(r)$ has been chosen to go to $0$ smoothly when $r=r_c$, being
$r_c$ lower or equal to the Wigner-Seitz radius of the cell. Note also
the cutoff $k_c$ on the momentum summation.

The best form of $\widetilde{Y}(k)$ and $W(r)$ is given by minimizing
\begin{equation}
  \chi^2=\frac{1}{\Omega}\int d\rv \left(V(\rv)-W(\rv)-
  \sum_{k\le k_c}\widetilde{Y}(k)e^{i\kv\rv}\right)^2
  \label{chi2r}
\end{equation}
or the reciprocal space equivalent
\begin{equation}
  \chi^2=\sum_{k\le k_c}(\tV(k)-\tW(k)-\tY(k))^2+\sum_{k>k_c}(\tV(k)-\tW(k))^2
  \label{chi2k}
\end{equation}
Eq.\ref{chi2k} follows from Eq.\ref{chi2r} and the unitarity
(norm conservation) of the Fourier transform.

This last condition is minimized by
\begin{equation}
\tY(k)=\tV(k)-\tW(k)\qquad \min_{\tW(k)}\sum_{k>k_c}(\tV(k)-\tW(k))^2
\label{mincond}
\end{equation}
We now use a set of basis function $c_i(r)$ vanishing smoothly at $r_c$
to expand $W(r)$ i.e.
\begin{equation}
W(r)=\sum_i t_i c_i(r)\qquad\text{or}\qquad \tW(k)=\sum_i t_i \tc_i(k)
\end{equation}
Inserting the reciprocal space expansion of $\tW$ in the second condition of
Eq.\ref{mincond} and minimizing with respect to $t_i$ leads immediately
to the linear system $\mathbf{A}\mathbf{t}=\mathbf{b}$ where
\begin{center}
\vskip 3mm
\begin{eqnarray}
A_{ij}=\sum_{k>k_c}\tc_i(k)\tc_j(k)\qquad b_j=\sum_{k>k_c} V(k) \tc_j(k)
\label{matrix_elements}
\end{eqnarray}

\end{center}
\vskip 3mm

\subsection{Basis functions}
The basis functions are splines. We define a uniform grid 
with $\Nk$ uniformly spaced knots at position $r_i=i\frac{r_c}{\Nk}$ 
where $i\in[0,\Nk-1]$. On each knot we center $m+1$ piecewise polynomials
$c_{i\alpha}(r)$ with $\alpha\in[0,m]$, defined as
\vskip 3mm
\begin{center}
\begin{eqnarray}
c_{i\alpha}(r)=\begin{cases}
\Delta^\alpha \sum_{n=0}^\mathcal{N} S_{\alpha n}(\frac{r-r_i}{\Delta})^n & r_i<r\le r_{i+1} \\
\Delta^{-\alpha} \sum_{n=0}^\mathcal{N} S_{\alpha n}(\frac{r_i-r}{\Delta})^n & r_{i-1}<r\le r_i \\
0 & |r-r_i| > \Delta
\end{cases}
\label{basisdef}
\end{eqnarray}

\end{center}
\vskip 3mm
These functions and their derivatives are, by construction, continuous and odd (even)
(with respect to $r-r_i\rightarrow r_i-r$) when $\alpha$ is odd (even).
We further ask them to satisfy
\begin{eqnarray}
\left.\frac{d^\beta}{dr^\beta} c_{i\alpha}(r)\right|_{r=r_i}=
\delta_{\alpha\beta} \quad \beta\in[0,m]\\
\left.\frac{d^{\beta}}{dr^{\beta}} c_{i\alpha}(r)\right|_{r=r_{i+1}}=0\quad \beta\in[0,m]
\label{constr}
\end{eqnarray}
(The parity of the functions guarantees that the second constraint is satisfied
at $r_{i-1}$ as well). These constraints have a simple interpretation: the basis functions
and their first $m$ derivatives are $0$ on the boundary of the subinterval where they
are defined; the only function to have a non zero $\beta$-th derivative in $r_i$ is $c_{i\beta}$.
These $2(m+1)$ constraints therefore impose $\mathcal{N}=2m+1$. 
Inserting the definitions of Eq.(\ref{basisdef}) in the constraints of Eq.(\ref{constr})
leads to the set of $2(m+1)$ linear equation that fixes the value of $S_{\alpha n}$: 
\begin{eqnarray}
\Delta^{\alpha-\beta} S_{\alpha\beta} \beta!=\delta_{\alpha\beta}
\label{Smatrix1}\\
\Delta^{\alpha-\beta}\sum_{n=\beta}^{2m+1} S_{\alpha n} \frac{n!}{(n-\beta)!}=0
\end{eqnarray}
One can further simplify inserting the first of these equations into the second and write
the linear system as
\begin{equation}
\sum_{n=m+1}^{2m+1} S_{\alpha n} \frac{n!}{(n-\beta)!}=\begin{cases}
-\frac{1}{(\alpha-\beta)!}& \alpha\ge \beta \\
0 & \alpha < \beta
\end{cases}
\label{Smatrix2}
\end{equation}

\subsection{Fourier components of the basis functions in 3D}
\subsubsection*{$k\ne 0$, non coulomb case.}
We now need to evaluate the Fourier transform $\tc_{i\alpha}(k)$. Let us start
by writing the definition
\begin{equation}
\tc_{i\alpha}(k)=\frac{1}{\omega}\int_\Omega d\rv  e^{-i\kv\rv} c_{i\alpha}(r)
\end{equation}
Because $c_{i\alpha}$ is different from zero only inside the spherical crown
defined by $r_{i-1}<r<r_i$ one can conveniently compute the integral in spherical
coordinates as
\vskip 3mm
\begin{center}
\begin{eqnarray}
\tc_{i\alpha}(k)=\Delta^\alpha\sum_{n=0}^\mathcal{N} S_{\alpha n} \left[
D_{in}^+(k) +\wk(-1)^{\alpha+n}D_{in}^-(k)\right]
\label{fourier_transform}
\end{eqnarray}
\end{center}
\vskip 3mm
where we used the definition $\wk=1-\delta_{i0}$ and
\begin{equation}
D_{in}^\pm(k)=\pm\frac{4\pi}{k\Omega}\Im\left[\int_{r_i}^{r_i\pm\Delta}
dr\left(\frac{r-r_i}{\Delta}\right)^n r e^{ikr}\right]
\label{D+-}
\end{equation}
obtained by integrating the angular part of the Fourier transform.
Using the identity
\begin{equation}
\left(\frac{r-r_i}{\Delta}\right)^n r=\Delta\left(\frac{r-r_i}{\Delta}\right)^{n+1}+\left(\frac{r-r_i}{\Delta}\right)^n r_i
\end{equation}
and the definition
\begin{equation}
E_{in}^\pm(k)=\int_{r_i}^{r_i\pm\Delta}
dr\left(\frac{r-r_i}{\Delta}\right)^n e^{ikr}
\end{equation}
we rewrite Eq.\ref{D+-} as
\begin{center}
\vskip 3mm
\begin{eqnarray}
D_{in}^\pm(k)=\pm\frac{4\pi}{k\Omega}\Im\left[\Delta E_{i(n+1)}^\pm(k)+
r_i E_{in}^\pm(k)\right]
\label{noncoulD+-}
\end{eqnarray}
\end{center}
\vskip 3mm

Finally, using integration by part, one can define $E^\pm_{in}$ recursively
\begin{center}
\vskip 3mm
\begin{eqnarray}
E^\pm_{in}(k)=\frac{1}{ik}\left[(\pm)^ne^{ik(r_i\pm\Delta)}-\frac{n}{\Delta}
E^\pm_{i(n-1)}(k)\right]
\label{nthEpm}
\end{eqnarray}
\end{center}
\vskip 3mm
\noindent
starting from the $n=0$ term
\vskip 3mm
\begin{center}
\begin{eqnarray}
E^\pm_{i0}(k)=\frac{1}{ik}e^{ikr_i}\left(e^{\pm ik\Delta}-1\right)
\label{0thEpm}
\end{eqnarray}
\end{center}
\vskip 3mm
\subsubsection{$k\ne 0$, coulomb case.}
To efficiently treat the coulomb divergence at the origin it is convenient to use
a basis set $c_{i\alpha}^\text{coul}$ of the form 
\begin{equation}
c_{i\alpha}^\text{coul}=\frac{c_{i\alpha}}{r}
\end{equation}
An equation identical to Eq.\ref{D+-} holds but with the modified definition
\begin{equation}
D_{in}^\pm(k)=\pm\frac{4\pi}{k\Omega}\Im\left[\int_{r_i}^{r_i\pm\Delta}
dr\left(\frac{r-r_i}{\Delta}\right)^n e^{ikr}\right]
\end{equation}
which can be simply expressed using $E^\pm_{in}(k)$ as
\vskip 3mm
\begin{center}
\begin{eqnarray}
D_{in}^\pm(k)=\pm\frac{4\pi}{k\Omega}\Im\left[E_{in}^\pm(k)\right]
\label{coulD+-}
\end{eqnarray}
\end{center}
\vskip 3mm
\subsubsection{$k=0$ coulomb and non coulomb case.}
The definitions of $D_{in}(k)$ given so far are clearly incompatible 
with the choice $k=0$ (they involve division by $k$). For the non-coulomb
case the starting definition is
\begin{equation}
D^\pm_{in}(0)=\pm\frac{4\pi}{\Omega}\int_{r_i}^{r_i\pm\Delta}r^2
\left(\frac{r-r_i}{\Delta}\right)^ndr
\end{equation}
Using the definition $I_n^\pm=(\pm)^{n+1}\Delta/(n+1)$ we can express this
as
\begin{center}
\vskip 3mm
\begin{eqnarray}
D^\pm_{in}(0)=\pm\frac{4\pi}{\Omega}\left[\Delta^2 I_{n+2}^\pm
+2r_i\Delta I_{n+1}^\pm+2r_i^2I_n^\pm\right]
\label{noncoul_k=0D+-}
\end{eqnarray}
\end{center}
\vskip 3mm
For the coulomb case one get

\vskip 3mm
\begin{center}
\begin{eqnarray}
D^\pm_{in}(0)=\pm\frac{4\pi}{\Omega}\left(
\Delta I^\pm_{n+1} + r_i I^\pm_n\right)
\label{coul_k=0D+-}
\end{eqnarray}
\end{center}
\vskip 3mm
\subsection{Fourier components of the basis functions in 2D}
Eq.\ref{fourier_transform} still holds provided we define  
\begin{equation}
D^\pm_{in}(k)=\pm\frac{2\pi}{\Omega \Delta^n} \sum_{j=0}^n \binom{n}{j}
(-r_i)^{n-j}\int_{r_i}^{r_i\pm \Delta}\negthickspace \negthickspace 
\negthickspace \negthickspace \negthickspace \negthickspace \negthickspace 
dr r^{j+1-C} J_0(kr)
\label{2DD+-}
\end{equation}
where $C=1(=0)$ for the coulomb(non coulomb) case.
Eq.\ref{2DD+-} is obtained using the integral definition of the 
zero order Bessel function of the first kind 
\begin{equation}
J_0(z)=\frac{1}{\pi}\int_0^\pi e^{iz\cos\theta}d\theta
\end{equation}
and the binomial expansion for $(r-r_i)^n$.
The integrals can be computed recursively using the following identities
\begin{center}
\begin{minipage}{0.7\textwidth}
\begin{align}
&\int dz J_0(z)=\frac{z}{2}\left[\pi J_1(z)H_0(z)+J_0(z)(2-\pi H_1(z))\right]
\label{0thmoment}\\
&\int dz z J_0(z)= z J_1(z)
\label{1stmoment}\\
&\int dz z^n J_0(z)= z^nJ_1(z)+(n-1)x^{n-1}J_0(z)
-(n-1)^2\int dz z^{n-2} J_0(z)
\label{nthmoment}
\end{align}
\end{minipage}
\end{center}
Eq.\ref{nthmoment} is obtained using Eq.\ref{1stmoment}, integration by part and 
the identity $\int J_1(z) dz =-J_0(z)$. In Eq.\ref{0thmoment} $H_0$ and $H_1$ are Struve functions.

\subsection{Construction of the matrix elements}
Using the above equations one can construct the matrix elements in Eq.\ref{matrix_elements}
and proceed solving for the $t_i$. It is sometimes desirable to put some constraints
on the value of $t_i$. For example, when the coulomb potential is concerned one may 
want to set $t_{0}=1$. If the first $g$ variable are constrained by $t_{m}=\gamma_m$ 
with $m=[1,g]$ one can simply redefine Eq.\ref{matrix_elements} as
\begin{equation}
\begin{split}
A_{ij}=&\sum_{k>k_c} \tc_i(k)\tc_j(k)  \quad i,j\notin[1,g] \\
b_j=&\sum_{k>k_c} \left(\tV(k)-\sum_{m=1}^g \gamma_m \tc_m(k)\right)\tc_j(k)\quad j\notin[1,g]
\end{split}
\label{modified_matrix_elements}
\end{equation}



% discussion below of (fortran) routines kept for now (18 Oct 2017)
% possibly these map onto routines in qmcpack also

%\subsection*{The routines}
%\subsubsection*{fitpnnew}
%This routine constructs the $t_i$ and $\tY(k)$. Previously a routine, 
%let us call it {\em shells}, generating a grid of $\kv$ points has to
%be called. {\em shells} stores $\kv$ vectors
%in order of increasing magnitude and defines a shell as the 
%set of vectors having the same magnitude $k$ (in practice their difference 
%in magnitude must be below a given threshold). The total number of
%shells $N_\text{shell}$ has to be large enough to represents $V(\rv)$
%accurately using $\tV(k)$. The number of vector
%in a given shell is called $w(k)$. The following variables are passed as
%input: $\tV(k),k,w(k),N_\text{shell},m,r_c,N_\text{knot},\Omega$ and are called
%\verb!v(0:nk),rk(0:nk),wt(0:nk),nk,m,rad,nknots!. Note that the vectors all
%start from $0$ which corresponds to $k=0$. The number of shells such that
%$k\le k_c$ is also passed as input and called \verb!nf!. Additional input variables 
%are  \verb!coul! a logical variable specifying if the potential is coulombic; 
%\verb!vmad! the exact value of the Madelung constant;
%\verb!t0,t1! logical variables specifying if a constraint has to be put
%on element $t_0$ or $t_1$ and \verb!vt0,vt1! the value at which $t_0$ and $t_1$
%have to be set if corresponding constraints are active. 
%The routine works in this way:
%\begin{itemize}
%\item it calls {\em basis} and gets the coefficients $S_{\alpha n}$ (the $n$-th
%      coefficient of the $\alpha$-th polynomials) for the desired value of $m$.
%\item for every $k$ point, knot $i$ and polynomial $\alpha$ compute $\tc_{i\alpha}(k)$
%      using Eq.\ref{fourier_transform}. $D^\pm_{in}(k)$ is provided by {\em splint3D}
%      or {\em splint2D}. The routine uses \verb!ialpha!$=i(m+1)+\alpha$ (the range of 
%      variability of $\alpha$ and $i$ is specified above Eq.\ref{basisdef}).
%\item Matrix elements are constructed according to Eq.\ref{matrix_elements}
%\item Matrix elements are modified according to Eq.\ref{modified_matrix_elements} 
%      if constraints are active
%\item $t_i$ are computed solving the linear system. $\tY(k)$ are computed.
%\item A comparison with the exact Madelung constant is performed.
%\end{itemize}
%
%\subsubsection*{splint3D}
%Called by {\em fitpnnew}. This routine compute $D^\pm_{in}(k)$ for given 
%$k$ and $i$ and for all
%$n$ (going from $0$ to $\mathcal{N}=2m+1$). $D^\pm_{in}(k)$ are called 
%\verb!ddplus(0:maxn)! and \verb!ddminus(0:maxn)! and are given as output
%by the routine. In input one is required to specify $\mathcal{N},r_i,\Delta,k,\Omega$,
%respectively named \verb!maxn,r,delta,k,vol!. A logical input flag called 
%\verb!coul! specify if the potential is coulombic or not. The routine works
%in this way:
%\begin{itemize}
%\item it checks if $k$ is equal to 0
%\item if $k\ne 0$ then
%  \begin{itemize}
%  \item it computes $E^\pm_{in}(k)$ for the specified $i$ and $k$ using Eqs.\ref{nthEpm} and
%      \ref{0thEpm}. $E^\pm_{in}(k)$ are called \verb!ee(0:maxn,!$\pm$\verb!1)! 
%      (\verb!ee(:,0)! are never used).
%  \item Depending on the value of \verb!coul! either Eq.\ref{noncoulD+-} or Eq.\ref{coulD+-} 
%      is used to construct $D^\pm_{in}(k)$. The prefactor $\frac{4\pi}{k \Omega}$ 
%      is precomputed and called \verb!dnorm!.
%  \end{itemize}
%\item if $k=0$ the code uses either Eq.\ref{noncoul_k=0D+-} or Eq.\ref{coul_k=0D+-}.
%\end{itemize}
%
%\subsubsection*{splint2D}
%Called by {\em fitpnnew}. This routine compute $D^\pm_{in}(k)$ in the 2D case.
%The \verb!i\o! format is identical to {\em splint3D}. Equations from \ref{0thmoment}
%to \ref{nthmoment} are used to generate the required integrals.
%
%
%\subsubsection*{basis}
%Called by {\em fitpnnew}. It computes the coefficients $S_{\alpha n}$ (the $n$-th 
%coefficient of the $\alpha$-th polynomials) using Eqs.\ref{Smatrix1} and \ref{Smatrix2}.
%These coefficients are stored in \verb!s(0:m,0:2m+1)!. $m$ (called \verb!m!) 
%is required in input.
%
%\subsubsection*{computespl}
%This compute $W(r)$ at any $r$. $r$ is named \verb!rpos! internally. It requires
%$m,2m+1,N_\text{knot},S_{\alpha n},r_i,t_i,\Delta$. These are internally called
%\verb!m,maxn,nknots,s(0:m,0:maxn),r(0:nknots),t(0:nknots(m+1)-1),delta!. 
%\verb!coul! is also needed: it is a logical variable 
%to specify if $c_{i\alpha}^\text{coul}(r)$ have to be used instead of $c_{i\alpha}(r)$.
%The value of $W(r)$ is stored in \verb!w!.




\newpage
\section{Feature: Cubic Spline Interpolation}
% Written by Kenneth P .Esler Jr.
% Originally titled ``Cubic Spline Interpolation in 1, 2 and 3 Dimensions''

\newenvironment{DMatrix}{\begin{array}|{*{20}{c}}|}{\end{array}}
\newenvironment{MyMatrix}{\begin{array}({*{20}{c}})}{\end{array}}
\newenvironment{LMatrix}{\begin{array}({*{20}{l}})}{\end{array}}

We present the basic equations and algorithms necessary to
construct and evaluate cubic interpolating splines in one, two, and
three dimensions.  Equations are provided for both natural and
periodic boundary conditions.

\subsection{One Dimension}
Let us consider the problem in which we have a function $y(x)$
specified at a discrete set of points $x_i$, such that $y(x_i) = y_i$.
We wish to construct a piece-wise cubic polynomial interpolating
function, $f(x)$, which satisfies the following conditions:
\begin{itemize}
\item $f(x_i) = y_i$
\item $f'(x_i^-) = f'(x_i^+)$
\item $f''(x_i^-) = f''(x_i+)$
\end{itemize}

\subsubsection{Hermite Interpolants}
In our piecewise representation, we wish to store only the values,
$y_i$, and first derivatives, $y'_i$, of our function at each point
$x_i$, which we call {\em knots}.  Given this data, we wish to
construct the piecewise cubic function to use between $x_i$ and
$x_{i+1}$ which satisfies the above conditions.  In particular, we
wish to find the unique cubic polynomial, $P(x)$ satisfying
\begin{eqnarray}
P(x_i)      & = & y_i      \label{eq:c1} \\
P(x_{i+1})  & = & y_{i+1}  \label{eq:c2} \\
P'(x_i)     & = & y'_i     \label{eq:c3} \\
P'(x_{i+1}) & = & y'_{i+1} \label{eq:c4}
\end{eqnarray}
\begin{eqnarray}
h_i & \equiv & x_{i+1} - {x_i} \\
t & \equiv & \frac{x-x_i}{h_i}.
\end{eqnarray}
We then define the basis functions,
\begin{eqnarray}
p_1(t) & = & (1+2t)(t-1)^2  \label{eq:p1}\\
q_1(t) & = & t (t-1)^2      \\
p_2(t) & = & t^2(3-2t)      \\
q_2(t) & = & t^2(t-1)      \label{eq:q2}
\end{eqnarray}
On the interval, $(x_i, x_{i+1}]$, we define the interpolating
function,
\begin{equation}
P(x) = y_i p_1(t) + y_{i+1}p_2(t) + h\left[y'_i q_1(t) + y'_{i+1} q_2(t)\right]
\end{equation}
It can be easily verified that $P(x)$ satisfies conditions (\ref{eq:c1})
through (\ref{eq:c4}).  It is now left to
determine the proper values for the $y'_i\,$s such that the continuity
conditions given above are satisfied.

By construction, the value of the function and derivative will match
at the knots, i.e.
\begin{equation}
P(x_i^-) = P(x_i^+), \ \ \ \ P'(x_i^-) = P'(x_i^+).
\end{equation}
Then we must now enforce only the second derivative continuity:
\begin{eqnarray}
P''(x_i^-) & = & P''(x_i^+) \\
\frac{1}{h_{i-1}^2}\left[\rule{0pt}{0.3cm}6 y_{i-1} -6 y_i + h_{i-1}\left(2 y'_{i-1} +4 y'_i\right) \right]& = &
\frac{1}{h_i^2}\left[\rule{0pt}{0.3cm}-6 y_i + 6 y_{i+1} +h_i\left( -4 y'_i -2 y'_{i+1} \right)\right] \nonumber
\end{eqnarray}
Let us define
\begin{eqnarray}
\lambda_i & \equiv & \frac{h_i}{2(h_i+h_{i-1})} \\
\mu_i & \equiv & \frac{h_{i-1}}{2(h_i+h_{i-1})}  = \frac{1}{2} - \lambda_i.
\end{eqnarray}
Then we may rearrange,
\begin{equation}
\lambda_i y'_{i-1} + y'_i + \mu_i y'_{i+1} = \underbrace{3 \left[\lambda_i \frac{y_i - y_{i-1}}{h_{i-1}} + \mu_i \frac{y_{i+1}
    - y_i}{h_i} \right] }_{d_i}
\end{equation}
This equation holds for all $0<i<(N-1)$, so we have a tridiagonal set of
equations.  The equations for $i=0$ and $i=N-1$ depend on the boundary
conditions we are using.  
\subsubsection{Periodic boundary conditions}
For periodic boundary conditions, we have
\begin{equation}
\begin{matrix}
y'_0           & +  & \mu_0 y'_1     &   &                   &            & \dots                   & +  \lambda_0 y'_{N-1} & = & d_0 \\
\lambda_1 y'_0 & +  & y'_1           & + &  \mu_1 y'_2       &            & \dots                   &                       & = & d_1 \\
               &    & \lambda_2 y'_1 & + &  y'_2           + & \mu_2 y'_3 & \dots                   &                       & = & d_2 \\
               &    &                &   &  \vdots           &            &                         &                       &   &     \\
\mu_{N-1} y'_0 &    &                &   &                   &            & +\lambda_{N-1} y'_{N-1} & +  y'_{N-2}           & = & d_3 
\end{matrix}
\end{equation}
Or, in matrix form, we have,
\begin{equation}
\begin{MyMatrix}
1         & \mu_0     &    0   &   0           & \dots         &      0        & \lambda_0 \\
\lambda_1 &  1        & \mu_1  &   0           & \dots         &      0        &     0     \\
0         & \lambda_2 &   1    & \mu_2         & \dots         &      0        &     0     \\
\vdots    & \vdots    & \vdots & \vdots        & \ddots        &   \vdots      &  \vdots   \\
0         &   0       &   0    & \lambda_{N-3} &      1        & \mu_{N-3}     &    0      \\
0         &   0       &   0    &   0           & \lambda_{N-2} &      1        & \mu_{N-2} \\
\mu_{N-1} &   0       &   0    &   0           &   0           & \lambda_{N-1} &  1     
\end{MyMatrix}
\begin{MyMatrix} y'_0 \\ y'_1 \\ y'_2 \\ \vdots \\ y'_{N-3} \\ y'_{N-2} \\ y'_{N-1} \end{MyMatrix} =
\begin{MyMatrix} d_0  \\  d_1 \\  d_2 \\ \vdots \\  d_{N-3} \\  d_{N-2} \\  d_{N-1} \end{MyMatrix} .
\end{equation}
The system is tridiagonal except for the two elements in the upper
right and lower left corners.  These terms complicate the solution a
bit, although it can still be done in $\mathcal{O}(N)$ time.  We first
proceed down the rows, eliminating the the first non-zero term in each
row by subtracting the appropriate multiple of the previous row.  At
the same time, we also eliminate the first element in the last row,
shifting the position of the first non-zero element to the right with
each iteration.  When we get to the final row, we will have the value
for $y'_{N-1}$.  We can then proceed back upward, backsubstituting
values from the rows below to calculate all the derivatives.

\subsubsection{Complete boundary conditions}
If we specify the first derivatives of our function at the end points,
we have what is known as {\em complete} boundary conditions.  The
equations in that case are trivial to solve:
\begin{equation}
\begin{MyMatrix}
1         &  0        &    0   &   0           & \dots         &      0        &     0     \\
\lambda_1 &  1        & \mu_1  &   0           & \dots         &      0        &     0     \\
0         & \lambda_2 &   1    & \mu_2         & \dots         &      0        &     0     \\
\vdots    & \vdots    & \vdots & \vdots        & \ddots        &   \vdots      &  \vdots   \\
0         &   0       &   0    & \lambda_{N-3} &      1        & \mu_{N-3}     &    0      \\
0         &   0       &   0    &   0           & \lambda_{N-2} &      1        & \mu_{N-2} \\
0         &   0       &   0    &   0           &   0           &      0        &  1     
\end{MyMatrix}
\begin{MyMatrix} y'_0 \\ y'_1 \\ y'_2 \\ \vdots \\ y'_{N-3} \\ y'_{N-2} \\ y'_{N-1} \end{MyMatrix} =
\begin{MyMatrix} d_0  \\  d_1 \\  d_2 \\ \vdots \\  d_{N-3} \\  d_{N-2} \\  d_{N-1} \end{MyMatrix} .
\end{equation}
This system is completely tridiagonal and we may solve trivially by
performing row eliminations downward, then proceeding upward as
before.

\subsubsection{Natural boundary conditions}
If we do not have information about the derivatives at the boundary
conditions, we may construct a {\em natural spline}, which assumes the
the second derivatives are zero at the end points of our spline.  In
this case our system of equations is the following:
\begin{equation}
\begin{MyMatrix}
1         & \frac{1}{2} &    0   &   0           & \dots         &      0        &     0     \\
\lambda_1 &  1          & \mu_1  &   0           & \dots         &      0        &     0     \\
0         & \lambda_2   &   1    & \mu_2         & \dots         &      0        &     0     \\
\vdots    & \vdots      & \vdots & \vdots        & \ddots        &   \vdots      &  \vdots   \\
0         &   0         &   0    & \lambda_{N-3} &      1        & \mu_{N-3}     &    0      \\
0         &   0         &   0    &   0           & \lambda_{N-2} &      1        & \mu_{N-2} \\
0         &   0         &   0    &   0           &   0           &  \frac{1}{2}  &  1     
\end{MyMatrix}
\begin{MyMatrix} y'_0 \\ y'_1 \\ y'_2 \\ \vdots \\ y'_{N-3} \\ y'_{N-2} \\ y'_{N-1} \end{MyMatrix} =
\begin{MyMatrix} d_0  \\  d_1 \\  d_2 \\ \vdots \\  d_{N-3} \\  d_{N-2} \\  d_{N-1} \end{MyMatrix} ,
\end{equation}
with
\begin{equation}
d_0 = \frac{3}{2} \frac{y_1-y_1}{h_0}, \ \ \ \ \ d_{N-1} = \frac{3}{2} \frac{y_{N-1}-y_{N-2}}{h_{N-1}}.
\end{equation}

\subsection{Bicubic Splines}
It is possible to extend the cubic spline interpolation method to
functions of two variables, i.e. $F(x,y)$.  In this case, we have a
rectangular mesh of points given by $F_{ij} \equiv F(x_i,y_j)$.  In
the case of 1D splines, we needed to store the value of the first
derivative of the function at each point, in addition to the value.
In the case of {\em bicubic splines}, we need to store four
quantities for each mesh point:  
\begin{eqnarray}
F_{ij}    & \equiv & F(x_i, y_i)            \\
F^x_{ij}  & \equiv & \partial_x F(x_i, y_i) \\
F^y_{ij}  & \equiv & \partial_y F(x_i, y_i) \\
F^{xy}    & \equiv & \partial_x \partial_y F(x_i, y_i)
\end{eqnarray}

Consider the point $(x,y)$ at which we wish to interpolate $F$.  We
locate the rectangle which contains this point, such that $x_i <= x <
x_{i+1}$ and $y_i <= x < y_{i+1}$.  Let 
\begin{eqnarray}
h & \equiv & x_{i+1}-x_i \\
l & \equiv & y_{i+1}-y_i \\
u & \equiv & \frac{x-x_i}{h} \\
v & \equiv & \frac{y-y_i}{l}
\end{eqnarray}
Then, we calculate the interpolated value as
\begin{equation}
F(x,y) = 
\begin{MyMatrix}
p_1(u) \\ p_2(u) \\ h q_1(u) \\ h q_2(u) 
\end{MyMatrix}^T
\begin{MyMatrix}
F_{i,j}     & F_{i+1,j}     & F^y_{i,j}      & F^y_{i,j+1}     \\
F_{i+1,j}   & F_{i+1,j+1}   & F^y_{i+1,j}    & F^y_{i+1,j+1}   \\
F^x_{i,j}   & F^x_{i,j+1}   & F^{xy}_{i,j}   & F^{xy}_{i,j+1}  \\
F^x_{i+1,j} & F^x_{i+1,j+1} & F^{xy}_{i+1,j} & F^{xy}_{i+1,j+1} 
\end{MyMatrix}
\begin{MyMatrix}
p_1(v)\\ p_2(v)\\ k q_1(v) \\ k q_2(v)
\end{MyMatrix}
\end{equation}
\subsubsection{Construction bicubic splines}
We now address the issue of how to compute the derivatives that are
needed for the interpolation.  The algorithm is quite simple.  For
every $x_i$, we perform the tridiagonal solution as we did in the 1D
splines to compute $F^y_{ij}$.  Similarly, we perform a tridiagonal
solve for every value of $F^x_{ij}$.  Finally, in order to compute the
cross-derivative we may {\em either} to the tridiagonal solve in the $y$
direction of $F^x_{ij}$, {\em or} solve in the $x$ direction for
$F^y_{ij}$ to obtain the cross-derivatives, $F^{xy}_{ij}$.  Hence,
only minor modifications to the $1D$ interpolations are necessary.

\subsection{Tricubic Splines}
Bicubic interpolation required two four-component vectors and a 4x4
matrix.  By extension, tricubic interpolation requires three
4-component vectors and a 4x4x4 tensor.  We summarize the forms of
these vectors below.
\begin{eqnarray}
h & \equiv & x_{i+1}-x_i \\
l & \equiv & y_{i+1}-y_i \\
m & \equiv & z_{i+1}-z_i \\
u & \equiv & \frac{x-x_i}{h} \\
v & \equiv & \frac{y-y_i}{l} \\
w & \equiv & \frac{z-z_i}{m}
\end{eqnarray}
\begin{eqnarray}
\vec{a} & = & 
\begin{MyMatrix}
p_1(u) & p_2(u) & h q_1(u) & h q_2(u) 
\end{MyMatrix}^T \\
\vec{b} & = & 
\begin{MyMatrix}
p_1(v) & p_2(v) & k q_1(v) & k q_2(v) 
\end{MyMatrix}^T \\
\vec{c} & = & 
\begin{MyMatrix}
p_1(w) & p_2(w) & l q_1(w) & l q_2(w) 
\end{MyMatrix}^T 
\end{eqnarray}
\begin{equation}
\begin{LMatrix}
A_{000} = F_{i,j,k}     & A_{001}=F_{i,j,k+1}     & A_{002}=F^z_{i,j,k}      & A_{003}=F^z_{i,j,k+1}      \\
A_{010} = F_{i,j+1,k}   & A_{011}=F_{i,j+1,k+1}   & A_{012}=F^z_{i,j+1,k}    & A_{013}=F^z_{i,j+1,k+1}    \\
A_{020} = F^y_{i,j,k}   & A_{021}=F^y_{i,j,k+1}   & A_{022}=F^{yz}_{i,j,k}   & A_{023}=F^{yz}_{i,j,k+1}   \\
A_{030} = F^y_{i,j+1,k} & A_{031}=F^y_{i,j+1,k+1} & A_{032}=F^{yz}_{i,j+1,k} & A_{033}=F^{yz}_{i,j+1,k+1} \\
                        &                         &                          &                            \\
A_{100} = F_{i+1,j,k}     & A_{101}=F_{i+1,j,k+1}     & A_{102}=F^z_{i+1,j,k}      & A_{103}=F^z_{i+1,j,k+1}      \\
A_{110} = F_{i+1,j+1,k}   & A_{111}=F_{i+1,j+1,k+1}   & A_{112}=F^z_{i+1,j+1,k}    & A_{113}=F^z_{i+1,j+1,k+1}    \\
A_{120} = F^y_{i+1,j,k}   & A_{121}=F^y_{i+1,j,k+1}   & A_{122}=F^{yz}_{i+1,j,k}   & A_{123}=F^{yz}_{i+1,j,k+1}   \\
A_{130} = F^y_{i+1,j+1,k} & A_{131}=F^y_{i+1,j+1,k+1} & A_{132}=F^{yz}_{i+1,j+1,k} & A_{133}=F^{yz}_{i+1,j+1,k+1} \\
                        &                         &                          &                            \\
A_{200} = F^x_{i,j,k}      & A_{201}=F^x_{i,j,k+1}      & A_{202}=F^{xz}_{i,j,k}      & A_{203}=F^{xz}_{i,j,k+1}    \\
A_{210} = F^x_{i,j+1,k}    & A_{211}=F^x_{i,j+1,k+1}    & A_{212}=F^{xz}_{i,j+1,k}    & A_{213}=F^{xz}_{i,j+1,k+1}  \\
A_{220} = F^{xy}_{i,j,k}   & A_{221}=F^{xy}_{i,j,k+1}   & A_{222}=F^{xyz}_{i,j,k}     & A_{223}=F^{xyz}_{i,j,k+1}   \\
A_{230} = F^{xy}_{i,j+1,k} & A_{231}=F^{xy}_{i,j+1,k+1} & A_{232}=F^{xyz}_{i,j+1,k}   & A_{233}=F^{xyz}_{i,j+1,k+1} \\
                        &                         &                          &                                      \\
A_{300} = F^x_{i+1,j,k}      & A_{301}=F^x_{i+1,j,k+1}      & A_{302}=F^{xz}_{i+1,j,k}    & A_{303}=F^{xz}_{i+1,j,k+1}   \\
A_{310} = F^x_{i+1,j+1,k}    & A_{311}=F^x_{i+1,j+1,k+1}    & A_{312}=F^{xz}_{i+1,j+1,k}  & A_{313}=F^{xz}_{i+1,j+1,k+1} \\
A_{320} = F^{xy}_{i+1,j,k}   & A_{321}=F^{xy}_{i+1,j,k+1}   & A_{322}=F^{xyz}_{i+1,j,k}   & A_{323}=F^{xyz}_{i+1,j,k+1}  \\
A_{330} = F^{xy}_{i+1,j+1,k} & A_{331}=F^{xy}_{i+1,j+1,k+1} & A_{332}=F^{xyz}_{i+1,j+1,k} & A_{333}=F^{xyz}_{i+1,j+1,k+1} 
\end{LMatrix}
\end{equation}
Now, we can write
\begin{equation}
F(x,y,z) = \sum_{i=0}^3 a_i \sum_{j=0}^3 b_j \sum_{k=0}^3 c_k \ A_{i,j,k} 
\end{equation}
The appropriate derivatives of $F$ may be computed by a generalization
of the method used for bicubic splines above.




\newpage
\section{Feature: B-spline Orbital Tiling (Band Unfolding)}

% Written by Kenneth P .Esler Jr.
% Originally titled ``Generalized band unfolding for quantum Monte Carlo simulation of solids''

In continuum quantum Monte Carlo simulations, it is necessary to
evaluate the electronic orbitals of a system at real-space positions
hundreds of millions of times.  It has been found that if
these orbitals are represented in a localized, B-spline basis, each
evaluation takes a small, constant time which is independent of system
size.

Unfortunately, the memory required for storing the B-spline grows with
the second power of the system size.  If we are studying perfect
crystals, however, this can be reduced to linear scaling if we {\em
  tile} the primitive cell.  In this approach, 
%implemented in the CASINO QMC simulation suite, 
a supercell is constructed by tiling the
primitive cell $N_1 \times N_2 \times N_3$ in the three lattice
directions.  The orbitals are then represented in real space only in
the primitive cell, and an $N_1 \times N_2 \times N_3$ k-point mesh.
To evaluate an orbital at any point in the supercell, it is only
necessary to wrap that point back into the primitive cell, evaluate
the spline, and then multiply the phase factor,
$e^{-i\mathbf{k}\cdot\mathbf{r}}$.  

Here, we show that this approach can be generalized to a tiling
constructed with a $3\times 3$ nonsingular matrix of integers, of which
the above approach is a special case.  This generalization brings with
it a number of advantages.  The primary reason for performing
supercell calculations in QMC is to reduce finite-size errors.  These
errors result from three sources:  1) the quantization of the crystal
momentum;  2) the unphysical periodicity of the exchange-correlation
hole of the electron; and 3) the kinetic-energy contribution from the
periodicity of the long-range jastrow correlation functions.  The first
source of error can be largely eliminated by twist averaging.  If the
simulation cell is large enough that XC hole does not ``leak'' out of
the simulation cell, the second source can be eliminated either
through use of the MPC interaction or the {\em a postiori} correction
of Chiesa et. al.  

The satisfaction of the leakage requirement is controlled by whether
the minimum distance, $L_\text{min}$ from one supercell image to the
next is greater than the width of the XC hole.  Therefore, given a
choice, it is best to use a cell which is as nearly cubic as possible,
since this choice maximizes $L_\text{min}$ for a given number of
atoms.  Most often, however, the primitive cell is not cubic.  In
these cases, if we wish to choose the optimal supercell to reduce
finite size effects, we cannot utilize the simple primitive tiling
scheme.  In the generalized scheme we present, it is possible to
choose far better supercells (from the standpoint of finite-size
errors), while retaining the storage efficiency of the original tiling
scheme.

\subsection{The mathematics}
\renewcommand{\vp}{\mathbf{a}^\text{p}}
\renewcommand{\vs}{\mathbf{a}^\text{s}} 
\renewcommand{\Smat}{\mathbf{S}}
Consider the set of primitive lattice vectors, $\{\vp_1, \vp_2,
\vp_3\}$.  We may write these vectors in a matrix, $\mathbf{L}_p$, whose
rows are the primitive lattice vectors.  Consider a non-singular
matrix of integers, $\Smat$.  A corresponding set of supercell lattice
vectors, $\{\vs_1, \vs_2, \vs_3\}$, can be constructed by the matrix
product 
\begin{equation}
\vs_i = S_{ij} \vp_j
\end{equation}
If the primitive cell contains $N_p$ atoms, the supercell will then
contain $N_s = |\det(\Smat)| N_p$ atoms.

\subsection{Example: FeO}
As an example, consider the primitive cell for antiferromagnetic FeO
(wustite) in the rocksalt structure.  The primitive vectors, given in
units of the lattice constant, are given by
\newcommand{\xv}{\hat{\mathbf{x}}} 
\newcommand{\yv}{\hat{\mathbf{y}}}
\newcommand{\zv}{\hat{\mathbf{z}}}
\begin{eqnarray}
\vs_1 & = & \frac{1}{2}\xv + \frac{1}{2}\yv +      \ \   \zv \\
\vs_2 & = & \frac{1}{2}\xv +      \ \   \yv + \frac{1}{2}\zv \\
\vs_3 & = &   \ \      \xv + \frac{1}{2}\yv + \frac{1}{2}\zv 
\end{eqnarray}
This primitive cell contains two iron atoms and two oxygen atoms. It
is a very elongated cell with acute angles, and thus has a short
minimum distance between adjacent images.

The smallest cubic cell consistent with the AFM ordering can be
constructed with the matrix
\begin{equation}
\Smat = \left[\begin{array}{rrr}
  -1 & -1 &  3 \\
  -1 &  3 & -1 \\
   3 & -1 & -1 
  \end{array}\right]
\end{equation}
This cell has $2\det(\Smat) = 32$ iron atoms and 32 oxygen atoms.  In
this example, we may perform the simulation in the 32-iron supercell,
while storing the orbitals only in the 2-iron primitive cell, for a
savings of a factor of 16.  
%On current multicore supercomputers, with
%1-2GB RAM per core, this is literally the difference between be able
%to perform the simulation or not.

\subsubsection{The k-point mesh}
In order to be able to use the generalized tiling scheme, we need to
have the appropriate number of bands to occupy in the supercell.
This may be achieved by appropriately choosing the k-point mesh.  In
this section, we explain how these points are chosen.  

For simplicity, let us assume that the supercell calculation will be
performed at the $\Gamma$-point.  We may lift this restriction very
easily later.  The fact that supercell calculation is performed at
$\Gamma$ implies that the k-points used in the primitive-cell
calculation must be $\mathbf{G}$-vectors of the superlattice.  This
still leaves us with an infinite set of vectors.  We may reduce this
set to a finite number by considering that the orbitals must form an
linearly independent set.  Orbitals with k-vectors $\mathbf{k}^p_1$
and $\mathbf{k}^p_2$ will differ by at most a constant factor if
$\mathbf{k}^p_1 - \mathbf{k}^p_2 = \mathbf{G}^p$, where $\mathbf{G}^p$
is a reciprocal lattice vector of the primitive cell.  

Combining these two considerations gives us a prescription for
generating our k-point mesh.  The mesh may be taken to be the set of
k-point which are G-vectors of the superlattice, reside within the
first Brillouin zone (FBZ) of the primitive lattice, whose members do
not differ a G-vector of the primitive lattice.  Upon constructing
such a set, we find that the number of included k-points is equal to
$|\det(\Smat)|$, precisely the number we need.  This can by considering
the fact that the supercell has a volume $|\det(\Smat)|$ times that of
the primitive cell.  This implies that the volume of the supercell's
FBZ is $|\det(\Smat)|^{-1}$ times that of the primitive cell.  Hence,
$|\det(\Smat)|$ G-vectors of the supercell will fit in the FBZ of the
primitive cell.  Removing duplicate k-vectors, which differ from
another by a reciprocal lattice vector, avoids double-counting vectors
which lie on zone faces.

\subsubsection{Formulae}
\newcommand{\Amat}{\mathbf{A}} 
\newcommand{\Bmat}{\mathbf{B}} 
\renewcommand{\vk}{\mathbf{k}}
\newcommand{\vt}{\mathbf{t}}

Let $\Amat$ be the matrix whose rows are the direct lattice vectors,
$\{\mathbf{a}_i\}$.  The, let the matrix $\Bmat$ be defined as
$2\pi(\Amat^{-1})^\dagger$.  Its rows are the primitive reciprocal
lattice vectors.  Let $\Amat_p$ and $\Amat_s$ represent the primitive
and superlattice matrices, respectively, and similarly for their
reciprocals.  Then we have
\begin{eqnarray}
\Amat_s & = & \Smat \Amat_p \\
\Bmat_s & = & 2\pi\left[(\Smat \Amat_p)^{-1}\right]^\dagger \\
        & = & 2\pi\left[\Amat_p^{-1} \Smat{-1}\right]^\dagger \\
        & = & 2\pi(\Smat^{-1})^\dagger (\Amat_p^{-1})^\dagger \\
        & = & (\Smat^{-1})^\dagger \Bmat_p
\end{eqnarray}  
Consider a k-vector, $\vk$.  It may be alternatively be written in
basis of reciprocal lattice vectors as $\vt$.  
\begin{eqnarray}
\vk & = & (\vt^\dagger \Bmat)^\dagger \\
    & = & \Bmat^\dagger \vt           \\
\vt & = & (\Bmat^\dagger)^{-1} \vk    \\
    & = & (\Bmat^{-1})^\dagger \vk    \\
    & = & \frac{\Amat \vk}{2\pi}
\end{eqnarray}
We may then express a twist vector of the primitive lattice, $\vt_p$ in terms
of the superlattice.
\begin{eqnarray}
\vt_s & = & \frac{\Amat_s \vk}{2\pi}                           \\
      & = & \frac{\Amat_s \Bmat_p^\dagger \vt_p}{2\pi}         \\
      & = & \frac{\Smat \Amat_p \Bmat_p^\dagger \vt_p}{2\pi}   \\
      & = & \frac{2\pi \Smat \Amat_p \Amat_p^{-1} \vt_p}{2\pi} \\
      & = & \Smat \vt_p
\end{eqnarray}
This gives the simple result that twist-vectors transform in precisely
the same way as direct lattice vectors.




\newpage
\section{Feature: Hybrid orbital representation}

% Written by Kenneth P. Esler, Jr.
% Document originally included in QMCPACK at src/QMCWaveFunctions/AtomicOrbital.tex
% Originally titled ``Hybrid orbital representation''

\renewcommand{\vr}{\mathbf{r}}

\begin{equation}
\phi(\vr) = \sum_{\ell=0}^{\ell_\text{max}} \sum_{m=-\ell}^\ell Y_\ell^m (\hat{\Omega})
u_{\ell m}(r),
\end{equation}
where $u_{lm}(r)$ are complex radial functions represented in some
radial basis (e.g. splines).

\subsection{Real Spherical Harmonics}
\renewcommand{\Re}{\rm Re}
\renewcommand{\Im}{\rm Im}
If $\phi(\vr)$ can be written as purely real, we can change the
representation so that
\begin{equation}
\phi(\vr) = \sum_{l=0}^{l_\text{max}} \sum_{m=-\ell}^\ell Y_{\ell m}(\hat{\Omega})
\bar{u}_{lm}(r),
\end{equation}
where $\bar{Y}_\ell^m$ are the {\em real} spherical harmonics defined by
\begin{equation}
Y_{\ell m} = \begin{cases}
Y_\ell^0 & \mbox{if } m=0\\
{1\over 2}\left(Y_\ell^m+(-1)^m \, Y_\ell^{-m}\right) \ = \Re\left[Y_\ell^m\right]
%\sqrt{2} N_{(\ell,m)} P_\ell^m(\cos \theta) \cos m\varphi 
& \mbox{if } m>0 \\
{1\over i 2}\left(Y_\ell^{-m}-(-1)^{m}\, Y_\ell^{m}\right) = \Im\left[Y_\ell^{-m}\right]
%\sqrt{2} N_{(\ell,m)} P_\ell^{-m}(\cos \theta) \sin m\varphi 
&\mbox{if } m<0.
\end{cases}
\end{equation}
We need then to relate $\bar{u}_{\ell m}$ to $u_{\ell m}$.  We wish
to express,
\begin{equation}
\Re\left[\phi(\vr)\right] = \sum_{\ell=0}^{\ell_\text{max}} \sum_{m=-\ell}^\ell
\Re\left[Y_\ell^m (\hat{\Omega}) u_{\ell m}(r)\right]
\end{equation}
in terms of $\bar{u}_{\ell m}(r)$ and $Y_{\ell m}$.
\begin{eqnarray}
\Re\left[Y_\ell^m u_{\ell m}\right] & = & \Re\left[Y_\ell^m\right]
\Re\left[u_{\ell m}\right] - \Im\left[Y_\ell^m\right] \Im\left[u_{\ell m}\right]
\end{eqnarray}
For $m>0$,
\begin{equation}
\Re\left[Y_\ell^m\right] = Y_{\ell m} \qquad \text{and} \qquad \Im\left[Y_\ell^m\right] = Y_{\ell\,-m}.
\end{equation}
For $m<0$,
\begin{equation}
\Re\left[Y_\ell^m\right] = (-1)^m Y_{\ell\, -m} \qquad \text and \qquad \Im\left[Y_\ell^m\right] = -(-1)^m Y_{\ell m}.
\end{equation}
Then for $m > 0$,
\begin{eqnarray}
\bar{u}_{\ell m} & = & \Re\left[u_{\ell m}\right] + (-1)^m \Re\left[u_{\ell\,-m}\right] \\
\bar{u}_{\ell\, -m} & = & -\Im\left[u_{\ell m}\right] + (-1)^m \Im\left[u_{\ell\,-m}\right].
\end{eqnarray}


\subsection{Projecting to atomic orbitals}

% Written by Ken Esler as part of the Common codebase used in wfconvert
% Originally titled ``Notes on projecting to atomic orbitals''
% Dated July 19, 2009

\renewcommand{\vr}{\mathbf{r}}
\newcommand{\vI}{\mathbf{I}}
\renewcommand{\vk}{\mathbf{k}}
\newcommand{\vG}{\mathbf{G}}

%\subsubsection{Form for orbitals}
Inside a muffin tin, orbitals are represented as product of spherical
harmonics and 1D radial functions, primarily represented by splines.
For a muffin tin centered at $\vI$, 
\begin{equation}
\phi_n(\vr) = \sum_{\ell,m} Y_\ell^m(\hat{\vr -\vI})
u_{lm}\left(\left|\vr - \vI\right|\right) \label{eq:ulm}
\end{equation}
Let use consider the case that our original representation for
$\phi(\vr)$ is of the form
\begin{equation}
\phi_{n,\vk}(\vr) = \sum_\vG c_{\vG+\vk}^n e^{i(\vG + \vk)\cdot \vr}
\end{equation}
Recall that
\begin{equation}
e^{i\vk\cdot\vr} = 4\pi \sum_{\ell,m} i^\ell j_\ell(|\vr||\vk|)
Y_\ell^m(\hat{\vk}) \left[Y_\ell^m(\hat{\vr})\right]^*.
\end{equation}
Conjugating,
\begin{equation}
e^{-i\vk\cdot\vr} = 4\pi\sum_{\ell,m} (-i)^\ell j_\ell(|\vr||\vk|)
\left[Y_\ell^m(\hat{\vk})\right]^* Y_\ell^m(\hat{\vr}).
\end{equation}
Setting $\vk \rightarrow -k$,
\begin{equation}
e^{i\vk\cdot\vr} = 4\pi\sum_{\ell,m} i^\ell j_\ell(|\vr||\vk|)
\left[Y_\ell^m(\hat{\vk})\right]^* Y_\ell^m(\hat{\vr}).
\end{equation}

Then,
\begin{equation}
e^{i\vk\cdot(\vr-\vI)} = 4\pi\sum_{\ell,m} i^\ell j_\ell(|\vr-\vI||\vk|)
\left[Y_\ell^m(\hat{\vk})\right]^* Y_\ell^m(\hat{\vr-\vI}).
\end{equation}

\begin{equation}
e^{i\vk\cdot\vr} = 4\pi e^{i\vk\cdot\vI} \-\sum_{\ell,m} i^\ell j_\ell(|\vr-\vI||\vk|)
\left[Y_\ell^m(\hat{\vk})\right]^* Y_\ell^m(\hat{\vr-\vI}).
\end{equation}

Then
\begin{equation}
\phi_{n,\vk}(\vr) =  \sum_\vG 4\pi c_{\vG+\vk}^n
e^{i(\vG+\vk)\cdot\vI} \sum_{\ell,m}
  i^\ell j_\ell(|\vG +\vk||\vr-\vI|)
  \left[Y_\ell^m(\hat{\vG+\vk})\right]^*
Y_\ell^m(\hat{\vr - \vI})
\end{equation}
Comparing to (\ref{eq:ulm}),
\begin{equation}
u_{\ell m}^n(r) = 4\pi i^\ell \sum_G c_{\vG+\vk}^n e^{i(\vG+\vk)\cdot\vI}  j_\ell\left(|\vG + \vk|r|\right)
\left[Y_\ell^m(\hat{\vG + \vk})\right]^*.
\end{equation}
If we had adopted the opposite sign convention for Fourier transforms
(as is unfortunately the case in wfconvert), we would have
\begin{equation}
u_{\ell m}^n(r) = 4\pi (-i)^\ell \sum_G c_{\vG+\vk}^n e^{-i(\vG+\vk)\cdot\vI}  j_\ell\left(|\vG + \vk|r|\right)
\left[Y_\ell^m(\hat{\vG + \vk})\right]^*.
\end{equation}




\newpage
\section{Feature: Electron-electron-ion Jastrow factor}

% Written by Kenneth P. Esler, Jr.
% Document originally included in QMCPACK at src/QMCWaveFunctions/Jastrow/eeI_Jastrow.tex
% Originally titled ``Electron-electron-ion Jastrow factor''

\newcommand{\riI}{r_{iI}}
\newcommand{\briI}{\mathbf{r}_{iI}}
\newcommand{\rjI}{r_{jI}}
\newcommand{\brjI}{\mathbf{r}_{jI}}
\newcommand{\rij}{r_{ij}}
\newcommand{\brij}{\mathbf{r}_{ij}}
%\section{Form of the Jastrow}
The general form of the 3-body Jastrow we describe here depends on the
three inter-particle distances, $(\rij, \riI, \rjI)$.
\begin{equation}
J_3 = \sum_{I\in\text{ions}} \sum_{i,j \in\text{elecs};i\neq j} U(\rij, \riI,
\rjI)
\end{equation}
Note that we constrain the form of $U$ such that
$U(\rij, \riI,\rjI) = U(\rij, \rjI,\riI)$, so as to preserve the
particle symmetry of the wave function.  We then compute the gradient as
\begin{equation}
\nabla_i J_3 =  \sum_{I\in\text{ions}} \sum_{j \neq i}
\left[\frac{\partial U(\rij, \riI,\rjI)}{\partial\rij}
  \frac{\mathbf{r}_i - \mathbf{r}_j}{|\mathbf{r}_i - \mathbf{r}_j|} 
+ \frac{\partial U(\rij, \riI,\rjI)}{\partial\riI}
  \frac{\mathbf{r}_i - \mathbf{I}}{|\mathbf{r}_i - \mathbf{I}|}  \right]
\end{equation}
To compute the laplacian, we take
\begin{eqnarray}
\nabla_i^2 J_3 & = & \nabla_i \cdot \left(\nabla_i J_3\right) \\
& = & \sum_{I\in\text{ions}} \sum_{j\neq i } \left[
\frac{\partial^2 U}{\partial \rij^2} + \frac{2}{\rij} \frac{\partial
  U}{\partial \rij} + 2 \frac{\partial^2 U}{\partial \rij \partial
  \riI}\frac{\brij\cdot\briI}{\rij\riI} +\frac{\partial^2 U}{\partial
  \riI^2}
+ \frac{2}{\riI}\frac{\partial U}{\partial \riI} \nonumber
\right]
\end{eqnarray}
We now wish to compute the gradient of these terms w.r.t. the ion position, $I$.
\begin{equation}
\nabla_I J_3 = -\sum_{j\neq i} \left[ \frac{\partial U(\rij, \riI,\rjI)}{\partial\riI}
  \frac{\mathbf{r}_i - \mathbf{I}}{|\mathbf{r}_i - \mathbf{I}|} 
+\frac{\partial U(\rij, \riI,\rjI)}{\partial\rjI}
  \frac{\mathbf{r}_j - \mathbf{I}}{|\mathbf{r}_j - \mathbf{I}|} \right]
\end{equation}
For the gradient w.r.t. $i$ of the gradient w.r.t. $I$, the result is a tensor,
\begin{eqnarray}
\nabla_I \nabla_i J_3 & = & \nabla_I \sum_{j \neq i}
\left[\frac{\partial U(\rij, \riI,\rjI)}{\partial\rij}
  \frac{\mathbf{r}_i - \mathbf{r}_j}{|\mathbf{r}_i - \mathbf{r}_j|} 
+ \frac{\partial U(\rij, \riI,\rjI)}{\partial\riI}
  \frac{\mathbf{r}_i - \mathbf{I}}{|\mathbf{r}_i - \mathbf{I}|}  \right] \\\nonumber \\\nonumber
& = & -\sum_{j\neq i} \left[ 
\frac{\partial^2 U}{\partial \rij \riI} \hat{\mathbf{r}}_{ij} \otimes
\hat{\mathbf{r}}_{iI} + \left(\frac{\partial^2 U}{\partial \riI^2} -
\frac{1}{\riI} \frac{\partial U}{\partial \riI}\right)
\hat{\mathbf{r}}_{iI} \otimes \hat{\mathbf{r}}_{iI} \right. + \\\nonumber
& & \left. \qquad \ \ \  \frac{\partial^U}{\partial \rij \rjI} \hat{\mathbf{r}}_{ij} \otimes \hat{\mathbf{r}}_{jI} + \frac{\partial^2 U}{\partial \riI \partial \rjI}
\hat{\mathbf{r}}_{iI}\otimes \hat{\mathbf{r}}_{jI}  +
\frac{1}{\riI} \frac{\partial U}{\partial \riI} \overleftrightarrow{\mathbf{1}}\right]
\end{eqnarray}

\begin{eqnarray}
\nabla_I \nabla_i J_3 & = & \nabla_I \sum_{j \neq i}
\left[\frac{\partial U(\rij, \riI,\rjI)}{\partial\rij}
  \frac{\mathbf{r}_i - \mathbf{r}_j}{|\mathbf{r}_i - \mathbf{r}_j|} 
+ \frac{\partial U(\rij, \riI,\rjI)}{\partial\riI}
  \frac{\mathbf{r}_i - \mathbf{I}}{|\mathbf{r}_i - \mathbf{I}|}  \right] \\\nonumber 
& = & \sum_{j\neq i} \left[ -\frac{\partial^2 U}{\partial \rij \partial \riI} \hat{\mathbf{r}}_{ij} \otimes \hat{\mathbf{r}}_{iI} +
\left(-\frac{\partial^2 U}{\partial \riI^2}  + \frac{1}{\riI}\frac{\partial U}{\partial \riI} \right) 
\hat{\mathbf{r}}_{iI} \otimes \hat{\mathbf{r}}_{iI} - \frac{1}{\riI}\frac{\partial U}{\partial \riI} \overleftrightarrow{\mathbf{1}}
\right]
\end{eqnarray}
For the laplacian,
\begin{eqnarray}
\nabla_I \nabla_i^2 J_3 & = & \nabla_I\left[\nabla_i \cdot \left(\nabla_i J_3\right)\right] \\
& = & \nabla_I \sum_{j\neq i } \left[
\frac{\partial^2 U}{\partial \rij^2} + \frac{2}{\rij} \frac{\partial
  U}{\partial \rij} + 2 \frac{\partial^2 U}{\partial \rij \partial
  \riI}\frac{\brij\cdot\briI}{\rij\riI} +\frac{\partial^2 U}{\partial
  \riI^2}
+ \frac{2}{\riI}\frac{\partial U}{\partial \riI} \nonumber
\right] \\
& = & \sum_{j\neq i } 
\left[ \frac{\partial^3 U}{\partial r_{iI} \partial^2 r_{ij}} +
\frac{2}{r_{ij}} \frac{\partial^2 U}{\partial r_{iI} \partial r_{ij}}
+ 2\left(\frac{\partial^3 U}{\partial \rij \partial^2 \riI} -\frac{1}{\riI} \frac{\partial^2 U}{\partial \rij \partial \riI}\right)\frac{\brij\cdot\briI}{\rij\riI} + \frac{\partial^3 U}{\partial^3 \riI} - \frac{2}{\riI^2} \frac{\partial U}{ \partial \riI} + \frac{2}{\riI} \frac{\partial^2 U}{\partial^2 \riI}
\right] \frac{\mathbf{I} - \mathbf{r}_i}{|\mathbf{I} - \mathbf{r}_i|} + \nonumber \\\nonumber 
 & & \sum_{j\neq i } \left[ \frac{\partial^3U}{\partial \rij^2 \partial \rjI} + \frac{2}{\rij}\frac{\partial^2 U}{\partial \rjI \partial \rij} 
+ 2\frac{\partial^3 U}{\partial \rij \partial \riI \partial \rjI}\frac{\brij\cdot\briI}{\rij\riI}
+\frac{\partial^3 U}{\partial \riI^2 \partial \rjI} + \frac{2}{\riI}\frac{\partial^2 U}{\partial \riI \partial \rjI} \right] 
\frac{\mathbf{I} - \mathbf{r}_j}{|\mathbf{r}_j - \mathbf{I}|} + \\\nonumber 
& & \sum_{j\neq i } \left[ -\frac{2}{\riI}\frac{\partial^2 U}{\partial \rij \partial \riI}\right] \frac{\mathbf{r}_{ij}}{r_{ij}}
\end{eqnarray}




\newpage
\section{Feature: Reciprocal-Space Jastrow Factors}

% Written by Kenneth P. Esler, Jr.
% Document originally included in QMCPACK at src/QMCWaveFunctions/Jastrow/kSpaceJastrowNotes.tex
% Originally titled ``Notes on Reciprocal-Space Jastrow Factors''

\renewcommand{\vG}{\mathbf{G}}
\renewcommand{\vr}{\mathbf{r}}
\renewcommand{\vI}{\mathbf{I}}

\subsection{Two-body Jastrow}
\begin{equation}
J_2 = \sum_{\vG\neq \mathbf{0}}\sum_{i\neq j} a_\vG e^{i\vG\cdot(\vr_i-\vr_j)}
\end{equation}
This may be rewritten as
\begin{eqnarray}
J_2 & = & \sum_{\vG\neq \mathbf{0}}\sum_{i\neq j} a_\vG e^{i\vG\cdot\vr_i}e^{-i\vG\cdot\vr_j} \\
& = & \sum_{\vG\neq \mathbf{0}} a_\vG \left\{
\underbrace{\left[\sum_i e^{i\vG\cdot\vr_i} \right]}_{\rho_\vG}
\underbrace{\left[\sum_j e^{-i\vG\cdot\vr_j} \right]}_{\rho_{-\vG}}  -1 \right\}
\end{eqnarray}
The $-1$ is just a constant term and may be subsumed into the $a_\vG$
coefficient by a simple redefinition.  This leaves a simple, but
general, form:
\begin{equation}
J_2 = \sum_{\vG\neq\mathbf{0}} a_\vG \rho_\vG \rho_{-\vG}
\end{equation}
We may now further constrain this on physical grounds.  First, we
recognize that $J_2$ should be real.  Since $\rho_{-\vG} =
\rho_\vG^*$, it follows that $\rho_{\vG}\rho_{-\vG} = |\rho_\vG|^2$ is
real, so that $a_\vG$ must be real.  Furthermore, we group the $\vG$'s
into $(+\vG, -\vG)$ pairs, and sum over only the positive vectors to
save time.

\subsection{One-body Jastrow}
The one-body Jastrow has a similar form, but depends on the
displacement from the electrons to the ions in the system.
\begin{equation}
J_1 = \sum_{\vG\neq\mathbf{0}} \sum_{\alpha}
\sum_{i\in\vI^\alpha}\sum_{j\in\text{elec.}} b^{\alpha}_\vG
  e^{i\vG\cdot(\vI^{\alpha}_i - \vr_j)},
\end{equation}
where $\alpha$ denotes the different ionic species.
We may rewrite this in terms of $\rho^{\alpha}_\vG$, 
\begin{equation}
J_1 = \sum_{\vG\neq\mathbf{0}} \left[\sum_\alpha b^\alpha_\vG
  \rho_\vG^\alpha\right] \rho_{-\vG},
\end{equation}
where
\begin{equation}
\rho^\alpha_\vG = \sum_{i\in\vI^\alpha} e^{i\vG\cdot\vI^\alpha_i}.
\end{equation}
We note that in the above equation, for a single configuration of the
ions, the sum in brackets can be rewritten as a single constant.  This
implies that the per-species one-body coefficients, $b^\alpha_\vG$, are
underdetermined for single configuration of the ions.  In general, if
we have $N$ species, we need $N$ linearly independent ion
configurations to uniquely determine $b^{\alpha}_\vG$.  For this
reason, we will drop the $\alpha$ superscript of $b_\vG$ for now.  

If we do desire to find a reciprocal space one-body Jastrow that is
transferable to systems with different ion positions and $N$ 
ionic species, we must perform compute $b_\vG$ for $N$ different ion
configurations.  We may then construct $N$ equations at each value of
$\vG$ to solve for the $N$ unknown values, $b^\alpha_\vG$.

In the two-body case, $a_\vG$ was constrained to be real by the fact
that $\rho_\vG \rho_{-\vG}$ was real.  However, in the one-body case,
there is no such guarantee about $\rho^\alpha_\vG \rho_\vG$.
Therefore, in general, $b_\vG$ may be complex.

\subsection{Symmetry considerations}
For a crystal, many of the $\vG$-vectors will be equivalent by
symmetry.  It is useful then, to divide the $\vG$-vectors into
symmetry-related groups and then to require that they share a common
coefficient.  Two vectors, $\vG$ and $\vG'$, may be considered
symmetry related if, for all $\alpha$ and $\beta$,
\begin{equation}
\rho^\alpha_\vG \rho^\beta_{-\vG} = \rho^\alpha_{\vG'} \rho^\beta_{-\vG'}. 
\end{equation}
For the one-body term, we may also omit from our list of $\vG$-vectors
those for which all species structure factors are zero.  This is
equivalent to saying that, if we are tiling a primitive cell, we
should include only the $\vG$-vectors of the primitive cell, and not
the supercell.  Note that this is not the case for the two-body term,
since the exchange-correlation hole should not have the periodicity of
the primitive cell.

\subsection{Gradients and Laplacians}
\begin{eqnarray}
\nabla_{\vr_i} J_2 & = & \sum_{\vG \neq 0} a_\vG \left[\left(\nabla_{\vr_i}\rho_\vG\right) \rho_{-\vG} + \text{c.c.}\right] \\
& = & \sum_{\vG\neq \mathbf{0}} 2\vG a_\vG \mathbf{Re}\left(i e^{i\vG\cdot\vr_i} \rho_{-\vG} \right) \\
& = & \sum_{\vG\neq \mathbf{0}} -2\vG a_\vG\mathbf{Im}\left(e^{i\vG\cdot\vr_i} \rho_{-\vG} \right)
\end{eqnarray}
The Laplacian is then given by
\begin{eqnarray}
  \nabla^2 J_2 & = & \sum_{\vG\neq\mathbf{0}} a_\vG \left[\left(\nabla^2 \rho_\vG\right) \rho_{-\vG} + \text{c.c.} 
  + 2\left(\nabla \rho_\vG)\cdot(\nabla \rho_{-\vG}\right)\right] \\
& = & \sum_{\vG\neq\mathbf{0}} a_\vG \left[ -2G^2\mathbf{Re}(e^{i\vG\cdot\vr_i}\rho_{-\vG}) + 
    2\left(i\vG e^{i\vG\cdot\vr_i}\right) \cdot \left(-i\vG e^{-i\vG\cdot\vr_i}\right)
\right] \\
& = & 2 \sum_{\vG\neq\mathbf{0}} G^2 a_\vG  \left[-\mathbf{Re}\left(e^{i\vG\cdot\vr_i}\rho_{-\vG}\right) + 1\right] 
%  \nabla^2_{\vr_i} J_2 & = & \nabla_{\vr_i} \cdot \nabla_{\vr_i} J_2 \\
%  & = & -2\sum_{\vG \neq \mathbf{0}} a_\vG \vG \cdot \nabla_{\vr_i} \mathbf{Im}\left(e^{i\vG\cdot\vr_i} \rho_{-\vG}\right)
%  & = & -2\sum_{\vG \neq \mathbf{0}} a_\vG \vG \cdot \mathbf{Im}\left(i\vG e^{i\vG\cdot\vr_i}\rho_{-\vG} -i\vG \right)
\end{eqnarray}


\chapter{Development Guide}
\label{chap:developguide}

The section gives guidance on how to extend the functionality of QMCPACK. Future examples will likely include topics such as the addition of a jastrow function or add a new QMC method.

For the mechanics of the github workflow and clang-format setup and deployment see the \href{https://github.com/QMCPACK/qmcpack/wiki}{\qmcpack wiki}.

\section{QMCPACK Coding Standards}

The following document presents best practices for the code at the current time. New development should follow these guidelines and contributors are expected to adhere to them for the sake of consistency in the code base. A well prepared PR should follow these standards before inclusion in the mainline. A Clang format file can be found in ? that should be run to fix and check the code style whenever possible.

Keep reformatting, feature, and bug commits separate whenever possible.  

\section{Files}
Each file should start with the header
\lstset{language=C++,style=C++}
\begin{lstlisting}
//////////////////////////////////////////////////////////////////////////////////////
// This file is distributed under the University of Illinois/NCSA Open Source License.
// See LICENSE file in top directory for details.
//
// Copyright (c) 2018 QMCPACK developers
//
// File developed by: Name, email, affiliation
//
// File created by: Name, email, affiliation
//////////////////////////////////////////////////////////////////////////////////////
\end{lstlisting}
If you make significant changes to an existing file, add yourself to the list of developed by authors.

\subsection{File organization}
Header files should be placed in the same directory as their implementations. 
Unit tests should be written for all new functionality. These tests should be placed in a \inlinecode{tests} subdirectory below the implementations.

\subsection{File Names}
Each class should be defined in a separate file whose name is the same as the class name. Use separate \inlinecode{.cpp} implementation files whenever possible to aid in incremental compilation. 

The filenames of tests are composed by the filename of the object tested and the prefix \inlinecode{test_}.
The filenames of \emph{fake} and \emph{mock} objects used in tests are composed by the prefixes \inlinecode{fake_} and \inlinecode{mock_}, respectively, and the filename of the object that is imitated.

\subsection{Header files}
All header files should be self-contained i.e. it is not dependent on following any other header when it is included. Nor should they include files that are not necessary for their use, i.e. headers only needed by the implementation. Implementation files should not include files only for the benefit of files they include.

There are many header files that violate this currently.
Each header must use \inlinecode{\#define} guards to prevent multiple inclusion.
The symbol name of the \inlinecode{\#define} guards should be \inlinecode{NAMESPACE(s)_CLASSNAME_H}.

\subsection{Includes}
Header files should be included with the full path based on the \verb|src| directory.
For example, the file \verb|qmcpack/src/QMCWaveFunctions/SPOSetBase.h| should be included as
\begin{lstlisting}
#include "QMCWaveFunctions/SPOSetBash.h"
\end{lstlisting}
Even if the included file is located in the same directory as the including file this rule should be obeyed. Header files from external projects and standard libraries should be includes using \inlinecode{<iostream>} convention, while headers that are part of the QMCPACK project should be included using the \verb|"our_header.h"| convention.

For readability it is suggested to use the standard order of includes:
\begin{enumerate}
	\item related header
	\item std C library headers
	\item std C++ library headers
	\item Other libraries' headers
	\item QMCPACK headers
\end{enumerate}

In each section the included files should be sorted in alphabetical order.

\section{Naming}
The balance between description and ease of implementation should be balanced such that the code remains self documenting within a single terminal window.  If an extremely short variable name is used its scope must be shorter than $\sim 40$ lines. An exception is made for template parameters which must be in all CAPS.

\subsection{Type and Class Names}

Type and class names should start with a capital letter and have a capital letter for each new word.
Underscores \inlinecode{_} are not allowed. 
\subsection{Namespace Names}

Namespace names should be one word, lowercase.

\subsection{Variable Names}

Variable names should not begin with a capital letter this is reserved for type and class names. Mixed caps or underscores to separate words are both permitable but only one convention should be used in a file.

\subsection{Class Data Members}

Class private/protected data members names should follow the convention of variable names with a trailing underscore \inlinecode{_}.

\subsection{Function Names}

Function names should start with a lowercase character and have a capital letter for each new word.

\subsection{Lambda Expressions}

Named lambda expressions follow the naming convention for variables:

\begin{lstlisting}[showspaces=false]
auto myWhatever = [](int i) { return i + 4; };
\end{lstlisting}

\subsection{Macro Names}

Macro names should be all uppercase and can include underscores (\inlinecode{_}).
The underscore is not allowed as first or last character.

\subsection{Test Case and Test Names}

\section{Comments}

\subsection{Comment Style}

Use the \inlinecode{// Comment} syntax for actual comments.
Use:
\begin{lstlisting}
/** base class for Single-particle orbital sets
 *
 * SPOSetBase stands for S(ingle)P(article)O(rbital)SetBase which contains
 * a number of single-particle orbitals with capabilities of
 * evaluating \f$ \psi_j({\bf r}_i)\f$
 */
\end{lstlisting}
or
\begin{lstlisting}
///index in the builder list of sposets
int builder_index;
\end{lstlisting}

\subsection{Documentation}
Doxygen will be used for source documentation. Doxygen commands should be used when appropriate guidance on this TBD.

\subsubsection{File Docs}
Do not put the file name after the \verb|\file| doxygen command. Doxygen will fill it in for the file the tag appears in.
\begin{lstlisting}
/** \file
 *  File level documentation 
 */
\end{lstlisting}

\subsubsection{Class Docs}
Every class should have a short description (in the header of the file) of what it is and what is does.
Comments for public class member functions follow the same rules as general function comments.
Comments for private members are allowed, but not mandatory.

\subsubsection{Function Docs}
For function parameters whose type is non-const reference or pointer to non-const memory,
it should be specified if they are input (In:), output (Out:) or input-output parameters (InOut:).

Example:
\begin{lstlisting}
/** Updates foo and computes bar using in_1 .. in_5.
 * \param[in] in_3
 * \param[in] in_5
 * \param[in,out] foo
 * \param[out] bar
 */

//This is probably not what our clang-format would do
void computeFooBar(Type in_1, const Type& in_2, Type& in_3,
                   const Type* in_4, Type* in_5, Type& foo,
                   Type& bar);
\end{lstlisting}

\subsubsection{Variable Documentation}
Name should be self-descriptive.  If you need documentation consider renaming first.

\subsubsection{Golden Rule of Comments}
If you modify a piece of code, also adapt the comments that belong to it if necessary.

\section{Formatting}
Use the provided clang-format style in \inlinecode{config/.clang-format} to format \verb|.h|, \verb|.hpp|, \verb|.cu| and \verb|.cpp| files. In general you should run clang-format on every file you modify, however if it hasn't been run on the file before, you should make a separate reformat PR and rebase your changes onto that.  We do not want reformatting and change commits mixed.

The following should cover the formats specified.
\subsection{Line Length}
The length of each line of your code should, in principle, be at most \emph{120} characters.

\subsection{Horizontal Spacing}
No trailing whitespaces should be added to any line.
Use no space before a comma (\inlinecode{,}) and a semicolon (\inlinecode{;}) and add a space after them if they are not at the end of a line.

\subsubsection{Binary Operators}
The assignment operator should always have spaces around it.
Other operators may have spaces around them, but it is not mandatory.

\subsubsection{Unary Operators}
Do not put any space between an unary operator and their argument.

\subsubsection{Types}
The \inlinecode{using} syntax is preferred to \inlinecode{typedef} for type aliases.
If the actual type is not excessively long or complex simple just use it.

The angle brackets of templates should not have any external and internal padding.
Pointer or Reference operators should go with the type.
Examples:
\begin{lstlisting}
Type* var;
Type& var;

Class1<Class2<type1>> object;
\end{lstlisting}

\subsection{Vertical Spacing}
Use empty lines when it helps to improve the readability of the code, but do not use too many.
Do not use empty lines after a brace which opens a scope,
or before a brace which closes a scope.
Each file should contain an empty line at the end of the file.
Some editors add an empty line automatically, some do not.

\subsection{Indentation}
Indentation consists of 2 spaces.
Do not use tabs in the code.

\subsection{Variable Declarations and Definitions}

\begin{itemize}
\item Avoid declaring multiple variables in the same declaration, especially if they are not fundamental types:

\begin{lstlisting}[showspaces=false]
int x, y;                        // Not recommended.
Matrix a("my-matrix"), b(size);  // Disallowed.

// Preferred way.
int x;
int y;
Matrix a("my-matrix");
Matrix b(10);
\end{lstlisting}

\item Use the following order for keywords and modifiers in  variable declarations:

\begin{lstlisting}[showspaces=false]
// General type
[static] [const/constexpr] Type variable_name;

// Pointer
[static] [const] Type* [const] variable_name;

// Integer
// int is optional, if a signedness or size modifier is present.
[static] [const/constexpr] [signedness] [size] int variable_name;

// Examples:
static const Matrix a(10);
const double* const d(3.14);
constexpr unsigned long l(42);
\end{lstlisting}

\end{itemize}

\subsection{Function Declarations and Definitions}

The return type should be on the same line as the function name.
Parameters should be on the same line, too, unless they do not fit on it.
Include the parameter names also in the declaration of a function, i.e.
\begin{lstlisting}
/** calculates a*b+c
 */
double function(double a, double b, double c);
\end{lstlisting}
avoid
\begin{lstlisting}
Type function(Type1, Type2, Type3);
\end{lstlisting}

\subsection{Conditionals}

Examples:
\begin{lstlisting}
if (condition)
  statement;
else
  statement;

if (condition) {
  statement;
}
else if (condition2) {
  statement;
}
else {
  statement;
}
\end{lstlisting}

\subsection{Switch statement}

Switch statements should always have a default case.

Example:
\begin{lstlisting}
switch (var) {
  case 0:
    statement1;
    statement2;
    break;

  case 1:
    statement1;
    statement2;
    break;

  default:
    statement1;
    statement2;
}
\end{lstlisting}

\subsection{Loops}

Examples:
\begin{lstlisting}
for (statement; condition; statement)
  statement;

for (statement; condition; statement) {
  statement1;
  statement2;
}

while (condition)
  statement;

while (condition) {
  statement1;
  statement2;
}

do {
  statement;
}
while (condition);
\end{lstlisting}

\subsection{Preprocessor Directives}
The preprocessor directives are not indented.
The hash is the first character of the line.

\subsection{Class Format}
\label{subsec:class_format}
\inlinecode{public}, \inlinecode{protected} and \inlinecode{private} keywords are not indented.
Do not to use \inlinecode{struct} as a way to avoid controlling access to the class. In rare cases where
a class is a fully public data structure \inlinecode{struct} is appropriate.

Example:
\begin{lstlisting}
class Foo : public Bar {
public:
  Foo();
  explicit Foo(int var);

  void function();
  void emptyFunction() {}

  void set_var(const int var) {
    var_ = var;
  }
  int get_var() const {
    return var_;
  }

private:
  bool privateFunction();

  int var_;
  int var2_;
};
\end{lstlisting}

\subsubsection{Constructor Initializer Lists}

Examples:
\begin{lstlisting}
// When everything fits on one line:
Foo::Foo(int var) : var_(var) {
  statement;
}

// If the signature and the initializer list do not
// fit on one line, the colon is indented by 4 spaces:
Foo::Foo(int var)
    : var_(var), var2_(var + 1) {
  statement;
}

// If the initializer list occupies more lines,
// they are aligned in the following way:
Foo::Foo(int var)
    : some_var_(var),
      some_other_var_(var + 1) {
  statement;
}

// No statements:
Foo::Foo(int var)
    : some_var_(var) {}
\end{lstlisting}

\subsection{Namespace Formatting}

The content of namespaces is not indented.
A comment should indicate when a namespace is closed.
If nested namespaces are used, a comment with the full namespace is required after opening a set of namespaces or an inner namespace.

Examples:
\begin{lstlisting}
namespace ns{
void foo();
}  // ns
\end{lstlisting}

\begin{lstlisting}
namespace ns1{
namespace ns2{
// ns1::ns2::
void foo();

namespace ns3{
// ns1::ns2::ns3::
void bar();
}  // ns3
}  // ns2

namespace ns4{
namespace ns5{
// ns1::ns4::ns5::
void foo();
}  // ns5
}  // ns4
}  // ns1
\end{lstlisting}


\section{Other C++ Features}

\subsection{Pre-increment and pre-decrement}

Use the pre-increment (pre-decrement) operator when a variable is incremented (decremented) and the value of the expression is not used.
In particular, use the pre-increment (pre-decrement) operator for loop counters where i is not used:

\begin{lstlisting}[showspaces=false]
for (int i = 0; i < N; ++i) {
  doSomething();
}

for (int i = 0; i < N; i++) {
  doSomething(i);
}
\end{lstlisting}

The post-increment and post-decrement operators create an unnecessary copy, that the compiler cannot optimize away in the case of iterators or other classes with overloaded increment and decrement operators.

\subsection{Alternative Operator Representations}

Alternative representations of operators and other tokens such as \inlinecode{and}, \inlinecode{or}, and \inlinecode{not} instead of \inlinecode{&&}, \inlinecode{||}, and \inlinecode{!} are not allowed.
For the reason of consistency, the far more common primary tokens should always be used.

\subsection{Use of const}

\begin{itemize}

\item Add the \inlinecode{const} qualifier to all function parameters that are not modified in the function body.
For parameters passed by value, only add the keyword in the function definition.

\begin{lstlisting}[showspaces=false]
// Declaration
int computeFoo(int bar, const Matrix& m) {

// Definition
int computeFoo(const int bar, const Matrix& m) {
  int foo = 42;

  // Compute foo without changing bar or m.
  // ...

  return foo;
}
\end{lstlisting}

\end{itemize}

\subsection{In-class member initialization}

In-class member initialization is recommended for members that are \emph{always} initialized with the same value to avoid code duplication.
In all other cases it is disallowed.

\begin{lstlisting}[showspaces=false]
std::string getNameFromId(const int id);

class Customer {
public:
  Customer(const std::string& name = "Doe",
           const std::string& address = default_address_)
    : name_(name), address_(address) {}
  Customer(const int id,
           const std::string& address = default_address_)
    : name_(getNameFromId(id)), address_(address) {}

private:
  static const std::string default_address_;

  std::string name_;
  std::string address_;
  int num_orders_ = 0;
};

const std::string Customer::default_address_ = "Zurich";
\end{lstlisting}


\section{Scalar estimator implementation}
\subsection{Introduction: Life of a specialized OperatorBase}

Almost all observables in QMCPACK are implemented as specialized derived classes of the OperatorBase base class. Each observable is instantiated in HamiltonianFactory and added to QMCHamiltonian for tracking. QMCHamiltonian tracks two types of observables: main and auxiliary. Main observables contribute to the local energy. These observables are elements of the simulated Hamiltonian such as kinetic or potential energy. Auxiliary observables are expectation values of matrix elements that do not contribute to the local energy. These Hamiltonians do not affect the dynamics of the simulation. In the code, the main observables are labeled by ``physical'' flag; the auxiliary observables have ``physical'' set to false.

\subsubsection{Initialization}
When an \verb|<estimator type="est_type" name="est_name" other_stuff="value"/>| tag is present in the \verb|<hamiltonian/>| section, it is first read by HamiltonianFactory. In general, the \verb|type| of the estimator will determine which specialization of OperatorBase should be instantiated, and a derived class with \verb|myName="est_name"| will be constructed. Then, the put() method of this specific class will be called to read any other parameters in the \verb|<estimator/>| XML node. Sometimes these parameters will instead be read by HamiltonianFactory because it can access more objects than OperatorBase.

\subsubsection{Cloning}
When \verb|OpenMP| threads are spawned, the estimator will be cloned by the \verb|CloneManager|, which is a parent class of many QMC drivers. 
\begin{lstlisting}[style=C++]
// In CloneManager.cpp
#pragma omp parallel for shared(w,psi,ham)
for(int ip=1; ip<NumThreads; ++ip)
{
  wClones[ip]=new MCWalkerConfiguration(w);
  psiClones[ip]=psi.makeClone(*wClones[ip]);
  hClones[ip]=ham.makeClone(*wClones[ip],*psiClones[ip]);
}
\end{lstlisting}
In the preceding snippet, \verb|ham| is the reference to the estimator on the master thread. If the implemented estimator does not allocate memory for any array, then the default constructor should suffice for the \verb|makeClone| method.
\begin{lstlisting}[style=C++]
// In SpeciesKineticEnergy.cpp
OperatorBase* SpeciesKineticEnergy::makeClone(ParticleSet& qp, TrialWaveFunction& psi)
{
  return new SpeciesKineticEnergy(*this);
}
\end{lstlisting}
If memory is allocated during estimator construction (usually when parsing the XML node in the \verb|put| method), then the \verb|makeClone| method should perform the same initialization on each thread.
\begin{lstlisting}[style=C++]
OperatorBase* LatticeDeviationEstimator::makeClone(ParticleSet& qp, TrialWaveFunction& psi)
{
  LatticeDeviationEstimator* myclone = new LatticeDeviationEstimator(qp,spset,tgroup,sgroup);
  myclone->put(input_xml);
  return myclone;
}
\end{lstlisting}

\subsubsection{Evaluate}
After the observable class (derived class of OperatorBase) is constructed and prepared (by the put() method), it is ready to be used in a QMCDriver. A QMCDriver will call \verb|H.auxHevaluate(W,thisWalker)| after every accepted move, where H is the QMCHamiltonian that holds all main and auxiliary Hamiltonian elements, W is a MCWalkerConfiguration, and thisWalker is a pointer to the current walker being worked on. As shown in the following, this function goes through each auxiliary Hamiltonian element and evaluates it using the current walker configuration. Under the hood, observables are calculated and dumped to the main particle set's property list for later collection.

\begin{lstlisting}[style=C++]
// In QMCHamiltonian.cpp
// This is more efficient. 
// Only calculate auxH elements if moves are accepted.
void QMCHamiltonian::auxHevaluate(ParticleSet& P, Walker_t& ThisWalker)
{
#if !defined(REMOVE_TRACEMANAGER)
  collect_walker_traces(ThisWalker,P.current_step);
#endif
  for(int i=0; i<auxH.size(); ++i)
  {
    auxH[i]->setHistories(ThisWalker);
    RealType sink = auxH[i]->evaluate(P);
    auxH[i]->setObservables(Observables);
#if !defined(REMOVE_TRACEMANAGER)
    auxH[i]->collect_scalar_traces();
#endif
    auxH[i]->setParticlePropertyList(P.PropertyList,myIndex);
  }
}
\end{lstlisting}

For estimators that contribute to the local energy (main observables), the return value of evaluate() is used in accumulating the local energy. For auxiliary estimators, the return value is not used (\verb|sink| local variable above); only the value of Value is recorded property lists by the setObservables() method as shown in the preceding code snippet. By default, the setObservables() method will transfer \verb|auxH[i]->Value| to \verb|P.PropertyList[auxH[i]->myIndex]|. The same property list is also kept by the particle set being moved by QMCDriver. This list is updated by \verb|auxH[i]->setParticlePropertyList(P.PropertyList,myIndex)|, where myIndex is the starting index of space allocated to this specific auxiliary Hamiltonian in the property list kept by the target particle set P.

\subsubsection{Collection}
The actual statistics are collected within the QMCDriver, which owns
an EstimatorManager object. This object (or a clone in the case of
multithreading) will be registered with each mover it owns. For each mover
(such as VMCUpdatePbyP derived from QMCUpdateBase), an accumulate() call
is made, which by default, makes an accumulate(walkerset) call to the
EstimatorManager it owns. Since each walker has a property set, EstimatorManager uses that local copy to calculate statistics. The EstimatorManager performs block averaging and file I/O.

\subsection{Single scalar estimator implementation guide}
Almost all of the defaults can be used for a single scalar observable. With any luck, only the put() and evaluate() methods need to be implemented. As an example, this section presents a step-by-step guide for implementing a \verb|SpeciesKineticEnergy| estimator that calculates the kinetic energy of a specific species instead of the entire particle set. For example, a possible input to this estimator can be:

\verb|<estimator type="specieskinetic" name="ukinetic" group="u"/>|

\verb|<estimator type="specieskinetic" name="dkinetic" group="d"/>|\:.

This should create two extra columns in the \inlinecode{scalar.dat} file that contains the kinetic energy of the up and down electrons in two separate columns. If the estimator is properly implemented, then the sum of these two columns should be equal to the default \verb|Kinetic| column.

\subsubsection{Barebone}

The first step is to create a barebone class structure for this simple scalar estimator. The goal is to be able to instantiate this scalar estimator with an XML node and have it print out a column of zeros in the \inlinecode{scalar.dat} file. 

To achieve this, first create a header file ``SpeciesKineticEnergy.h" in the QMCHamiltonians folder, with only the required functions declared as follows: 

\begin{lstlisting}[style=C++]
// In SpeciesKineticEnergy.h
#ifndef QMCPLUSPLUS_SPECIESKINETICENERGY_H
#define QMCPLUSPLUS_SPECIESKINETICENERGY_H

#include <Particle/WalkerSetRef.h>
#include <QMCHamiltonians/OperatorBase.h>

namespace qmcplusplus
{

class SpeciesKineticEnergy: public OperatorBase
{
public:
  
  SpeciesKineticEnergy(ParticleSet& P):tpset(P){ };
  
  bool put(xmlNodePtr cur);         // read input xml node, required
  bool get(std::ostream& os) const; // class description, required
  
  Return_t evaluate(ParticleSet& P);
  inline Return_t evaluate(ParticleSet& P, std::vector<NonLocalData>& Txy)
  { // delegate responsity inline for speed
    return evaluate(P);
  } 
  
  // pure virtual functions require overrider
  void resetTargetParticleSet(ParticleSet& P) { }                         // required
  OperatorBase* makeClone(ParticleSet& qp, TrialWaveFunction& psi); // required

private:
  ParticleSet& tpset;

}; // SpeciesKineticEnergy

} // namespace qmcplusplus
#endif
\end{lstlisting}

Notice that a local reference \verb|tpset| to the target particle set \verb|P| is saved in the constructor. The target particle set carries much information useful for calculating observables. Next, make ``SpeciesKineticEnergy.cpp," and make vacuous definitions.
\begin{lstlisting}[style=C++]
// In SpeciesKineticEnergy.cpp
#include <QMCHamiltonians/SpeciesKineticEnergy.h>
namespace qmcplusplus
{

bool SpeciesKineticEnergy::put(xmlNodePtr cur)
{
  return true;
} 

bool SpeciesKineticEnergy::get(std::ostream& os) const
{ 
  return true;
}

SpeciesKineticEnergy::Return_t SpeciesKineticEnergy::evaluate(ParticleSet& P)
{
  Value = 0.0;
  return Value;
}

OperatorBase* SpeciesKineticEnergy::makeClone(ParticleSet& qp, TrialWaveFunction& psi)
{
  // no local array allocated, default constructor should be enough
  return new SpeciesKineticEnergy(*this);
}

} // namespace qmcplusplus
\end{lstlisting}

Now, head over to HamiltonianFactory and instantiate this observable if an XML node is found requesting it. Look for ``gofr" in HamiltonianFactory.cpp, for example, and follow the if block.
\begin{lstlisting}[style=C++]
// In HamiltonianFactory.cpp
#include <QMCHamiltonians/SpeciesKineticEnergy.h>
else if(potType =="specieskinetic")
{        
  SpeciesKineticEnergy* apot = new SpeciesKineticEnergy(*target_particle_set);
  apot->put(cur);
  targetH->addOperator(apot,potName,false);
}
\end{lstlisting}
The last argument of addOperator() (i.e., the \verb|false| flag) is \textbf{crucial}. This tells QMCPACK that the observable we implemented is not a physical Hamiltonian; thus, it will not contribute to the local energy. Changes to the local energy will alter the dynamics of the simulation. Finally, add ``SpeciesKineticEnergy.cpp" to HAMSRCS in ``CMakeLists.txt" located in the QMCHamiltonians folder. Now, recompile QMCPACK and run it on an input that requests \verb|<estimator type="specieskinetic" name="ukinetic"/>| in the \verb|hamiltonian| block. A column of zeros should appear in the \inlinecode{scalar.dat} file under the name ``ukinetic."

\subsubsection{Evaluate}
The evaluate() method is where we perform the calculation of the desired observable. In a first iteration, we will simply hard-code the name and mass of the particles.
\begin{lstlisting}[style=C++]
// In SpeciesKineticEnergy.cpp
#include <QMCHamiltonians/BareKineticEnergy.h> // laplaician() defined here
SpeciesKineticEnergy::Return_t SpeciesKineticEnergy::evaluate(ParticleSet& P)
{
  std::string group="u";
  RealType minus_over_2m = -0.5;
  
  SpeciesSet& tspecies(P.getSpeciesSet());
  
  Value = 0.0;
  for (int iat=0; iat<P.getTotalNum(); iat++)
  {
    if (tspecies.speciesName[ P.GroupID(iat) ] == group)
    {
      Value += minus_over_2m*laplacian(P.G[iat],P.L[iat]);
    }
  }
  return Value;
  
  // Kinetic column has:
  // Value = -0.5*( Dot(P.G,P.G) + Sum(P.L) );
}
\end{lstlisting}
\textit{Voila}---you should now be able to compile QMCPACK, rerun, and see that the values in the ``ukinetic'' column are no longer zero. Now, the only task left to make this basic observable complete is to read in the extra parameters instead of hard-coding them.

\subsubsection{Parse extra input}
The preferred method to parse extra input parameters in the XML node is to implement the put() function of our specific observable. Suppose we wish to read in a single string that tells us whether to record the kinetic energy of the up electron (group=``u") or the down electron (group=``d"). This is easily achievable using the OhmmsAttributeSet class,
\begin{lstlisting}[style=C++]
// In SpeciesKineticEnergy.cpp
#include <OhmmsData/AttributeSet.h>
bool SpeciesKineticEnergy::put(xmlNodePtr cur)
{ 
  // read in extra parameter "group"
  OhmmsAttributeSet attrib;
  attrib.add(group,"group");
  attrib.put(cur);
  
  // save mass of specified group of particles
  SpeciesSet& tspecies(tpset.getSpeciesSet());
  int group_id  = tspecies.findSpecies(group);
  int massind   = tspecies.getAttribute("mass");
  minus_over_2m = -1./(2.*tspecies(massind,group_id));
  
  return true;
}
\end{lstlisting}
where we may keep ``group'' and ``minus\_over\_2m'' as local variables to our specific class.
\begin{lstlisting}[style=C++]
// In SpeciesKineticEnergy.h
private:
  ParticleSet& tpset;
  std::string  group;
  RealType minus_over_2m;
\end{lstlisting}
Notice that the previous operations are made possible by the saved reference to the target particle set. Last but not least, compile and run a full example (i.e., a short DMC calculation) with the following XML nodes in your input:

\verb|<estimator type="specieskinetic" name="ukinetic" group="u"/>|

\verb|<estimator type="specieskinetic" name="dkinetic" group="d"/>|\:.\\

Make sure the sum of the ``ukinetic" and ``dkinetic" columns is \textbf{exactly} the same as the Kinetic columns at \textbf{every block}.

For easy reference, a summary of the complete list of changes follows:
\begin{lstlisting}[style=C++]
// In HamiltonianFactory.cpp
#include "QMCHamiltonians/SpeciesKineticEnergy.h"
else if(potType =="specieskinetic")
{
	SpeciesKineticEnergy* apot = new SpeciesKineticEnergy(*targetPtcl);
	apot->put(cur);
	targetH->addOperator(apot,potName,false);
}
\end{lstlisting}
\begin{lstlisting}[style=C++]
// In SpeciesKineticEnergy.h
#include <Particle/WalkerSetRef.h>
#include <QMCHamiltonians/OperatorBase.h>

namespace qmcplusplus
{

class SpeciesKineticEnergy: public OperatorBase
{
public:

  SpeciesKineticEnergy(ParticleSet& P):tpset(P){ };

  // xml node is read by HamiltonianFactory, eg. the sum of following should be equivalent to Kinetic
  // <estimator name="ukinetic" type="specieskinetic" target="e" group="u"/>
  // <estimator name="dkinetic" type="specieskinetic" target="e" group="d"/>
  bool put(xmlNodePtr cur);         // read input xml node, required
  bool get(std::ostream& os) const; // class description, required
  
  Return_t evaluate(ParticleSet& P);
  inline Return_t evaluate(ParticleSet& P, std::vector<NonLocalData>& Txy)
  { // delegate responsity inline for speed
    return evaluate(P);
  } 
  
  // pure virtual functions require overrider
  void resetTargetParticleSet(ParticleSet& P) { }                         // required
  OperatorBase* makeClone(ParticleSet& qp, TrialWaveFunction& psi); // required
  
private:
  ParticleSet&       tpset; // reference to target particle set
  std::string        group; // name of species to track
  RealType   minus_over_2m; // mass of the species !! assume same mass
  // for multiple species, simply initialize multiple estimators
  
}; // SpeciesKineticEnergy

} // namespace qmcplusplus
#endif
\end{lstlisting}
\begin{lstlisting}[style=C++]
// In SpeciesKineticEnergy.cpp
#include <QMCHamiltonians/SpeciesKineticEnergy.h>
#include <QMCHamiltonians/BareKineticEnergy.h> // laplaician() defined here
#include <OhmmsData/AttributeSet.h>

namespace qmcplusplus
{

bool SpeciesKineticEnergy::put(xmlNodePtr cur)
{
  // read in extra parameter "group"
  OhmmsAttributeSet attrib;
  attrib.add(group,"group");
  attrib.put(cur);
  
  // save mass of specified group of particles
  int group_id  = tspecies.findSpecies(group);
  int massind   = tspecies.getAttribute("mass");
  minus_over_2m = -1./(2.*tspecies(massind,group_id)); 

  return true;
}

bool SpeciesKineticEnergy::get(std::ostream& os) const
{ // class description
  os << "SpeciesKineticEnergy: " << myName << " for species " << group;
  return true;
}

SpeciesKineticEnergy::Return_t SpeciesKineticEnergy::evaluate(ParticleSet& P)
{
  Value = 0.0;
  for (int iat=0; iat<P.getTotalNum(); iat++)
  {
    if (tspecies.speciesName[ P.GroupID(iat) ] == group)
    {
      Value += minus_over_2m*laplacian(P.G[iat],P.L[iat]);
    }
  }
  return Value;
}

OperatorBase* SpeciesKineticEnergy::makeClone(ParticleSet& qp, TrialWaveFunction& psi)
{ //default constructor
  return new SpeciesKineticEnergy(*this);
}

} // namespace qmcplusplus
\end{lstlisting}

\subsection{Multiple scalars}
It is fairly straightforward to create more than one column in the \inlinecode{scalar.dat} file with a single observable class. For example, if we want a single SpeciesKineticEnergy estimator to simultaneously record the kinetic energies of all species in the target particle set, we only have to write two new methods: addObservables() and setObservables(), then tweak the behavior of evaluate(). First, we will have to override the default behavior of addObservables() to make room for more than one column in the \inlinecode{scalar.dat} file as follows,
\begin{lstlisting}[style=C++]
// In SpeciesKineticEnergy.cpp
void SpeciesKineticEnergy::addObservables(PropertySetType& plist, BufferType& collectables)
{
  myIndex = plist.size();
  for (int ispec=0; ispec<num_species; ispec++)
  { // make columns named "$myName_u", "$myName_d" etc.
    plist.add(myName + "_" + species_names[ispec]);
  }
}
\end{lstlisting}
where ``num\_species'' and ``species\_name'' can be local variables initialized in the constructor. We should also initialize some local arrays to hold temporary data.
\begin{lstlisting}[style=C++]
// In SpeciesKineticEnergy.h
private:
  int num_species;
  std::vector<std::string> species_names;
  std::vector<RealType> species_kinetic,vec_minus_over_2m;
  
// In SpeciesKineticEnergy.cpp
SpeciesKineticEnergy::SpeciesKineticEnergy(ParticleSet& P):tpset(P)
{
  SpeciesSet& tspecies(P.getSpeciesSet());
  int massind = tspecies.getAttribute("mass");

  num_species = tspecies.size();
  species_kinetic.resize(num_species);
  vec_minus_over_2m.resize(num_species);
  species_names.resize(num_species);
  for (int ispec=0; ispec<num_species; ispec++)
  {
    species_names[ispec] = tspecies.speciesName[ispec];
    vec_minus_over_2m[ispec] = -1./(2.*tspecies(massind,ispec));   
  }
}
\end{lstlisting}
Next, we need to override the default behavior of \icode{setObservables()} to transfer multiple values to the property list kept by the main particle set, which eventually goes into the \inlinecode{scalar.dat} file.
\begin{lstlisting}[style=C++]
// In SpeciesKineticEnergy.cpp
void SpeciesKineticEnergy::setObservables(PropertySetType& plist)
{ // slots in plist must be allocated by addObservables() first
  copy(species_kinetic.begin(),species_kinetic.end(),plist.begin()+myIndex);
}
\end{lstlisting}
Finally, we need to change the behavior of evaluate() to fill the local vector ``species\_kinetic'' with appropriate observable values.
\begin{lstlisting}[style=C++]
SpeciesKineticEnergy::Return_t SpeciesKineticEnergy::evaluate(ParticleSet& P)
{
  std::fill(species_kinetic.begin(),species_kinetic.end(),0.0);

  for (int iat=0; iat<P.getTotalNum(); iat++)
  {
    int ispec = P.GroupID(iat);
    species_kinetic[ispec] += vec_minus_over_2m[ispec]*laplacian(P.G[iat],P.L[iat]);
  }
  
  Value = 0.0; // Value is no longer used
  return Value;
}
\end{lstlisting}
That's it! The SpeciesKineticEnergy estimator no longer needs the ``group'' input and will automatically output the kinetic energy of every species in the target particle set in multiple columns. You should now be able to run with 
\verb|<estimator type="specieskinetic" name="skinetic"/>| and check that the sum of all columns that start with ``skinetic'' is equal to the default ``Kinetic'' column.

\subsection{HDF5 output}
If we desire an observable that will output hundreds of scalars per simulation step (e.g., SkEstimator), then it is preferred to output to the \inlinecode{stat.h5} file instead of the \inlinecode{scalar.dat} file for better organization. A large chunk of data to be registered in the \inlinecode{stat.h5} file is called a ``Collectable'' in QMCPACK. In particular, if a OperatorBase object is initialized with \verb|UpdateMode.set(COLLECTABLE,1)|, then the ``Collectables'' object carried by the main particle set will be processed and written to the \inlinecode{stat.h5} file, where ``UpdateMode'' is a bit set (i.e., a collection of flags) with the following enumeration:
\begin{lstlisting}[style=C++]
// In OperatorBase.h
///enum for UpdateMode
enum {PRIMARY=0,
  OPTIMIZABLE=1,
  RATIOUPDATE=2,
  PHYSICAL=3,
  COLLECTABLE=4,
  NONLOCAL=5,
  VIRTUALMOVES=6
};
\end{lstlisting}

As a simple example, to put the two columns we produced in the previous section into the \inlinecode{stat.h5} file, we will first need to declare that our observable uses ``Collectables.''
\begin{lstlisting}[style=C++]
// In constructor add: 
hdf5_out = true;
UpdateMode.set(COLLECTABLE,1);
\end{lstlisting}
Then make some room in the ``Collectables'' object carried by the target particle set.
\begin{lstlisting}[style=C++]
// In addObservables(PropertySetType& plist, BufferType& collectables) add:
if (hdf5_out)
{
  h5_index = collectables.size();
  std::vector<RealType> tmp(num_species);
  collectables.add(tmp.begin(),tmp.end());
}
\end{lstlisting}
Next, make some room in the \inlinecode{stat.h5} file by overriding the registerCollectables() method.
\begin{lstlisting}[style=C++]
// In SpeciesKineticEnergy.cpp
void SpeciesKineticEnergy::registerCollectables(std::vector<observable_helper*>& h5desc, hid_t gid) const
{
  if (hdf5_out)
  {
    std::vector<int> ndim(1,num_species);
    observable_helper* h5o=new observable_helper(myName);
    h5o->set_dimensions(ndim,h5_index);
    h5o->open(gid);
    h5desc.push_back(h5o);
  }
}
\end{lstlisting}
Finally, edit evaluate() to use the space in the ``Collectables'' object.
\begin{lstlisting}[style=C++]
// In SpeciesKineticEnergy.cpp
SpeciesKineticEnergy::Return_t SpeciesKineticEnergy::evaluate(ParticleSet& P)
{
  RealType wgt = tWalker->Weight; // MUST explicitly use DMC weights in Collectables!
  std::fill(species_kinetic.begin(),species_kinetic.end(),0.0);

  for (int iat=0; iat<P.getTotalNum(); iat++)
  {
    int ispec = P.GroupID(iat);
    species_kinetic[ispec] += vec_minus_over_2m[ispec]*laplacian(P.G[iat],P.L[iat]);
    P.Collectables[h5_index + ispec] += vec_minus_over_2m[ispec]*laplacian(P.G[iat],P.L[iat])*wgt;
  }

  Value = 0.0; // Value is no longer used
  return Value;
}
\end{lstlisting}
There should now be a new entry in the \inlinecode{stat.h5} file containing the same columns of data as the \inlinecode{stat.h5} file. After this check, we should clean up the code by
\begin{enumerate}
\item making ``hdf5\_out'' and input flag by editing the put() method and
\item disabling output to \inlinecode{scalar.dat} when the ``hdf5\_out'' flag is on.
\end{enumerate}



\section{Estimator Output}
\subsection{Estimator Definition}
For simplicity, consider a local property $O(\bs{R})$, where $\bs{R}$ is the collection of all particle coordinates. An \textit{estimator} for $O(\bs{R}) $ is a weighted average over walkers
\begin{align}
E[O] = \left(\sum\limits_{i=1}^{N^{tot}_{walker}} w_i O(\bs{R}_i) \right) / \left( \sum \limits_{i=1}^{N^{tot}_{walker}} w_i \right). \label{eq:estimator}
\end{align}
$N^{tot}_{walker}$ is the total number of walkers collected in the entire simulation. Notice, $N^{tot}_{walker}$ is typically far larger than the number of walkers held in memory at any given simulation step. $w_i$ is the weight of walker $i$.

In a VMC simulation, the weight of every walkers is 1.0. Further, the number of walkers is constant at each step. Therefore, eq.~(\ref{eq:estimator}) simplifies to
\begin{align}
E_{VMC}[O] = \frac{1}{N_{step}N_{walker}^{ensemble}} \sum_{s,e} O(\bs{R}_{s,e}).
\end{align}
Each walker $\bs{R}_{s,e}$ is labeled by \textit{step index} s, and \textit{ensemble index} e.

In a DMC simulation, the weight of each walker is different and may change from step to step. Further, the ensemble size varies from step to step. Therefore, eq.~(\ref{eq:estimator}) simplifies to
\begin{align}
E_{DMC}[O] = \frac{1}{N_{step}} \sum_{s} \left\{ \left(\sum_e w_{s,e} O(\bs{R}_{s,e})  \right) / \left( \sum \limits_{e} w_{s,e} \right)  \right\}.
\end{align}

I will refer to the average in the $\{\}$ as \textit{ensemble average} and the remaining averages \textit{block average}. The process of calculating $O(\bs{R})$ is \textit{evaluate}.

\subsection{Class Relations}
A large number of classes are involved in the estimator collection process. They often have misleading class name or method name. Document gotchas in the following list:
\begin{enumerate}
\item \verb|EstimatorManager| is an unused copy of \verb|EstimatorManagerBase|. \verb|EstimatorManagerBase| is the class used in the QMC drivers. (PR \#371 explains this)
\item \verb|EstimatorManagerBase::Estimators| is completely different from \verb|QMCDriver::Estimators|, which is subtly different from \verb|QMCHamiltonianBase::Estimators|. The first is a list of pointers to \verb|ScalarEstimatorBase|. The second is the master estimator (one per MPI group). The third is the slave estimator that exists one per OpenMP thread.
\item \verb|QMCHamiltonian| is NOT a parent class of \verb|QMCHamiltonianBase|. Instead, \verb|QMCHamiltonian| owns two lists of \verb|QMCHamiltonianBase| named \verb|H| and \verb|auxH|.
\item \verb|QMCDriver::H| is NOT the same as \verb|QMCHamiltonian::H|. The first is a pointer to a \verb|QMCHamiltonian|. \verb|QMCHamiltonian::H| is a list.
\item \verb|EstimatorManager::stopBlock(std::vector)| is completely different from \verb|EstimatorManager::|
\verb|stopBlock(RealType)|, which is the same as \verb|stopBlock(RealType, true)|, but is subtly different from \verb|stopBlock(RealType, false)|. The first three methods are intended to be called by the master estimator which exists one per MPI group. The last method is intended to be called by the slave estimator which exists one per OpenMP thread.
\end{enumerate}

\subsection{Estimator Output Stages}
%In QMCPACK, evaluation is done by \verb|QMCHamiltonianBase|; ensemble average is done either by a ``CloneDriver'' (e.g. \verb|VMCSingleOMP|, \verb|DMCOMP|) or \verb|ScalarEstimatorBase|; block average is done by \verb|ScalarEstimatorBase| or \verb|EstimatorManagerBase|. Walkers can be accessed by ``CloneDriver'' and \verb|QMCHamiltonianBase| but not by \verb|EstimatorManagerBase| or \verb|ScalarEstimatorBase|. Output files can be accessed by the latter two classes but not the former two. Therefore, in order to output estimators to file, data must be transferred from \textit{evaluate} classes to \textit{average} classes.

Estimators take four conceptual stages to propagate to the output files: evaluate, load ensemble, unload ensemble, and collect. They are easier to understand in reverse order.

\subsubsection{Collect Stage}
File output is performed by the master \verb|EstimatorManager| owned by \verb|QMCDriver|. The first 8+ entries in \verb|EstimatorManagerBase::AverageCache| will be written to scalar.dat. The remaining entries in \verb|AverageCache| will be written to stat.h5. File writing is triggered by \verb|EstimatorManagerBase|\\ \verb|::collectBlockAverages| inside \verb|EstimatorManagerBase::stopBlock|.

\begin{lstlisting}
// In EstimatorManagerBase.cpp::collectBlockAverages
  if(Archive)
  {
    *Archive << std::setw(10) << RecordCount;
    int maxobjs=std::min(BlockAverages.size(),max4ascii);
    for(int j=0; j<maxobjs; j++)
      *Archive << std::setw(FieldWidth) << AverageCache[j];
    for(int j=0; j<PropertyCache.size(); j++)
      *Archive << std::setw(FieldWidth) << PropertyCache[j];
    *Archive << std::endl;
    for(int o=0; o<h5desc.size(); ++o)
      h5desc[o]->write(AverageCache.data(),SquaredAverageCache.data());
    H5Fflush(h_file,H5F_SCOPE_LOCAL);
  }
\end{lstlisting}

\verb|EstimatorManagerBase::collectBlockAverages| is triggered from master-thread estimator via either \verb|stopBlock(std::vector)| or \verb|stopBlock(RealType, true)|. Notice, file writing is NOT triggered by the slave-thread estimator method \verb|stopBlock(RealType, false)|.

\begin{lstlisting}
// In EstimatorManagerBase.cpp
void EstimatorManagerBase::stopBlock(RealType accept, bool collectall)
{
  //take block averages and update properties per block
  PropertyCache[weightInd]=BlockWeight;
  PropertyCache[cpuInd] = MyTimer.elapsed();
  PropertyCache[acceptInd] = accept;
  for(int i=0; i<Estimators.size(); i++)
    Estimators[i]->takeBlockAverage(AverageCache.begin(),SquaredAverageCache.begin());
  if(Collectables)
  { 
    Collectables->takeBlockAverage(AverageCache.begin(),SquaredAverageCache.begin());
  }
  if(collectall)
    collectBlockAverages(1);
}
\end{lstlisting}

\begin{lstlisting}
// In ScalarEstimatorBase.h
template<typename IT>
inline void takeBlockAverage(IT first, IT first_sq)
{
  first += FirstIndex;
  first_sq += FirstIndex;
  for(int i=0; i<scalars.size(); i++)
  {
    *first++ = scalars[i].mean();
    *first_sq++ = scalars[i].mean2();
    scalars_saved[i]=scalars[i]; //save current block
    scalars[i].clear();
  }
}
\end{lstlisting}

At the collect stage, \verb|ScalarEstimatorBase::scalars| must be populated with ensemble-averaged data. Two derived classes of \verb|ScalarEstimatorBase| are crucial: \verb|LocalEnergyEstimator| will carry \verb|Properties|, where as \verb|CollectablesEstimator| will carry \verb|Collectables|.

\subsubsection{Unload Ensemble Stage}
\verb|LocalEnergyEstimator::scalars| are populated by \verb|ScalarEstimatorBase::accumulate|, whereas \verb|CollectablesEstimator::scalars| are populated by \verb|CollectablesEstimator::|
\verb|accumulate_all|. Both accumulate methods are triggered by \verb|EstimatorManagerBase::accumulate|. One confusing aspect about the unload stage is that \verb|EstimatorManagerBase::accumulate| has a master and a slave call signature. A slave estimator such as \verb|QMCUpdateBase::Estimators| should unload a subset of walkers. Thus, the slave estimator should call \verb|accumulate(W,it,it_end)|. However, the master estimator, such as \verb|SimpleFixedNodeBranch::myEstimator| should unload data from the entire walker ensemble. This is achieved by calling \verb|accumulate(W)|.

\begin{lstlisting}
void EstimatorManagerBase::accumulate(MCWalkerConfiguration& W)
{ // intended to be called by master estimator only
  BlockWeight += W.getActiveWalkers();
  RealType norm=1.0/W.getGlobalNumWalkers();
  for(int i=0; i< Estimators.size(); i++)
    Estimators[i]->accumulate(W,W.begin(),W.end(),norm);
  if(Collectables)//collectables are normalized by QMC drivers
    Collectables->accumulate_all(W.Collectables,1.0);
}
\end{lstlisting}

\begin{lstlisting}
void EstimatorManagerBase::accumulate(MCWalkerConfiguration& W
 , MCWalkerConfiguration::iterator it
 , MCWalkerConfiguration::iterator it_end)
{ // intended to be called slaveEstimator only
  BlockWeight += it_end-it;
  RealType norm=1.0/W.getGlobalNumWalkers();
  for(int i=0; i< Estimators.size(); i++)
    Estimators[i]->accumulate(W,it,it_end,norm);
  if(Collectables)
    Collectables->accumulate_all(W.Collectables,1.0);
}
\end{lstlisting}

\begin{lstlisting}
// In LocalEnergyEstimator.h
inline void accumulate(const Walker_t& awalker, RealType wgt)
{ // ensemble average W.Properties
  // expect ePtr to be W.Properties; expect wgt = 1/GlobalNumberOfWalkers
  const RealType* restrict ePtr = awalker.getPropertyBase();
  RealType wwght= wgt* awalker.Weight;
  scalars[0](ePtr[LOCALENERGY],wwght);
  scalars[1](ePtr[LOCALENERGY]*ePtr[LOCALENERGY],wwght);
  scalars[2](ePtr[LOCALPOTENTIAL],wwght);
  for(int target=3, source=FirstHamiltonian; target<scalars.size(); ++target, ++source)
    scalars[target](ePtr[source],wwght);
}
\end{lstlisting}

\begin{lstlisting}
// In CollectablesEstimator.h
inline void accumulate_all(const MCWalkerConfiguration::Buffer_t& data, RealType wgt)
{ // ensemble average W.Collectables
  // expect data to be W.Collectables; expect wgt = 1.0
  for(int i=0; i<data.size(); ++i)
    scalars[i](data[i], wgt);
}
\end{lstlisting}

At the unload ensemble stage, the data structures \verb|Properties| and \verb|Collectables| must be populated by appropriately normalized values so that the ensemble average can be correctly taken. \verb|QMCDriver| is responsible for the correct loading of data onto the walker ensemble.

\subsubsection{Load Ensemble Stage}
\verb|Properties| in the Monte Carlo ensemble of walkers \verb|QMCDriver::W| is populated by \verb|QMCHamiltonian|\\ \verb|::saveProperties|. The master \verb|QMCHamiltonian::LocalEnergy|, \verb|::KineticEnergy|, and \verb|::Observables| must be properly populated at the end of the evaluate stage.
\begin{lstlisting}
// In QMCHamiltonian.h
  template<class IT>
  inline
  void saveProperty(IT first)
  { // expect first to be W.Properties
    first[LOCALPOTENTIAL]= LocalEnergy-KineticEnergy;
    copy(Observables.begin(),Observables.end(),first+myIndex);
  }
\end{lstlisting}

\verb|Collectables|'s load stage is combined with its evaluate stage.

\subsubsection{Evaluate Stage}

The master \verb|QMCHamiltonian::Observables| is populated by slave \verb|QMCHamiltonianBase|
\verb|::setObservables|. However, the call signature must be \verb|QMCHamiltonianBase::setObservables(|
\verb|QMCHamiltonian::Observables)|. This call signature is enforced by \verb|QMCHamiltonian::evaluate| and \verb|QMCHamiltonian::auxHevaluate|.

\begin{lstlisting}
// In QMCHamiltonian.cpp
QMCHamiltonian::Return_t
QMCHamiltonian::evaluate(ParticleSet& P)
{
  LocalEnergy = 0.0;
  for(int i=0; i<H.size(); ++i)
  {
    myTimers[i]->start();
    LocalEnergy += H[i]->evaluate(P);
    H[i]->setObservables(Observables);
#if !defined(REMOVE_TRACEMANAGER)
    H[i]->collect_scalar_traces();
#endif
    myTimers[i]->stop();
    H[i]->setParticlePropertyList(P.PropertyList,myIndex);
  }
  KineticEnergy=H[0]->Value;
  P.PropertyList[LOCALENERGY]=LocalEnergy;
  P.PropertyList[LOCALPOTENTIAL]=LocalEnergy-KineticEnergy;
  // auxHevaluate(P);
  return LocalEnergy;
}
\end{lstlisting}

\begin{lstlisting}
// In QMCHamiltonian.cpp
void QMCHamiltonian::auxHevaluate(ParticleSet& P, Walker_t& ThisWalker)
{
#if !defined(REMOVE_TRACEMANAGER)
  collect_walker_traces(ThisWalker,P.current_step);
#endif
  for(int i=0; i<auxH.size(); ++i)
  {
    auxH[i]->setHistories(ThisWalker);
    RealType sink = auxH[i]->evaluate(P);
    auxH[i]->setObservables(Observables);
#if !defined(REMOVE_TRACEMANAGER)
    auxH[i]->collect_scalar_traces();
#endif
    auxH[i]->setParticlePropertyList(P.PropertyList,myIndex);
  }
}
\end{lstlisting}

\subsection{Estimator Use Cases}

\subsubsection{VMCSingleOMP pseudo code}
\begin{lstlisting}
bool VMCSingleOMP::run()
{
  masterEstimator->start(nBlocks);
  for (int ip=0; ip<NumThreads; ++ip)
    Movers[ip]->startRun(nBlocks,false);  // slaveEstimator->start(blocks, record)
  
  do // block
  {
    #pragma omp parallel
    {
      Movers[ip]->startBlock(nSteps);  // slaveEstimator->startBlock(steps)
      RealType cnorm = 1.0/static_cast<RealType>(wPerNode[ip+1]-wPerNode[ip]);
      do // step
      {
        wClones[ip]->resetCollectables();
        Movers[ip]->advanceWalkers(wit, wit_end, recompute);
        wClones[ip]->Collectables *= cnorm;
        Movers[ip]->accumulate(wit, wit_end);
      } // end step
      Movers[ip]->stopBlock(false);  // slaveEstimator->stopBlock(acc, false)
    } // end omp
    masterEstimator->stopBlock(estimatorClones);  // write files
  } // end block
  masterEstimator->stop(estimatorClones);
}
\end{lstlisting}

\subsubsection{DMCOMP  pseudo code}
\begin{lstlisting}
bool DMCOMP::run()
{
  masterEstimator->setCollectionMode(true);
  
  masterEstimator->start(nBlocks);
  for(int ip=0; ip<NumThreads; ip++)
    Movers[ip]->startRun(nBlocks,false);  // slaveEstimator->start(blocks, record)
  
  do // block
  {
    masterEstimator->startBlock(nSteps);
    for(int ip=0; ip<NumThreads; ip++)
      Movers[ip]->startBlock(nSteps);  // slaveEstimator->startBlock(steps)
    
    do // step
    {
      #pragma omp parallel
      {
      wClones[ip]->resetCollectables();
      // advanceWalkers
      } // end omp
      
      //branchEngine->branch
      { // In WalkerControlMPI.cpp::branch
      wgt_inv=WalkerController->NumContexts/WalkerController->EnsembleProperty.Weight;
      walkers.Collectables *= wgt_inv;
      slaveEstimator->accumulate(walkers);
      }
      masterEstimator->stopBlock(acc)  // write files
    }  // end for step
  }  // end for block
  
  masterEstimator->stop();
}
\end{lstlisting}

\subsection{Summary}

Two ensemble-level data structures \verb|ParticleSet::Properties| and \verb|::Collectables| serve as intermediaries between evaluate classes and output classes to scalar.dat and stat.h5. \verb|Properties| appears in both scalar.dat and stat.h5, whereas \verb|Collectables| appears only in stat.h5. \verb|Properties| is overwritten by \verb|QMCHamiltonian::Observables| at the end of each step. \verb|QMCHamiltonian::Observables| is filled upon call to \verb|QMCHamiltonian::evaluate| and \verb|::auxHevaluate|. \verb|Collectables| is zeroed at the beginning of each step and accumulated upon call to \verb|::auxHevaluate|.

Data are outputted to scalar.dat in 4 stages: evaluate, load, unload, and collect. In the evaluate stage, \verb|QMCHamiltonian::Observables| is populated by a list of \verb|QMCHamiltonianBase|. In the load stage, \verb|QMCHamiltonian::Observables| is transfered to \verb|Properties| by \verb|QMCDriver|. In the unload stage, \verb|Properties| is copied to \verb|LocalEnergyEstimator::scalars|. In the collect stage, \verb|LocalEnergyEstimator::scalars| is block-averaged to \verb|EstimatorManagerBase|\\ \verb|::AverageCache| and dumped to file. For \verb|Collectables|, the evaluate and load stages are combined in a call to \verb|QMCHamiltonian::auxHevaluate|. In the unload stage, \verb|Collectables| is copied to \verb|CollectablesEstimator::scalars|. In the collect stage, \verb|CollectablesEstimator|\\ \verb|::scalars| is block-averaged to \verb|EstimatorManagerBase::AverageCache| and dumped to file.

\subsection{Appendix: dmc.dat}

There is an additional data structure \verb|ParticleSet::EnsembleProperty|, which is managed by \verb|WalkerControlBase::EnsembleProperty| and directly dumped to dmc.dat via its own averaging procedure. dmc.dat is written by \verb|WalkerControlBase::measureProperties|, which is called by \verb|WalkerControlBase::branch|, which is called by \verb|SimpleFixedNodeBranch|\\ \verb|::branch| for example.

\newcommand{\bs}{\boldsymbol}
\newcommand{\tr}{\text{tr}}
\section{Slater-Backflow Wavefunction Implementation Details}

For simplicity, consider $N$ identical fermions of the same spin (e.g. up electrons) at spatial locations $\{\bs{r}_1,\bs{r}_2,\dots,\bs{r}_{N}\}$. Then the Slater determinant can be written as
\begin{align}
S=\det M,
\end{align}
where each entry in the determinant is a single-particle orbital evaluated at a particle position
\begin{align}
M_{ij} = \phi_i(\bs{r}_j).
\end{align}

When backflow transformation is applied to the Determinant, the particle coordinates $\bs{r}_i$ that go into the single-particle orbitals are replaced by quasi-particle coordinates $\bs{x}_i$
\begin{align}
M_{ij} = \phi_i(\bs{x}_j), \label{eq:psiM}
\end{align}
where
\begin{align}
\bs{x}_i=\bs{r}_i+\sum\limits_{j=1,j\neq i}^N\eta(r_{ij})(\bs{r}_i-\bs{r}_j). \label{eq:quasi}
\end{align}
$r_{ij}=\vert\bs{r}_i-\bs{r}_j\vert$. The integers i,j label the particle/quasi-particle. There is a one-to-one correspondence between the particles and the quasi-particles, which is simplest when $\eta=0$.

\subsection{Value}
The evaluation of the Slater-Backflow wavefunction is almost identical to that of a Slater wavefunction. The only difference is that the quasi-particles coordinates are used to evaluate the single-particle orbitals. The actual value of the determinant is stored during the inversion of the matrix $M$ (\verb|cgetrf|$\rightarrow$\verb|cgetri|). Suppose $M=LU$, then $S=\prod\limits_{i=1}^N L_{ii} U_{ii}$. \\

\begin{lstlisting}
// In DiracDeterminantWithBackflow::evaluateLog(P,G,L)
Phi->evaluate(BFTrans->QP, FirstIndex, LastIndex, psiM,dpsiM,grad_grad_psiM);
psiMinv = psiM;
LogValue=InvertWithLog(psiMinv.data(),NumPtcls,NumOrbitals
  ,WorkSpace.data(),Pivot.data(),PhaseValue);
\end{lstlisting}

QMCPACK represents the complex value of the wavefunction in polar coordinates $S=e^Ue^{i\theta}$. Specifically, \verb|LogValue| $U$ and \verb|PhaseValue| $\theta$ are handled separately. In the following, I will consider derivatives of the log value only.

\subsection{Gradient}
To evaluate particle gradient of the log value of the Slater-Backflow wavefunction, we can use the $\log\det$ identity in eq.~(\ref{eq:logdet}). This identity maps the derivative of $\log\det M$ with respect to a real variable $p$ to a trace over $M^{-1}dM$
\begin{align}
\frac{\partial}{\partial p}\log\det M = \tr\left( M^{-1} \frac{\partial M}{\partial p} \right) \label{eq:logdet}.
\end{align}

Following Kwon, Ceperley, and Martin~\cite{Kwon1993backflow}, the particle gradient
\begin{align}
G_i^\alpha \equiv \frac{\partial}{\partial r_i^\alpha} \log\det M = \sum\limits_{j=1}^N \sum\limits_{\beta=1}^3 F_{jj}^\beta A_{jj}^{\alpha\beta}, \label{eq:grad}
\end{align}
where the quasi-particle gradient matrix
\begin{align}
A_{ij}^{\alpha\beta} \equiv \frac{\partial x_j^\beta}{\partial r_i^\alpha},
\end{align}
and the intermediate matrix
\begin{align}
F_{ij}^\alpha\equiv\sum\limits_k M^{-1}_{ik} dM_{kj}^\alpha,
\end{align}
with the single-particle orbital derivatives (w.r. to quasi-particle coordinates)
\begin{align}
dM_{ij}^\alpha \equiv \frac{\partial M_{ij}}{\partial x_j^\alpha}.
\end{align}
Notice, I have made the name change of $\phi\rightarrow M$ from the notations of ref.~\cite{Kwon1993backflow}. This name change is intended to help the reader associate M with the QMCPACK variable \verb|psiM|.
\begin{lstlisting}
// In DiracDeterminantWithBackflow::evaluateLog(P,G,L)
for(int i=0; i<num; i++) // k in above formula
{
  for(int j=0; j<NumPtcls; j++)
  {
    for(int k=0; k<OHMMS_DIM; k++) // alpha in above formula
    {
      myG(i) += dot(BFTrans->Amat(i,FirstIndex+j),Fmat(j,j));
    }
  }
}
\end{lstlisting}

Eq. (\ref{eq:grad}) is still relatively simple to understand. The $A$ matrix maps changes in particle coordinates $d\bs{r}$ to changes in quasi-particle coordinates $d\bs{x}$. Dotting A into F propagates $d\bs{x}$ to $dM$. Thus $F\cdot A$ is the term inside the trace operator of eq.~(\ref{eq:logdet}). Finally, performing the trace completes the evaluation of the derivative.

\subsection{Laplacian}
The particle laplacian is given in ref.~\cite{Kwon1993backflow} as
\begin{align}
L_i \equiv \sum\limits_{\beta} \frac{\partial^2}{\partial (r_i^\beta)^2} \log\det M = \sum\limits_{j\alpha} B_{ij}^\alpha F_{jj}^\alpha - \sum\limits_{jk}\sum\limits_{\alpha\beta\gamma} A_{ij}^{\alpha\beta}A_{ik}^{\alpha\gamma}\times\left(F_{kj}^\alpha F_{jk}^\gamma -\delta_{jk}\sum\limits_m M^{-1}_{jm} d2M_{mj}^{\beta\gamma}\right), \label{eq:lap}
\end{align}
where the quasi-particle laplacian matrix
\begin{align}
B_{ij}^{\alpha} \equiv \sum\limits_\beta \frac{\partial^2 x_j^\alpha}{\partial (r_i^\beta)^2},
\end{align}
with the second derivatives of the single-particles orbitals being
\begin{align}
d2M_{ij}^{\alpha\beta} \equiv \frac{\partial^2 M_{ij}}{\partial x_j^\alpha\partial x_j^\beta}.
\end{align}

Schematically, $L_i$ has contributions from 3 terms of the form $BF, AAFF, \tr(AA,Md2M)$, respectively. $A, B, M ,d2M,$ and $F$ can be calculated and stored before the calculations of $L_i$. The first $BF$ term can be directly calculated in a loop over quasi-particle coordinates $j\alpha$.
\begin{lstlisting}
// In DiracDeterminantWithBackflow::evaluateLog(P,G,L)
for(int j=0; j<NumPtcls; j++)
  for(int a=0; a<OHMMS_DIM; k++)
    myL(i) += BFTrans->Bmat_full(i,FirstIndex+j)[a]*Fmat(j,j)[a];
\end{lstlisting}
Notice $B_{ij}^\alpha$ is stored in \verb|Bmat_full|, NOT \verb|Bmat|. 

The remaining two terms both involve $AA$. Thus, it is best to define a temporary tensor $AA$
\begin{align}
{}_iAA_{jk}^{\beta\gamma} \equiv \sum\limits_\alpha A_{ij}^{\alpha\beta} A_{ij}^{\alpha\gamma},
\end{align}
which we will overwrite for each particle $i$. Similarly, define $FF$
\begin{align}
FF_{jk}^{\alpha\gamma} \equiv F_{kj}^\alpha F_{jk}^\gamma,
\end{align}
which is simply the outer product of $F\otimes F$. Then the $AAFF$ term can be calculated by fully contracting $AA$ with $FF$.
\begin{lstlisting}
// In DiracDeterminantWithBackflow::evaluateLog(P,G,L)
for(int j=0; j<NumPtcls; j++)
  for(int k=0; k<NumPtcls; k++)
    for(int i=0; i<num; i++)
    {
      Tensor<RealType,OHMMS_DIM> AA = dot(transpose(BFTrans->Amat(i,FirstIndex+j)),BFTrans->Amat(i,FirstIndex+k));
      HessType FF = outerProduct(Fmat(k,j),Fmat(j,k));
      myL(i) -= traceAtB(AA,FF);
    }
\end{lstlisting}
Finally, define the single-particle orbital derivative term
\begin{align}
Md2M_j^{\beta\gamma} \equiv \sum\limits_m M^{-1}_{jm} d2M_{mj}^\beta,
\end{align}
then the last term is given by the contraction of $Md2M$ (\verb|q_j|) with the diagonal of $AA$.
\begin{lstlisting}
for(int j=0; j<NumPtcls; j++)
{
  HessType q_j;
  q_j=0.0;
  for(int k=0; k<NumPtcls; k++)
    q_j += psiMinv(j,k)*grad_grad_psiM(j,k);
  for(int i=0; i<num; i++)
  {
    Tensor<RealType,OHMMS_DIM> AA = dot(
      transpose(BFTrans->Amat(i,FirstIndex+j)),
      BFTrans->Amat(i,FirstIndex+j)
    );
    myL(i) += traceAtB(AA,q_j);
  }
}
\end{lstlisting}

\subsection{Wavefunction Parameter Derivative}
In order to use the robust linear optimization method of ref.~\cite{Toulouse2007linear}, the trial wavefunction needs to know its contributions to the overlap and hamiltonian matrices. In particular, we need derivatives of these matrices with respect to wavefunction parameters. As a consequence, the wavefunction $\psi$ needs to be able to evaluate $\frac{\partial}{\partial p} \ln \psi$ and $\frac{\partial}{\partial p} \frac{\mathcal{H}\psi}{\psi}$, where $p$ is a parameter.

When two-body backflow is considered, a wavefunction parameter $p$ enters the $\eta$ function only (eq.~(\ref{eq:quasi})). $\bs{r}$, $\phi$, and $M$ are do not explicitly dependent on $p$. Derivative of the log value is almost identical to particle gradient. Namely, eq. (\ref{eq:grad}) applies upon the substitution $r_i^\alpha\rightarrow p$
\begin{align}
\frac{\partial}{\partial p} \ln\det M = \sum\limits_{j=1}^N \sum\limits_{\beta=1}^3 F_{jj}^\beta \left({}_pC_{j}^{\beta}\right),
\end{align}
where the quasi-particle derivatives are stored in \verb|Cmat|
\begin{align}
{}_pC_{i}^{\alpha} \equiv \frac{\partial}{\partial p} x_{i}^{\alpha}.
\end{align}

The change in local kinetic energy is a lot more difficult to calculate
\begin{align}
\frac{\partial T_{\text{local}}}{\partial p} = \frac{\partial}{\partial p} \left\{ \left( \sum\limits_{i=1}^N \frac{1}{2m_i} \nabla^2_i \right) \ln \det M \right\} = \sum\limits_{i=1}^N \frac{1}{2m_i} \frac{\partial}{\partial p} L_i, \label{eq:dK}
\end{align}
where $L_i$ is the particle laplacian defined in eq.~(\ref{eq:lap}). In order to evaluate eq.~(\ref{eq:dK}), we need to calculate parameter derivatives of all three terms defined in the laplacian evalaution. Namely $(B)(F)$, $(AA)(FF)$, and $\tr(AA,Md2M)$, where I have put parentheses around previously identified data structures. After $\frac{\partial}{\partial p}$ hits, each of the 3 terms will split into two terms by the product rule. Each smaller term will contain a contraction of two data structures. Therefore, we will need to calculate the parameter derivatives of each data structure defined in the laplacian evaluation
\begin{align}
{}_pX_{ij}^{\alpha\beta} \equiv \frac{\partial}{\partial p} A_{ij}^{\alpha\beta} \\
{}_pY_{ij}^{\alpha} \equiv \frac{\partial}{\partial p} B_{ij}^{\alpha} \\
{}_pdF_{ij}^{\alpha} \equiv \frac{\partial}{\partial p} F_{ij}^{\alpha} \\
{}_{pi}{AA'}_{jk}^{\beta\gamma} \equiv \frac{\partial}{\partial p}  {}_iAA_{jk}^{\beta\gamma} \\
{}_p {FF'}_{jk}^{\alpha\gamma} \equiv \frac{\partial}{\partial p} FF_{jk}^{\alpha\gamma} \\
{}_p {Md2M'}_{j}^{\beta\gamma} \equiv \frac{\partial}{\partial p} Md2M_j^{\beta\gamma}
\end{align}
X and Y are stored as \verb|Xmat|, and \verb|Ymat_full| (NOT \verb|Ymat|) in the code. dF is \verb|dFa|. $AA'$ is not fully store, intermediate values are stored in \verb|Aij_sum| and \verb|a_j_sum|. $FF'$ is calculated on-the-fly as $dF\otimes F+F\otimes dF$. $Md2M'$ is not stored, intermediate values are stored in \verb|q_j_prime|.



%% \renewcommand{\chaptername}{}
%% \renewcommand{\thechapter}{}
\chapter*{References}
\addcontentsline{toc}{chapter}{References}
\begin{btSect}{bibliography}
\btPrintCited
\end{btSect}
\end{btUnit}

\appendix
\chapter{Derivation of twist averaging efficiency}
\label{sec:app_ta_efficiency}
In this appendix we derive the relative statistical efficiency of 
twist averaging with an irreducible (weighted) set of k-points 
versus using uniform weights over an unreduced set of k-points 
(\emph{e.g.} a full Monkhorst-Pack mesh).

Consider the weighted average of a set of statistical variables 
$\{x_m\}$ with weights $\{w_m\}$:
\begin{align}
  x_{TA} = \frac{\sum_mw_mx_m}{\sum_mw_m}
\end{align} 
If produced by a finite quantum Monte Carlo run at a set of 
twist angles/k-points $\{k_m\}$, each variable mean $\mean{x_m}$ 
has a statistical error bar $\sigma_m$ and we can also obtain 
the statistical error bar of the mean of the twist averaged 
quantity $\mean{x_{TA}}$:
\begin{align}
  \sigma_{TA} = \frac{\left(\sum_mw_m^2\sigma_m^2\right)^{1/2}}{\sum_mw_m}
\end{align}
The error bar of each individual twist $\sigma_m$, is related to the 
autocorrelation time $\kappa_m$,  intrinsic variance $v_m$, and number 
of post-equilibration Monte Carlo steps $N_{step}$ in the following way:
\begin{align}
  \sigma_m^2=\frac{\kappa_mv_m}{N_{step}}
\end{align}
In the setting of twist averaging, the autocorrelation time and 
variance for different twist angles are often very similar across 
twists and we have
\begin{align}
  \sigma_m^2=\sigma^2=\frac{\kappa v}{N_{step}}
\end{align} 
If we define the total weight as $W$, \emph{i.e.} $W\equiv\sum_{m=1}^Mw_m$, 
for the weighted case with $M$ irreducible twists the error bar is
\begin{align}
  \sigma_{TA}^{weighted}=\frac{\left(\sum_{m=1}^Mw_m^2\right)^{1/2}}{W}\sigma
\end{align}
For uniform weighting with $w_m=1$ the number of twists is $W$ and 
we have
\begin{align}
  \sigma_{TA}^{uniform}=\frac{1}{\sqrt{W}}\sigma
\end{align}
We are interested in comparing the efficiency of choosing weights 
uniformly or based on the irreducible multiplicity of each twist angle 
for a given target error bar $\sigma_{target}$.  The number of Monte Carlo  
steps required to reach this target for uniform weighting is
\begin{align}
  N_{step}^{uniform} = \frac{1}{W}\frac{\kappa v}{\sigma_{target}^2}
\end{align}
while for non-uniform weighting we have
\begin{align}\label{eq:weighted_step}
  N_{step}^{weighted} &= \frac{\sum_{m=1}^Mw_m^2}{W^2}\frac{\kappa v}{\sigma_{target}^2} \nonumber\\
                  &=\frac{\sum_{m=1}^Mw_m^2}{W}N_{step}^{uniform}
\end{align}
The Monte Carlo efficiency is defined as 
\begin{align}
  \xi = \frac{1}{\sigma^2t}
\end{align}
where $\sigma$ is the error bar and $t$ is the total cpu time required 
for the Monte Carlo run.  

The main advantage made possible by irreducible twist weighting is to 
reduce the equilibration time overhead by having fewer twists, and 
hence fewer Monte Carlo runs to equilibrate.  In the context of twist 
averaging, the total cpu time for a run can be considered to be
\begin{align}
  t=N_{twist}(N_{eq}+N_{step})t_{step}
\end{align}
where $N_{twist}$ is the number of twists, $N_{eq}$ is the number of Monte 
Carlo steps required to reach equilibrium, $N_{step}$ is the number 
of Monte Carlo steps included in the statisical averaging as before, 
and $t_{step}$ is the wall clock time required to complete a single 
Monte Carlo step. For uniform weighting $N_{twist}=W$ while for irreducible 
weighting $N_{twist}=M$.

We can now calculate the relative efficiency ($\eta$) of irreducible vs. 
uniform twist weighting with the aim of obtaining a target error bar 
$\sigma_{target}$:
\begin{align}
  \eta &= \frac{\xi_{TA}^{weighted}}{\xi_{TA}^{uniform}} \nonumber \\
       &= \frac{\sigma_{target}^2t_{TA}^{uniform}}{\sigma_{target}^2t_{TA}^{weighted}} \nonumber \\
       &= \frac{W(N_{eq}+N_{step}^{uniform})}{M(N_{eq}+N_{step}^{weighted})} \nonumber \\
       &= \frac{W(N_{eq}+N_{step}^{uniform})}{M(N_{eq}+\frac{\sum_{m=1}^Mw_m^2}{W}N_{step}^{uniform})} \nonumber \\
       &= \frac{W}{M}\frac{1+f}{1+\frac{\sum_{m=1}^Mw_m^2}{W}f}
\end{align}
In this last expression, $f$ is the ratio of the number of usable 
Monte Carlo steps to the number that must be discarded during equilibration 
($f=N_{step}^{uniform}/N_{eq}$) and as before $W=\sum_mw_m$, which is the number of 
twist angles in the uniform weighting case.  It is important to recall 
that $N_{step}^{uniform}$ in $f$ is defined relative to uniform weighting; it is 
the number of Monte Carlo steps required to reach a target accuracy in the 
case of uniform twist weights.

The formula for $\eta$ above can be easily changed with the help of 
Eq. \ref{eq:weighted_step} to reflect the 
number of Monte Carlo steps obtained in an irreducibly weighted run 
instead.  A good exercise is to consider runs one has already completed 
with either uniform or irreducible weighting and calculate the 
expected efficiency change had the opposite type of weighting been used.

The break even point $(\eta=1)$ can be found at a usable step fraction of 
\begin{align}
  f=\frac{W-M}{M\frac{\sum_{m=1}^Mw_m^2}{W}-W}
\end{align}

The relative efficiency $(\eta)$ above is useful to consider in view of certain 
scenarios.  An important case is where the number of required sampling 
steps is no larger than the number of equilibration steps, \emph{i.e.} 
$f\approx 1$.  For a very simple case with 8 uniform twists with 
irreducible multiplicities of $w_m\in\{1,3,3,1\}$ ($W=8$, $M=4$), the 
relative efficiency of irreducible vs. uniform weighting is 
$\eta=\frac{8}{4}\frac{2}{1+20/8}\approx 1.14$.  In this case, 
irreducible weighting is about $14$\% more efficient than uniform weighting.

Another interesting case is where the number of sampling steps you can 
reach with uniform twists before wall clock time runs out is small 
relative to the number of equilibration steps ($f\rightarrow 0$). 
In this limit $\eta\approx W/M$.  For our 8 uniform twist example, this would 
result in a relative efficiency of $\eta=8/4=2$ making irreducible 
weighting twice as efficient.

A final case of interest is where the equilibration time is short 
relative to the available sampling time $(f\rightarrow\infty)$, 
giving $\eta\approx W^2/(M\sum_{m=1}^Mw_m^2)$.  Again for our simple example 
we find $\eta=8^2/(4\times 20)\approx 0.8$, with uniform weighting being 
$25$\% more efficient than irreducible.  

For this simple example, the crossover point for irreducible weighting being 
more efficient than uniform weighting is $f<2$, \emph{i.e.} when the 
available sampling period is less than twice the length of the equilibration 
period.  The expected efficiency ratio and crossover point should be checked 
for the particular case under consideration to inform the choice between   
twist averaging methods.




\begin{btUnit}
\chapter{QMCPACK papers}
% Uset the bibtopic package to generate a bibliography local to this btsection below.
% btPrintAll creates the bibliography citing all entries in the files qmcpack_papers.bib
\begin{btSect}{qmcpack_papers}

  The following is a list of all papers, theses, and book chapters
  known to use QMCPACK. Please let the developers know if your paper
  is missing, if you know of other works, or an entry is incorrect. We
  list papers whether they cite QMCPACK directly or not. This list
  will be placed on the \url{http://www.qmcpack.org} website.

\btPrintAll

\end{btSect}
 
\end{btUnit}
\end{document}
