\begin{DoxyAuthor}{Author}
Jeongnim Kim 
\end{DoxyAuthor}
\begin{DoxyDate}{Date}
2013 May 30
\end{DoxyDate}
\begin{DoxyCopyright}{Copyright}
  
  Creative Commons Attribution-ShareAlike 3.0 Unported License
  
\end{DoxyCopyright}
A pdf verion is available {\tt here}.

Q\-M\-C\-P\-A\-C\-K, an open-\/source Q\-M\-C simulation code written in C++, implements advanced continuum Q\-M\-C algorithms for large-\/scale parallel computers. Its main applications are electronic structure calculatinos of molecular, quasi-\/2\-D and solid-\/state systems. Designed with the modularity afforded by object-\/oriented architecture, it makes extensive use of template metaprogramming to achieve high computational efficiency through inlined specializations. It utilizes a hybrid (Open\-M\-P,C\-U\-D\-A)/\-M\-P\-I approach to parallelization to take advantage of the growing number of cores per S\-M\-P node or G\-P\-Us. Finally, it utilizes standard file formats for input and output in X\-M\-L and H\-D\-F5 to facilitate data exchange.\section{About Development}\label{index_qmcpack_about}
Main contributing authors are Jeongnim Kim and the past and current members of the electron structure group of Profs. Ceperley and Martin at University of Illinois. To cite Q\-M\-C\-P\-A\-C\-K, use these references\-:


\begin{DoxyItemize}
\item J. Kim, K. P. Esler, J. Mc\-Minis, M. A. Morales, B. K. Clark, L. Shulenburger, and D. M. Ceperley, “\-Hybrid algorithms in quantum monte carlo,” Journal of Physics\-: Conference Series, vol. 402, no. 1, p. 012008, 2012.
\item K. P. Esler, J. Kim, L. Shulenburger, and D. M. Ceperley, “\-Fully accelerating quantum monte carlo simulations of real materials on gpu clusters,” Computing in Science and Engineering, vol. 14, p. 40, 2012.
\item Q\-M\-C\-P\-A\-C\-K, {\tt http\-://qmcpack.\-cmscc.\-org/}
\end{DoxyItemize}

Different parts of the code are written by different people, whose names are usually signed at the top of each file. If you are interested in participating in Q\-M\-C\-P\-A\-C\-K development, contact Jeongnim Kim {\tt jeongnim.\-kim@gmail.\-com}. Q\-M\-C\-P\-A\-C\-K is available under University of Illinois/\-N\-C\-S\-A Open Source License (O\-S\-I approved) {\tt http\-://www.\-opensource.\-org/licenses/\-Uo\-I-\/\-N\-C\-S\-A.\-php}.\section{Acknowledgements}\label{index_funding}
We welcome anyone who is interested in Q\-M\-C development and in using Q\-M\-C\-P\-A\-C\-K to their research. Contact any of the owners of this project.

The development of Q\-M\-C\-P\-A\-C\-K has been funded by


\begin{DoxyItemize}
\item Q\-M\-C network supported through the Materials Genome initiatives (D\-O\-E-\/\-B\-E\-S P\-T\-M)
\item Materials Computational Center, supported by the U.\-S. National Science Foundation (N\-S\-F)
\item Q\-M\-C Endstation, supported by the U.\-S. Department of Energy (D\-O\-E)
\item Peta\-Apps, supported by the U. S. National Science Foundation
\end{DoxyItemize}

We also acknowledge the O\-L\-C\-F and A\-L\-C\-F for help and support in accessing their resources, as part of a D\-O\-E I\-N\-C\-I\-T\-E allocation grant supported by U\-S Department of Energy and N\-C\-S\-A, T\-A\-C\-C and N\-I\-C\-S for providing resources as a part of N\-S\-F Tera\-Grid allocation grant.

These guides are not meant to discuss Quantum Monte Carlo methods and is written on the assumption that the users are familiar with various Q\-M\-C algorithms. There are many excellent tutorials and talks on Q\-M\-C methods and numerous published works. A short list includes


\begin{DoxyItemize}
\item {\tt Home page of Prof. David M Ceperley}
\item {\tt Home page of C\-A\-S\-I\-N\-O Q\-M\-C Package}
\item {\tt 2012 Summer School on Computational Materials Science\-: Quantum Monte Carlo\-: Theory and Fundamentals}
\item {\tt Quantum Monte Carlo from Minerals and Materials to Molecules, 2007 Summer School on Computational Materials Science} 
\end{DoxyItemize}